\begin{table}[htpb!]\caption{`Plausibly Exogenous' Bounds} 
\label{TWINtab:Conley}\begin{center}\begin{tabular}{lcccc}
\toprule \toprule 
&\multicolumn{2}{c}{UCI: $\gamma\in [0,\delta]$}&\multicolumn{2}{c}{LTZ: $\gamma \sim U(0,\delta)$}\\ 
\cmidrule(r){2-3} \cmidrule(r){4-5}
&Lower Bound&Upper Bound&Lower Bound&Upper Bound\\
Two Plus&-0.1827&0.0173&-0.1584&-0.0008\\
Three Plus&-0.1659&-0.0019&-0.1485&-0.0155\\
Four Plus&-0.1486&-0.0062&-0.1332&-0.0186\\
Five Plus&-0.1329&0.0203&-0.1182&0.0076\\
\midrule\multicolumn{5}{p{11.6cm}}{\begin{footnotesize}\textsc{Notes:} This table presents upper and lower bounds of a 95\% confidence interval for the effects of family size on (standardised) children's education attainment. These are estimated by the methodology of \citet{Conleyetal2012}  under various priors about the direct effect that being from a twin family has on educational outcomes ($\gamma$). In the UCI (union of confidence interval) approach, it is assumed the true $\gamma\in[0,\delta]$, while in the LTZ (local to zero) approach it is assumed that $\gamma\sim U(0,\delta)$.  In each case $\delta$ is estimated by including twinning in the first stage  equation and observing the effect size $\hat\gamma$.  Estimated $\hat\gamma$'s are (respectively for two plus to five plus):   0.1077, 0.0931, 0.0810, 0.0870.\end{footnotesize}}  
\\ \bottomrule \end{tabular}\end{center}\end{table} 
