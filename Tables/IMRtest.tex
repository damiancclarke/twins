\begin{table}[htpb!]
\caption{Test of hypothesis that women who bear twins have better prior health}\label{TWINtab:IMR}\begin{center}\begin{tabular}{lccc}
\toprule \toprule 
\textsc{Infant Mortality Rate}& Base & +S\&H & Observations \\ \midrule 
\begin{footnotesize}\end{footnotesize}& 
\begin{footnotesize}\end{footnotesize}& 
\begin{footnotesize}\end{footnotesize}& 
\begin{footnotesize}\end{footnotesize}\\ 
Treated (2+)\hspace{5mm}\hspace{5mm}\hspace{5mm}\hspace{5mm}\hspace{5mm}\hspace{5mm}&-1.396**&-1.402**&337378\\
&(0.639)&(0.638)&\\
Treated (3+)\hspace{5mm}&-1.565***&-1.570***&455372\\
&(0.362)&(0.363)&\\
Treated (4+)&-0.634***&-0.620***&443165\\
&(0.066)&(0.066)&\\
Treated (5+)&-0.423***&-0.412***&382405\\
&(0.139)&(0.139)&\\
\midrule\multicolumn{4}{p{10.6cm}}{\begin{footnotesize}\textsc{Notes:} The sample for these regressions consist of all children who have been entirely exposed to the risk of infant mortality (ie those over 1 year of age). Subsamples 2+, 3+, 4+ and 5+ are generated to allow comparison of children born at similar birth orders.  For a full description of these groups see the the body of the paper or notes to tables \ref{TWINtab:IVTwoplus}, \ref{TWINtab:IVThreeplus}, \ref{TWINtab:IVFourplus} or \ref{TWINtab:IVFiveplus} respectively. Treated=1 refers to children who are born before a twin while Treated=0 refers to children of similar birth orders not born before a twin.  Base and S+H controls are described in table \ref{TWINtab:IVTwoplus}.$^{*}$p$<$0.1; $^{**}$p$<$0.05; $^{***}$p$<$0.01 
\end{footnotesize}} \\ \bottomrule 
\end{tabular}\end{center}\end{table}