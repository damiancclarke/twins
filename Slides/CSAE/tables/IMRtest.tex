\begin{table}[htpb!]
%\caption{Test of hypothesis that women who bear twins have better prior health}\label{TWINtab:IMR}
\begin{center}
\scalebox{0.6}{
\begin{tabular}{lccc}
\toprule \toprule 
\textsc{Infant Mortality (per 100 births)}& Base & +S\&H & Observations \\ \midrule 
\begin{footnotesize}\end{footnotesize}& 
\begin{footnotesize}\end{footnotesize}& 
\begin{footnotesize}\end{footnotesize}& 
\begin{footnotesize}\end{footnotesize}\\ 
Treated (2+)\hspace{5mm}\hspace{5mm}\hspace{5mm}\hspace{5mm}\hspace{5mm}\hspace{5mm}&-2.065***&-2.110***&503785\\
&(0.212)&(0.213)&\\
Treated (3+)\hspace{5mm}&-4.619***&-4.632***&686931\\
&(0.201)&(0.201)&\\
Treated (4+)&-4.257***&-4.243***&676303\\
&(0.183)&(0.183)&\\
Treated (5+)&-3.353***&-3.324***&587919\\
&(0.183)&(0.183)&\\
\midrule\multicolumn{4}{p{12.1cm}}{\begin{footnotesize}\textsc{Notes:} The sample for these regressions consist of all children who have been entirely exposed to the risk of infant mortality (ie those over 1 year of age). Subsamples 2+, 3+, 4+ and 5+ are generated to allow comparison of children born at similar birth orders.  For a full description of these groups see the the body of the paper or notes to table \ref{TWINtab:IVAll}. Treated=1 refers to children who are born before a twin while Treated=0 refers to children of similar birth orders not born before a twin.  Base and S+H controls are described in table \ref{TWINtab:IVAll}.$^{*}$p$<$0.1; $^{**}$p$<$0.05; $^{***}$p$<$0.01 
\end{footnotesize}} \\ \bottomrule 
\end{tabular}}\end{center}\end{table}
