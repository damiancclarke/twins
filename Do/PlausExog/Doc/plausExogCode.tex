\documentclass{article}[11pt] % can change class to book, report, letter, etc

\usepackage{amsmath}
\usepackage{amssymb}
\usepackage{appendix}
\usepackage[english]{babel}
\usepackage{bm}
\usepackage{booktabs}
\usepackage[usenames, dvipsnames]{color}
\usepackage{cool}
\usepackage[justification=centering]{caption}
\usepackage{epsfig}
\usepackage{epstopdf}
\usepackage{etex}
\usepackage{framed}
\usepackage{hyperref}
\usepackage[utf8]{inputenc}
\usepackage{lastpage}
\usepackage{lineno}
\usepackage{natbib}
\usepackage{pictex}
\usepackage{placeins}
\usepackage{setspace}
\usepackage{soul}
\usepackage{subfigure}
\usepackage{subfloat}
\usepackage{textcomp}
\usepackage{tikz}
\usepackage{upquote}
\usepackage{verbatim}

\bibliographystyle{abbrvnat}
\bibpunct{(}{)}{;}{a}{,}{,}

\setlength\topmargin{-0.375in}
\setlength\textheight{8.5in}
\setlength\textwidth{6.25in}
\setlength\oddsidemargin{0.1in}
\setlength\evensidemargin{-0.35in}
\setlength\parindent{0.25in}
\setlength\parskip{0.25in}



\title{Example of Extended Conley et al.\ Stata code}
\author{Damian Clarke\thanks{Contact \href{mailto:damian.clarke@economics.ox.ac.uk}
{damian.clarke@economics.ox.ac.uk}.}}
\date{\today}


\begin{document}
\maketitle
\begin{spacing}{1.35}
The following describes an extended version of the Stata code from Conely et al's
(2012) REStat paper on plausibly exogenous IV estimation.  The heart of the code
comes from the excellent implementations on Christian Hansen's site, with a number of 
additional elements. These elements have no major effect on functionality however 
the code now includes extensive error capture, the option to work with missing 
observations and factor variables, and the inclusion of automatic graphing to allow 
for the generation of figures such as Conley et al.'s figures 1 and 2.  A Stata help 
file has also been written describing how to run the code, the return values issued, 
and a number of examples based on the ``Plausibly Exogenous'' dataset.

The code---an ado file called \texttt{plausexog.ado} and \texttt{plausexog.hlp}---largely
follows the syntax in the original Hansen code, however some slight tweaks have been
incorporated to include additional options, and to fall precisely in line with the 
syntax for Stata's inbuilt \texttt{ivregress} command.   The \texttt{plausexog} code
uses Stata's \texttt{ivregress} command rather than the (no longer documented) 
\texttt{ivreg}.  This has a very slight effect on the confidence intervals produced, as
Stata's old command resulted in (very) slightly wider confidence intervals than 
\texttt{ivregress}\footnote{For example, the following gives identical point estimates
but different confidence intervals while in theory it is 2sls estimation in each case:\\
\texttt{. clear all}\\
\texttt{. set obs 1000}\\
\texttt{. gen y=rnormal()}\\
\texttt{. gen x=rnormal()}\\
\texttt{. gen z=rnormal()}\\
\texttt{. ivreg y (x=z)}\\
\texttt{. ivregress 2sls y (x=z)}
}.  As \texttt{ivreg} is no longer documented (and standard error
calculation is compiled code), it is not entirely clear why this is so, but given
the small magnitude of the difference I suspect this may be a degree of freedom 
correction not incorporated into \texttt{ivreg}.

Below I provide three brief examples of the functionality of \texttt{plausexog}.  These
are (in order) the union of confidence approach using the 401(k) data, the local-to-zero
approach with the same data, and finally an example with graphical output.  Full details
of syntax and so forth are available in the help file if interested.

\section{Union of Confidence Interval Example (plus setup)}
\end{spacing}
\begin{spacing}{1}
\begin{verbatim}
.  webuse set http://users.ox.ac.uk/~ball3491/
(prefix now "http://users.ox.ac.uk/~ball3491")
.  webuse Conleyetal2012
.  local xvar i2 i3 i4 i5 i6 i7 age age2 fsize hs smcol col marr twoearn 
>  db pira hown
.  plausexog uci net_tfa `xvar' (p401  = e401), gmin(-10000) gmax(10000) 
>  grid(2) level(.95) vce(robust);

Estimating Conely et al.'s uci method
Exogenous variables: i2 i3 i4 i5 i6 i7 age age2 fsize hs smcol col marr
 twoearn db pira hown
Endogenous variables: p401
Instruments: e401


Conley et al (2012)'s UCI results                     Number of obs =     9915
------------------------------------------------------------------------------
Variable    Lower Bound     Upper Bound
------------------------------------------------------------------------------
i2          -521.86846      2458.0638
i3          -104.87266      4488.6619
i4          2312.9554       8309.6845
i5          6354.6884       14456.23
i6          17131.861       26601.305
i7          49954.621       74997.276
age         -2923.7451      -696.17742
age2        14.900429       42.438197
fsize       -1471.6129      20.443792
hs          181.41554       5334.5844
smcol       -656.13757      6145.4425
col         1187.9477       9119.9383
marr        622.05162       8280.6488
twoearn     -19344.348      -10750.815
db          -5695.9016      186.59616
pira        27566.006       35785.257
hown        2473.4966       5931.7854
p401        -4886.9191      31332.203
_cons       -1781.7058      39584.939
------------------------------------------------------------------------------
\end{verbatim}


\section{Local to Zero Approach (symmetrical around zero, normality)}
\begin{verbatim}
. matrix omega_eta = J(19,19,0)

. matrix omega_eta[1,1] = 5000^2

. matrix mu_eta = J(19,1,0)

. plausexog ltz net_tfa `xvar' (p401 = e401), omega(omega_eta) mu(mu_eta) 
> level(.95) vce(robust)

Estimating Conely et al.'s ltz method
Exogenous variables: i2 i3 i4 i5 i6 i7 age age2 fsize hs smcol col marr 
  twoearn db pira hown
Endogenous variables: p401
Instruments: e401


Conley et al. (2012)'s LTZ results                    Number of obs =      9915
------------------------------------------------------------------------------
             |      Coef.   Std. Err.      z    P>|z|     [95% Conf. Interval]
-------------+----------------------------------------------------------------
        p401 |   13222.14   7423.859     1.78   0.075    -1328.353    27772.64
          i2 |   962.1541   702.5647     1.37   0.171    -414.8473    2339.156
          i3 |   2190.277   1006.502     2.18   0.030     217.5697    4162.983
          i4 |   5313.626   1423.773     3.73   0.000     2523.082     8104.17
          i5 |   10400.47    2018.09     5.15   0.000     6445.082    14355.85
          i6 |   21859.43   2245.767     9.73   0.000     17457.81    26261.05
          i7 |   62464.83   5893.325    10.60   0.000     50914.12    74015.53
         age |  -1811.558   537.0111    -3.37   0.001    -2864.081   -759.0358
        age2 |   28.68893   6.718589     4.27   0.000     15.52074    41.85712
       fsize |  -724.4649   378.4223    -1.91   0.056    -1466.159    17.22925
          hs |   2761.253   1245.912     2.22   0.027     319.3096    5203.196
       smcol |   2750.739   1646.118     1.67   0.095    -475.5927     5977.07
         col |   5161.979   1929.035     2.68   0.007     1381.139    8942.818
        marr |   4453.186   1855.549     2.40   0.016      816.376    8089.996
     twoearn |  -15051.59   2126.671    -7.08   0.000    -19219.79   -10883.39
          db |   -2750.19   1240.347    -2.22   0.027    -5181.226   -319.1539
        pira |   31667.72    1766.43    17.93   0.000     28205.58    35129.86
        hown |   4200.889   775.1211     5.42   0.000     2681.679    5720.098
       _cons |   18929.86    9784.79     1.93   0.053     -247.981    38107.69
------------------------------------------------------------------------------

\end{verbatim}


\section{Local to Zero Approach with Graphing (uniform positive)}
\begin{verbatim}
. foreach num of numlist 0(1)5 {
      matrix om`num' = J(19,19,0)
      matrix om`num'[1,1] = ((`num'/5)*10000/sqrt(12))^2
      matrix mu`num' = J(19,1,0)
      matrix mu`num'[1,1] = (`num'/5)*10000/2
      local d`num' = (`num'/5)*10000
  }

. qui plausexog ltz net_tfa `xvar' (p401 = e401), omega(omega_eta) mu(mu_eta)
> level(.95) vce(robust) graph(p401) graphomega(om0 om1 om2 om3 om4 om5)
> graphmu(mu0 mu1 mu2 mu3 mu4 mu5) graphdelta(`d0' `d1' `d2' `d3' `d4' `d5')

\end{verbatim} 

\begin{figure}[htpb]
\begin{center}
\includegraphics[scale=0.89]{ConleyetalLTZ.eps}
\caption{Automatically produced graph (as per Conely et al figure 2)}
\label{fig:LTZ}
\end{center}
\end{figure}



\end{spacing}
\end{document}
