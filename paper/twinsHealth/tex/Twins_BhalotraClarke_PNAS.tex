\documentclass[11pt]{article}
\usepackage{scicite}
\usepackage{times}


\usepackage{bm}
\usepackage{booktabs}
\usepackage{graphicx}
\usepackage[capposition=bottom]{floatrow}
\usepackage{subfloat}
\usepackage{subcaption}
\usepackage{lineno}
\usepackage{longtable}
\usepackage{lscape}
\usepackage{colortbl}
\usepackage{setspace}

\definecolor{LightCyan}{rgb}{0.7,0.9,1}


\topmargin -1.0cm
\oddsidemargin 0cm
\textwidth 16.4cm
\textheight 22cm
\footskip 1.0cm


\newenvironment{sciabstract}{%
  \begin{quote} \bf}
{\end{quote}}

\renewcommand\refname{References and Notes} 

\newcounter{lastnote}
 \newenvironment{scilastnote}{%
   \setcounter{lastnote}{\value{enumiv}}%
   \addtocounter{lastnote}{+1}%
   \begin{list}%
     {\arabic{lastnote}.}
     {\setlength{\leftmargin}{.22in}}
     {\setlength{\labelsep}{.5em}}}
                {\end{list}}
                
               
% Include your paper's title here

\title{Twin Births and Maternal Condition}
\author{Sonia Bhalotra$^{1\ast}$ \& Damian Clarke$^2$      \\
  \normalsize{$^{1}$Department of Economics and Institute for Social and Economic Research,} \\
  \normalsize{University of Essex, Essex, United Kingdom.}\\
      \normalsize{$^{2}$ Department of Economics, University of Santiago de Chile, Chile.}\\
      \\
      \normalsize{$^\ast$To whom correspondence should be addressed; E-mail:  srbhal@essex.ac.uk}
    }
    
\date{}


\begin{document}

\baselineskip24pt

\maketitle

\noindent\textbf{Abstract}: Twin births are often construed as a natural experiment in the social and natural sciences on the premise that the occurrence of twins is quasi-random.  We present new population-level evidence that challenges this premise. Using individual data for more than 18 million births (of which more than half a million are twins) in 72 countries, we demonstrate that mother's health is systematically positively associated with the probability of a twin birth. This holds for many indicators of mother's health including not only height and weight but also anemia, hypertension, diabetes, asthma, diet, smoking and the consumption of alcohol and drugs during pregnancy, and indicators of prenatal care availability. We are agnostic on the question of whether twin conceptions are random, our hypothesis being that carrying twins to term is particularly demanding, so stressors of maternal health lead to selective miscarriage of twins. We establish this using US vital statistics data on foetal death. The estimated associations of maternal health and twin births are sizeable, evident in richer and poorer countries, evident even when we use a sample of women who do not use IVF, and they hold conditional upon mother's age and parity. The findings are of significance given that social and environmental factors that modify maternal health are correlated with socio-economic status and preferences, implying re-assessment of research seeking to understand child development, career costs of children and the role of nature vs nurture.
\vspace{4mm}

\noindent\textbf{Significance Statement}: This paper demonstrates that giving birth to twins is positively associated with a wide range of measures of maternal condition and health-seeking behaviours. Using population-level data on 18 million births in 72 countries we show that the distribution of twins is skewed in favour of healthy women, marked by nutritional status, medical conditions and pregnancy behaviours including drug and alcohol consumption. We present evidence of selective miscarriage as a mechanism. The findings are of significance given that social and environmental factors that modify maternal health are correlated with socio-economic status and preferences, implying re-assessment of a wide range of research seeking to understand child development, career costs of children and the role of nature vs nurture.



\newpage
\section*{Introduction}
%DD is 2.89% the percent of birth-events (mother level statistic) or the percent of births (child level stat). strictly speaking for former i'd use the language ```twin births''' and for latter i'd use ```twins'''. i am changing below to twins but we need to correct that dep how 2.89 is defined.
In behavioural genetics, demography and psychology, monozygotic twins are studied to assess the importance of nurture relative to nature\cite{Thorndike1905,Boomsmaetal2002,Poldermanetal2015,Phillips1993,BouchardPropping1993,McClearnetal1997,Nisen2013}. In the social sciences, twin births are also used to denote an unexpected increase in family size which assists causal identification of the impact of fertility on investments in children and on women's labour market participation\cite{WolpinRosenzweig2000,RosenzweigWolpin1980,BronarsGrogger1994}. A premise of these studies is that twin births are quasi-random, or independent of characteristics of the mother that influence the environment in which children are reared, or the mother's preferences over labour supply.  This assumption of quasi-randomness is central in the validity of natural experiments which use twins as either an exogenous shock to family size, or as a natural treatment--control pair.  If twins are non-random for reasons which are not perfectly observable, this implies that any studies based on natural experiments with twins will lead to inconsistent estimates, or findings which are incorrectly applied to the population at large.

While twins are known to occur more frequently among older mothers, mothers with greater prior fertility, and among certain races and ethnicities\cite{Hall2003}, these variables are in practice all observable.  Similarly, the well-documented fact that women who engage in artificial reproductive technologies (ART) are much more likely to give birth to twins\cite{Vitthalaetal2009} is, in theory at least, an observable variable which is measured in many birth registries and can be included in statistical models.  Beyond a small number of genetic predictors, no population-level evidence exists to suggest that twins may depend more widely on maternal health or health-seeking behaviours.

In this paper we cast considerable doubt on the conditional-randomness assumption for twinning.  We document that twin births are positively associated with a wide range of maternal health stocks prior to pregnancy and behaviours prior to and during pregnancy including smoking, diet, drug-taking and chronic disease burden.  Over a wide range of time periods and in both developed and developing countries we document that twin mothers are considerably healthier than non-twin mothers. Further, this leads to a socio-demographic gradient in twinning, which we document when demonstrating that twin mothers on average have higher levels of completed education than non-twin mothers. This is true even among a population of women not engaging in ART.  To document this, we collected data on 18,652,028 births, of which 539,544 (2.89\%) are twins. We accessed data that contain indicators of maternal health in addition to age, parity, race and ethnicity, which are known determinants of twin birth\cite{Bulmer1970}.

We document that selective miscarriage acts as the mechanism to (at least partially) explain why greater rates of twin births are observed among healthier women.  If less healthy women who conceive twins are selectively more likely to suffer miscarriage, and if this gradient is steeper than that among singleton conceptions, we will observe a population of twins which is concentrated among healty mothers.
%SONIA: Should we link at this point to Trivers-Willard?
Pooling all fetal deaths and live births observed in United States Vital Statistics data we observe precisely this pattern, both when considering pre-pregnancy health stocks such as suffering from anemia, and pregnancy behaviours such as smoking and alcohol consumption.  These findings are relevant for a twin literature which stretches back for more than a century\cite{Thorndike1905}, and which is cited in the definition of large-scale population policies.

\section*{Results}
Table 1 presents evidence from four rich countries and 68 developing countries that twin births occur disproportionately among healthier mothers, where health refers to health stocks prior to pregnancy and health-seeking and risk-avoiding behaviours during pregnancy. Geographic coverage of the data is presented in supplementary figure S1, data sources are described in Materials and Methods and summary statistics are in table S1. The displayed coefficients express the effect of a 1 standard deviation ($\sigma$) increase in an indicator of maternal health on the change in the likelihood, in percentage points (pp), of a mother giving (live) birth to twins (unstandardised estimates are in table S2). For every reported indicator, p$<$0.01 unless otherwise stated. We observe that being underweight, having morbidities prior to conception (hypertension, diabetes, kidney disease), smoking, and drug or alcohol consumption during pregnancy, all significantly reduce the probability of a twin birth. On the other hand, a healthy diet in pregnancy, height (an indicator of stock of health\cite{Silventoinen2003,BhalotraRawlings2013}) and greater access to prenatal and medical care raise the likelihood of giving birth to twins. Maternal education is also a significant predictor of twins, consistent with education promoting health, for example by increasing the ease with which individuals acquire and process health-related information\cite{Kenkel1991,CutlerLlerasMuney2010}. Statistical significance of the health indicators in Table 1 is robust to running multivariate regressions which condition on all available indicators of the mother's health, including education (table S3).  Across samples and health indicators, a 1$\sigma$ improvement in the indicator increases the likelihood of twinning by between 0.24 (fresh fruit consumption) and 22.0 percent (height), with most effects tending to be in the region of 6-12\%.

Previous research has documented that twins have different endowments from singleton children, for example, twins are more likely to have low birth weight and congenital anomalies\cite{Hall2003}. We focus not on differences between twins and singletons but rather on differences between mothers of twins and singletons, which indicate whether occurrence of twin births is quasi-random. We now elaborate the evidence in Table 1 and then provide evidence of the hypothesized mechanism of selective miscarriage.

There is a documented positive association of Artificial Reproductive Technology (ART) with the likelihood of twin births\cite{Vitthalaetal2009}, and a positive socioeconomic gradient in selection into ART. To demonstrate that our hypothesis holds independently of ART use we provide estimates for the USA using the universe of mothers giving birth between 2009-2013, excluding women who reported using ART. A 1$\sigma$ increase in rates of smoking in each trimester is associated with a 0.20-0.25 pp reduction in the risk of a twin birth (p$<$0.001 in each case). Smoking in the third trimester imposes the largest reduction, consistent with evidence that adverse effects of smoking on birth weight (a marker of foetal health) are largest in the third trimester\cite{Bernsteinetal2005}; also see supplementary table S4.  These effects of smoking are sizeable, being about 10\% of the mean rate of twinning. 

Diabetes or hypertension prior to pregnancy has similar standardized effects, reducing the likelihood of twin birth by between 0.2 and 0.3 pp. Height and education have larger standardized effects, of 0.63 and 0.81 pp respectively, and being underweight is associated with a 0.15 pp decrease in the probability of twins. Estimates for the 1.6 percent of women using ART are in table S5 and are, with the exception of being underweight, larger and statistically significant for every indicator, underlining the additional sensitivity of birth outcomes in this group.

Analysis of birth registers from Sweden for years 1993-2012 indicates similar standardised effect sizes for smoking, diabetes, height and underweight to those for US women. The effect of hypertension is smaller, at close to 0.10 pp. These data additionally record asthma, which reduces the risk of twin births by 0.015 pp.


Survey data from Avon county UK 1991-1992, Chile 2006-2009 and 68 developing countries in 1961-2012 exhibit similar patterns for anthropometric indicators of health, risky behaviours and pre-pregnancy illnesses (see Table 1). Here we report estimates for indicators specific to these additional datasets. The UK data indicate that the standardised effect of eating healthily during pregnancy is a 0.54 pp increase in the likelihood of having twins, and the Chilean data indicate that frequent drug use during pregnancy reduces the probability of twins by 0.17 pp, an estimate similar to that for frequent alcohol consumption in this sample. In the developing country sample we observe availability of medical facilities and prenatal care at the geographic cluster level, which is pertinent since health service coverage in low-income countries is far from universal. We estimate that a 1$\sigma$ increase in  availability of doctors or nurses is associated with a 0.092 pp and 0.065 pp increase in the likelihood of twins respectively. 

In figure \ref{fig:countryEsts} we display estimates for height for every country for which heights data are available, unravelling countries in the developing world sample. Height is the indicator of health most widely measured in birth and demographic data and several studies show that it responds to infection and nutritional scarcity in the growing years, for instance individuals exposed to famine and war have been shown to have lower stature in adulthood, other things equal\cite{Silventoinen2003,Bozzolietal2009,Wangetal2010,Akreshetal2012}. Moreover, previous research has shown widespread associations of short stature among mothers with the risk of low birth weight and infant mortality among their children\cite{BhalotraRawlings2013}.  Figure \ref{fig:countryEsts} shows that in 68 of 70 countries, twin mothers are on average significantly taller than non-twin mothers. As the comparison is within country, it nets out country differences including differences in the genetic pool. A similar result obtains for mother's education\cite{Kenkel1991,CutlerLlerasMuney2010} (see figure S2).

Since many women in poorer countries are under-nourished, it seems plausible that their resources are particularly challenged in carrying twins to term. As a result, we may expect that income growth and poverty reduction attenuate the association of mother's health and twin births. To assess this, we need a comparable index of mother's health for countries that span a range of income levels. As height is widely available, we plot the point estimates from figure \ref{fig:countryEsts} against GDP per capita in figure \ref{fig:GDPEsts}. The estimates lie above the zero line, indicating that the relationship persists in high income countries. There are no comparable data for other indicators but while nutritional status of women is worse in poorer countries, it is unclear that risky behaviours in pregnancy are.  Figure S2 shows that the positive association of education with twinning is also evident across income levels. As in figure \ref{fig:GDPEsts}, the estimates for most countries lie above the zero line and, overall, there is a slight positive association of the education difference with country income, consistent with studies showing that education is most likely to benefit health when medical technology is changing quickly\cite{LlerasMuneyGlied2008}.

We hypothesize that the process by which the association of a woman's health and her likelihood of a twin birth emerges is selective miscarriage. It has been documented that the biological demands of twin pregnancies are higher than the demands of non-twin pregnancies\cite{Shinagawaetal2005,Kahnetal2003} and also that, in general, healthier mothers are less likely to miscarry\cite{Garciaetal2002}. Using vital statistics data for the United States we present new evidence on the interaction of these two effects, summarized in figure \ref{fig:mech}, with full results in tables S6. The evidence confirms previous research showing that the spontaneous abortion rate among twins is about three times that among singletons\cite{Boklage1990}. It adds new evidence of steeper gradients in indicators of mother's health for twins than for singletons. For example, a 1$\sigma$ increase in rates of smoking during pregnancy whilst carrying a singleton elevates the risk of miscarriage by 0.431 fetal deaths per 1,000 live births. The corresponding risk elevation among mothers pregnant with twins is an increase of 0.787 fetal deaths, which is almost twice the risk. Alcohol consumption in pregnancy is similarly almost twice as risky for women carrying twins, and the risks associated with anemia are about three times as high. We also show that a college education modifies the difference in miscarriage probabilities more than three times as much when the mother is carrying twins than when she is carrying a singleton.  Unstandardised results are reported in figure S3.


Importantly, these results establish a plausible mechanism for the associations that we document in Table 1. As miscarriage is conditional upon conception, figure \ref{fig:mech} demonstrates the empirical relevance of health and health-related behaviours that modify the probability of taking twins to term. We are agnostic on the question of whether maternal health modifies the probability of a twin conception.

\section*{Discussion}
Our research relates to a vast literature documenting that a mother's health and her environmental exposure to nutritional or other stresses during pregnancy influence birth outcomes, with many studies documenting lower birth weight\cite{CurrieMoretti2007,Bernsteinetal2005,SerranoDomeque2014}. If birth weight is the intensive margin, we may think of miscarriage as the extensive margin response, or the limiting case of low birth weight. Trivers and Willard (1973) made an argument similar to ours but pertaining to the distribution of sons across women\cite{TriversWillard1973}. They observed that since the male foetus is more vulnerable to adverse health conditions, sons are more likely to be born of healthy mothers. As for twins, so for sons, selective miscarriage is the suggested mechanism.

While other critiques have been directed at twin studies, the systematic tendency for maternal health and pregnancy behaviours to influence the distribution of twins across families appears not to have been previously documented using population-level data. Since health and healthy behaviours are positively associated with socioeconomic status, our results imply that twin births are more prevalent among women of high socio-economic status, and we have demonstrated this using education as a marker. Moreover, we show that this is a widespread association, independent of the positive association of IVF use and twin births with socioeconomic status.

Studies using twin births can adjust for demographic determinants but it is virtually impossible to fully adjust for the range of potential indicators of maternal health since factors like anxiety, detailed diet or pollution exposure, to take a few examples, are not commonly recorded. This limits the extent to which the findings of twin studies can be generalised, for example studies of `nature versus nurture' should be interpreted as illustrating the influence of nurture in selectively healthy environments. Social science studies that deploy the occurrence of twin births as an exogenous shock to family size will tend to under-estimate the effects of additional births on sibling outcomes resulting from potentially diluted parental investment and similarly under-estimate the impact of additional children on women's careers if, as is common, mother's health and health-related behaviours are correlated with these outcomes. This is relevant, amongst other things, to understanding the full import of policies designed to incentivize increased fertility in Europe, and policies such as the One Child Policy in China that penalized fertility.

\section*{Materials and Methods}
\subsection*{Data Collection and Generation}
We collect microdata from multiple sources seeking data that contain measures of a woman's fertility history and measures of her health stocks before pregnancy and/or health behaviors during or prior to pregnancy.  The data sources in table 1 are: the United States Vital Statistics Data (``National Vital Statistics System''), 2009-2013; the Swedish Medical Birth Register, 1993-2012; the Demographic and Health Surveys (DHS) for all countries, 1961-2012; the Chilean Survey of Early Infancy (ELPI), 2006-2009; and the Avon Longitudinal Study of Parents and Children (ALSPAC, United Kingdom), 1991-1992.  The years given with each set of data refer to the birth years covered.  The first two of these data sources (USA and Sweden) are complete vital statistics data, while the remainging sources come from hosuehold surveys.

Many other publicly available vital statistics data including from Mexico, Spain, and India, do not contain  measures of mother's health.  The geographic distribution and availability of birth records linked to maternal health measures is described in figure S1. In all cases we retain mothers who are aged 18-49 years at time of birth, and remove births with multiplicity of three or above.  Where women's anthropometric measures are included, we trim the sample to exclude women with reported heights of less than 70cm or greater than 240cm, or BMI of greater than 50.  Conditional regressions are always estimated controlling for fixed effects for mother's age, completed fertility, and child's year of birth. Further information relating to each dataset is provided in Extended Materials in Supplementary Information and summary statistics are in Tables S1 and S8.  

\subsection*{Methods}

\paragraph{Modelling Twin Predictors}
In Table 1 for each of the countries for which we have measures of twin birth and maternal health, we estimate the following regression:
\begin{equation}
  twin_{ijt}=\alpha_0 + \alpha_1 Health_j + \phi_t + f(age_j) + b(order_i) + \varepsilon_{ijt}.
\end{equation}
For each birth $i$ occurring to mother $j$ in year $t$, we regress the twin status of the birth (100 if birth $i$ is a twin, 0 otherwise), on the mother's health indicator $Health_j$. We estimate separate regressions for each available health measure $Health_j$. Fixed effects are included for mother's age at birth and the birth order of the child, denoted $f(\cdot)$ and $b(\cdot)$ respectively, to take into account the well-known established biological relationships of twinning with maternal age, and birth order. In US and Swedish data where gestational length is reported, we control for it given that twin gestation is on average shorter, and potentially correlated with maternal health condition (\emph{29}).

Standard errors are clustered at the level of the mother to allow for arbitrary correlations of stochastic elements ($\varepsilon_{ijt}$) across children born to the same mother. In cases where survey data are used, observations are weighted to be representative of the population from which they are drawn. If twins are random (conditional on the well known  hormonal predictors which are correlated with age, race and completed fertility (\emph{16})), then once conditioning on these variables, the probability of twin birth should be uncorrelated with maternal health. For each of the $Health_j$ measures used we are thus interested in testing the null hypothesis: $H_0: \alpha_1=0$ versus the alternative $H_1: \alpha_1\neq0$.  We present each of these point estimates and 95\% confidence intervals in figure S5 which constitutes a summary representation of Table 1. 

As a supplementary test, we present a table which displays \emph{conditional} results, where a child's twin status is simultaneously regressed upon all maternal health conditions or behaviours.  In this case, the estimated equations are:
\begin{equation}
  \label{reg:twincond}
  twin_{ijt}=\alpha'_0 + \bm{\alpha'_1} \bm{Behaviours}_j + \bm{\alpha'_2} \bm{Conditions}_j + \phi_t + f(age_j) + b(order_i) + \varepsilon'_{ijt}.
\end{equation}

The vectors of variables $\bm{Behaviours}_j$ and $\bm{Conditions}_j$ refer to all available health variables, and now, along with tests for the significance of each variable separately, we are interested in the joint (F-)test that $H_0:\bm{\alpha'_1}=\bm{\alpha'_2}=0$.  As before, rejection of the null casts doubt on the veracity of the ``as good as random'' assumption regarding twin births.

\paragraph{Mother's height (Figure 1 and Figure 2)}
In comparing twin mothers to non-twin mothers (figure 1), for each country the following models are estimated:
\begin{eqnarray}
  Height_{jt}&=&\beta'_0 + \beta'_1 Twin_{jt} + f(age_j) + b(fertility_j) + \varepsilon_{jt}.\\
  Education_{jt}&=&\beta_0 + \beta_1 Twin_{jt} + f(age_j) + b(fertility_j) + \varepsilon_{ijt}
\end{eqnarray}

The mother's height in centimeters ($Height_{jt}$) is regressed on an indicator of whether she has ever given birth to a twin.  Once again, controls for mother's average age at time of birth, and total completed fertility are included.  For each country a separate coefficient and confidence interval is produced.  The coefficient is interpreted as the conditional difference in mother's outcomes between twin and non-twin mothers.  If mothers who give birth to twins have similar health stocks to mothers who do not give birth to twins, in each case we should fail to reject the null hypothesis: $H_0: \beta_1=0$.  The estimates are displayed in figure 1, and plotted against each country's PPP adjusted GDP per capita in figure 2.  Similar plots for education are in figure S2.

\paragraph{Modelling Selective Miscarriage by Birth Type}
To test whether twin  pregnancies are more likely to terminate in foetal death then singleton pregnancies when subject to similar stresses, we estimate the following regressions using United States Vital Statistics data.  The data used combine all live births from the vital statistics with all observed foetal deaths (occurring at 20 weeks or greater of gestation).  We estimate:
\begin{eqnarray}
  FoetalDeath_{ijt} &=& \gamma_0 + \gamma_1 Twin_{ijt} + \gamma_2 Health_{jt} + \gamma_3 Twin\times Health_{ijt} + \nonumber \\ && \phi_t + f(age_j) + b(fertility_j) + \nu_{ijt}.
\end{eqnarray}

$FoetalDeath_{ijt}$ is a binary variable (multiplied by 1,000) indicating whether a birth was taken to term (coded as 0) or resulted in a miscarriage (coded as 1).  As above, $i$ indicates a conception leading to birth or fetal death, $j$ a mother, and $t$ the time period. This is then regressed on the twin status of the pregancy (1 if twins, 0 if singleton), a variable recording an indicator of maternal health and an interaction between twins and mother's health. The interaction term is the focus of our interest and this what there is relatively scarce evidence on.

Estimated coefficients represent the average rate of observed miscarriage (per 1,000 live births registered in US Vital Statistics) for each of the four groups: twins and singletons whose mothers do and do not engage in health behaviour or have the stated health characteristic. The coefficient of interest $\gamma_3$ is the differential effect of the variable $Health_{jt}$ on twin conceptions. The measures of health are behaviours observed entirely before birth (eg smoking or drinking) or health conditions observed entirely before conception (eg anemia). If $\gamma_3=0$, this suggests that twin fetuses are as likely to miscarry as singleton fetuses when exposed to health (dis)amenity $Health_{jt}$.  This then leads to the null hypothesis that $H_0: \gamma_3=0$. Full regression results are presented in table S6, and average rates of miscarriage for each of the four groups (healthy singleton mothers, unhealthy singleton mothers, healthy twin mothers, and unhealthy twin mothers) are displayed in figure 3 of the text. Figure S4 replicates these results, conditioning on fixed effects for mother's age, total fertility and child year of birth.

\paragraph{Code Availability}
Custom computer code to generate all results from raw data is publicly available. The code and data is available at \texttt{http://dx.doi.org/10.7910/DVN/KH06MG} on the Harvard Dataverse, including documentation describing the code and its usage. Two of the data sources used (The Swedish Medical Birth Registry and the UK ALSPAC sample) need to be applied for. In these cases we provide all generating and estimation code which can be run once the data are available. Along with this source code, we provide log files documenting analysis with the original data.  The rest of the data sources are publicly available, and as such our files are freely available for download without restriction from the project repository.



\bibliographystyle{naturemag}
\bibliography{refs}

%\vspace{4mm}
%\noindent\textbf{Supplementary Materials}\\
%Materials and Methods \\
%Figures S1-S5 \\
%Tables S1-S9 \\
%References 30-32
%\nocite{Morrison2005}
%\nocite{Martinetal2013}
%\nocite{EPC2003}
%\nocite{LlerasMuneyGlied2008}



\section*{Tables and Figures}
%\clearpage
%\thispagestyle{empty}
\input{twinEffectsUncond.tex}
\clearpage

\begin{spacing}{1}
\begin{figure}[htpb!]
%\begin{subfigure}{.5\textwidth}
  \includegraphics[scale=1.1]{./HeightDif.eps}
%\end{subfigure}%
%\begin{subfigure}{.5\textwidth}
%  \includegraphics[scale=0.6]{./smokeCoefs.eps}
%\end{subfigure}
%\captionsetup{labelformat=empty}
\vspace{5mm}
\caption{\textbf{Positive association of mother's height with the probability of a twin birth}: {\footnotesize Point estimates of the average difference in height between mothers of twin and singleton births are presented along with the 95\% confidence intervals for each country for which the required microdata are available. Sources of data and estimation techniques are described in the Methods section. When based on survey data, each point is weighted to be nationally representative, and if based on vital statistics data, the universe of births is included. The estimates are conditioned upon total fertility, mother's age and child year of birth using multivariate regression. Supplementary figure S2 presents a similar plot replacing height with the education of the mother.}}
\label{fig:countryEsts}
\end{figure}
%Damian : Are UK and Chile in Fig 1? I could only spot US and Sweden.
%%%%% Sonia: In Chile we unfortunately don't have height.  In UK I didn't include it as it's non-representative of the country. In general, the  figure has a very wide confidence 


\begin{figure}[htpb!]
%\begin{subfigure}{.5\textwidth}
  \includegraphics[scale=0.85]{./heightGDP.png}
%\end{subfigure}%
%\begin{subfigure}{.5\textwidth}
%  \includegraphics[scale=0.45]{./educGDP.png}
%\end{subfigure}
%\captionsetup{labelformat=empty}
\vspace{5mm}
\caption{\textbf{The positive association of mother's height and twin birth is evident at all income levels}: {\footnotesize The correlation of the average height differential between twin and singleton mothers in a country with the country's log GDP per capita is plotted.  Estimates for the height differential are calculated using the same controls and methodology as in figure 1.
%Damian The log of GDP per capita comes from the World Bank Data Bank (indicator NY.GDP.PCAP.PP.KD), and is expressed at Purchasing Power Parity. I took this out, it can go in Methods/data.  
Each circle represents a country and the size of the circle indicates the proportion of births in the country that are twins. Circles above the horizontal dotted line imply that mothers of twins are taller on average. The global correlation between the height difference and GDP conditional on continent fixed effects is 0.259 (t-statistic 1.95).}} %Extended data figure 4 replicates these figures using standardized (Z-scores) in each country.}}
\label{fig:GDPEsts}
\end{figure}

%Damain: In Fig 2, Notes to do not mention conditioning vars but in fig 1 they mention fertlity and age so I assume these controls are same? Should state them. 
%%%%% Sonia: Thanks, I have added this.

\begin{figure}[htpb!]
\begin{subfigure}{.5\textwidth}
  \includegraphics[scale=0.59]{./DeathsZ_smokes_Uncond.eps}
\end{subfigure}%
\begin{subfigure}{.5\textwidth}
  \includegraphics[scale=0.59]{./DeathsZ_drinks_Uncond.eps}
\end{subfigure}
\begin{subfigure}{.5\textwidth}
  \includegraphics[scale=0.59]{./DeathsZ_noCollege_Uncond.eps}
\end{subfigure}%
\begin{subfigure}{.5\textwidth}
  \includegraphics[scale=0.59]{./DeathsZ_anemic_Uncond.eps}
\end{subfigure}
%\captionsetup{labelformat=empty}
\vspace{5mm}
\caption{\textbf{Rates of miscarriage are increasing in poor maternal health and risky pregnancy behaviours and more so for twins than singletons (standardised effects)}: {\footnotesize Each plot displays rates of miscarriage (recorded foetal deaths occurring at or after 20 weeks of gestation per 1,000 live births) for twin and singleton births partitioned by whether the mother has a particular condition or health-related behaviour. Coefficients on health behaviours are standardised to represent the effect of a $1\sigma$ change in the behaviour for ease of comparison with table 1 (unstandardised results are presented in figure S3). Data consists of the full universe of births and foetal deaths recorded in the USA National Vital Statistics System  (see Materials and Methods). $\beta_{single}$ refers to the standardised difference in miscarriage rates between singleton mothers who smoke and singleton mothers who do not smoke (top left panel). $\beta_{twin}$ refers to the same standardised difference (smokers$-$non-smokers) for twin mothers.  Black error bars represent 95\% confidence intervals. In each case the difference in miscarriage rates associated with a particular behaviour is statistically larger for twin mothers.  Estimated regression-adjusted coefficients along with their statistical significance are reported in table S6. A similar plot with coefficients obtained after adjustment for mother's age, total fertility, and child's year of birth are presented in figure S4.}}
\label{fig:mech}
\end{figure}
\end{spacing}

%\begin{scilastnote}
%  \item We thank Judith Hall, Paul Devereux, James Fenske, Atheendar Venkataramani for comments, and Pietro Biroli and Hanna M\"uhlrad for assistance.  All authors contributed equally to this paper. No competing interests are declared. 
%\end{scilastnote}


\end{document}
