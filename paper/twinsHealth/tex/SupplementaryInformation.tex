\documentclass{nature}
\bibliographystyle{naturemag}
\usepackage{graphicx}
\usepackage[capposition=bottom]{floatrow}
\usepackage{subfloat}
\usepackage{subcaption}
\usepackage{lineno}
\usepackage{lscape}
\usepackage{bm}
\usepackage{longtable}
\usepackage{booktabs}
\usepackage{colortbl}

\definecolor{LightCyan}{rgb}{0.7,0.9,1}
\renewcommand\thetable{S\arabic{table}}  
\renewcommand\thefigure{S\arabic{figure}}  


\title{Supplementary Information}

\begin{document}

\maketitle

\begin{linenumbers}

The supplementary information for ``Improvements in Maternal Behaviors and Conditions Increase Twinning'' provides additional details regarding data and estimation methodologies to support the findings in the paper that twinning depends upon positive maternal health investments, behaviours, and stocks.  A full list of DHS survey data and timing is presented in table S1, while tables S2 and S3 provide regression results to accompany figure 4 and extended data figure 4.


%
% to a twin.
%
%y in DHS clusters is now associated with an increase in the probability of twinning (p$<$0.001).  The only unexpected move in coefficients occurs with hypertension in the USA.  While in conditional regressions this had a negative insignificant effect on twin rates, in unconditional tests it has positive significant effects on twinning: a sign that this variable is likely to be associated with other health variables, and that mothers who have hypertensions may decide to embrace \emph{more} health seeking behaviours to mitigate this condition.
%
%bly less likely to twin.


\clearpage
\begin{spacing}{1}
\input{Countries.tex}
\begin{landscape}
\input{FDeath_Uncond.tex}
\end{landscape}
\begin{landscape}
\input{FDeath_Cond.tex}
\end{landscape}
\end{spacing}



\end{linenumbers}
\end{document}
