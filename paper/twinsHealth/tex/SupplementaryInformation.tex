\documentclass{nature}
\bibliographystyle{naturemag}
\usepackage{graphicx}
\usepackage[capposition=bottom]{floatrow}
\usepackage{subfloat}
\usepackage{subcaption}
\usepackage{lineno}
\usepackage{lscape}
\usepackage{bm}
\usepackage{longtable}
\usepackage{booktabs}
\usepackage{colortbl}

\definecolor{LightCyan}{rgb}{0.7,0.9,1}
\renewcommand\thetable{S\arabic{table}}  
\renewcommand\thefigure{S\arabic{figure}}  


\title{Supplementary Information}

\begin{document}

\maketitle

\begin{linenumbers}

The supplementary information for ``Improvements in Maternal Behaviors and Conditions Increase Twinning'' provides additional details regarding data and estimation methodologies to support the findings in the paper that twinning depends upon positive maternal health investments, behaviours, and stocks.  A full list of DHS survey data and timing is presented in table S1, while tables S2 and S3 provide regression results to accompany figure 4 and extended data figure 4.


%
% to a twin.
%
%y in DHS clusters is now associated with an increase in the probability of twinning (p$<$0.001).  The only unexpected move in coefficients occurs with hypertension in the USA.  While in conditional regressions this had a negative insignificant effect on twin rates, in unconditional tests it has positive significant effects on twinning: a sign that this variable is likely to be associated with other health variables, and that mothers who have hypertensions may decide to embrace \emph{more} health seeking behaviours to mitigate this condition.
%
%bly less likely to twin.


\clearpage
\begin{spacing}{1}
\end{spacing}\begin{spacing}{1} 
\begin{longtable}{llccccccc}\caption{Full Survey Countries and Years} \\ 
\toprule\label{TWINtab:countries} 
& & \multicolumn{7}{c}{Survey Year} \\ \cmidrule(r){3-9} 
\textsc{Country}&\textsc{Income}&1&2&3&4&5&6&7\\ \midrule 
Albania&Middle&2008&&&&&&\\
Armenia&Low&2000&2005&2010&&&&\\
Azerbaijan&Middle&2006&&&&&&\\
Bangladesh&Low&1994&1997&2000&2004&2007&2011&\\
Benin&Low&1996&2001&2006&&&&\\
Bolivia&Middle&1989&1994&1998&2003&2008&&\\
Brazil&Middle&1986&1991&1996&&&&\\
Burkina Faso&Low&1993&1999&2003&2010&&&\\
Burundi&Low&1987&2010&&&&&\\
Cambodia&Low&2000&2005&2010&&&&\\
Cameroon&Middle&1991&1998&2004&2011&&&\\
Central African Republic&Low&1994&&&&&&\\
Chad&Low&1997&2004&&&&&\\
Colombia&Middle&1986&1990&1995&2000&2005&2010&\\
Comoros&Low&1996&&&&&&\\
Congo Brazzaville&Middle&2005&2011&&&&&\\
Congo, Dem&Low&2007&&&&&&\\
Cote d'Ivoire&Low&1994&1998&2005&2012&&&\\
Dominican Republic&Middle&1986&1991&1996&1999&2002&2007&\\
Ecuador&Middle&1987&&&&&&\\
Egypt, Arab Rep&Middle&1988&1992&1995&2000&2005&2008&\\
El Salvador&Middle&1985&&&&&&\\
Ethiopia&Low&2000&2005&2011&&&&\\
Gabon&Middle&2000&2012&&&&&\\
Ghana&Low&1988&1993&1998&2003&2008&&\\
Guatemala&Middle&1987&1995&&&&&\\
Guinea&Low&1999&2005&&&&&\\
Guyana&Middle&2005&2009&&&&&\\
Haiti&Low&1994&2000&2006&2012&&&\\
Honduras&Middle&2005&2011&&&&&\\
India&Low&1993&1999&2006&&&&\\
Indonesia&Low&1987&1991&1994&1997&2003&2007&2012\\
Jordan&Middle&1990&1997&2002&2007&&&\\
Kazakhstan&Middle&1995&1999&&&&&\\
Kenya&Low&1989&1993&1998&2003&2008&&\\
Kyrgyz Republic&Low&1997&&&&&&\\
Lesotho&Low&2004&2009&&&&&\\
Liberia&Low&1986&2007&&&&&\\
Madagascar&Low&1992&1997&2004&2008&&&\\
Malawi&Low&1992&2000&2004&2010&&&\\
Maldives&Middle&2009&&&&&&\\
Mali&Low&1987&1996&2001&2006&&&\\
Mexico&Middle&1987&&&&&&\\
Moldova&Middle&2005&&&&&&\\
Morocco&Middle&1987&1992&2003&&&&\\
Mozambique&Low&1997&2003&2011&&&&\\
Namibia&Middle&1992&2000&2006&&&&\\
Nepal&Low&1996&2001&2006&2011&&&\\
Nicaragua&Low&1998&2001&&&&&\\
Niger&Low&1992&1998&2006&&&&\\
Nigeria&Low&1990&1999&2003&2008&&&\\
Pakistan&Low&1991&2006&&&&&\\
Paraguay&Middle&1990&&&&&&\\
Peru&Middle&1986&1992&1996&2000&&&\\
Philippines&Middle&1993&1998&2003&2008&&&\\
Rwanda&Low&1992&2000&2005&2010&&&\\
Sao Tome and Principe&Middle&2008&&&&&&\\
Senegal&Low&1986&1993&1997&2005&2010&&\\
Sierra Leone&Low&2008&&&&&&\\
South Africa&Middle&1998&&&&&&\\
Sri Lanka&Low&1987&&&&&&\\
Sudan&Low&1990&&&&&&\\
Swaziland&Middle&2006&&&&&&\\
Tanzania&Low&1992&1996&1999&2004&2007&2010&2012\\
Thailand&Middle&1987&&&&&&\\
Togo&Low&1988&1998&&&&&\\
Trinidad and Tobago&Middle&1987&&&&&&\\
Tunisia&Middle&1988&&&&&&\\
Turkey&Middle&1993&1998&2003&&&&\\
Uganda&Low&1988&1995&2000&2006&2011&&\\
Ukraine&Middle&2007&&&&&&\\
Uzbekistan&Middle&1996&&&&&&\\
Vietnam&Low&1997&2002&&&&&\\
Yemen, Rep&Low&1991&&&&&&\\
Zambia&Low&1992&1996&2002&2007&&&\\
Zimbabwe&Middle&1988&1994&1999&2005&2010\\
\midrule\multicolumn{9}{p{13.3cm}}{\begin{footnotesize}\textsc{Notes:} Country income status is based upon World Bank classifications described at http://data.worldbank.org/about/country-classifications and available for download at http://siteresources.worldbank.org/DATASTATISTICS/Resources/OGHIST.xls (consulted 1 April, 2014).  Income status varies by country and time.  Where a country's status changed between DHS waves only the most recent status is listed above.  Middle refers to both lower-middle and upper-middle income countries, while low refers just to those considered to be low-income economies.\end{footnotesize}}  
\\ \bottomrule \end{longtable}\end{spacing}\begin{spacing}{1.5}
\begin{landscape}
\input{FDeath_Uncond.tex}
\end{landscape}
\begin{landscape}
\input{FDeath_Cond.tex}
\end{landscape}
\end{spacing}



\end{linenumbers}
\end{document}
