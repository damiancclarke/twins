\documentclass{nature}
\bibliographystyle{naturemag}
\usepackage{booktabs}
\usepackage{graphicx}
\usepackage[capposition=bottom]{floatrow}
\usepackage{subfloat}
\usepackage{subcaption}
\usepackage{lineno}
\usepackage{colortbl}

\definecolor{LightCyan}{rgb}{0.7,0.9,1}


\title{Improvements in Maternal Health Behaviors and Conditions Increase Twinning}
\author{Sonia Bhalotra$^{1}$ \& Damian Clarke$^2$}


\begin{document}

\maketitle

\begin{affiliations}
 \item Institute for Social and Economic Research, University of Essex, Essex, United Kingdom.
 \item Department of Economics, The University of Oxford, Manor Road, Oxford, United Kingdom.
\end{affiliations}

\begin{linenumbers}
\begin{abstract}
Twin births are frequently employed as a natural experiment in social and natural sciences\cite{Boomsmaetal2002,Poldermanetal2015}.  In biology, behavioural genetics, demography and psychology, twins are used to examine the importance of environment on human outcomes\cite{Boomsmaetal2002,Poldermanetal2015,Phillips1993,BouchardPropping1993,McClearnetal1997,Nisen2013}. In the social sciences, twins are used as an unexpected increase in family size to estimate the causal effect of additional births on child and family outcomes\cite{WolpinRosenzweig2000,RosenzweigWolpin1980,AngristEvans1988}.  If twin births occur non-randomly, findings from twin studies will not hold in the general population or will result in biased estimates of the effect of additional births on family outcomes.  While twins are known to depend upon a number of observable measures such as maternal age and In Vitro Fertilization (IVF) use\cite{Hall2003,Hoekstraetal2008,Vitthalaetal2009}, the degree to which twins depend upon a wider range of the mother's behaviours and health stocks during and prior to pregnancy has not been documented.  Here we show that healthier women are considerably more likely to give birth to twins in a large range of contexts, even among non-IVF users.  Using microdata covering millions of births in 72 different (low- and high-income) countries we show that mothers who engage in positive behaviours during pregnancy and mothers who have greater health stocks prior to pregnancy are much more likely to give live birth to twins. We show that less healthy mothers who conceive twins are selectively more likely to miscarry, even when compared to unhealthy mothers who conceive singletons.  Our finding that twinning depends upon maternal health hold when examining a wide range of health behaviours and conditions in both low and high-income settings.  These results have important implications regarding the degree to which findings from twins studies in a range of scientific fields can be thought of as representative or unbiased estimates in the population of all children.
\end{abstract}

%DHS: 2058324/43175   (2.10)
%USA: 15345243/466518 (3.14)
%CHI: 14915/370       (2.48)
%SWE: 1233546/29482   (2.39)
%



We collect data covering 18,652,028 births, of which 539,544 (2.89\%) are twin births.  %This data is collated from nationally- or regionally-representative surveys or complete vital statistics data which provide records of birth types, as well as indicators of the mother's health behaviours (such as drug consumption and diet), and stocks (such as height and disease burden).  
The data sources consulted (see Supplementary Information for a full discussion of this data) allow us to test whether mothers who give live birth to twins look different---controlling for age, race, IVF use, and other known twin determinants\cite{Hoekstraetal2008}---to mothers who give birth to singletons.  Table 1 presents evidence that in a wide variety of contexts, and using a wide range of health measures, mothers who give birth to twins are healthier than their non-twin counterparts.  While it is has been documented that the biological demands of twin pregnancy are higher than the demands of non-twin pregancy\cite{Shinagawaetal2005,Kahnetal2003} and that healthier mothers are less likely to miscarry in general\cite{Frettsetal1995,Garciaetal2002}, the evidence presented in table 1 demonstrates that an additional maternal health-twin birth gradient exists. The likelihood of twin births increases as the health of mothers increases over a large range of dimensions: both in terms of health-seeking and risk-avoiding behaviours during pregnancy, and health stocks prior to pregnancy.

Each panel of table 1 presents data from a different context in which both birth type and mother's health are observed.  The twin status of each birth in our data is regressed on measures of each mother's behaviour and health indicators entirely before the twin birth occurs.  Each coefficient is expressed as the effect that a 1 standard deviation ($\sigma$) increase in the variable of interest has on the change in likelihood, in percentage points, of a mother giving live birth to twins, and hence appearing in birth records.  In the contexts examined, mothers who smoke, consume drugs, consume alcohol, who are taller (an underlying indicator of health\cite{Silventoinen2003,BhalotraRawlings2013}) and who have higher body mass indices are more likely to twin.  Mothers with a greater disease burden (hypertension, diabetes, and kidney disease) are found to be less likely to twin, and mothers in the developing world with greater access to adequate medical care in their region have higher rates of twin births.  In all contexts examined, maternal education is a significant predictor of twinning.  This finding is consistent with education's role in producing health, for example through its link to health access, and the ease with which an individual can acquire and process health information.\cite{Kenkel1991,CutlerLlerasMuney2010}

In the USA, the full universe of mothers who did not use Artificial Reproductive Technology (ART) between 2009-2013 is used.  Increasing rates of smoking in each trimester by 1$\sigma$ is associated with a 0.15-0.25\% reduction in twinning (p$<$0.001 in each case), with third trimester smoking imposing the largest twin reduction.  Increasing disease burden by 1$\sigma$ nearly universally reduces twinning by between 0.02\% to 0.2\% depending on the condition.  The only exception is hypertension during (rather than preceding) gestation, which may itself be associated with conceiving the twin fetuses (rather than a predictor of live twin births). When compared to rates of twinning in the general (non-ART) population during this period of 3.14\%, this evidence suggests---nearly universally--that having worse health indicators results in a considerable reduction in the likelihood of taking two fetuses to term.  These findings are robust to running regressions on each variable individually with only fixed effects, rather than including all health variables together in the same specification (see Supplementary Information table S.1)  


The result of non-random twin-births is found in all contexts examined. When pooling Demographic and Health Surveys for 68 low- and middle-income countries in panel B, taller mothers and mothers with higher BMI are observed to be significantly more likely to twin (p$<$0.001).  In this data we also observe the availability of pre-natal care in the geographic cluster where the mother lives.  A 1$\sigma$ increase in the availability of nurses is associated with a 0.44\% increase in the likelihood of twinning, while a larger proportion of individuals receiving no medical care during pregnancy is assocated with a 0.31\% reduction in the likelihood of twinning (p=0.051).  We focus on medical availability rather than medical use as the mother's decision to seek medical care is likely endogenous to her conception status (twin or singleton), while medical availability in her area of residence is not.  A similar maternal health--twin birth gradient is found in a high-income (Sweden) and a middle-income (Chile).  In Sweden's Medical Birth Registry, the largest behavioural predictor of twinning is smoking in the third trimester, where a 1$\sigma$ increase in rates of smoking during the third trimester reduces the likelihood of twin births by 0.31\% (smoking in the second trimester is not observed).  Disease burden results suggest that women suffering from diseases prior to pregnancy are considerably less likely to give live birth to twins, with diabetes, kidney disease and hypertension reducing twin births significantly (p$<$0.01).  Finally, in a Chilean Early Infancy Survey we observe additional behavioural measures.  These results suggest that high (but not moderate) reported use of drugs and alcohol by mothers is associated with lower rates of twins in the population.  In each case increasing rates of substance abuse by 1$\sigma$ reduces the probability of a twin birth by 0.12\%.

%%We present results from alternative data sources in different panels.  A full description of each data source and coverage is provided in the Supplementary Information to this letter.  Panels A and B present regressions based on universal vital statistics data of all births in USA and Sweden (respectively), panels C-E are based on pooled data from the demographic and health survey data from all available countries in Africa, Latin America, and Asia, and panels F and G use panel survey data which follows women and their children over time (from Chile and the United Kingdom respectively).  In each case we condition on fixed effects for mother's age, total fertility, and race and total weeks of gestation where possible.  Table 1 presents regression results where all variables enter the regression together, and unconditional results for each specification are displayed in Supplementary Information.
%% 
%%In the United States, vital statistics data records a number of maternal behaviours, as well as whether the mother has suffered from a series of diseases.  We focus only on births ocurring between 2009 and 2013, as in these years women are asked whether or not they have used artifical reproductive technologies (ART), and we are able to remove these women (who have much higher rates of MZ twins) from the regression sample\cite{Vitthalaetal2009}.  In this sample, we find that smoking during pregnancy (in any trimester) is associated with a -0.2 to -0.6\% reduction in the likelihood of twinning, even conditional on mother's age, parity and race.  Compared to the mean rate of twinning in the non-ART sample of x.xx\%, these are considerable effects [change this].  Similarly, disease burden prior to pregnancy: and indicator of poor prevailing health, reduces the likelihood of twinning by between 1 and 2\%, with all of these results being highly statististically significant (p$<0.01$).  Results from Swedish vital statistics registers suggest quantitatively similar effects: women who smoke during the third trimester of pregnancy are nearly 1\% less likely to give birth to twins (though smoking in the first trimester has no statistically significant effect on twin births), and those who suffer from chronic kidney disease prior to pregnancy are also less likely to twin.  Full details on variable construction and inclusion are provided in the Supplementary Information.
%%

If this positive selection of twin mothers holds only in very low-income countries where health outcomes are on average worse, the degree to which twin families and non-twin families differ will become less relevant as income levels increase, both across countries and across time.  However, in figure 3 we find little evidence to suggest that the twin--health gradient diminishes with development.  In 70 countries we are able to examine both country income level and the difference in height (panel A) or education (panel B) between twin and non-twin mothers.  In all countries (with the exception of one), twin mothers are on average taller than non-twin mothers, and in all but four countries twin mothers are more educated than non-twin mothers.  What's more, if anything there seems to be a slight positive gradient between selection and income, with the difference in height between non-twin mothers and twin mothers growing as GDP per capita rises.  This result is consistent with the finding that height is a better indicator of health in high income countries than in low income countries\cite{Deaton2007}.  Similarly, the effect of education on rates of twinning is not ameliorated by increases in development.  The unconditional , consistent with the fact that education the link between education and health is most pronounced when medical technology changes most quickly.\cite{LlerasMuneyGlied2008}.

Given the higher biological demands of twin pregnancies\cite{Shinagawaetal2005,Kahnetal2003}, it seems likely that insults to health \emph{in utero} may be magnified in the case of twins.  It has been documented, for example, that smoking in utero has a larger effect on the birth weight of twin fetuses than a singleton fetus\cite{Pollacketal2000}.  Figure \ref{fig:mech} documents that this is the case when considering survival of twins to full term.  When twin fetuses are exposed to adverse health conditions, they are considerably more likely to miscarry than singletons exposed to the same adverse conditions.  In panel A we observe that for a mother pregnant with a single child who smokes during pregnancy, the rate of miscarriage increases by 1.394 fetal deaths per live births (from approximately 5 to 6 fetal deaths per 1,000 recorded live births).  On the other hand, for a mother who is pregnant with twins, smoking is associated with an increase in 2.548 fetal deaths per 1,000 live births.  Similar effects are found for each health condition or behaviour available.  Results for continuous measures such as years of education, number of cigarettes, and number of standard drinks are presented in supplementary tables 6-7.  

Unlike the case where fully observable characteristics of mothers result in an increased likelihood of twinning, if healthier mothers are universally more likely to give live birth to twins, this has important, implications for all types of twin studies.  In fields where environmental differences between dizygotic twins are used to infer the effect of different experiences in the presence of identical genes, non-randomness of twins has implications for the degree to which findings can be generalised to the entire population of both children \emph{and} families.  The non-randomness of twins has been long-recognised\cite{Recordetal1970}, and, as such, the interpretation of twin studies must case with respects to the population from which twins are drawn.  Our results suggest that beyond simply coming from older and larger families, twins are---all else constant---more likely to be born into and brought up in healthier environments.  Any `nature versus nurture' type interpretations should thus be interpreted as the effect of nurture in particularly healthy environments.

Secondly, non-randomness in twin births has considerable implications for estimates from social science research.  Twins are used as an exogenous shock to family size in studies examining the effects of additional births on a range of outcomes including sibling outcomes resulting from potentially diluted parental time and investment, and parental education and labour market outcomes.  The logic behind these so called `instrumental variables' estimates is that if a twin birth is unexpected at the time of conception, the random variation in children per birth can be used to isolate the causal response of having an additional child.  However, if, as we demonstrate, twins are non-random for multitude reasons, rather than isolating the causal response of an additional child, this response may be confounded with the fact that women who have twins are healthier, and likely to have more positive other baseline outcomes.

\clearpage
%%%%\begin{figure}
%%%%\includegraphics[scale=0.65]{twinsEffects}
%%%%\vspace{-1cm}
%%%%\captionsetup{labelformat=empty}
%%%%\caption{\textbf{Healthier Mothers are More Likely to Twin}: See figure note 1.}
%%%%\end{figure}
%%%%
%%%%\begin{figure}
%%%%\includegraphics[scale=0.65]{twinsEffects_two}
%%%%\vspace{-1cm}
%%%%\captionsetup{labelformat=empty}
%%%%\caption{\textbf{Healthier Mothers are More Likely to Twin}: See figure note 1.}
%%%%\end{figure}

\input{twinEffectsUncond.tex}

\begin{spacing}{1}
\begin{figure}
\begin{center}
  \includegraphics{./forest/forestCrop}
\end{center}
  \caption{\textbf{Effect of Maternal Health on Twinning (Unconditional Results)} This table displays results from Ordinary Least Square regressions of a child's birth type (twin or singleton) on the mother's health behaviours and conditions. Each cell represents a seperate regression, where only the variable of interest and fixed effects for control variables are included. The outcome variable is a binary variable for twin (=1) or singleton (=0) multiplied by 100, so all coefficients are expressed in terms of the percent incrrease in twinning.  Height is measured in centimetres, BMI is measured in  $\frac{kilograms}{metres^2}$, availability measures in panel B refer to theproportion of births in the women's survey cluster which were attended/unattended, and all remaining variables are binary.  In each case the interpretion of the coefficient is the effect that a 1 unit increase of the variable will have on the probability that a woman gives birth to twins. All models include fixed effects for age and birth order, and where possible, for gestation of the birth in weeks (panels A and C). Stars next to the coefficientsnts indicate significance levels, with: *p$<$0.1  **p$<$0.05  ***p$<$0.01. 95\% confidence intervals are displayed in parentheses. Further details regarding estimation samples and variable construction can be found in the Supplementary Information provided above.}
\end{figure}




\begin{figure}[htpb!]
\begin{subfigure}{.5\textwidth}
  \includegraphics[scale=0.59]{./HeightDif.eps}
\end{subfigure}%
\begin{subfigure}{.5\textwidth}
  \includegraphics[scale=0.59]{./EducDif.eps}
\end{subfigure}
\begin{subfigure}{.5\textwidth}
  \includegraphics[scale=0.45]{./heightGDP.png}
\end{subfigure}%
\begin{subfigure}{.5\textwidth}
  \includegraphics[scale=0.45]{./educGDP.png}
\end{subfigure}
%\captionsetup{labelformat=empty}
\vspace{5mm}
\caption{\textbf{Twin Effects exist at all Income Levels}: Each plot displays the difference in height (left-hand side) or total years of education (right-hand side) between twin and non-twin mothers.  The top two plots present multivariate regression estimates of the difference between twin and non-twin mothers along with their 95\% confidence intervals for each country in which this microdata is available.  All estimates are conditional on total fertility and mother's age.  The bottom two panels plot these differences against log GDP per capita, where log GDP per capita data comes from the World Bank Data Bank (indicator NY.GDP.PCAP.PP.KD), and is expressed at Purchasing Power Parity.  Each circle represents a particular country.  In the left-hand panel, circles above the horizontal dotted line imply that twin mothers are taller than non-twin mothers, while in the right-hand panel circles above the dotted line imply that twin mothers have more education than non-twin mothers.  Both measures are taken at the time of the survey, not the time of the birth.  The size of the circle indicates the proportion of all births in the country which are twins, with larger points implying a larger proportion of twins. The global correlation between height difference and GDP conditional on continent fixed effects is 0.259 (t-statistic 1.95), and between education and GDP is 0.198 (t-statistic 1.47). Supplementary figure S2 replicates these figures using standardized (Z-scores) for education and mother's height in each country.}
\end{figure}


\begin{figure}[htpb!]
\begin{subfigure}{.5\textwidth}
  \includegraphics[scale=0.59]{./Deathssmokes_Uncond.eps}
\end{subfigure}%
\begin{subfigure}{.5\textwidth}
  \includegraphics[scale=0.59]{./Deathsdrinks_Uncond.eps}
\end{subfigure}
\begin{subfigure}{.5\textwidth}
  \includegraphics[scale=0.59]{./DeathsnoCollege_Uncond.eps}
\end{subfigure}%
\begin{subfigure}{.5\textwidth}
  \includegraphics[scale=0.59]{./Deathsanemic_Uncond.eps}
\end{subfigure}
%\captionsetup{labelformat=empty}
\vspace{5mm}
\caption{\textbf{Rates of Miscarriage are higher for twins with unhealthy mothers}: Each plot displays rates of miscarriage (recorded fetal deaths per 1,000 live births) for twin and singleton births partitioned by whether the mother has a particular condition, health stock, or health behaviour.  Data consists of all births and fetal deaths recorded in the National Vital Statistics System (USA) for the final four years preceding the changed reporting on birth and fetal death certificates (1999-2002) in which information on health behaviour is available for births \emph{and} fetal deaths.  Further information on the sample and data description is available in Supplementary Information.  Each panel presents the unadjusted rate of miscarriage for each group.  $\beta_{twin}$ refers to the difference in miscarriage rates between twin mothers who smoke (top left panel) and twin mothers who do not smoke. $\beta_{single}$ refers to the same rate for mothers of singleton children.  Health/socioeconomic variables examined are smoking during pregnancy (top left), alcohol consumption during pregnancy (top right), whether a mother has any college education (bottom left), and whether a mother is anemic (bottom left).  Estimated coefficients along with their statistical significance are reported in supplementary infomrmation table SXXX.  The same graphs conditioning on mother's age, total fertility, and child's year of birth are presented in supplementary information figure SXXX.}
\label{fig:mech}
\end{figure}
\end{spacing}

\clearpage
\bibliography{refs}

\clearpage
\begin{addendum}
 \item We acknowledge ...  Clarke acknowledges financial support received from CONICYT of the Government of Chile.
 \item[Author Contributions] All authors contributed equally to this paper.
 \item[Author Information] No competing interests are declared.  Corresponding author: Professor Sonia Bhalotra, Institute for Social and Economic Research, University of Essex, Essex, United Kingdom.
\end{addendum}

%%
%% TABLES
%%
%% If there are any tables, put them here.
%%
\end{linenumbers}
\end{document}
