\documentclass[a4paper, 11pt]{article}

\usepackage{amsfonts}
\usepackage{amsmath}
\usepackage{amsthm}
\usepackage{amssymb}
\usepackage{appendix}
\usepackage{blindtext}
\usepackage{bm}
\usepackage{booktabs}
\usepackage{caption}
\usepackage[usenames, dvipsnames]{color}
\usepackage{dcolumn}
\usepackage{graphicx}
\usepackage{epsfig}
\usepackage{epstopdf}
\epstopdfsetup{update}
\usepackage[capposition=top]{floatrow}
\usepackage{helvet}
\usepackage{hyperref}
\usepackage{indentfirst}
\usepackage{longtable}
\usepackage{lscape}
\usepackage{morefloats}
\usepackage{multirow}
\usepackage{natbib} \bibliographystyle{abbrvnat}\bibpunct{(}{)}{;}{a}{,}{,}
%\bibliographystyle{abbrvnat}\bibpunct{(}{)}{;}{a}{,}{,}
\usepackage{setspace}
\usepackage{subcaption}
\usepackage[capposition=top]{floatrow}
\usepackage{subfloat}
\usepackage[latin1]{inputenc}
\usepackage{tikz}
%\usepackage[pdf]{pstricks}
\usepackage{xr}
\externaldocument{BhalotraClarke_TwinQQ}
 
%\usepackage{pdfpages}
%\usepackage{rotating}
%\usepackage{setspace}
%\usepackage{subcaption}
%\usepackage{subfloat}
%\usepackage{url}
%\usepackage{wrapfig}



\usetikzlibrary{trees}
\usetikzlibrary{decorations.markings}
\newcommand{\twinfolder}{/home/damian/investigacion/Activa/Twins}


\theoremstyle{plain}
\newtheorem{thm}{Theorem}
\newtheorem{cor}{Corollary}
\newtheorem{lem}[thm]{Lemma}
\newtheorem{proposition}{Proposition}
\newtheorem{assumption}{Assumption}
\newtheorem{definition}{Definition}

%MARGINS
 \topmargin   =  -0.5in
 \headsep     =  0.7in
 \oddsidemargin= -0.4in
 \evensidemargin=-0.2in
 \textheight  =  9.4in
 \textwidth   =  7.25in


\newcommand{\fmt}{.eps}
%\newcommand{\fmt}{.png}
\hypersetup{
    colorlinks=true,
    linkcolor=BlueViolet,
    citecolor=BlueViolet,
    filecolor=BlueViolet,
    urlcolor=BlueViolet
}




\setcounter{page}{0}
\begin{document}

%\begin{doublespace}
 

\renewcommand\thesection{\Alph{section}}
\renewcommand*{\thepage}{A\arabic{page}}
\setcounter{section}{0}

\setcounter{table}{0}
\renewcommand{\thetable}{A\arabic{table}}
\setcounter{figure}{0}
\renewcommand{\thefigure}{A\arabic{figure}}


\setcounter{page}{1}
  \begin{center}
    \textbf{ONLINE APPENDIX} \\
    \vspace{4mm}
    For the paper: \\
     \vspace{6mm}
    {\large \textsc{The Twin Instrument}} \\
    Sonia Bhalotra and Damian Clarke
  \end{center}


\tableofcontents
\setlength{\parskip}{1em}
\setlength{\parindent}{4em}

\newpage
\section{Appendix Figures}
\setcounter{figure}{0}
\renewcommand{\thefigure}{A\arabic{figure}}

\begin{figure}[htpb!]
\begin{center}
\caption{Height and Selective Survival}
\label{TWINfig:bord}
\includegraphics[scale=0.92]{\twinfolder/Figures/MMRcuts.eps} 
\end{center}
\end{figure}




\clearpage
\section{Appendix Tables}
\begin{table}[htpb!]\caption{Summary Statistics} 
\label{TWINtab:sumstats}\begin{center}\scalebox{0.99}{\begin{tabular}{lccccc}
\toprule \toprule 
&\multicolumn{2}{c}{Low Income}&\multicolumn{2}{c}{Middle Income}\\ 
\cmidrule(r){2-3} \cmidrule(r){4-5}
& Single & Twins & Single & Twins & All \\ \midrule 
\textsc{Fertility} & & & & & \\ 
Fertility&3.749&6.223&3.412&5.584&3.689\\
&(2.392)&(2.622)&(2.308)&(2.687)&(2.406)\\
Desired Family Size&4.193&5.328&3.380&4.190&3.921\\
&(2.530)&(2.885)&(2.130)&(2.555)&(2.440)\\
Fraction Twin & \multicolumn{2}{c}{  0.0200}& \multicolumn{2}{c}{  0.0179 } &  0.0191\\
& \multicolumn{2}{c}{(0.1402)}& \multicolumn{2}{c}{(0.1326)} & (0.1370)\\
Birth Order Twin & \multicolumn{2}{c}{   4.664}& \multicolumn{2}{c}{   4.016 }&   4.410\\
& \multicolumn{2}{c}{(2.465)}& \multicolumn{2}{c}{(2.374)}& (2.450)\\
\textsc{Mother's Characteristics}&&&&&\\ Age
&31.22&34.52&32.32&35.61&31.72\\
&(8.238)&(7.381)&(8.356)&(7.428)&(8.293)\\
Education&3.859&3.222&6.690&5.906&4.885\\
&(4.327)&(3.991)&(4.795)&(5.023)&(4.706)\\
Height&155.5&157.6&155.6&157.2&155.6\\
&(7.093)&(7.065)&(6.966)&(6.945)&(7.053)\\
BMI&21.90&22.50&25.90&26.63&23.39\\
&(4.027)&(4.175)&(5.118)&(5.512)&(4.867)\\
Pr(BMI)$<$18.5&0.175&0.125&0.0346&0.0276&0.122\\
&(0.380)&(0.331)&(0.183)&(0.164)&(0.327)\\
Actual Births$>$Desired&0.310&0.526&0.324&0.575&0.321\\
&(0.463)&(0.499)&(0.468)&(0.494)&(0.467)\\
\textsc{Children's Outcomes}&&&&&\\ Education (Years)
&3.660&3.204&5.445&5.043&4.446\\
&(3.576)&(3.293)&(3.867)&(3.760)&(3.810)\\
Education (Z-Score)&-0.00843&-0.0156&0.0119&-0.0428&0.000144\\
&(1.001)&(0.963)&(0.998)&(0.987)&(1.000)\\
No Education (Percent)&0.207&0.222&0.0649&0.0786&0.144\\
&(0.405)&(0.416)&(0.246)&(0.269)&(0.351)\\
Infant Mortality&0.0158&0.0917&0.00946&0.0497&0.0141\\
&(0.125)&(0.289)&(0.0968)&(0.217)&(0.118)\\
Child Mortality&0.0239&0.108&0.0122&0.0535&0.0199\\
&(0.153)&(0.310)&(0.110)&(0.225)&(0.140)\\
\midrule
Number of Countries & 39&39  & 28&28  & 67 \\
Number of Children &2,231,844 &45,654 &1,614,358 &29,430 & 3,921,286 \\
Number of Mothers &875,587 &12,908 &653,969 &8,605 & 1,586,899 \\
\midrule
\multicolumn{6}{p{13.2cm}}{\begin{footnotesize}\textsc{Notes:}  Group means are presented with standard deviation below in parenthesis.  Education is reported as total years attained, and Z-score presents educational attainment relative to country and cohort (mean 0, std deviation 1).  Infant mortality refers to the proportion of children who die before 1 year of age,  while child mortality refers to the proportion who die before 5 years.  Maternal height is reported in cm, and BMI is weight in kg over height in metres squared.  Summary statistics are for the full sample of 1,586,899
 mothers responding to any publicly available DHS survey.  For a full list of country and years of survey, see appendix table \ref{TWINtab:countries}.\end{footnotesize}} \\ \bottomrule \end{tabular}}\end{center}\end{table}
\clearpage
\input{\twinfolder/Tables/NHISstats.tex}
\input{\twinfolder/Tables/Stress.tex}
\input{\twinfolder/Tables/OLS_bounds.tex}
\input{\twinfolder/Tables/literature.tex}
\input{\twinfolder/Tables/LeeBounds.tex}
%\begin{table}
\caption{Twins, Miscarriage and Maternal Health (Administrative Data from Spain)}
\begin{center}
\scalebox{0.5}{
\begin{tabular}{lc}
\hline
VARIABLES	&	Fetal Death*100 \\	\hline
	&	(1)	\\
Primary	&	  -0.60179***	\\
	&	 (0.01456)	\\
Primary*Twin	&	   -0.6618***	\\
	&	 (0.20926)	\\
Secondary	&	  -0.71998*	\\
	&	  (0.0151)	\\
Secondary*Twin	&	  -0.55901***	\\
	&	 .0020978	\\
Tertiary	&	  -0.80019***	\\
	&	(0.01582)	\\
Tertiary*Twin	&	  -0.65091***	\\
	&	(0.20866)	\\
Immigration	&	  -0.07223***	\\
	&	 (0.0171)	\\
Immigration*Twin	&	  0.22871	\\
	&	(0.29614)	\\
Citm	&	  -0.00321	\\
	&	(0.01584)	\\
Citm*Twin	&	  0.09566	\\
	&	 (0.26876)	\\
Married	&	  -0.07354***	\\
	&	 (0.00759)	\\
Married*Twin	&	  -0.08978	\\
	&	 (0.11893)	\\
No Father	&	0.68626***	\\
	&	 (0.23825)	\\
No Father*Twin	&	3.25232	\\
	&	(4.09309)	\\
Constant	&	-1.35966***	\\
	&	(0.33674)	\\
	& \\
	Obs & 2869329 \\
	$R^2$ & 0.0044 \\ \hline
	\multicolumn{2}{p{5cm}}{Note: Spanish administrative births: 2007-2012.}
\end{tabular}}
\end{center}
\end{table}

%\begin{table}[htpb!]\caption{Principal IV Results}
\label{TWINtab:IVAll}
\begin{center}\scalebox{0.55}{
\begin{tabular}{lcccp{2mm}cccp{2mm}ccc}
\toprule \toprule 
&\multicolumn{3}{c}{2+}&&\multicolumn{3}{c}{3+}&&\multicolumn{3}{c}{4+}\\ \cmidrule(r){2-4} \cmidrule(r){6-8} \cmidrule(r){10-12} 
\textsc{School Z-Score}&Base&+H&+S\&H&&Base&+H&+S\&H&&Base&+H&+S\&H\\ \midrule 
\begin{footnotesize}\end{footnotesize}& 
\begin{footnotesize}\end{footnotesize}& 
\begin{footnotesize}\end{footnotesize}& 
\begin{footnotesize}\end{footnotesize}& 
\begin{footnotesize}\end{footnotesize}& 
\begin{footnotesize}\end{footnotesize}& 
\begin{footnotesize}\end{footnotesize}& 
\begin{footnotesize}\end{footnotesize}& 
\begin{footnotesize}\end{footnotesize}& 
\begin{footnotesize}\end{footnotesize}& 
\begin{footnotesize}\end{footnotesize}& 
\begin{footnotesize}\end{footnotesize}\\ 
\multicolumn{12}{l}{\textbf{All}}\\ 
Fertility&0.006&-0.026&-0.026&&-0.004&-0.036&-0.038*&&-0.017&-0.036&-0.035*\\
&(0.029)&(0.027)&(0.026)&&(0.024)&(0.022)&(0.021)&&(0.025)&(0.023)&(0.021)\\
\begin{footnotesize}\end{footnotesize}&\begin{footnotesize}\end{footnotesize}&\begin{footnotesize}\end{footnotesize}&\begin{footnotesize}\end{footnotesize}&\begin{footnotesize}\end{footnotesize}&\begin{footnotesize}\end{footnotesize}&\begin{footnotesize}\end{footnotesize}&\begin{footnotesize}\end{footnotesize}&\begin{footnotesize}\end{footnotesize}&\begin{footnotesize}\end{footnotesize}&\begin{footnotesize}\end{footnotesize}&\begin{footnotesize}\end{footnotesize}\\Observations&249536&249536&249536&&375987&375987&375987&&385389&385389&385389\\
\multicolumn{12}{l}{\textbf{Low-Income}}\\ 
Fertility&0.035&0.008&0.012&&0.016&-0.016&-0.027&&-0.011&-0.031&-0.024\\
&(0.034)&(0.032)&(0.031)&&(0.030)&(0.028)&(0.026)&&(0.029)&(0.027)&(0.025)\\
\begin{footnotesize}\end{footnotesize}&\begin{footnotesize}\end{footnotesize}&\begin{footnotesize}\end{footnotesize}&\begin{footnotesize}\end{footnotesize}&\begin{footnotesize}\end{footnotesize}&\begin{footnotesize}\end{footnotesize}&\begin{footnotesize}\end{footnotesize}&\begin{footnotesize}\end{footnotesize}&\begin{footnotesize}\end{footnotesize}&\begin{footnotesize}\end{footnotesize}\\Observations&149602&149602&149602&&232371&232371&232371&&246622&246622&246622\\
\multicolumn{12}{l}{\textbf{Middle-Income}}\\ 
Fertility&-0.065&-0.087*&-0.093**&&-0.046&-0.079**&-0.067*&&-0.027&-0.048&-0.054\\
&(0.053)&(0.049)&(0.047)&&(0.040)&(0.036)&(0.035)&&(0.043)&(0.040)&(0.037)\\
\begin{footnotesize}\end{footnotesize}&\begin{footnotesize}\end{footnotesize}&\begin{footnotesize}\end{footnotesize}&\begin{footnotesize}\end{footnotesize}&\begin{footnotesize}\end{footnotesize}&\begin{footnotesize}\end{footnotesize}&\begin{footnotesize}\end{footnotesize}&\begin{footnotesize}\end{footnotesize}&\begin{footnotesize}\end{footnotesize}&\begin{footnotesize}\end{footnotesize}\\Observations&99934&99934&99934&&143616&143616&143616&&138767&138767&138767\\
\multicolumn{12}{l}{\textbf{Adjusted Fertility}}\\ 
Fertility&0.017&-0.052&-0.055&&-0.013&-0.073*&-0.077*&&-0.033&-0.068&-0.066*\\
&(0.065)&(0.056)&(0.054)&&(0.047)&(0.043)&(0.040)&&(0.045)&(0.042)&(0.039)\\
\begin{footnotesize}\end{footnotesize}&\begin{footnotesize}\end{footnotesize}&\begin{footnotesize}\end{footnotesize}&\begin{footnotesize}\end{footnotesize}&\begin{footnotesize}\end{footnotesize}&\begin{footnotesize}\end{footnotesize}&\begin{footnotesize}\end{footnotesize}&\begin{footnotesize}\end{footnotesize}&\begin{footnotesize}\end{footnotesize}&\begin{footnotesize}\end{footnotesize}\\Observations&249505&249505&249505&&375957&375957&375957&&385363&385363&385363\\
\multicolumn{12}{l}{\textbf{Twins and Pre-Twins}}\\ 
Fertility&-0.021&-0.073***&-0.078***&&-0.019&-0.062***&-0.067***&&-0.018&-0.039**&-0.046**\\
&(0.024)&(0.021)&(0.020)&&(0.020)&(0.018)&(0.018)&&(0.021)&(0.019)&(0.018)\\
\begin{footnotesize}\end{footnotesize}&\begin{footnotesize}\end{footnotesize}&\begin{footnotesize}\end{footnotesize}&\begin{footnotesize}\end{footnotesize}&\begin{footnotesize}\end{footnotesize}&\begin{footnotesize}\end{footnotesize}&\begin{footnotesize}\end{footnotesize}&\begin{footnotesize}\end{footnotesize}&\begin{footnotesize}\end{footnotesize}&\begin{footnotesize}\end{footnotesize}\\Observations&488815&488815&488815&&563177&563177&563177&&523197&523197&523197\\
\bottomrule
\end{tabular}}\end{center}\end{table}

%\input{\twinfolder/Tables/OLSPlus.tex}
\clearpage
\input{\twinfolder/Tables/DHS-togetherGirls.tex}
\input{\twinfolder/Tables/DHS-togetherBoys.tex}
\input{\twinfolder/Tables/DHS-togetherMidIncome.tex}
\input{\twinfolder/Tables/DHS-togetherLowIncome.tex} 
%\begin{landscape}\begin{table}[htpb!]\caption{First Stage Results} 
\label{TWINtab:FS}\begin{center}\begin{tabular}{lcccp{2mm}cccp{2mm}ccc}
\toprule \toprule 
&\multicolumn{3}{c}{2+}&&\multicolumn{3}{c}{3+}&&\multicolumn{3}{c}{4+}\\ \cmidrule(r){2-4} \cmidrule(r){6-8} \cmidrule(r){10-12} 
\textsc{Fertility}&Base&+H&+S\&H&&Base&+H&+S\&H&&Base&+H&+S\&H\\ \midrule 
\begin{footnotesize}\end{footnotesize}& 
\begin{footnotesize}\end{footnotesize}& 
\begin{footnotesize}\end{footnotesize}& 
\begin{footnotesize}\end{footnotesize}& 
\begin{footnotesize}\end{footnotesize}& 
\begin{footnotesize}\end{footnotesize}& 
\begin{footnotesize}\end{footnotesize}& 
\begin{footnotesize}\end{footnotesize}& 
\begin{footnotesize}\end{footnotesize}& 
\begin{footnotesize}\end{footnotesize}\\ 
\multicolumn{12}{l}{\textbf{All}}\\ 
Twin&0.776***&0.821***&0.822***&&0.794***&0.827***&0.826***&&0.840***&0.859***&0.861***\\
&(0.031)&(0.029)&(0.028)&&(0.027)&(0.027)&(0.026)&&(0.027)&(0.027)&(0.026)\\
\begin{footnotesize}\end{footnotesize}&\begin{footnotesize}\end{footnotesize}&\begin{footnotesize}\end{footnotesize}&\begin{footnotesize}\end{footnotesize}&\begin{footnotesize}\end{footnotesize}&\begin{footnotesize}\end{footnotesize}&\begin{footnotesize}\end{footnotesize}&\begin{footnotesize}\end{footnotesize}&\begin{footnotesize}\end{footnotesize}&\begin{footnotesize}\end{footnotesize}&\begin{footnotesize}\end{footnotesize}&\begin{footnotesize}\end{footnotesize}\\Observations&249536&249536&249536&&249536&249536&249536&&249536&249536&249536\\
\begin{footnotesize}\end{footnotesize}&\begin{footnotesize}\end{footnotesize}&\begin{footnotesize}\end{footnotesize}&\begin{footnotesize}\end{footnotesize}&\begin{footnotesize}\end{footnotesize}&\begin{footnotesize}\end{footnotesize}&\begin{footnotesize}\end{footnotesize}&\begin{footnotesize}\end{footnotesize}&\begin{footnotesize}\end{footnotesize}&\begin{footnotesize}\end{footnotesize}&\begin{footnotesize}\end{footnotesize}&\begin{footnotesize}\end{footnotesize}\\\multicolumn{12}{l}{\textbf{Low-Income}}\\ 
Twin&0.826***&0.853***&0.848***&&0.810***&0.828***&0.834***&&0.867***&0.873***&0.869***\\
&(0.038)&(0.038)&(0.037)&&(0.033)&(0.033)&(0.032)&&(0.033)&(0.033)&(0.033)\\
\begin{footnotesize}\end{footnotesize}&\begin{footnotesize}\end{footnotesize}&\begin{footnotesize}\end{footnotesize}&\begin{footnotesize}\end{footnotesize}&\begin{footnotesize}\end{footnotesize}&\begin{footnotesize}\end{footnotesize}&\begin{footnotesize}\end{footnotesize}&\begin{footnotesize}\end{footnotesize}&\begin{footnotesize}\end{footnotesize}&\begin{footnotesize}\end{footnotesize}&\begin{footnotesize}\end{footnotesize}&\begin{footnotesize}\end{footnotesize}\\Observations&149602&149602&149602&&149602&149602&149602&&149602&149602&149602\\
\begin{footnotesize}\end{footnotesize}&\begin{footnotesize}\end{footnotesize}&\begin{footnotesize}\end{footnotesize}&\begin{footnotesize}\end{footnotesize}&\begin{footnotesize}\end{footnotesize}&\begin{footnotesize}\end{footnotesize}&\begin{footnotesize}\end{footnotesize}&\begin{footnotesize}\end{footnotesize}&\begin{footnotesize}\end{footnotesize}&\begin{footnotesize}\end{footnotesize}&\begin{footnotesize}\end{footnotesize}&\begin{footnotesize}\end{footnotesize}\\\multicolumn{12}{l}{\textbf{Middle-Income}}\\ 
Twin&0.718***&0.774***&0.784***&&0.757***&0.817***&0.801***&&0.783***&0.831***&0.839***\\
&(0.050)&(0.045)&(0.043)&&(0.046)&(0.045)&(0.043)&&(0.047)&(0.044)&(0.042)\\
\begin{footnotesize}\end{footnotesize}&\begin{footnotesize}\end{footnotesize}&\begin{footnotesize}\end{footnotesize}&\begin{footnotesize}\end{footnotesize}&\begin{footnotesize}\end{footnotesize}&\begin{footnotesize}\end{footnotesize}&\begin{footnotesize}\end{footnotesize}&\begin{footnotesize}\end{footnotesize}&\begin{footnotesize}\end{footnotesize}&\begin{footnotesize}\end{footnotesize}&\begin{footnotesize}\end{footnotesize}&\begin{footnotesize}\end{footnotesize}\\Observations&99934&99934&99934&&99934&99934&99934&&99934&99934&99934\\
\begin{footnotesize}\end{footnotesize}&\begin{footnotesize}\end{footnotesize}&\begin{footnotesize}\end{footnotesize}&\begin{footnotesize}\end{footnotesize}&\begin{footnotesize}\end{footnotesize}&\begin{footnotesize}\end{footnotesize}&\begin{footnotesize}\end{footnotesize}&\begin{footnotesize}\end{footnotesize}&\begin{footnotesize}\end{footnotesize}&\begin{footnotesize}\end{footnotesize}&\begin{footnotesize}\end{footnotesize}&\begin{footnotesize}\end{footnotesize}\\\multicolumn{12}{l}{\textbf{Adjusted Fertility}}\\ 
Twin&0.354***&0.393***&0.395***&&0.403***&0.428***&0.427***&&0.453***&0.467***&0.468***\\
&(0.028)&(0.028)&(0.028)&&(0.026)&(0.026)&(0.026)&&(0.027)&(0.027)&(0.027)\\
\begin{footnotesize}\end{footnotesize}&\begin{footnotesize}\end{footnotesize}&\begin{footnotesize}\end{footnotesize}&\begin{footnotesize}\end{footnotesize}&\begin{footnotesize}\end{footnotesize}&\begin{footnotesize}\end{footnotesize}&\begin{footnotesize}\end{footnotesize}&\begin{footnotesize}\end{footnotesize}&\begin{footnotesize}\end{footnotesize}&\begin{footnotesize}\end{footnotesize}&\begin{footnotesize}\end{footnotesize}&\begin{footnotesize}\end{footnotesize}\\Observations&249505&249505&249505&&249505&249505&249505&&249505&249505&249505\\
\begin{footnotesize}\end{footnotesize}&\begin{footnotesize}\end{footnotesize}&\begin{footnotesize}\end{footnotesize}&\begin{footnotesize}\end{footnotesize}&\begin{footnotesize}\end{footnotesize}&\begin{footnotesize}\end{footnotesize}&\begin{footnotesize}\end{footnotesize}&\begin{footnotesize}\end{footnotesize}&\begin{footnotesize}\end{footnotesize}&\begin{footnotesize}\end{footnotesize}&\begin{footnotesize}\end{footnotesize}&\begin{footnotesize}\end{footnotesize}\\\multicolumn{12}{l}{\textbf{Twins and Pre-Twins}}\\ 
Twin&0.727***&0.782***&0.788***&&0.809***&0.828***&0.832***&&0.853***&0.855***&0.859***\\
&(0.027)&(0.025)&(0.025)&&(0.027)&(0.026)&(0.026)&&(0.027)&(0.025)&(0.025)\\
\begin{footnotesize}\end{footnotesize}&\begin{footnotesize}\end{footnotesize}&\begin{footnotesize}\end{footnotesize}&\begin{footnotesize}\end{footnotesize}&\begin{footnotesize}\end{footnotesize}&\begin{footnotesize}\end{footnotesize}&\begin{footnotesize}\end{footnotesize}&\begin{footnotesize}\end{footnotesize}&\begin{footnotesize}\end{footnotesize}&\begin{footnotesize}\end{footnotesize}&\begin{footnotesize}\end{footnotesize}&\begin{footnotesize}\end{footnotesize}\\Observations&488815&488815&488815&&488815&488815&488815&&488815&488815&488815\\

\midrule\multicolumn{12}{p{19.2cm}}{\begin{footnotesize}\textsc{Notes:} Each cell represents the coefficient from the first-stage of a two-stage regression.  The first-stage represents the effect of twinning at parity $N$ on total fertility where $N$ is 2, 3 or 4 for 2+, 3+ and 4+ groups respectively.  The 2+ group includes all first borns in families with at least 2 births, the 3+ group includes first and second borns in families with at least 3 births, and the 4+ group includes all first to third borns in families with at least four births.  In each regressions the sample is made up of all children aged between 6-18 years from families in the DHS who fulfill these birth order conditions.  Controls in each case are identical to those described in table \ref{TWINtab:IVAll}.  Standard errors are clustered at the level of the mother.$^{*}$p$<$0.1; $^{**}$p$<$0.05; $^{***}$p$<$0.01 
\end{footnotesize}} \\ \bottomrule 
\end{tabular}\end{center}\end{table}\end{landscape}
\begin{table}[htpb!]\caption{Q-Q IV Estimates by Gender} 
\label{TWINtab:gend}\begin{center}\begin{tabular}{lcccccccc}
\toprule \toprule 
&\multicolumn{4}{c}{Females}&\multicolumn{4}{c}{Males}\\ 
\cmidrule(r){2-5} \cmidrule(r){6-9} 
&Base&Socioec&Health&Obs.&Base&Socioec&Health&Obs. \\ \midrule 
\begin{footnotesize}\end{footnotesize}&\begin{footnotesize}\end{footnotesize}&\begin{footnotesize}\end{footnotesize}&\begin{footnotesize}\end{footnotesize}&\begin{footnotesize}\end{footnotesize}&\begin{footnotesize}\end{footnotesize}&\\Two Plus &0.005&-0.039&-0.037&122,414&0.010&-0.010&-0.015&127,122\\
&(0.043)&(0.039)&(0.038)&&(0.040)&(0.038)&(0.036)&\\
\begin{footnotesize}\end{footnotesize}&\begin{footnotesize}\end{footnotesize}&\begin{footnotesize}\end{footnotesize}&\begin{footnotesize}\end{footnotesize}&\begin{footnotesize}\end{footnotesize}&\begin{footnotesize}\end{footnotesize}&\\Three Plus &-0.024&-0.056*&-0.052*&187,098&0.016&-0.015&-0.022&188,889\\
&(0.033)&(0.030)&(0.029)&&(0.030)&(0.028)&(0.027)&\\
\begin{footnotesize}\end{footnotesize}&\begin{footnotesize}\end{footnotesize}&\begin{footnotesize}\end{footnotesize}&\begin{footnotesize}\end{footnotesize}&\begin{footnotesize}\end{footnotesize}&\begin{footnotesize}\end{footnotesize}&\\Four Plus &-0.029&-0.052*&-0.053**&192,714&-0.005&-0.020&-0.018&192,675\\
&(0.032)&(0.029)&(0.027)&&(0.030)&(0.028)&(0.027)&\\
\midrule\multicolumn{9}{p{14.2cm}}{\begin{footnotesize}\textsc{Notes:} Female or male refers to the gender of the index child of the regression. 
All regressions include full controls including socioeconomic and maternal health variables.  The full lis of controls are available in 
the notes to table \ref{TWINtab:IVAll}.  Full IV results for male and female children are presented in table \ref{TWINtab:IVgend}. Standard errors are clustered 
 by mother.$^{*}$p$<$0.1; $^{**}$p$<$0.05; $^{***}$p$<$0.01
\end{footnotesize}} \\ \bottomrule 
\end{tabular}\end{center}\end{table}
\input{\twinfolder/Tables/GenderUSA.tex}
\end{spacing}\begin{spacing}{1} 
\begin{longtable}{llccccccc}\caption{Full Survey Countries and Years} \\ 
\toprule\label{TWINtab:countries} 
& & \multicolumn{7}{c}{Survey Year} \\ \cmidrule(r){3-9} 
\textsc{Country}&\textsc{Income}&1&2&3&4&5&6&7\\ \midrule 
Albania&Middle&2008&&&&&&\\
Armenia&Low&2000&2005&2010&&&&\\
Azerbaijan&Middle&2006&&&&&&\\
Bangladesh&Low&1994&1997&2000&2004&2007&2011&\\
Benin&Low&1996&2001&2006&&&&\\
Bolivia&Middle&1989&1994&1998&2003&2008&&\\
Brazil&Middle&1986&1991&1996&&&&\\
Burkina Faso&Low&1993&1999&2003&2010&&&\\
Burundi&Low&1987&2010&&&&&\\
Cambodia&Low&2000&2005&2010&&&&\\
Cameroon&Middle&1991&1998&2004&2011&&&\\
Central African Republic&Low&1994&&&&&&\\
Chad&Low&1997&2004&&&&&\\
Colombia&Middle&1986&1990&1995&2000&2005&2010&\\
Comoros&Low&1996&&&&&&\\
Congo Brazzaville&Middle&2005&2011&&&&&\\
Congo, Dem&Low&2007&&&&&&\\
Cote d'Ivoire&Low&1994&1998&2005&2012&&&\\
Dominican Republic&Middle&1986&1991&1996&1999&2002&2007&\\
Ecuador&Middle&1987&&&&&&\\
Egypt, Arab Rep&Middle&1988&1992&1995&2000&2005&2008&\\
El Salvador&Middle&1985&&&&&&\\
Ethiopia&Low&2000&2005&2011&&&&\\
Gabon&Middle&2000&2012&&&&&\\
Ghana&Low&1988&1993&1998&2003&2008&&\\
Guatemala&Middle&1987&1995&&&&&\\
Guinea&Low&1999&2005&&&&&\\
Guyana&Middle&2005&2009&&&&&\\
Haiti&Low&1994&2000&2006&2012&&&\\
Honduras&Middle&2005&2011&&&&&\\
India&Low&1993&1999&2006&&&&\\
Indonesia&Low&1987&1991&1994&1997&2003&2007&2012\\
Jordan&Middle&1990&1997&2002&2007&&&\\
Kazakhstan&Middle&1995&1999&&&&&\\
Kenya&Low&1989&1993&1998&2003&2008&&\\
Kyrgyz Republic&Low&1997&&&&&&\\
Lesotho&Low&2004&2009&&&&&\\
Liberia&Low&1986&2007&&&&&\\
Madagascar&Low&1992&1997&2004&2008&&&\\
Malawi&Low&1992&2000&2004&2010&&&\\
Maldives&Middle&2009&&&&&&\\
Mali&Low&1987&1996&2001&2006&&&\\
Mexico&Middle&1987&&&&&&\\
Moldova&Middle&2005&&&&&&\\
Morocco&Middle&1987&1992&2003&&&&\\
Mozambique&Low&1997&2003&2011&&&&\\
Namibia&Middle&1992&2000&2006&&&&\\
Nepal&Low&1996&2001&2006&2011&&&\\
Nicaragua&Low&1998&2001&&&&&\\
Niger&Low&1992&1998&2006&&&&\\
Nigeria&Low&1990&1999&2003&2008&&&\\
Pakistan&Low&1991&2006&&&&&\\
Paraguay&Middle&1990&&&&&&\\
Peru&Middle&1986&1992&1996&2000&&&\\
Philippines&Middle&1993&1998&2003&2008&&&\\
Rwanda&Low&1992&2000&2005&2010&&&\\
Sao Tome and Principe&Middle&2008&&&&&&\\
Senegal&Low&1986&1993&1997&2005&2010&&\\
Sierra Leone&Low&2008&&&&&&\\
South Africa&Middle&1998&&&&&&\\
Sri Lanka&Low&1987&&&&&&\\
Sudan&Low&1990&&&&&&\\
Swaziland&Middle&2006&&&&&&\\
Tanzania&Low&1992&1996&1999&2004&2007&2010&2012\\
Thailand&Middle&1987&&&&&&\\
Togo&Low&1988&1998&&&&&\\
Trinidad and Tobago&Middle&1987&&&&&&\\
Tunisia&Middle&1988&&&&&&\\
Turkey&Middle&1993&1998&2003&&&&\\
Uganda&Low&1988&1995&2000&2006&2011&&\\
Ukraine&Middle&2007&&&&&&\\
Uzbekistan&Middle&1996&&&&&&\\
Vietnam&Low&1997&2002&&&&&\\
Yemen, Rep&Low&1991&&&&&&\\
Zambia&Low&1992&1996&2002&2007&&&\\
Zimbabwe&Middle&1988&1994&1999&2005&2010\\
\midrule\multicolumn{9}{p{13.3cm}}{\begin{footnotesize}\textsc{Notes:} Country income status is based upon World Bank classifications described at http://data.worldbank.org/about/country-classifications and available for download at http://siteresources.worldbank.org/DATASTATISTICS/Resources/OGHIST.xls (consulted 1 April, 2014).  Income status varies by country and time.  Where a country's status changed between DHS waves only the most recent status is listed above.  Middle refers to both lower-middle and upper-middle income countries, while low refers just to those considered to be low-income economies.\end{footnotesize}}  
\\ \bottomrule \end{longtable}\end{spacing}\begin{spacing}{1.5}
\clearpage
%\subsection{Twin Regressions in Other Data}
%\begin{table}[htpb!]
\caption{Probability of Twin Births: Brazil Administrative Data}
\begin{center}
\scalebox{0.56}{

\begin{tabular}{llcc}\hline

&&(1)&(2)\\ \cmidrule(r){3-4}
&&\multicolumn{2}{c}{Probability for twin}\\ \hline
&&&\\

Female	&		&	0.0012***	&	0.0012***	\\
	&		&	(0.0001)	&	(0.0001)	\\
Age	&		&	0.0008***	&	0.0009***	\\
	&		&	(0.0000)	&	(0.0000)	\\
Marital status	&	Married	&	0.0008***	&	0.0009***	\\
	&		&	(0.0001)	&	(0.0001)	\\
	&	Divorced	&	0.0002	&	0.0002	\\
	&		&	(0.0006)	&	(0.0006)	\\
Years of schooling	&		&	-0.0000**	&	-0.0000**	\\
	&		&	(0.0000)	&	(0.0000)	\\
Number of live births&	&		0.0006***	&	0.0005***	\\
	&		&	(0.0000)	&	(0.0000)	\\
Number of still births&		&	-0.0001**	&	-0.0002***	\\
	&		&	(0.0001)	&	(0.0001)	\\
Race	&	Black	&	-0.0011***	&	-0.0014***	\\
	&		&	(0.0002)	&	(0.0002)	\\
	&	Asian	&	-0.0014***	&	-0.0019***	\\
	&		&	(0.0005)	&	(0.0005)	\\
	&	Mixed	&	-0.0016***	&	-0.0019***	\\
	&		&	(0.0001)	&	(0.0001)	\\
	&	Indigenous	&	-0.0057***	&	-0.0065***	\\
	&		&	(0.0004)	&	(0.0004)	\\
Number of prenatal visits	&		&		&	-0.0005***	\\
	&		&		&	(0.0000)	\\
Constant	&		&	-0.0035***	&	-0.0009***	\\
	&		&	(0.0001)	&	(0.0002)	\\
R-squared	&		&	0.0020	&	0.0021	\\ \hline
\multicolumn{4}{p{12cm}}{\begin{footnotesize}Notes: Standard errors are clustered at the municipality level. Number of observations = 20,013,814. *** $p<0.01$, ** $p<0.05$, * $p<0.1$ \end{footnotesize}}\\




\end{tabular}}


\end{center}
\end{table}

%\begin{table}[htpb!] 
\caption{Probability of Giving Birth to Twins (Chile)} 
\label{TWINtab:Chile} 
\begin{center}\begin{tabular}{lclc} \toprule \toprule 
&(1)&&\\
Twin$\times$100&&&\\\midrule
\multicolumn{2}{l}{\textsc{Pre-Pregnancy}}&\multicolumn{2}{l}{\textsc{Pregnancy}}\\\begin{footnotesize}\end{footnotesize}&\begin{footnotesize}\end{footnotesize}&\begin{footnotesize}\end{footnotesize}&\begin{footnotesize}\end{footnotesize}\\
Income p.c.&-0.006&Smoked&-0.573\\
&	(0.011)
&&	(0.416)
\\
Income p.c. squared&0.000&Drugs (infrequent)&-0.119\\
&	(0.000)
&&	(1.646)
\\
Secondary Education&0.142&Drugs (frequent)&-1.872***\\
&	(0.300)
&&	(0.344)
\\
Tertiary Education&1.507***&Alcohol (infrequent)&-0.002\\
&	(0.583)
&&	(0.570)
\\
Low Weight&-0.589&Alcohol (frequent)&-1.891***\\
&	(0.471)
&&	(0.290)
\\
Obese&-1.997***&No Check-ups&-1.031\\
&	(0.766)
&&	(0.966)
\\
Mother's Age&0.410***&Hospital Birth&0.939***\\
&	(0.133)
&&	(0.344)
\\
Mother's Age Squared&-0.007***&Diabetes&-0.255\\
&	(0.002)
&&	(0.505)
\\
Indigenous&-1.027***&Depression&0.031\\
&	(0.395)
&&	(0.416)
\\
&&&\\
Observations&14268&R-squared&0.00\\
\midrule\multicolumn{4}{p{11cm}}{\begin{footnotesize}\textsc{Notes:} Data comes from the Encuesta Longitudinal de Primera Infancia (ELPI) from Chile. Education at each level are dummy variables, primary education is the omitted base. Regional controls and child age fixed effects are omitted for clarity. Heteroscedasticity robust standard errors are presented in parenthesis.$^{*}$p$<$0.1; $^{**}$p$<$0.05; $^{***}$p$<$0.01\end{footnotesize}}\\ \hline \normalsize \end{tabular}\end{center}\end{table}

%\begin{table}[htpb!] 
\caption{Probability of Giving Birth to Twins (Chile)} 
\label{TWINtab:Chile} 
\begin{center}\begin{tabular}{lclc} \toprule \toprule 
&(1)&&\\
Twin$\times$100&&&\\\midrule
\multicolumn{2}{l}{\textsc{Pre-Pregnancy}}&\multicolumn{2}{l}{\textsc{Pregnancy}}\\\begin{footnotesize}\end{footnotesize}&\begin{footnotesize}\end{footnotesize}&\begin{footnotesize}\end{footnotesize}&\begin{footnotesize}\end{footnotesize}\\
Income p.c.&-0.006&Smoked&-0.573\\
&	(0.011)
&&	(0.416)
\\
Income p.c. squared&0.000&Drugs (infrequent)&-0.119\\
&	(0.000)
&&	(1.646)
\\
Secondary Education&0.142&Drugs (frequent)&-1.872***\\
&	(0.300)
&&	(0.344)
\\
Tertiary Education&1.507***&Alcohol (infrequent)&-0.002\\
&	(0.583)
&&	(0.570)
\\
Low Weight&-0.589&Alcohol (frequent)&-1.891***\\
&	(0.471)
&&	(0.290)
\\
Obese&-1.997***&No Check-ups&-1.031\\
&	(0.766)
&&	(0.966)
\\
Mother's Age&0.410***&Hospital Birth&0.939***\\
&	(0.133)
&&	(0.344)
\\
Mother's Age Squared&-0.007***&Diabetes&-0.255\\
&	(0.002)
&&	(0.505)
\\
Indigenous&-1.027***&Depression&0.031\\
&	(0.395)
&&	(0.416)
\\
&&&\\
Observations&14268&R-squared&0.00\\
\midrule\multicolumn{4}{p{11cm}}{\begin{footnotesize}\textsc{Notes:} Data comes from the Encuesta Longitudinal de Primera Infancia (ELPI) from Chile. Education at each level are dummy variables, primary education is the omitted base. Regional controls and child age fixed effects are omitted for clarity. Heteroscedasticity robust standard errors are presented in parenthesis.$^{*}$p$<$0.1; $^{**}$p$<$0.05; $^{***}$p$<$0.01\end{footnotesize}}\\ \hline \normalsize \end{tabular}\end{center}\end{table}

%\begin{table}[htpb!] 
\caption{Probability of Giving Birth to Twins (Scotland)} 
\label{TWINtab:Scotland} 
\begin{center}\begin{tabular}{lclc} \toprule \toprule 
&(1)&&\\
Twin$\times$100&&&\\\midrule
\multicolumn{2}{l}{\textsc{Pre-Pregnancy}}&\multicolumn{2}{l}{\textsc{Pregnancy}}\\\begin{footnotesize}\end{footnotesize}&\begin{footnotesize}\end{footnotesize}&\begin{footnotesize}\end{footnotesize}&\begin{footnotesize}\end{footnotesize}\\
Deprivation Index (Quintile 2)&-1.628**
&Smoker&0.001
\\
&(0.958)
&&(0.669)
\\
Deprivation Index (Quintile 3)&-0.188
&Previous Smoker&1.717**
\\
&(0.967)
&&(0.877)
\\
Deprivation Index (Quintile 4)&-0.421
&Alcohol (1-2 per week)&-4.498*
\\
&(0.934)
&&(1.935)
\\
Deprivation Index (Quintile 5)&-1.132
&Alcohol (3+ per week)&-3.030*
\\
&(0.920)
&&(1.543)
\\
Height&0.306***
&Overweight&-0.092
\\
&(0.044)
&&(0.643)
\\
Married&3.272***
&Obese&1.350**
\\
&(0.878)
&&(0.746)
\\
Age&-0.337
&Diabetes&-0.188
\\
&(0.400)
&&(0.967)
\\
Age Squared&0.020***
&&\\
&(0.007)
&&\\
&&&\\
Observations&193254
&R-squared&0.01\\
\midrule\multicolumn{4}{p{11cm}}{\begin{footnotesize}\textsc{Notes:} Data comes from the ADD NOTE HERE!.$^{*}$p$<$0.1; $^{**}$p$<$0.05; $^{***}$p$<$0.01\end{footnotesize}}\\ \hline \normalsize \end{tabular}\end{center}\end{table}

%\begin{table}
\caption{Probability of Twin Birth: Spain Administrative Data 2007--2012}
\begin{center}
\scalebox{0.6}{
\begin{tabular}{lc}
\toprule
VARIABLES	&	Twin*100 \\	\midrule
	&	(1)	\\
Primary Education (Mother)	&	  -0.047**	\\
	&	 (0.024)	\\
      Secondary Education (Mother)	&	  -0.116***	\\
	&	(0.025)	\\
   Tertiary Education (Mother)	&	  -0.031	\\
	&	(0.027)	\\
  Parents are Immigrants	&	  -0.170***	\\
	&	(0.033)	\\
        City	&	   0.381***	\\
	&	(0.032)	\\
 Married	&	   0.638***	\\
	&	(0.018)	\\
    Mother's Age	&	   1.63***	\\
	&	(0.096)	\\
    Mother's Age$^2$	&	  -0.058***	\\
	&	(0.003)	\\
   Father's Age	&	  -0.173***	\\
	&	(0.047)	\\
   Father's Age$^2$	&	   0.006***	\\
	&	(0.001)	\\
   No Father	&	  -0.483	\\
	&	(0.539)	\\
    Constant	&	  -13.20	\\
	&	(0.940)	\\
	& \\
	Observations & 2,869,329 \\
	$R^2$ & 0.01 \\ \bottomrule
\end{tabular}}
\end{center}
\end{table}

%\begin{table}[htpb!] 
\caption{Probability of Giving Birth to Twins (Sweden Birth Registry)} \label{TWINtab:twinregS} 
\begin{center}
\scalebox{0.64}{
\begin{tabular}{lccc} \toprule \toprule 
&(1)&(2)&(3)\\
Twin*100&All&\multicolumn{2}{c}{Time}\\
 \cmidrule(r){3-4} 
&&1991-2010&1983-1990\\\midrule
\begin{footnotesize}\end{footnotesize}&\begin{footnotesize}\end{footnotesize}&\begin{footnotesize}\end{footnotesize}&\begin{footnotesize}\end{footnotesize}\\
Age&0.257***&0.483***&0.277***\\
&(0.031)&(0.047)&(0.040)\\
Age Squared&-0.002***&-0.005***&-0.003***\\
&(0.001)&(0.001)&(0.001)\\
Education (years)&0.014&0.011&-0.013\\
&(0.009)&(0.014)&(0.012)\\
Height&0.038***&0.047***&0.019***\\
&(0.003)&(0.003)&(0.004)\\
Smoked&0.252***&0.237***&0.208***\\
&(0.045)&(0.069)&(0.056)\\
Alcohol&0.160&-2.381***&0.137\\
&(1.176)&(0.414)&(1.192)\\
&&&\\R-squared&0.01&0.01&0.01\\
Observations &1,628,737&1,098,076&530,661\\
\hline\hline\multicolumn{4}{p{8.3cm}}{\begin{footnotesize}\textsc{Notes:} Conditional on birth cohorts and marriage status FEs.  Linear probability estimates.  Standard errors in parenthesis are clustered at the level of the mother.
$^{*}$p$<$0.1; $^{**}$p$<$0.05; $^{***}$p$<$0.01
 \end{footnotesize}}\\ \hline \normalsize \end{tabular}}\end{center}\end{table} 

%\begin{table}[htpb!] 
\caption{Probability of Giving Birth to Twins (UK)} 
\label{TWINtab:UK} 
\begin{center}
\scalebox{0.6}{
\begin{tabular}{lclc} \toprule \toprule 
&(1)&&\\
Twin$\times$100&&&\\\midrule
\midrule\multicolumn{4}{p{11cm}}{\begin{footnotesize}\textsc{Notes:}$^{*}$p$<$0.1; $^{**}$p$<$0.05; $^{***}$p$<$0.01\end{footnotesize}}\\ \hline \normalsize \end{tabular}}\end{center}\end{table}

%\begin{table}[htpb!]
\caption{United States Birth Registry (Administrative 2003-2012)}
\begin{center}
\scalebox{0.54}{
\begin{tabular}{lcccc} \hline
 & (1) & (2) & (3) & (4) \\
VARIABLES & twin100 & twin100 & twin100 & twin100 \\ \hline
\vspace{4pt} & \begin{footnotesize}\end{footnotesize} & \begin{footnotesize}\end{footnotesize} & \begin{footnotesize}\end{footnotesize} & \begin{footnotesize}\end{footnotesize} \\
africanAmerican & 0.554*** & 0.437*** & 0.437*** & 0.331*** \\
\vspace{4pt} & \begin{footnotesize}(0.00911)\end{footnotesize} & \begin{footnotesize}(0.0142)\end{footnotesize} & \begin{footnotesize}(0.0142)\end{footnotesize} & \begin{footnotesize}(0.0143)\end{footnotesize} \\
otherRace & 0.0433*** & 0.135*** & 0.135*** & -0.680*** \\
\vspace{4pt} & \begin{footnotesize}(0.00886)\end{footnotesize} & \begin{footnotesize}(0.0148)\end{footnotesize} & \begin{footnotesize}(0.0148)\end{footnotesize} & \begin{footnotesize}(0.0200)\end{footnotesize} \\
meducSecondary & 0.00124 & 0.0124 & 0.0124 & 0.810*** \\
\vspace{4pt} & \begin{footnotesize}(0.00881)\end{footnotesize} & \begin{footnotesize}(0.0159)\end{footnotesize} & \begin{footnotesize}(0.0159)\end{footnotesize} & \begin{footnotesize}(0.0203)\end{footnotesize} \\
meducTertiary & 1.124*** & 1.110*** & 1.110*** & 2.063*** \\
\vspace{4pt} & \begin{footnotesize}(0.00930)\end{footnotesize} & \begin{footnotesize}(0.0158)\end{footnotesize} & \begin{footnotesize}(0.0158)\end{footnotesize} & \begin{footnotesize}(0.0209)\end{footnotesize} \\
tobaccoUse & -0.327*** & -0.424*** & -0.424*** & -0.422*** \\
\vspace{4pt} & \begin{footnotesize}(0.0126)\end{footnotesize} & \begin{footnotesize}(0.0181)\end{footnotesize} & \begin{footnotesize}(0.0181)\end{footnotesize} & \begin{footnotesize}(0.0182)\end{footnotesize} \\
alcoholUse &  & -1.233*** & -1.233*** & -1.182*** \\
\vspace{4pt} & \begin{footnotesize}\end{footnotesize} & \begin{footnotesize}(0.0648)\end{footnotesize} & \begin{footnotesize}(0.0648)\end{footnotesize} & \begin{footnotesize}(0.0651)\end{footnotesize} \\
anemia &  &  &  & -1.349*** \\
\vspace{4pt} & \begin{footnotesize}\end{footnotesize} & \begin{footnotesize}\end{footnotesize} & \begin{footnotesize}\end{footnotesize} & \begin{footnotesize}(0.0335)\end{footnotesize} \\
diabetes &  &  &  & -0.408*** \\
\vspace{4pt} & \begin{footnotesize}\end{footnotesize} & \begin{footnotesize}\end{footnotesize} & \begin{footnotesize}\end{footnotesize} & \begin{footnotesize}(0.0280)\end{footnotesize} \\
chyper &  &  &  & -0.883*** \\
\vspace{4pt} & \begin{footnotesize}\end{footnotesize} & \begin{footnotesize}\end{footnotesize} & \begin{footnotesize}\end{footnotesize} & \begin{footnotesize}(0.0528)\end{footnotesize} \\
phyper &  &  &  & -4.128*** \\
\vspace{4pt} & \begin{footnotesize}\end{footnotesize} & \begin{footnotesize}\end{footnotesize} & \begin{footnotesize}\end{footnotesize} & \begin{footnotesize}(0.0267)\end{footnotesize} \\
eclamp &  &  &  & -5.686*** \\
\vspace{4pt} & \begin{footnotesize}\end{footnotesize} & \begin{footnotesize}\end{footnotesize} & \begin{footnotesize}\end{footnotesize} & \begin{footnotesize}(0.0896)\end{footnotesize} \\
Constant & 5.461*** & 5.326*** & 5.326*** & 32.75*** \\
 & \begin{footnotesize}(0.0525)\end{footnotesize} & \begin{footnotesize}(0.0806)\end{footnotesize} & \begin{footnotesize}(0.0806)\end{footnotesize} & \begin{footnotesize}(0.293)\end{footnotesize} \\
\vspace{4pt} & \begin{footnotesize}\end{footnotesize} & \begin{footnotesize}\end{footnotesize} & \begin{footnotesize}\end{footnotesize} & \begin{footnotesize}\end{footnotesize} \\
Observations & 38,910,055 & 16,605,619 & 16,605,619 & 12,219,256 \\
 $R^2$ & 0.008 & 0.008 & 0.008 & 0.012 \\ \hline
\multicolumn{5}{c}{\begin{footnotesize} Standard errors in parentheses\end{footnotesize}} \\
\multicolumn{5}{c}{\begin{footnotesize} *** p$<$0.01, ** p$<$0.05, * p$<$0.1\end{footnotesize}} \\
\end{tabular}}
\end{center}
\end{table}


\newpage
\section{Data Appendix}
\label{TWINscn:dataApp}
Main IV and OLS results for this paper are based on DHS and NHIS data described
in section \ref{TWINscn:data}.  These data are downloaded directly off the web
and merged to form the estimation samples of interest. For DHS data, we use two
surveys: the Individual (woman) Recode (IR), and the Household Recode (HR)
providing education for each household member.  For NHIS data, we merge three
of the datafiles made available by the CDC: familyxx, household, and person.
In each case, full generating code for this process is made available on the
authors' websites.  This code downloads, merges and cleans DHS and NHIS data to
produce the datasets (one line per child) used in analysis.

In auxiliary regressions examining the characterstics of mothers and the
relationship these characteristics and twin births and miscarriage, we consult
a large number of other datasets.  These are the following:
\begin{itemize}
\item United States National Vital Statistics Birth Data
\item United States National Vital Statistics Fetal Death Data
\item Spanish Vital Statistics (INE)
\item The Swedish Medical Birth Register
\item Scottish Vital Statistics
\item Longitudinal Early Life Survey, Chile (ELPI)
\end{itemize}

In the case of the first 5 datasets (administrative records of births and/or
fetal deaths), we use all recorded instances, focusing on twins as our
outcome variable of interest.  Depending upon the data source, we use all
available measures of pre-determined maternal health stocks or family
socioeconomic indicators.  The ELPI survey from Chile focuses on child early
life, and records mother's behaviours before, during and after pregnancy,
along with child birth outcomes.  We use all children from the first wave of
this survey to run the twin regression included in the appendix tables.
Further notes regarding each dataset and the particular years and number of
births can be found in the notes to each table.

\section{Resampling and Simulation Based Estimation of $\gamma$}
\label{ATWINscn:gammaSim}
\subsection{Bootstrap Confidence Intervals}
The methodology to estimate $\gamma$ in equations (\ref{TWINeqn:realgamma}) and
(\ref{TWINeqn:realgammaN}) is described in section \ref{TWINscn:gamma} of the
paper.  In the case of \citeauthor{Conleyetal2012}'s UCI approach, this estimate
is then sufficient to produce bounds on $\beta_1$, assuming that:
$\gamma\in[0,2\hat\gamma]$. We scale $\hat\gamma$ by the factor of 2 in order for
this value to fall precisely in the middle of the range. \citet{Conleyetal2012}
provide a similar example to calculate the returns to education using the UCI
approach.  In the case of the more precise LTZ approach (our preferred method)
the logic is similar, however now we must form a prior over the entire
distribution of $\gamma$.  Calculating the variance of $\gamma$ is not as
straightforward as using the variance-covariance matrix corresponding to each of
the estimates $\hat\phi^t$ and $\hat\phi^q$.  In this case however we can use
bootstrapping to calculate $J$ replications of $\hat\phi^t\times\hat\phi^q$, and
from these estimates construct an estimated distribution of $\hat\gamma$, which
allows us to determine our prior for the distribution of $\gamma$.  From this
empirical distribution, we observe the estimated mean and standard deviation, and
finally test whether the distribution is normal using a Shapiro Wilk test for
normality. We also use Kolmogorov-Smirnov tests for equality of distributions to
test whether the distribution is more likely to be log normal, uniform, and a
number of other known analytical distributions. In order to do this, we first
estimate the empirical distribution as described previously.  We then observe the
mean $\hat\mu$ and the standard deviation $\hat\sigma$, and run a one-sample test
to determine whether the observed empirical distribution is is significantly
different to each analytical distribution $\mathcal{N}(\hat\mu,\hat\sigma^2)$,
$U(\hat\mu,\hat\sigma^2)$ or $\ln\mathcal{N}(\hat\mu,\hat\sigma^2)$.

\subsection{Simulation-Based Estimation for non-Normal Distributions}
Estimates of the full distribution of $\gamma$ are presented in Figures
\ref{TWINfig:gammaBootsN} and \ref{TWINfig:gammaBootsL}.  These are
the estimated $\hat\gamma_j$ from $j \in \{1,\ldots,100\}$ bootstrap
replications for $\gamma$ in Nigeria and the United States.  In all cases,
when the underlying empirical distribution is tested for equality against
the overlaid analytical distribution (uniform, normal, log normal, $\chi^2$),
the normal distribution provides the best fit of the analytical with the
empirical distribution.\footnote{In the US, We cannot reject that $\gamma$
  is normal with a p-value of 0.xxxx.  In this case, although we can't reject
  that $\gamma$ is log normal, the p-value is much lower, at 0.xxxx.  Similar
  values for Nigeria suggest xxxx.
}

However, the underlying distribution appears to not be perfectly normal,
and it appears doubtful that this would be the case in distribution.
Fortunately, \citet{Conleyetal2012} describe a simulation-based estimation
method to calculate $\gamma$ in the case of a non-normal distribution for
$\gamma$.  We have followed this methodology using the empirical distribution
calculated bootstrapping for $\gamma$.  This code has been publicly released
as \texttt{plausexog} for Stata \citep{Clarke2014}.  The simulation-based
estimation procedure is described fully in \citet{Conleyetal2012} p.\ 265
as a five step algorithm.  The procedure consists of taking repeated draws
from the variance-covariance matrix estimated using IV with the plausibly
exogenous instrument, and in each case adding to it a draw from the
distribution of $\gamma$, scaled by a quantity which depends
on the strength of the instrument. \citeauthor{Conleyetal2012} refer to the
underlying distibution of $\gamma$ as $F$, and the scale parameter as $A$,
where $A=(X^\prime Z(Z^\prime Z)^{-1}Z^\prime X)^{-1}(X^\prime Z)$.
These repeated draws then lead to a large number of estimates for $\beta$,
the parameter of interest, and a 95\% confidence interval is taken by
forming $[\hat\beta-c_{1-\alpha/2},\hat\beta+c_{\alpha/2}]$, where $c$ are
percentiles of the distribution of simulated estimates.

Thus, as well as estimating the LTZ case where we assume that $\gamma$
is distributed $\sim \mathcal{N}(\mu_{\hat\gamma},\sigma^2_{\hat\gamma})$,
we can estimate a version fully utilizing the bootstrapped distribution
of $\hat\gamma$ described in the previous sub-section.  In this case, we
use as $F$, the distribution of $\gamma$, the empirically estimated
distribution of $\gamma$.  The simulation based algorithm then consists
of taking $b\in 1,\ldots,B$ draws from the empirically estimated $F$, as
well as $B$ draws from the variance-covariance matrix, and defining the
95\% confidence interval based on the 2.5 and 97.5\% quintiles of the
resulting simulated values for $\beta$.


%\end{doublespace}

\newpage
\bibliography{./BiBBase1}
\end{document}



  \begin{figure}[htpb!]
    \begin{center}
      \caption{Temperature and Good Season (40-45 ``Teachers'' vs ``Non-Teachers'')}
      \label{bqFig:coldTeach4045}
      \begin{subfigure}{.5\textwidth}
        \centering
        \includegraphics[scale=0.55]{./../results/hispall/ipums/graphs/StateTemp_4045Teacher_cold_weight.eps}
        \caption{``Teachers''}
        \label{fig:Educ3}
      \end{subfigure}%
      \begin{subfigure}{.5\textwidth}
        \centering
        \includegraphics[scale=0.55]{./../results/hispall/ipums/graphs/StateTemp_4045NonTeacher_cold_weight.eps}
        \caption{``Non-Teachers''}
        \label{fig:NonEduc3}
      \end{subfigure}
    \end{center}
    \floatfoot{\textsc{Notes to figure}: State averages of good season are plotted against the
      coldest average monthly temperature in the state. Panel A includes all workers who are in
      ``Education, Training and Library Occupations'', while Panel B includes all other workers. }
  \end{figure}

  \begin{figure}[htpb!]
    \begin{center}
      \centering
      \caption{ART Conceptions by Month}
      \includegraphics[scale=0.74]{./../results/hisp/graphs/proportionMonthART.eps}
      \label{fig:ARTMonth}
    \end{center}
    \floatfoot{\textsc{Notes to figure \ref{fig:ARTMonth}}: Proportion of ART births
      are calculated using data from 2009-2013 for our main sample.  The proportion
      is calculated as: (ART conceptions)/(Non-ART Conceptions + ART Conceptions).}
  \end{figure}

