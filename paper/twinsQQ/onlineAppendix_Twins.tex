\documentclass[a4paper, 11pt]{article}

\usepackage{amsfonts}
\usepackage{amsmath}
\usepackage{amsthm}
\usepackage{amssymb}
\usepackage{appendix}
\usepackage{blindtext}
\usepackage{bm}
\usepackage{booktabs}
\usepackage{caption}
\usepackage[usenames, dvipsnames]{color}
\usepackage{dcolumn}
\usepackage{graphicx}
\usepackage{epsfig}
\usepackage{epstopdf}
\epstopdfsetup{update}
\usepackage[capposition=top]{floatrow}
\usepackage{helvet}
\usepackage{hyperref}
\usepackage{indentfirst}
\usepackage{longtable}
\usepackage{lscape}
\usepackage{morefloats}
\usepackage{multirow}
\usepackage{natbib} \bibliographystyle{abbrvnat}\bibpunct{(}{)}{;}{a}{,}{,}
%\bibliographystyle{abbrvnat}\bibpunct{(}{)}{;}{a}{,}{,}
\usepackage{setspace}
\usepackage{subcaption}
\usepackage[capposition=top]{floatrow}
\usepackage{subfloat}
\usepackage[latin1]{inputenc}
\usepackage{tikz}
%\usepackage[pdf]{pstricks}
\usepackage{xr}
\externaldocument{BhalotraClarke_TwinQQ}
 
%\usepackage{pdfpages}
%\usepackage{rotating}
%\usepackage{setspace}
%\usepackage{subcaption}
%\usepackage{subfloat}
%\usepackage{url}
%\usepackage{wrapfig}



\usetikzlibrary{trees}
\usetikzlibrary{decorations.markings}
\newcommand{\twinfolder}{/home/damian/investigacion/Activa/Twins}


\theoremstyle{plain}
\newtheorem{thm}{Theorem}
\newtheorem{cor}{Corollary}
\newtheorem{lem}[thm]{Lemma}
\newtheorem{proposition}{Proposition}
\newtheorem{assumption}{Assumption}
\newtheorem{definition}{Definition}

%MARGINS
 \topmargin   =  -0.5in
 \headsep     =  0.7in
 \oddsidemargin= -0.4in
 \evensidemargin=-0.2in
 \textheight  =  9.4in
 \textwidth   =  7.25in


\newcommand{\fmt}{.eps}
%\newcommand{\fmt}{.png}
\hypersetup{
    colorlinks=true,
    linkcolor=BlueViolet,
    citecolor=Black,
    filecolor=BlueViolet,
    urlcolor=BlueViolet
}


 \renewcommand{\familydefault}{\sfdefault}

\setcounter{page}{0}
\begin{document}

%\begin{doublespace}
 

\renewcommand\thesection{\Alph{section}}
\renewcommand*{\thepage}{A\arabic{page}}
\setcounter{section}{0}

\setcounter{table}{0}
\renewcommand{\thetable}{A\arabic{table}}
\setcounter{figure}{0}
\renewcommand{\thefigure}{A\arabic{figure}}


\setcounter{page}{1}
  \begin{center}
    \textbf{ONLINE APPENDIX} \\
    \vspace{4mm}
    For the paper: \\
     \vspace{6mm}
    {\large \textsc{The Twin Instrument}} \\
    Sonia Bhalotra and Damian Clarke
  \end{center}


\tableofcontents
\setlength{\parskip}{1em}
\setlength{\parindent}{4em}

\newpage
\section{Appendix Figures}
\setcounter{figure}{0}
\renewcommand{\thefigure}{A\arabic{figure}}

\begin{figure}[htpb!]
\begin{center}
\caption{Height and Selective Survival}
\label{TWINfig:bord}
\includegraphics[scale=0.92]{\twinfolder/Figures/MMRcuts.eps} 
\end{center}
\end{figure}




\clearpage
\section{Appendix Tables}
\input{\twinfolder/Tables/summaryStatsWorld.tex}
\input{\twinfolder/Tables/twinEffectsCond.tex}
\input{\twinfolder/Tables/twinEffectsUncondUnstand.tex}
\input{\twinfolder/Tables/BalanceAll.tex}
\input{\twinfolder/Tables/Smoking.tex}
\input{\twinfolder/Tables/twinEffectsIVF.tex}
\begin{landscape}\begin{table}[htpb!] 
\caption{Probability of Giving Birth to Twins} \label{TWINtab:twinreg1} 
\begin{center}\begin{tabular}{lcccccc} \toprule \toprule 
&(1)&(2)&(3)&(4)&(5)&(6)\\
Twin$\times$100&All&\multicolumn{2}{c}{Income}&\multicolumn{2}{c}{Time}&Prenatal\\
 \cmidrule(r){3-4} \cmidrule(r){5-6} 
&&Low inc&Middle inc&1990-2013&1972-1989&\\\midrule
\begin{footnotesize}\end{footnotesize}&\begin{footnotesize}\end{footnotesize}&\begin{footnotesize}\end{footnotesize}&\begin{footnotesize}\end{footnotesize}&\begin{footnotesize}\end{footnotesize}&\begin{footnotesize}\end{footnotesize}&\begin{footnotesize}\end{footnotesize}\\
Age&0.491***&0.489***&0.498***&0.587***&0.168***&0.632***\\
&(0.026)&(0.033)&(0.045)&(0.030)&(0.064)&(0.040)\\
Age Squared&-0.006***&-0.006***&-0.007***&-0.008***&-0.000&-0.009***\\
&(0.000)&(0.001)&(0.001)&(0.001)&(0.001)&(0.001)\\
Age First Birth&-0.051***&-0.082***&-0.002&-0.050***&-0.051***&-0.041***\\
&(0.008)&(0.010)&(0.013)&(0.009)&(0.015)&(0.013)\\
Education (years)&0.027*&0.065***&-0.008&0.044**&-0.008&-0.071**\\
&(0.016)&(0.021)&(0.027)&(0.019)&(0.028)&(0.028)\\
Education squared&-0.001&-0.005**&0.001&-0.002&0.002&0.003\\
&(0.001)&(0.002)&(0.002)&(0.001)&(0.002)&(0.002)\\
Height&0.057***&0.056***&0.058***&0.063***&0.038***&0.059***\\
&(0.004)&(0.005)&(0.006)&(0.005)&(0.007)&(0.007)\\
BMI&0.050***&0.059***&0.043***&0.046***&0.056***&0.045***\\
&(0.006)&(0.008)&(0.008)&(0.007)&(0.009)&(0.011)\\
Prenatal (Doctor)&&&&&&0.917***\\
&&&&&&(0.129)\\
Prenatal (Nurse)&&&&&&0.076\\
&&&&&&(0.109)\\
Prenatal (None)&&&&&&-0.479***\\
&&&&&&(0.133)\\
&&&&&&\\R-squared&0.01&0.01&0.01&0.01&0.00&0.01\\
Observations &2271948&1430703&841245&1660253&611695&624990\\
\hline\hline\multicolumn{7}{p{14.3cm}}{\begin{footnotesize}\textsc{Notes:} All specifications include a full set of year of birth and  country dummies, and are estimated as linear probability models.  Twin is multiplied by 100 for presentation.  Height is measured in cm  and BMI is weight in kg divided by height in metres squared. l  Prenatal care variables are only recoreded for recent births.  As  such, column (6) is estimated only for that subset of births where  these observations are made.
$^{*}$p$<$0.1; $^{**}$p$<$0.05; $^{***}$p$<$0.01
 \end{footnotesize}}\\ \hline \normalsize \end{tabular}\end{center}\end{table}\end{landscape} 

\begin{table}[htpb]
\caption{Probability of Giving Births to Twins (NHIS, USA)}
\begin{center}
\scalebox{0.64}{
\begin{tabular}{lcc} \toprule
&(1)&(2) \\
VARIABLES&Twin$\times$100&Twin$\times$100 \\ \midrule
&& \\
Mother's Height&0.0416**&0.0406** \\
&(0.0201)&(0.0201) \\
Mother's Education&0.0084&0.0033 \\
&(0.0162)&(0.0164) \\
Smokes (pre-Pregnancy)&-0.119&-0.0983 \\
&(0.115)&(0.116) \\
Mother's Age&0.0121&0.0108 \\
&(0.0446)&(0.0446) \\
Mother's Age$^2$ &-0.0008&-0.0008 \\
&(0.0006)&(0.0006) \\
Age First Birth &0.166***&0.164*** \\
&(0.0135)&(0.0136) \\
BMI &0.0123***&0.0130*** \\
&(0.0034)&(0.0034) \\
Mother Good Health&&0.203* \\
&&(0.116) \\
Mother Poor Health&&-0.00284 \\
&&(0.189) \\
Constant&-4.091***&-4.101*** \\
&(1.542)&(1.543) \\
&& \\
Observations&105,879&105,879 \\
R-squared&0.004&0.004 \\ \midrule
%\multicolumn{3}{ p{5cm} }{\begin{footnotesize}\textsc{Notes:} Standard errors in parentheses. *** p$<$0.01; ** p$<$0.05; * p$<$0.1\end{footnotesize}}\bottomrule
\end{tabular}}
\end{center}
\end{table}


%\input{\twinfolder/Tables/OLS_bounds.tex}
\input{\twinfolder/Tables/literature.tex}
\input{\twinfolder/Tables/LeeBounds.tex}
\begin{landscape}
\input{\twinfolder/Tables/FDeath_Cond.tex}
\end{landscape}

%\begin{table}
\caption{Twins, Miscarriage and Maternal Health (Administrative Data from Spain)}
\begin{center}
\scalebox{0.5}{
\begin{tabular}{lc}
\hline
VARIABLES	&	Fetal Death*100 \\	\hline
	&	(1)	\\
Primary	&	  -0.60179***	\\
	&	 (0.01456)	\\
Primary*Twin	&	   -0.6618***	\\
	&	 (0.20926)	\\
Secondary	&	  -0.71998*	\\
	&	  (0.0151)	\\
Secondary*Twin	&	  -0.55901***	\\
	&	 .0020978	\\
Tertiary	&	  -0.80019***	\\
	&	(0.01582)	\\
Tertiary*Twin	&	  -0.65091***	\\
	&	(0.20866)	\\
Immigration	&	  -0.07223***	\\
	&	 (0.0171)	\\
Immigration*Twin	&	  0.22871	\\
	&	(0.29614)	\\
Citm	&	  -0.00321	\\
	&	(0.01584)	\\
Citm*Twin	&	  0.09566	\\
	&	 (0.26876)	\\
Married	&	  -0.07354***	\\
	&	 (0.00759)	\\
Married*Twin	&	  -0.08978	\\
	&	 (0.11893)	\\
No Father	&	0.68626***	\\
	&	 (0.23825)	\\
No Father*Twin	&	3.25232	\\
	&	(4.09309)	\\
Constant	&	-1.35966***	\\
	&	(0.33674)	\\
	& \\
	Obs & 2869329 \\
	$R^2$ & 0.0044 \\ \hline
	\multicolumn{2}{p{5cm}}{Note: Spanish administrative births: 2007-2012.}
\end{tabular}}
\end{center}
\end{table}

%\begin{table}[htpb!]\caption{Principal IV Results}
\label{TWINtab:IVAll}
\begin{center}\scalebox{0.55}{
\begin{tabular}{lcccp{2mm}cccp{2mm}ccc}
\toprule \toprule 
&\multicolumn{3}{c}{2+}&&\multicolumn{3}{c}{3+}&&\multicolumn{3}{c}{4+}\\ \cmidrule(r){2-4} \cmidrule(r){6-8} \cmidrule(r){10-12} 
\textsc{School Z-Score}&Base&+H&+S\&H&&Base&+H&+S\&H&&Base&+H&+S\&H\\ \midrule 
\begin{footnotesize}\end{footnotesize}& 
\begin{footnotesize}\end{footnotesize}& 
\begin{footnotesize}\end{footnotesize}& 
\begin{footnotesize}\end{footnotesize}& 
\begin{footnotesize}\end{footnotesize}& 
\begin{footnotesize}\end{footnotesize}& 
\begin{footnotesize}\end{footnotesize}& 
\begin{footnotesize}\end{footnotesize}& 
\begin{footnotesize}\end{footnotesize}& 
\begin{footnotesize}\end{footnotesize}& 
\begin{footnotesize}\end{footnotesize}& 
\begin{footnotesize}\end{footnotesize}\\ 
\multicolumn{12}{l}{\textbf{All}}\\ 
Fertility&0.006&-0.026&-0.026&&-0.004&-0.036&-0.038*&&-0.017&-0.036&-0.035*\\
&(0.029)&(0.027)&(0.026)&&(0.024)&(0.022)&(0.021)&&(0.025)&(0.023)&(0.021)\\
\begin{footnotesize}\end{footnotesize}&\begin{footnotesize}\end{footnotesize}&\begin{footnotesize}\end{footnotesize}&\begin{footnotesize}\end{footnotesize}&\begin{footnotesize}\end{footnotesize}&\begin{footnotesize}\end{footnotesize}&\begin{footnotesize}\end{footnotesize}&\begin{footnotesize}\end{footnotesize}&\begin{footnotesize}\end{footnotesize}&\begin{footnotesize}\end{footnotesize}&\begin{footnotesize}\end{footnotesize}&\begin{footnotesize}\end{footnotesize}\\Observations&249536&249536&249536&&375987&375987&375987&&385389&385389&385389\\
\multicolumn{12}{l}{\textbf{Low-Income}}\\ 
Fertility&0.035&0.008&0.012&&0.016&-0.016&-0.027&&-0.011&-0.031&-0.024\\
&(0.034)&(0.032)&(0.031)&&(0.030)&(0.028)&(0.026)&&(0.029)&(0.027)&(0.025)\\
\begin{footnotesize}\end{footnotesize}&\begin{footnotesize}\end{footnotesize}&\begin{footnotesize}\end{footnotesize}&\begin{footnotesize}\end{footnotesize}&\begin{footnotesize}\end{footnotesize}&\begin{footnotesize}\end{footnotesize}&\begin{footnotesize}\end{footnotesize}&\begin{footnotesize}\end{footnotesize}&\begin{footnotesize}\end{footnotesize}&\begin{footnotesize}\end{footnotesize}\\Observations&149602&149602&149602&&232371&232371&232371&&246622&246622&246622\\
\multicolumn{12}{l}{\textbf{Middle-Income}}\\ 
Fertility&-0.065&-0.087*&-0.093**&&-0.046&-0.079**&-0.067*&&-0.027&-0.048&-0.054\\
&(0.053)&(0.049)&(0.047)&&(0.040)&(0.036)&(0.035)&&(0.043)&(0.040)&(0.037)\\
\begin{footnotesize}\end{footnotesize}&\begin{footnotesize}\end{footnotesize}&\begin{footnotesize}\end{footnotesize}&\begin{footnotesize}\end{footnotesize}&\begin{footnotesize}\end{footnotesize}&\begin{footnotesize}\end{footnotesize}&\begin{footnotesize}\end{footnotesize}&\begin{footnotesize}\end{footnotesize}&\begin{footnotesize}\end{footnotesize}&\begin{footnotesize}\end{footnotesize}\\Observations&99934&99934&99934&&143616&143616&143616&&138767&138767&138767\\
\multicolumn{12}{l}{\textbf{Adjusted Fertility}}\\ 
Fertility&0.017&-0.052&-0.055&&-0.013&-0.073*&-0.077*&&-0.033&-0.068&-0.066*\\
&(0.065)&(0.056)&(0.054)&&(0.047)&(0.043)&(0.040)&&(0.045)&(0.042)&(0.039)\\
\begin{footnotesize}\end{footnotesize}&\begin{footnotesize}\end{footnotesize}&\begin{footnotesize}\end{footnotesize}&\begin{footnotesize}\end{footnotesize}&\begin{footnotesize}\end{footnotesize}&\begin{footnotesize}\end{footnotesize}&\begin{footnotesize}\end{footnotesize}&\begin{footnotesize}\end{footnotesize}&\begin{footnotesize}\end{footnotesize}&\begin{footnotesize}\end{footnotesize}\\Observations&249505&249505&249505&&375957&375957&375957&&385363&385363&385363\\
\multicolumn{12}{l}{\textbf{Twins and Pre-Twins}}\\ 
Fertility&-0.021&-0.073***&-0.078***&&-0.019&-0.062***&-0.067***&&-0.018&-0.039**&-0.046**\\
&(0.024)&(0.021)&(0.020)&&(0.020)&(0.018)&(0.018)&&(0.021)&(0.019)&(0.018)\\
\begin{footnotesize}\end{footnotesize}&\begin{footnotesize}\end{footnotesize}&\begin{footnotesize}\end{footnotesize}&\begin{footnotesize}\end{footnotesize}&\begin{footnotesize}\end{footnotesize}&\begin{footnotesize}\end{footnotesize}&\begin{footnotesize}\end{footnotesize}&\begin{footnotesize}\end{footnotesize}&\begin{footnotesize}\end{footnotesize}&\begin{footnotesize}\end{footnotesize}\\Observations&488815&488815&488815&&563177&563177&563177&&523197&523197&523197\\
\bottomrule
\end{tabular}}\end{center}\end{table}

%\input{\twinfolder/Tables/OLSPlus.tex}
\clearpage
\input{\twinfolder/Tables/DHS-togetherMidIncome.tex}
\input{\twinfolder/Tables/DHS-togetherLowIncome.tex} 
\input{\twinfolder/Tables/DHS-togetherGirls.tex}
\input{\twinfolder/Tables/DHS-togetherBoys.tex}
%\begin{table}[htpb!]\caption{Q-Q IV Estimates by Gender} 
\label{TWINtab:gend}\begin{center}\begin{tabular}{lcccccccc}
\toprule \toprule 
&\multicolumn{4}{c}{Females}&\multicolumn{4}{c}{Males}\\ 
\cmidrule(r){2-5} \cmidrule(r){6-9} 
&Base&Socioec&Health&Obs.&Base&Socioec&Health&Obs. \\ \midrule 
\begin{footnotesize}\end{footnotesize}&\begin{footnotesize}\end{footnotesize}&\begin{footnotesize}\end{footnotesize}&\begin{footnotesize}\end{footnotesize}&\begin{footnotesize}\end{footnotesize}&\begin{footnotesize}\end{footnotesize}&\\Two Plus &0.005&-0.039&-0.037&122,414&0.010&-0.010&-0.015&127,122\\
&(0.043)&(0.039)&(0.038)&&(0.040)&(0.038)&(0.036)&\\
\begin{footnotesize}\end{footnotesize}&\begin{footnotesize}\end{footnotesize}&\begin{footnotesize}\end{footnotesize}&\begin{footnotesize}\end{footnotesize}&\begin{footnotesize}\end{footnotesize}&\begin{footnotesize}\end{footnotesize}&\\Three Plus &-0.024&-0.056*&-0.052*&187,098&0.016&-0.015&-0.022&188,889\\
&(0.033)&(0.030)&(0.029)&&(0.030)&(0.028)&(0.027)&\\
\begin{footnotesize}\end{footnotesize}&\begin{footnotesize}\end{footnotesize}&\begin{footnotesize}\end{footnotesize}&\begin{footnotesize}\end{footnotesize}&\begin{footnotesize}\end{footnotesize}&\begin{footnotesize}\end{footnotesize}&\\Four Plus &-0.029&-0.052*&-0.053**&192,714&-0.005&-0.020&-0.018&192,675\\
&(0.032)&(0.029)&(0.027)&&(0.030)&(0.028)&(0.027)&\\
\midrule\multicolumn{9}{p{14.2cm}}{\begin{footnotesize}\textsc{Notes:} Female or male refers to the gender of the index child of the regression. 
All regressions include full controls including socioeconomic and maternal health variables.  The full lis of controls are available in 
the notes to table \ref{TWINtab:IVAll}.  Full IV results for male and female children are presented in table \ref{TWINtab:IVgend}. Standard errors are clustered 
 by mother.$^{*}$p$<$0.1; $^{**}$p$<$0.05; $^{***}$p$<$0.01
\end{footnotesize}} \\ \bottomrule 
\end{tabular}\end{center}\end{table}
\input{\twinfolder/Tables/AllNHIS-female.tex}
\input{\twinfolder/Tables/AllNHIS-male.tex}
\end{spacing}\begin{spacing}{1} 
\begin{longtable}{llccccccc}\caption{Full Survey Countries and Years} \\ 
\toprule\label{TWINtab:countries} 
& & \multicolumn{7}{c}{Survey Year} \\ \cmidrule(r){3-9} 
\textsc{Country}&\textsc{Income}&1&2&3&4&5&6&7\\ \midrule 
Albania&Middle&2008&&&&&&\\
Armenia&Low&2000&2005&2010&&&&\\
Azerbaijan&Middle&2006&&&&&&\\
Bangladesh&Low&1994&1997&2000&2004&2007&2011&\\
Benin&Low&1996&2001&2006&&&&\\
Bolivia&Middle&1989&1994&1998&2003&2008&&\\
Brazil&Middle&1986&1991&1996&&&&\\
Burkina Faso&Low&1993&1999&2003&2010&&&\\
Burundi&Low&1987&2010&&&&&\\
Cambodia&Low&2000&2005&2010&&&&\\
Cameroon&Middle&1991&1998&2004&2011&&&\\
Central African Republic&Low&1994&&&&&&\\
Chad&Low&1997&2004&&&&&\\
Colombia&Middle&1986&1990&1995&2000&2005&2010&\\
Comoros&Low&1996&&&&&&\\
Congo Brazzaville&Middle&2005&2011&&&&&\\
Congo, Dem&Low&2007&&&&&&\\
Cote d'Ivoire&Low&1994&1998&2005&2012&&&\\
Dominican Republic&Middle&1986&1991&1996&1999&2002&2007&\\
Ecuador&Middle&1987&&&&&&\\
Egypt, Arab Rep&Middle&1988&1992&1995&2000&2005&2008&\\
El Salvador&Middle&1985&&&&&&\\
Ethiopia&Low&2000&2005&2011&&&&\\
Gabon&Middle&2000&2012&&&&&\\
Ghana&Low&1988&1993&1998&2003&2008&&\\
Guatemala&Middle&1987&1995&&&&&\\
Guinea&Low&1999&2005&&&&&\\
Guyana&Middle&2005&2009&&&&&\\
Haiti&Low&1994&2000&2006&2012&&&\\
Honduras&Middle&2005&2011&&&&&\\
India&Low&1993&1999&2006&&&&\\
Indonesia&Low&1987&1991&1994&1997&2003&2007&2012\\
Jordan&Middle&1990&1997&2002&2007&&&\\
Kazakhstan&Middle&1995&1999&&&&&\\
Kenya&Low&1989&1993&1998&2003&2008&&\\
Kyrgyz Republic&Low&1997&&&&&&\\
Lesotho&Low&2004&2009&&&&&\\
Liberia&Low&1986&2007&&&&&\\
Madagascar&Low&1992&1997&2004&2008&&&\\
Malawi&Low&1992&2000&2004&2010&&&\\
Maldives&Middle&2009&&&&&&\\
Mali&Low&1987&1996&2001&2006&&&\\
Mexico&Middle&1987&&&&&&\\
Moldova&Middle&2005&&&&&&\\
Morocco&Middle&1987&1992&2003&&&&\\
Mozambique&Low&1997&2003&2011&&&&\\
Namibia&Middle&1992&2000&2006&&&&\\
Nepal&Low&1996&2001&2006&2011&&&\\
Nicaragua&Low&1998&2001&&&&&\\
Niger&Low&1992&1998&2006&&&&\\
Nigeria&Low&1990&1999&2003&2008&&&\\
Pakistan&Low&1991&2006&&&&&\\
Paraguay&Middle&1990&&&&&&\\
Peru&Middle&1986&1992&1996&2000&&&\\
Philippines&Middle&1993&1998&2003&2008&&&\\
Rwanda&Low&1992&2000&2005&2010&&&\\
Sao Tome and Principe&Middle&2008&&&&&&\\
Senegal&Low&1986&1993&1997&2005&2010&&\\
Sierra Leone&Low&2008&&&&&&\\
South Africa&Middle&1998&&&&&&\\
Sri Lanka&Low&1987&&&&&&\\
Sudan&Low&1990&&&&&&\\
Swaziland&Middle&2006&&&&&&\\
Tanzania&Low&1992&1996&1999&2004&2007&2010&2012\\
Thailand&Middle&1987&&&&&&\\
Togo&Low&1988&1998&&&&&\\
Trinidad and Tobago&Middle&1987&&&&&&\\
Tunisia&Middle&1988&&&&&&\\
Turkey&Middle&1993&1998&2003&&&&\\
Uganda&Low&1988&1995&2000&2006&2011&&\\
Ukraine&Middle&2007&&&&&&\\
Uzbekistan&Middle&1996&&&&&&\\
Vietnam&Low&1997&2002&&&&&\\
Yemen, Rep&Low&1991&&&&&&\\
Zambia&Low&1992&1996&2002&2007&&&\\
Zimbabwe&Middle&1988&1994&1999&2005&2010\\
\midrule\multicolumn{9}{p{13.3cm}}{\begin{footnotesize}\textsc{Notes:} Country income status is based upon World Bank classifications described at http://data.worldbank.org/about/country-classifications and available for download at http://siteresources.worldbank.org/DATASTATISTICS/Resources/OGHIST.xls (consulted 1 April, 2014).  Income status varies by country and time.  Where a country's status changed between DHS waves only the most recent status is listed above.  Middle refers to both lower-middle and upper-middle income countries, while low refers just to those considered to be low-income economies.\end{footnotesize}}  
\\ \bottomrule \end{longtable}\end{spacing}\begin{spacing}{1.5}

\newpage
\section{Data Appendix}
\label{TWINscn:dataApp}
Main IV and OLS results for this paper are based on DHS and NHIS data described
briefly in section \ref{TWINscn:data} and further below.  These data are 
downloaded directly off the web and merged to form the estimation samples of 
interest. For DHS data, we use two surveys: the Individual (woman) Recode (IR), 
and the Household Recode (HR) providing education for each household member.  
For NHIS data, we merge three of the datafiles made available by the CDC: familyxx, 
household, and person. In each case, full generating code for this process is 
made available on the authors' websites.  This code downloads, merges and cleans 
DHS and NHIS data to produce the datasets (one line per child) used in analysis.

\subsection{The DHS}
The DHS are a set of nationally representative surveys which have been 
administered in low- and middle-income countries between 1985 and the present. 
Women aged between 15--49 in surveyed households respond to an in-depth series 
of questions reporting their full fertility history (listing all surviving and 
non-surviving children), their actual and desired contraceptive use and number 
of births, education level, marital status, plus the measurement of a number of 
health endowments such as height and body mass index. For all other members 
living in the household, a shorter series of responses are recorded, including 
the individual's educational attainment.

This results in two distinct sets of data to be merged. One database contains 
one line for each birth reported by every 15--49 year-old woman surveyed with a 
limited number of child-level covariates such as the child's date of birth, type 
of birth (single or multiple), and the child's survival status. The other 
database contains one line for each member currently living in the survey
household. This database includes each member's educational status. We merge
these two databases where all children who live in the same household as their 
mother merge without loss.  We are thus able to generate data for the educational 
attainment of each of a woman's children currently residing in the household as 
well as their mother's health and educational status. This database is selected in 
two ways: firstly it only contains children who have survived up until the survey 
date, and secondly it only contains children who have remained living in the same 
household as their mother. We drop from our sample children aged 18 and over, due
to concerns that these will \emph{not} be representative of the general 
population.

We pool all publicly available DHS data resulting in microdata on 3,297,318 
children ever-born to women who responded fully to any DHS survey. A full list of 
the DHS countries and years of surveys which make up this sample is provided in
the online appendix (table \ref{TWINtab:countries}).  Of the 3,297,318 offspring 
reported in survey data, 2,033,510 remain living in the same household as their 
mother.  The majority of these 2,033,510 children are aged 18 and under (92.96\%) 
and hence make up our principal estimation sample (in future we will refer to this 
as the `household sample'). The remaining 1,263,808 offspring were not recorded as 
living in the same household as their mother.  Of these children not in the 
household, and hence for whom education is not recorded, the majority (53.9\%) 
were aged over 18 or had died prior to the date of survey.\footnote{Children aged 
under 18 who are alive but not living in the same household as their mother are 
statistically quite different to those children who do remain in the household. 
In our data sample, they are on average 2.7 years older, born to less educated 
and younger mothers, and are slightly more likely to be males.}

\subsection{The NHIS}
The National Health Interview Survey (NHIS) is a yearly survey, conducted from 
1957 and ongoing as at 2015, with participants drawn from each of the 50 US 
States as well as the District of Columbia each year.\footnote{The NHIS has a
survey design to oversample Hispanic and African American people.  We use NHIS-%
specific probability weights in all analyses.}  We pool all survey data from 
2004 until 2014, resulting in data on 119,111 mothers and 227,213 children. We 
focus on this period given that prior to 2004, changes in a number of key 
variables make it difficult to compare between years, and post-1996 the survey 
was considerably revised.

Each set of surveys is collected at the level of the household.  For our 
analysis we use all households which consist of a biological mother and her 
children, whether or not any father is present.  For all children who remain in
the household, the survey records total fertility.  We infer twin status by
assuming that all children who share a birth month, birth year and biological
mother must be twins.  For each child and mother, we have a number of measures
of usage of health care along with a self-reported measure for health status, 
whether or not the mother smokes, and the level of completed education (at the 
time of the survey) of mothers and children.  Once again, we subset to children
aged below 18, and for education measures, children who are aged above 6 years
old, and hence who are able to be enrolled in school.  Descriptive statistics of
this and DHS data are provided in section \ref{TWINsscn:descriptives}.

\subsection{Regressions of Twinning on Maternal Health}
In auxiliary regressions examining the characterstics of mothers and the
relationship these characteristics and twin births and miscarriage, we consult
a large number of other datasets.  These are the following:
\begin{itemize}
\item United States National Vital Statistics Birth Data
\item United States National Vital Statistics Fetal Death Data
\item Spanish Vital Statistics (INE)
\item The Swedish Medical Birth Register
\item Longitudinal Early Life Survey, Chile (ELPI)
\item The Avon Longitudinal Study of Parents and Children (ALSPAC)
\end{itemize}

In the case of the first 4 datasets (administrative records of births and/or
fetal deaths), we use all recorded instances for mothers aged 18-49, focusing 
on twins as our outcome variable of interest.  The only exception is the United
States Vital Statistics data, in which case we observe Artifical Reproductive
Technology (ART) use, and remove the 1.6\% of ART users from the estimation sample. 
Depending upon the data source, we use all available measures of pre-determined 
maternal health stocks or family socioeconomic indicators.  The ELPI survey from 
Chile focuses on child early life, and records mother's behaviours before, during 
and after pregnancy, along with child birth outcomes.  We use all children from the 
first wave of this survey to run the twin regression included in table 
\ref{tab:twinsMain}. Finally, for the ALSPAC survey follows mothers and their
children who were born in the early 1990's in the county of Avon, UK.  We use all
mothers from the principal wave of enrolled children.

\section{Resampling and Simulation Based Estimation of $\gamma$}
\label{ATWINscn:gammaSim}
\subsection{Bootstrap Confidence Intervals}
The methodology to estimate $\gamma$ in equations (\ref{TWINeqn:realgamma}) and
(\ref{TWINeqn:realgammaN}) is described in section \ref{TWINscn:gamma} of the
paper.  In the case of \citeauthor{Conleyetal2012}'s UCI approach, this estimate
is then sufficient to produce bounds on $\beta_1$, assuming that:
$\gamma\in[0,2\hat\gamma]$. We scale $\hat\gamma$ by the factor of 2 in order for
this value to fall precisely in the middle of the range. \citet{Conleyetal2012}
provide a similar example to calculate the returns to education using the UCI
approach.  In the case of the more precise LTZ approach (our preferred method)
the logic is similar, however now we must form a prior over the entire
distribution of $\gamma$.  Calculating the variance of $\gamma$ is not as
straightforward as using the variance-covariance matrix corresponding to each of
the estimates $\hat\phi^t$ and $\hat\phi^q$.  In this case however we can use
bootstrapping to calculate $J$ replications of $\hat\phi^t\times\hat\phi^q$, and
from these estimates construct an estimated distribution of $\hat\gamma$, which
allows us to determine our prior for the distribution of $\gamma$.  From this
empirical distribution, we observe the estimated mean and standard deviation, and
finally test whether the distribution is normal using a Shapiro Wilk test for
normality. We also use Kolmogorov-Smirnov tests for equality of distributions to
test whether the distribution is more likely to be log normal, uniform, and a
number of other known analytical distributions. In order to do this, we first
estimate the empirical distribution as described previously.  We then observe the
mean $\hat\mu$ and the standard deviation $\hat\sigma$, and run a one-sample test
to determine whether the observed empirical distribution is is significantly
different to each analytical distribution $\mathcal{N}(\hat\mu,\hat\sigma^2)$,
$U(\hat\mu,\hat\sigma^2)$ or $\ln\mathcal{N}(\hat\mu,\hat\sigma^2)$.

\subsection{Simulation-Based Estimation for non-Normal Distributions}
Estimates of the full distribution of $\gamma$ are presented in Figures
\ref{TWINfig:gammaBootsN} and \ref{TWINfig:gammaBootsL}.  These are
the estimated $\hat\gamma_j$ from $j \in \{1,\ldots,100\}$ bootstrap
replications for $\gamma$ in Nigeria and the United States.  In all cases,
when the underlying empirical distribution is tested for equality against
the overlaid analytical distribution (uniform, normal, log normal, $\chi^2$),
the normal distribution provides the best fit of the analytical with the
empirical distribution.\footnote{In the US, We cannot reject that $\gamma$
  is normal with a p-value of 0.xxxx.  In this case, although we can't reject
  that $\gamma$ is log normal, the p-value is much lower, at 0.xxxx.  Similar
  values for Nigeria suggest xxxx.
}

However, the underlying distribution appears to not be perfectly normal,
and it appears doubtful that this would be the case in distribution.
Fortunately, \citet{Conleyetal2012} describe a simulation-based estimation
method to calculate $\gamma$ in the case of a non-normal distribution for
$\gamma$.  We have followed this methodology using the empirical distribution
calculated bootstrapping for $\gamma$.  This code has been publicly released
as \texttt{plausexog} for Stata \citep{Clarke2014}.  The simulation-based
estimation procedure is described fully in \citet{Conleyetal2012} p.\ 265
as a five step algorithm.  The procedure consists of taking repeated draws
from the variance-covariance matrix estimated using IV with the plausibly
exogenous instrument, and in each case adding to it a draw from the
distribution of $\gamma$, scaled by a quantity which depends
on the strength of the instrument. \citeauthor{Conleyetal2012} refer to the
underlying distibution of $\gamma$ as $F$, and the scale parameter as $A$,
where $A=(X^\prime Z(Z^\prime Z)^{-1}Z^\prime X)^{-1}(X^\prime Z)$.
These repeated draws then lead to a large number of estimates for $\beta$,
the parameter of interest, and a 95\% confidence interval is taken by
forming $[\hat\beta-c_{1-\alpha/2},\hat\beta+c_{\alpha/2}]$, where $c$ are
percentiles of the distribution of simulated estimates.

Thus, as well as estimating the LTZ case where we assume that $\gamma$
is distributed $\sim \mathcal{N}(\mu_{\hat\gamma},\sigma^2_{\hat\gamma})$,
we can estimate a version fully utilizing the bootstrapped distribution
of $\hat\gamma$ described in the previous sub-section.  In this case, we
use as $F$, the distribution of $\gamma$, the empirically estimated
distribution of $\gamma$.  The simulation based algorithm then consists
of taking $b\in 1,\ldots,B$ draws from the empirically estimated $F$, as
well as $B$ draws from the variance-covariance matrix, and defining the
95\% confidence interval based on the 2.5 and 97.5\% quintiles of the
resulting simulated values for $\beta$.


%\end{doublespace}

\newpage
\bibliography{./BiBBase1}
\end{document}



  \begin{figure}[htpb!]
    \begin{center}
      \caption{Temperature and Good Season (40-45 ``Teachers'' vs ``Non-Teachers'')}
      \label{bqFig:coldTeach4045}
      \begin{subfigure}{.5\textwidth}
        \centering
        \includegraphics[scale=0.55]{./../results/hispall/ipums/graphs/StateTemp_4045Teacher_cold_weight.eps}
        \caption{``Teachers''}
        \label{fig:Educ3}
      \end{subfigure}%
      \begin{subfigure}{.5\textwidth}
        \centering
        \includegraphics[scale=0.55]{./../results/hispall/ipums/graphs/StateTemp_4045NonTeacher_cold_weight.eps}
        \caption{``Non-Teachers''}
        \label{fig:NonEduc3}
      \end{subfigure}
    \end{center}
    \floatfoot{\textsc{Notes to figure}: State averages of good season are plotted against the
      coldest average monthly temperature in the state. Panel A includes all workers who are in
      ``Education, Training and Library Occupations'', while Panel B includes all other workers. }
  \end{figure}

  \begin{figure}[htpb!]
    \begin{center}
      \centering
      \caption{ART Conceptions by Month}
      \includegraphics[scale=0.74]{./../results/hisp/graphs/proportionMonthART.eps}
      \label{fig:ARTMonth}
    \end{center}
    \floatfoot{\textsc{Notes to figure \ref{fig:ARTMonth}}: Proportion of ART births
      are calculated using data from 2009-2013 for our main sample.  The proportion
      is calculated as: (ART conceptions)/(Non-ART Conceptions + ART Conceptions).}
  \end{figure}

