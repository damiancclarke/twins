The use of twin births to address the problem of endogeneity in the Q-Q model seems to have been initially proposed by \citet{RosenzweigWolpin1980}.  
They derive the theoretical requirements to estimate the size of the trade-off when the shadow price of child quality depends on the number 
of children and vice versa.  By relying on the assumption that multiple births are an exogenous shock to family size (once accounting for 
the total number of a mother's pregnancies) they estimate the effect of a twin birth upon the educational attainment of children in the twins' 
family.  This and alternate empirical specifications discussed in this section are decribed in table \ref{tab:litrev} in appendix \ref{scn:litrev}.  

Subsequent papers employing a twin-birth methodology have proposed a number of strategies which enable them to obtain consistent estimates of 
the Q-Q trade-off while relaxing Rosenzweig and Wolpin's exogeneity assumption.  \citet{Blacketal2005} extend the controls to account for the 
fact that the probability of multiple birth increases with maternal age as described by \citet{Jacobsenetal1999} and others.  They include a 
set of parental age and education controls, however note that they are unable to reject the hypothesis that parental education has no effect 
on the probability of multiple birth.  Likewise, \citet{Caceres2006} includes controls for mother's age, race, and education, suggesting that 
the use of these and pre-1980 US Census data should be sufficient to approximate conditional exogeneity\footnote{The use of pre-1980 Census 
data seems important as this predates the widespread introduction of fertility drugs.  The use of fertility drugs is associated with higher a 
probability of multiple births, and resulting concerns that the orthogonality assumption will be violated if users of fertility treatment are 
non-random.}.  Finally, \citet{Angristetal2010} recognise that twin birth varies with maternal age at birth and race, including twinning as 
one of three instruments. 

Angrist et al.\ join recent work from \citet{RosenzweigZhang2009} in questioning the validity of twin instrumentation in another sense.  These 
authors suggest that the error term in the Q-Q equation is unlikely to be orthogonal to the instrument given that twinning imposes predictable 
and unobserved family responses in investment decisions.  Particularly, these studies question the effect that close birth spacing and an endowment 
effect responding to the lower health at birth of twins compared to single births\footnote{Using data from the United States, \citet{Almondetal2005} 
document that twins have substantially lower birth weight, lower APGAR scores, higher use of assisted ventilation at birth and lower gestion period 
than singletons.} has on investments in pre-twin siblings.  Despite this critique, Rosenzweig and Zhang suggest that it is still possible to compute 
an upper and a lower bound of the Q-Q trade-off.

More recently, instrumentation using twin births has been applied to estimate a Q-Q model in the developing world. \cite{SouzaPonczek2012}, 
\cite{FitzsimonsMalde2010}, \citet{Sanhueza2009} and \citet{Lietal2008} have applied a similar methodology to that of Angrist et al., 
examining twin births in Brazil, Mexico, Chile and China respectively.  These studies find mixed results depending upon the country under 
examination and once again, while considering the invalidity of the twin exclusion restriction in the Q-Q model in terms of maternal education, 
do not examine this in terms of non-random twin births due to maternal health. 
