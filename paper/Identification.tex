\documentclass{article}[11pt]

\usepackage{amsmath}


\newcommand{\mathsc}[1]{{\small\textsc{#1}}}

\setlength\parindent{0.25in}
\setlength\parskip{0.25in}

\title{Draft Notes: Twin Identification}
\date{}

\begin{document}
\maketitle

Consider a case where child educational outcomes in family $j$, $\mathsc{Y}_{ij}$, 
are determined by family size $\mathsc{Q}_{ij}$, and a vector of child- and 
family-specific covariates $\textbf{X}_{i}$ and $\textbf{X}_{ij}$.  We define 
$\mathsc{Y}_{ij}(q,z)$ as the potential outcome for child $i$, where we follow 
the potential outcomes notation of Angrist and Imbens (1995).  Here a child's 
`potential outcome' depends upon their treatment status, $\mathsc{Q}_{ij}=q$, 
and instrument value $\mathsc{Z}_{ij}=z$. We are thus interested in estimating:
\begin{equation}
\mathsc{Y}_{ij}(\mathsc{Q}_{ij}+1,\mathsc{Z}_{ij})-\mathsc{Y}_{ij}(\mathsc{Q}_{ij},\mathsc{Z}_{ij}), 
\end{equation}
the causal effect of an additional sibling on child outcomes.  For simplicity
we will assume that $\mathsc{Y}_{ij}(\mathsc{Q}_{ij}+1)-\mathsc{Y}_{ij}(\mathsc{Q}_{ij})
=\rho$ for all $i$ and $j$,\footnote{Referred to as a linear constant-effects model
by Angrist et al.\ (2010).  \texttt{We should come back to this.}} however this will 
later be relaxed.

If twinning were truly randomly assigned,
2SLS estimates of $\hat{\rho}$ driven by the twin instrument would provide consistent
estimates of the effect of an additional birth on child outcomes.  Here we could 
consider a Wald estimate of $\rho$:
\begin{equation}
 \frac{E[\mathsc{Y}_{ij}|\mathsc{Z}_{ij}=1]-E[\mathsc{Y}_{ij}|\mathsc{Z}_{ij}=0]}{E[\mathsc{Q}_{ij}|\mathsc{Z}_{ij}=1]-E[\mathsc{Q}_{ij}|\mathsc{Z}_{ij}=0]}=E[Y_{1ij}-Y_{0ij}|\mathsc{Q}_{1ij}>\mathsc{Q}_{0ij}].
\end{equation}
This is the well known Local Average Treatment Effect (LATE) of the effect of an
additional birth on
compliers.  As Angrist et al.\ (2010) suggest, if we accept the strong assumption
that twinning is as good as randomly assigned, causal effects for the population
with twin births are the same as for the population without twin births, and this
parameter is then the average effect for the entire population.\footnote{We do 
\emph{not} question their assumption that all families who receive twin births are 
compliers, however do question the more important assumption of random assignation 
of twin births.}

However, as we discuss, twins are not randomly assigned, but rather depend upon
maternal and family characteristics such as maternal age, health, and parity:
\begin{equation}
\label{eqn:twinZ}
 Z_{ij} = \left\{
  \begin{array}{l l}
    1 & \quad \text{if} \quad Z_{ij}^*=\textbf{X}_{ij}^\prime \beta + \varepsilon_{ij} > 0 \\
    0 & \quad \text{if} \quad Z_{ij}^*=\textbf{X}_{ij}^\prime \beta + \varepsilon_{ij} \leq 0
  \end{array} \right.
\end{equation}
Typically, prior literature assumes that even though twins are not entirely an 
exogenous shock, they are ``as good as randomly assigned'' (Angrist et al.\ 2010, 
p.\ 788).  This is tantamount to assuming that all relevant twin predictor variables 
$\textbf{X}_{ij}$ are observed and can be included in a 2SLS specification.  This
then results in an estimable specification:
\begin{equation}
\label{eqn:qqz}
y_{ij}=\textbf{X}_{ij}^\prime \beta+\textbf{X}_{i}^\prime \mu + \rho Q_{ij} + \nu_{ij}
\end{equation}
which is consistent under 2SLS due to the inclusion of the vector $\textbf{X}_{ij}$ 
in the first and second stages.

Our results suggest that even an `as good as randomly assigned' assumption may
be questionable.  We are concerned that unobserved maternal characteristics driving
twinning in (\ref{eqn:twinZ}) are also relevant (unobserved) predictors of child 
outcomes in (\ref{eqn:qqz}).  In this case $E[\nu_{ij}\varepsilon_{ij}]\neq 0$, a 
failure of the conditions for instrumental validity.





\end{document}
