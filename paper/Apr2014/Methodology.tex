Typically, empirical analyses of the quality-quantity trade-off focus on 
producing consistent estimates of $\beta_1$ in the following specification:
\begin{equation}
\label{TWINeqn:secondstage}
educ_{ij}=\beta_0+\beta_1 fert_{j} + \bm{X}\bm{\beta}_2+u_{ij}.
\end{equation}
Here, quality is proxied by the educational attainment of child $i$ in family 
$j$, ($educ$) and fertility ($fert$) is measured as the total births in a child's
family.  A vector of family and child controls is included, denoted $\bm{X}$.  As
has been extensively discussed in prior literature, estimation of $\beta_1$ using
OLS and cross-sectional data will result in biased coefficients given that child 
quality and quantity are jointly determined \citep{BeckerLewis1973,BeckerTomes1976}, 
and given that unobservable parental behaviours and attributes influence both 
fertility decisions, and investments in children's education \citep{Qian2009}.

