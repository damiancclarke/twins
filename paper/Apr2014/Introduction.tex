%********************************************************************************
Since the pioneering work of \citet{RosenzweigWolpin1980}, economists have 
attempted to leverage the occurrence of twin births to estimate the effect of 
family size on child outcomes. If twin births occur at random, the occurrence of 
twin births constitutes a fertility shock that is uncorrelated with family 
characteristics including parental preferences, and other unobservables which may 
be related to child quality. This provides the exogenous variation (in quantity) 
required to estimate the quantity-quality model of \citet{Becker1960,
BeckerLewis1973,BeckerTomes1976}.  The essential idea is that the shadow price of 
child quality is increasing in child quality and \emph{vice versa} i.e.\ the 
shadow price of quantity is increasing in quality.  By comparing those families who 
unexpectedly produced additional children to those who always produced one child
per birth, we can isolate the effect of an individual's sibling size on her human 
capital attainment.

However, the consistent estimation of this effct is based upon the untestable 
assumption that twinning is exogenous. This requires not only that twin conceptions 
are randomly assigned to families, but also that taking a twin conception to term 
does not depend upon a woman's behaviours during pregnancy or on her endowments
prior to pregnancy. This is at odds with the evidence.

We show that endowments and behaviours affect the chances of twins being born (see 
section \ref{TWINscn:results} below and \citet{BhalotraClarke2014}). In data from 
a large sample of developing countries, we show that taller women and women with a 
higher body-mass index are significantly more likely to give birth to twins.\footnote{
Height is an index of the stock of health of a woman and is a function of investments 
over her growth period and, especially, the early years of her life (see references 
in \citet{BhalotraRawlings2013})} Maternal health is likely to be an especially significant 
determinant of birth outcomes in poorer countries where many women are chronically 
under-nourished (an indicator of which is their final stature), exhibit anemia and 
low-BMI and are prone to infections. In these conditions only relatively healthy 
women will have the resources to support a successful twin pregnancy. However, our 
critique applies to richer countries too. Using data from the UK, Scotland, the USA 
and Chile, we find that women who are taller and less likely to engage in risky
behaviours such as smoking, drug taking or alchol consumption during pregnancy are 
significantly more likely to have a (live) twin birth.\footnote{Simiarly, it is 
well known that fertility treatments increase the likelihood of twinning. While we 
do not discuss this extensively here, we note that this is likely to be less of an 
identification challenge given that IVF treatment is an entirely observable 
behaviour.}

Overall, our argument is that women who give birth to twins are positively selected and 
that the common tendency to ignore this will result in under-estimation of the QQ 
trade-off. We focus upon maternal health because no previous work has highlighted it 
as a determinant of twinning and because it is inherently impossible to fully control 
for. Even if we had data that included all of the indicators of health we mention 
above, we may not observe whether women skip breakfast \citep{MazumderSeeskin2014}, 
whether they are stressed (\citet{Blacketal2014} and references therein), whether they 
seek and adhere to antenatal care and so on.  Our contention adds a novel twist to a 
recent literature which suggests that mothers' health and fetal environment matter and 
may alter the birth weight of children, the sex ratio at birth and a range of future 
human capital outcomes \citep{Almondetal2011,BhalotraRawlings2013,Barker1995}. Like 
birth weight and the probability of a boy relative to a girl birth, the probability 
of a twin birth is increasing in the health of the mother/ the fetal environment.  

The emerging consensus in the literature using twin births to test the Q-Q model is 
that there is in fact no significant or substantial trade-off between fertility and 
investments in children. In this paper we suggest that this result may, in principle, 
follow from the bias created by twin-mothers being positively selected. We re-examine 
the validity of these results, accounting for the innovation discussed previously.
If twin births are not truly exogenous, but instead depend upon maternal health stocks 
and behaviours, there is an identification challenge.  Specifically, if healthier 
mothers are both more likely to give birth to twins, and more likely to invest in child 
human capital in later life, existing estimates may significantly underestimate the
size of the Q-Q tradeoff.  When ignoring these considerations we find that the effect 
of an additional birth on human capital attainment (education) is minor, or even 
weakly positive. However, when taking into account this innovation we find that a 
significant trade-off does exist, and that an additional birth reduces standardised 
schooling behaviour by at least 4\% of a standard deviation.

the estimated effect of $\sim 4\%$ of an s.d.\ is at most the lower bound for the true 
size of the Q-Q tradeoff. Given that we suggest that maternal health predicts twinning,
and given that maternal health is multidimensional in nature and difficult to observe 
fully, we will only ever be able to include a partial set of controls to account for 
the inconsistency in IV estimates. As such, we examine a number of methods to estimate 
plausible bounds on the Q-Q tradeoff. First, we argue that the true estimate is 
bounded by the OLS and IV estimtes. We also follow \citet{Conleyetal2012} in
conceptualizing twins as a plausibly exogenous event, and derive estimates by assuming 
that the traditional exclusion restriction is `close to' holding.

This work which further probes the Q-Q tradeoff is important, especially in the 
developing country setting on which we focus. In order to conduct empirical tests, 
we construct a large microdata set from 68 developing countries with observations on 
more than 2.5 million children and nearly 1 million mothers. The macro level trends in 
this data suggest that educational attainment has risen considerably while completed 
and desired fertility has fallen sharply over the past 50 years (see figures 
\ref{TWINfig:fertrend} and \ref{TWINfig:eductrend}). Similar effects have been described 
extensively in the empirical literature (ie \citet{Hanushek1992}). It is of considerable 
relevance to researchers and to policy makers to determine whether such a trend is (at 
least partially) causal. A significant number of countries have engaged in aggresive 
fertility control policies over the previous century, with family planning still 
considered a concern according to policy makers in the developing world.\footnote{A 
recent survey of national governments suggests that fertility was perceived as too 
high in 50\% of developing countries, with this figure rising to 86\% among the least 
developed countries \citet{UN2010}.} As discussed, our evidence suggests that the same 
identification problem arises with using twins as an instrument for fertility in 
analysis of data from richer countries. Future work is merited that assesses the size 
of the bias.\footnote{The data for Israel and China analysed in the twin-IV studies of 
\citet{Angristetal2010} and \citet{RosenzweigZhang2009} respectively do not contain 
information on maternal health. The data for Norway analysed in the twin-IV study of 
\citet{Blacketal2005} are not accessible outside Norway.}

This paper unfolds as follows.  In the next section we discuss the existing 
literature which estimates the Q-Q model using twins.  We then discuss the large
dataset that we will use to examine twin exogeneity and to bound the Q-Q tradeoff.
Section \ref{TWINscn:method} discusses our methodology, while section 
\ref{TWINscn:results} presents results.  We briefly conclude in the final section.
