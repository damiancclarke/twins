The data used here come from the Demographic and Health Surveys (DHS).  The DHS 
are a set of nationally representative surveys which have been administered in 
low- and middle-income countries between 1985 and the present.  Women aged 
between 15--49 in surveyed households respond to an in-depth series of questions 
reporting their full fertility history (listing all surviving and non-surviving 
children), their actual and desired contraceptive use and number of births, 
education level, marital status, plus the measurement of a number of health 
endowments such as height and body mass index.  For all other members living in 
the household a shorter series of responses are recorded, including the 
individual's educational attainment.

This results in two distinct sets of data to be merged.  One database contains
one line for each birth reported by every 15--49 year-old woman surveyed with
a limited number of child-level covariates such as the child's date of birth,
type of birth (single or multiple), and the child's survival status.  The other
database contains one line for each member currently living in the survey 
household.  This database includes each member's educational status.  By 
crossing these two databases we are thus able to generate data for the
educational attainment of each of a woman's children currently residing in the 
household as well as their mother's health and educational status.  This 
database is selected in two ways: firstly it only contains children who have 
survived up until the survey date, and secondly it only contains children who 
have remained living in the same household as their mother.  In order to ensure 
that this sample is as representative as possible, we restrict our analysis to 
those children aged 18 and under.

We pool all publicly available DHS data resulting in microdata on 3,297,318
children ever-born to women who responded fully to any DHS survey.  A full list 
of the DHS countries and years of surveys which make up this sample is provided 
as appendix table \ref{TWINtab:countries}.  Of the 3,297,318 offspring reported in 
survey data, 2,033,510 remain living in the same household as their mother.  The 
majority of these 2,033,510 children are aged 18 and under (92.96\%) and hence make 
up our principal estimation sample (in future we will refer to this as the 
`household sample').  The remaining 1,263,808 offspring were not recorded as 
living in the same household as their mother.  Of these children \emph{not} in the 
household, and hence for whom education is not recorded, the majority (53.9\%) were 
aged over 18 or had died prior to the date of survey.\footnote{Children aged under 
18 who are alive but not living in the same household as their mother are 
statistically quite different to those children who do remain in the household.  
In our data sample, they are on average 2.7 years older, born to less educated 
and younger mothers, and are slightly more likely to be males or twins.}
%********************************************************************************

