%********************************************************************************
The DHS are a nationally representative set of surveys which have been 
administered in low- and middle-income countries between 1985 and the present.
Women aged between 15--49 in surveyed households respond to an in-depth series 
of questions reporting their full fertility history (both surviving and 
non-surviving children), their actual and desired contraceptive use and number
of births, education level, marital status, plus the measurement of a 
number of health endowments such as height and body mass index.  For all other 
members living in the household a shorter series of responses are recorded, 
including the individual's educational attainment.

This results in two distinct sets of data to be merged.  One database contains
one line for each birth reported by every 15--49 year-old woman surveyed with
a limited number of child-level covariates such as the child's date of birth,
type of birth (single or multiple), and the child's survival status.  The other
database contains one line for each member currently living in the survey 
household.  This database includes each member's educational status.  By 
crossing these two databases we are thus able to generate data for the
educational attainment of each of a woman's children currently residing in the 
household as well as their mother's health and educational status.  This 
database is thus selected in two ways: firstly it only contains children who
have survived up until the survey date, and secondly it only contains children
who have remained living in the same household as their mother.  In order to
ensure that this sample is as representative as possible, we restrict our
analysis to those children aged 18 and under.

We pool all publicly available DHS data resulting in microdata on 6,697,397 
children ever born.  Of these offspring, 3,803,796 remain living in the same
household as their mother.  The majority of these 3,803,796 children are aged
18 and under (91.45\%) and hance make up our principal estimation sample.  The
remaining 2,893,601 offspring were not recorded as living in the same household
as their mother.  Of these children \emph{not} in the household, and hence for 
whom education is not recorded, the majority (53.4\%) were aged over 18 or had 
died prior to the date of survey.  The full sample of children aged 18 and 
under (with and without education) are used in tests which involve infant or
child mortality as outcome variables.

Table \ref{TWINtab:sumstats} provides summary statistics of this data.  
Fertility and maternal characteristics are described at the level of the 
mother, while child education and survival are described at the level of the
child (the education and full samples respectively).  The number of 
observations at each level is provided at the bottom of the table.

%measures of child outcomes and mother 
