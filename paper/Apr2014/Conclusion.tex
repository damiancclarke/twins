%********************************************************************************
Estimation of the Q-Q model using twin births to instrument fertility relies on 
the fact that twin births are exogenous. This implies that both conceiving twins 
\emph{and} taking them to term cannot depend on maternal behaviours and health 
stocks. In this paper we suggest that such an assumption may be too strong to 
hold in practice. We show that healthier mothers are more likely to give birth to 
twins even if twin conception is random, and that the size of this effect is 
considerable.

We re-estimate the Q-Q tradeoff accounting for such an effect.  We thus show
that the existing wisdom (which suggests that the effect of additional births 
on human capital might be minor), likely underestimates the true magnitude of
the Q-Q tradeoff.  Our IV estimates move point estimates from approximately 
0\% of a standard deviation (using the old methodology which does not account
for maternal health) to a significant $-4\%$ of a standard deviation of 
educational attainment.  Further, we suggest that this is likely
a lower bound of the true effect.  Depending upon the degree to which maternal
characteristics predict twin births, the true trade-off may be considerably
larger, perhaps even doubling these IV estimates.  The availability of richer 
maternal health and behaviour data in surveys would allow for us to restrict
the size of these bounds and produce more precise estimates.

