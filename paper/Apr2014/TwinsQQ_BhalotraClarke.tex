\documentclass{article}[11pt,subeqn]

\title{The Twin Instrument\footnote{We are grateful to Paul Devereux, James Fenske, 
Atheen Venkataramani and Cheti Nicoletti,  [WHO ELSE TO ADD?] along with seminar 
audiences and discussants at CMPO Bristol, 
ESPE, NEUDC, CSAE, and The University of Essex for helpful comments.  We make full data and
source code available for examination and replication.  This is available at
\texttt{https://bitbucket.org/damiancclarke/twins}.}}
\author{Sonia Bhalotra\thanks{The University of Essex.  Contact: srbhal@essex.ac.uk} 
\and Damian Clarke\thanks{The University of Oxford.  Contact: damian.clarke@economics.ox.ac.uk}}
\date{\today}

\usepackage[capposition=top]{floatrow}
\usepackage{url}
\usepackage{longtable}
\usepackage{booktabs}
\usepackage{rotating}
\usepackage{dcolumn}
\usepackage{color}
\pagecolor{white}
\usepackage{pdfpages}
\usepackage{lastpage}
\usepackage{lscape}
\usepackage{setspace}
\usepackage{amsmath}
\usepackage{amssymb}
\usepackage{breqn}
\usepackage{appendix}
\usepackage{natbib}
\bibliographystyle{abbrvnat}
\bibpunct{(}{)}{;}{a}{,}{,}

\usepackage{epsfig}
\usepackage{epstopdf}
\usepackage{multirow}

\usepackage{wrapfig}
\usepackage{blindtext}

\setlength\topmargin{-0.375in}
\setlength\textheight{8.8in}
\setlength\textwidth{5.8in}
\setlength\oddsidemargin{0.4in}
\setlength\evensidemargin{-0.5in}
\setlength\parindent{0.25in}
\setlength\parskip{0.25in}

\newcommand{\person}{we\ }
\newcommand{\Person}{We\ }

\bibliographystyle{abbrvnat}
\bibpunct{(}{)}{;}{a}{,}{,}

\newcommand\independent{\protect\mathpalette{\protect\independenT}{\perp}}
\def\independenT#1#2{\mathrel{\rlap{$#1#2$}\mkern2mu{#1#2}}}

\usepackage{changepage}% http://ctan.org/pkg/changepage
\makeatletter
\newenvironment{chapabstract}{%
    \begin{center}%
      \bfseries Abstract
    \end{center}}%
   {\par}
\makeatother


\newcommand{\twinfolder}{./../../../Twins}
\newcommand{\biblioinc}{\newpage \bibliography{./../../Document/ThesisRefs}}

\begin{document}
\begin{spacing}{1.4}

\maketitle
\begin{abstract}
 The incidence of twins has been used to identify the impact of changes in fertility 
 on measures of investment in children born prior to the twins, and the emerging 
 consensus in this literature is that there is no evidence of a quantity-quality 
 trade-off. We argue that the standard approach is flawed for two reasons. First, 
 even if twin conception is random, bringing twins to term is a function of maternal 
 health which is difficult to fully observe and which tends to be correlated with 
 child quality, rendering the instrument invalid. Second, twins will only constitute 
 a shock to family size if their occurrence takes family size across the desired 
 level. The neglect of both of these considerations in the literature will tend to 
 lead to under-estimation of the quality-quantity (Q-Q) trade-off and so could 
 contribute to explaining the negative results in the literature. Using a large sample
 of microdata from developing countries, we show that a significant trade-off emerges
 upon correcting for these biases.  We show that this result is likely to be only a 
 \emph{lower} bound of the true Q-Q trade-off and discuss how to estimate the size of
 these bounds. \\
\end{abstract}

\hspace{4mm}\textbf{\small JEL codes}: J13,J16,J18,I15,O15. \\
\newpage
\section{Introduction}
%********************************************************************************
Since the pioneering work of \citet{RosenzweigWolpin1980}, economists have 
attempted to leverage the occurrence of twin births to estimate the effect of 
family size on child outcomes. If twin births occur at random, the occurrence of 
twin births constitutes a fertility shock that is uncorrelated with family 
characteristics including parental preferences, and other unobservables which may 
be related to child quality. This provides the exogenous variation (in quantity) 
required to estimate the quantity-quality model of \citet{Becker1960,
BeckerLewis1973,BeckerTomes1976}.  The essential idea is that the shadow price of 
child quality is increasing in child quality and \emph{vice versa} i.e.\ the 
shadow price of quantity is increasing in quality.  By comparing those families who 
unexpectedly produced additional children to those who always produced one child
per birth, we can isolate the effect of an individual's sibling size on her human 
capital attainment.

However, the consistent estimation of this effct is based upon the untestable 
assumption that twinning is exogenous. This requires not only that twin conceptions 
are randomly assigned to families, but also that taking a twin conception to term 
does not depend upon a woman's behaviours during pregnancy or on her endowments
prior to pregnancy. This is at odds with the evidence.

We show that endowments and behaviours affect the chances of twins being born (see 
section \ref{TWINscn:results} below and \citet{BhalotraClarke2014}). In data from 
a large sample of developing countries, we show that taller women and women with a 
higher body-mass index are significantly more likely to give birth to twins.\footnote{
Height is an index of the stock of health of a woman and is a function of investments 
over her growth period and, especially, the early years of her life (see references 
in \citet{BhalotraRawlings2013})} Maternal health is likely to be an especially significant 
determinant of birth outcomes in poorer countries where many women are chronically 
under-nourished (an indicator of which is their final stature), exhibit anemia and 
low-BMI and are prone to infections. In these conditions only relatively healthy 
women will have the resources to support a successful twin pregnancy. However, our 
critique applies to richer countries too. Using data from the UK, Scotland, the USA 
and Chile, we find that women who are taller and less likely to engage in risky
behaviours such as smoking, drug taking or alchol consumption during pregnancy are 
significantly more likely to have a (live) twin birth.\footnote{Simiarly, it is 
well known that fertility treatments increase the likelihood of twinning. While we 
do not discuss this extensively here, we note that this is likely to be less of an 
identification challenge given that IVF treatment is an entirely observable 
behaviour.}

Overall, our argument is that women who give birth to twins are positively selected and 
that the common tendency to ignore this will result in under-estimation of the QQ 
trade-off. We focus upon maternal health because no previous work has highlighted it 
as a determinant of twinning and because it is inherently impossible to fully control 
for. Even if we had data that included all of the indicators of health we mention 
above, we may not observe whether women skip breakfast \citep{MazumderSeeskin2014}, 
whether they are stressed (\citet{Blacketal2014} and references therein), whether they 
seek and adhere to antenatal care and so on.  Our contention adds a novel twist to a 
recent literature which suggests that mothers' health and fetal environment matter and 
may alter the birth weight of children, the sex ratio at birth and a range of future 
human capital outcomes \citep{Almondetal2011,BhalotraRawlings2013,Barker1995}. Like 
birth weight and the probability of a boy relative to a girl birth, the probability 
of a twin birth is increasing in the health of the mother/ the fetal environment.  

The emerging consensus in the literature using twin births to test the Q-Q model is 
that there is in fact no significant or substantial trade-off between fertility and 
investments in children. In this paper we suggest that this result may, in principle, 
follow from the bias created by twin-mothers being positively selected. We re-examine 
the validity of these results, accounting for the innovation discussed previously.
If twin births are not truly exogenous, but instead depend upon maternal health stocks 
and behaviours, there is an identification challenge.  Specifically, if healthier 
mothers are both more likely to give birth to twins, and more likely to invest in child 
human capital in later life, existing estimates may significantly underestimate the
size of the Q-Q tradeoff.  When ignoring these considerations we find that the effect 
of an additional birth on human capital attainment (education) is minor, or even 
weakly positive. However, when taking into account this innovation we find that a 
significant trade-off does exist, and that an additional birth reduces standardised 
schooling behaviour by at least 4\% of a standard deviation.

the estimated effect of $\sim 4\%$ of an s.d.\ is at most the lower bound for the true 
size of the Q-Q tradeoff. Given that we suggest that maternal health predicts twinning,
and given that maternal health is multidimensional in nature and difficult to observe 
fully, we will only ever be able to include a partial set of controls to account for 
the inconsistency in IV estimates. As such, we examine a number of methods to estimate 
plausible bounds on the Q-Q tradeoff. First, we argue that the true estimate is 
bounded by the OLS and IV estimtes. We also follow \citet{Conleyetal2012} in
conceptualizing twins as a plausibly exogenous event, and derive estimates by assuming 
that the traditional exclusion restriction is `close to' holding.

This work which further probes the Q-Q tradeoff is important, especially in the 
developing country setting on which we focus. In order to conduct empirical tests, 
we construct a large microdata set from 68 developing countries with observations on 
more than 2.5 million children and nearly 1 million mothers. The macro level trends in 
this data suggest that educational attainment has risen considerably while completed 
and desired fertility has fallen sharply over the past 50 years (see figures 
\ref{TWINfig:fertrend} and \ref{TWINfig:eductrend}). Similar effects have been described 
extensively in the empirical literature (ie \citet{Hanushek1992}). It is of considerable 
relevance to researchers and to policy makers to determine whether such a trend is (at 
least partially) causal. A significant number of countries have engaged in aggresive 
fertility control policies over the previous century, with family planning still 
considered a concern according to policy makers in the developing world.\footnote{A 
recent survey of national governments suggests that fertility was perceived as too 
high in 50\% of developing countries, with this figure rising to 86\% among the least 
developed countries \citet{UN2010}.} As discussed, our evidence suggests that the same 
identification problem arises with using twins as an instrument for fertility in 
analysis of data from richer countries. Future work is merited that assesses the size 
of the bias.\footnote{The data for Israel and China analysed in the twin-IV studies of 
\citet{Angristetal2010} and \citet{RosenzweigZhang2009} respectively do not contain 
information on maternal health. The data for Norway analysed in the twin-IV study of 
\citet{Blacketal2005} are not accessible outside Norway.}

This paper unfolds as follows.  In the next section we discuss the existing 
literature which estimates the Q-Q model using twins.  We then discuss the large
dataset that we will use to examine twin exogeneity and to bound the Q-Q tradeoff.
Section \ref{TWINscn:method} discusses our methodology, while section 
\ref{TWINscn:results} presents results.  We briefly conclude in the final section.


\section{The Twin Literature}
The occurrence of twin births is a commonly used for identification in economic
studies.  Specifically, twin births as a shock to total fertility have been used
to motivate the estimation of the quantity-quality trade-off 
\citep{RosenzweigWolpin1980,Blacketal2005,Angristetal2010,Caceres2006}, the 
effects of childbearing on female labour force participation 
\citep{RosenzweigWolpin1980b,Jacobsenetal1999,AngristEvans1998}, and the effects 
of unwed childbearing on marriage market outcomes, poverty and welfare receipt 
\citep{BronarsGrogger1994}.\footnote{Twins have been widely used in the economic, 
medical, biology and psychology literature in a number of ways.  In this paper we 
focus only on the use of twin births as an instrument for total fertility, and 
not on the so called `twin studies', which base inference on between-twin 
comparisons using maternal fixed effects.}

The methodology employed in all of these studies requires that twinning be (at 
least conditionally) exogenous.  Typically then, the previous literature has made 
one of a series of assumptions.  The earliest twin studies of the Q-Q trade-off 
\citep{RosenzweigWolpin1980,RosenzweigWolpin1980b} pointed out that twinning 
is both a positive function of total parity, and of maternal age.  By focusing
on twins at first birth and conditioning on maternal age, they thus produced 
consistent estimates of the tradeoff under the assumption that beyond maternal 
age and parity, twin births were entirely random.  A more recent wave of studies 
including \citet{Blacketal2005,Caceres2006,Lietal2008,Angristetal2010} take this 
one step further, assuming that twinning is conditionally exogenous after 
controlling for additional family characteristics such as a mother's race and her 
educational attainment.  They thus condition on a mother's age, race, and 
education, as well as on the birth order of the child in first and second stage 
regressions.

In some cases the validity of such assumptions is probed by regressing twinning
on observable family outcomes, or testing for the equality of means of certain 
characteristics between twin and non-twin families. \citet{Blacketal2005}, 
\citet{Lietal2008} and \citet{Sanhueza2009} report joint $F$-tests suggesting that
twinning is not related to parental education in their data samples, while 
\citet{RosenzweigZhang2009} report $t$-tests showing equality of means across twin 
and non-twin groups. However, as is well known and acknowledged in each case, any 
such tests are at best partial evidence in support of instrumental validity.  While 
twins can be shown to be unrelated to observable or measured characteristics, 
similar tests cannot be run for variables which are either unobservable, or not 
recorded in survey data. We return to this point in the following sections.

Finally, recent studies aim to control for the fact that multiple births are 
correlated with fertility treatments.  Typically such a treatment requires either 
focusing on offspring born before the introduction of fertility treatments 
\citep{Caceres2006, Angristetal2010}, or, in the case of sufficiently rich data, 
removing families undergoing fertility treatment from estimation samples 
\citep{Braakman2014}.  Once again, consistent estimation in this case is based
on the assumption that beyond fertility treatment and family controls listed 
above, twin births are as good as random.

Critiques of the twin instrument have focused on parental behaviours in response
to twins, rather than on the likelihood that parental behaviours (or endowments)
may affect the likelihood of twinning.  \citet{RosenzweigZhang2009}  question the
effect that close (or indeed no) birth spacing and an endowment effect---where 
parental behaviours respond to the lower health at birth of twins compared to single 
births\footnote{Using data from the United States, \citet{Almondetal2005} document 
that twins have substantially lower birth weight, lower APGAR scores, higher use of 
assisted ventilation at birth and lower gestion period than singletons.}---has on 
investments in pre-twin siblings.  They demonstrate that if parents behave in such
a manner, bounds for the Q-Q trade-off can be calculated.  This hypothesis is 
tested in \citet{Angristetal2010}, and applied in \citet{FitzsimonsMalde2010}.



%REMOVED.  SHOULD MAKE THIS POINT MORE EXTENSIVELY IN INTRO.
%Previous twin literature has not controlled for maternal health variables which
%may increase the likelihood of live twin births.  There is, however, considerable
%evidence that maternal health and maternal behaviours during pregnancy have a 
%substantial effect on birth and later life outcomes \citep{Almondetal2011,
%BhalotraRawlings2013,Barker1995}.  As we will discuss in the coming sections,



\section{Data}
%********************************************************************************
In this study we use matched data of mothers and their children coming from the 
Demographic and Health Surveys.  These surveys allow mothers to be linked to all
children living in the same household
%allow mothers to be linked with all children living in the same household, plus
%all children born in the 

%measures of child outcomes and mother 


\section{Methodology}
Typically, empirical analyses of the quality-quantity trade-off focus on 
producing consistent estimates of $\beta_1$ in the following specification:
\begin{equation}
\label{TWINeqn:secondstage}
educ_{ij}=\beta_0+\beta_1 fert_{j} + \bm{X}\bm{\beta}_2+u_{ij}.
\end{equation}
Here, quality is proxied by the educational attainment of child $i$ in family 
$j$, ($educ$) and fertility ($fert$) is measured as the total births in a child's
family.  A vector of family and child controls is included, denoted $\bm{X}$.  As
has been extensively discussed in prior literature, estimation of $\beta_1$ using
OLS and cross-sectional data will result in biased coefficients given that child 
quality and quantity are jointly determined \citep{BeckerLewis1973,BeckerTomes1976}, 
and given that unobservable parental behaviours and attributes influence both 
fertility decisions, and investments in children's education \citep{Qian2009}.



\section{Results}
\subsection{Twin Exogeneity}
\label{TWINsscn:twinRes}
%*******************************************************************************
Table \ref{TWINtab:comp} presents a test of balance of maternal characteristics
between mothers who have and who have not given birth to twins.  This table
presents a comparison of means, and a two-tailed $t$-test of the null that 
mothers of twins are not statistically different from mothers of non-twins.
These comparisons suggest that twin mothers are older at the moment of birth,
begin child-bearing earlier, have more children, and are more (less) likely to
come from top (bottom) wealth quintiles.  These comparisons of means, however,
are not surprising given the well established fact that twins occur with higher
frequency as mothers age, and at higher parities \citep{Hall2003}. Given that
mother's who have more births (and hence who are more likely to have twins) are
different in other dimensions, these tests may capture the effect of selection
into higher fertility, rather than twinning itself.  It is, however, worth 
pointing out that table \ref{TWINtab:comp} suggests that healthier mothers as
proxied by height and body mass index (BMI) are unconditionally more likely to 
have twins.  Similar comparisons of all births suggest that healthier mothers
on average have less births.

In order to examine partial correlations, we regress a child's twin status
(1 if a twin, 0 otherwise) on their mother's health, education, and a range
of other demographic and family characteristics.\footnote{In our principal
specification, the full set of controls are country, child year of birth, and
age dummies; a cubic function of mother's age at time of birth; mother's age
at time of first birth; mother's education and education squared; and mother's
height and BMI.  We cluster standard errors at the level of the mother.}  In 
table \ref{TWINtab:twinreg1} we report the results from specification 
(\ref{TWINeqn:twinreg}). These results suggest that twin births are not random, 
even after conditioning on maternal age and child birth order as is typical in 
recent twin literature. The inclusion of a full set of country and year-of-birth 
dummies (not displayed in table \ref{TWINtab:twinreg1}) will capture any 
systematic trend in the frequency of twin births across time or regions, and 
country dummies will absorb all time invariant differences in the probability 
of a twin birth across countries.  The estimated coefficients and signs support 
the idea discussed in section \ref{TWINscn:method} that higher `investments' 
(for example in maternal health) required to maintain multiple healthy fetuses 
\emph{in utero} may result in non-random twin births. Initially results from the 
pooled DHS data are presented as this provides a particularly large sample with 
which to test the hypothesis of twin exogeneity.  This is represented in table 
\ref{TWINtab:twinreg1} column (1) and provides considerable evidence that live 
multiple births respond to family `choice' variables such as education (tests 
for the joint significance of both socioeconomic variables and health variables 
are rejected with p-values of 0.0000).

These results are robust to all alternative specifications examined. Significant
and quantitatively similar results are found if a logit model is estimated 
rather than a linear probability model, and when running separate models for 
twinning at each birth order.  Similarly, if we run the regression at the level
of the mother or include any combination of fertility measures, similar 
patterns are observed.

The fact that maternal health is correlated with twinning is supported by 
medical literature, although is not a point which has been incorporated into 
prior economic studies of twinning.  \citet{Hall2003} for example suggests that 
follicle-stimulating hormone (FSH) is associated with an increased likelihood 
of twinning, and is found in higher concentrations in older, heavier and taller 
mothers.  Further, she suggests ``that adequate maternal folic acid consumption 
could affect the number of twins coming to term'' (see p.\ 741, and further 
discussion in \citet{Lietal2003}).  Given that twinning also increases in cases 
where the mother undergoes fertility treatment, we run a similar regression for 
children born in a period not potentially affected by IVF.\footnote{In order to 
be conservative, we estimate for the period preceeding 1990, the date which 
coincides with the first reported successful use of IVF in South Africa, an 
early-adopter among DHS countries.}  These results are included in columns (4) 
and (5).  Pre- and post-1990 results are qualitatively similar although education 
is no longer significant prior to 1990 (in the smaller sample). Mother's height 
and weight, and family socioeconomic variables remain economically and 
statistically significant.

If the reason non-random twin births are observed is due to insufficient investment 
in the developing fetus, it seems likely that twin `selection' will be more 
pronounced in lower income settings, and settings where the mother is less well 
resourced during gestation.  This is tested in columns (2) and (3), where it is 
shown that the violation of the twin exclusion restriction is particularly strong 
in low income countries.  Here maternal health is a more important predictor, and 
the explained portion of this set of variables is larger that in middle-income 
countries.\footnote{The low $R^2$ in these regressions is not at all surprising 
given that twin conception can be thought of as an approximately random process.
The fact that socioeconomic and health variables have \emph{any} power in 
explaining twin birth however is sufficient to invalidate IV estimations if these 
or other relevant predictors are not controlled for.} Finally, we add measures for 
the availability of prenatal care (for the subset of children for whom this variable 
is recorded).  The omitted case are those who report prenatal care from relatives
or traditional health care providers.  We find evidence to suggest that those 
receiving prenatal care from a doctor are more likely to give live birth to twins,
and those who receive no prenatal care are less likely to twin.  We treat these
results with precaution however, given that accessing prenatal care is a choice
variable, and likely endogenous.

These results call into question the veracity of the conditional exogeneity (or 
`as good as random') assumption regarding twin births required to estimate 
$\beta_1$ consistently in (\ref{TWINeqn:secondstage}).  This implies that omitting 
factors such as family socioeconomic and maternal health variables is likely to 
result in inconsistent IV estimates if these variables are also correlated with 
child `quality' outcomes.  It seems unlikely, furthermore, that controlling for 
the factors that we have shown to predict twinning will allow us to recover 
consistent IV estimates.  Maternal health stocks are both difficult to measure, 
and multi-dimensional in nature.\footnote{In ongoing work, we find that the 
probability of twinning falls with maternal alcohol consumption, smoking, drug 
taking and a range of other behaviours in data from the USA, Chile, UK and 
Scotland.}  In the following subsections we examine the effect that omitting these
factors is likely to have on IV estimates of the Q-Q trade-off.


\subsection{Twinning and Quantity-Quality}
\label{TWINsscn:QQRes}
%********************************************************************************
As is typically found in empirical studies of the Q-Q tradeoff, correlations 
between family size and child outcome variables are negative, and strongly 
significant. Table \ref{TWINtab:OLS} shows OLS estimates of child `quality' on
total fertility. These results suggest that an additional sibling is associated
with an approximatelty 0.1 sd decrease in standardised schooling outcomes.


Of course, this empirically observed relationship between an individual's sibship 
size and their measured `quality' does not necessarily imply that such a 
trade-off exists if parental decisions regarding the production of child quality 
and quantity are jointly made and possibly influenced by unobserved factors 
\citep{BeckerTomes1976}.  Principally here we are concerned with unobserved 
parental behaviours which may favour both lower family size and higher child 
quality. \citet{Qian2009} suggests that such a mechanism will exist where parents 
who value education more highly also decide to have less children.  The OLS 
results are consistent with such a result, as the inclusion of maternal education 
and maternal health controls---likely correlated with desires for smaller 
family size and higher investments per child---reduce the magnitude of this 
observed tradeoff.

%********************************************************************************
\subsubsection{Q-Q Tradeoff: Estimates Using Twin Births}
Rather than focusing on OLS estimates which are likely to be biased, we turn 
to estimates which rely on twin births to identify the Q-Q tradeoff.  As we 
outline in section \ref{TWINsscn:twinRes}, the assumption of `as good as random' 
twin births is unlikely to hold, even when conditioning on the augmented set of 
controls proposed in (\ref{TWINeqn:twinreg}).  If this is the case, we will be 
unable to consistently estimate $\beta$ using twin births.  

However, it is likely that the $\beta$ that we estimate using twin births will 
provide us with a strict lower bound of the magnitude of the Q-Q trade-off. 
Given that we expect that the bias in this estimate is due to those mothers who 
invest more in their children in utero (or with greater initial health endowments) 
being more likely to give birth to twins (and hence having larger family sizes), 
and at the same time we expect healthier mothers to invest more in their children
after birth resulting in higher quality children, then relegating health 
variables to the error term will result in a positive bias on the fertility 
coefficient.  In order to determine the effect that these omitted variables have 
on estimates of the Q-Q trade-off, we turn to results for equation
(\ref{TWINeqn:secondstage}), both omitting and not omitting maternal health and
socioeconomic variables.

The main specification is presented in row one of tables 
\ref{TWINtab:IVTwoplus}-\ref{TWINtab:IVFourplus}.  These results suggest that
the inclusion of maternal health and socioeconomic controls may be of 
considerable importance.  In each case (for 2+, 3+ and 4+ groups), the base
case which includes no twin predictor variables as controls results in an 
insignificant (and at times weakly positive) coefficient on fertility.  However,
once including the full set of health and socioeconomic variables, this 
coefficient becomes significantly more negative, resulting (at least in the
3+ and 4+ case) in point estimates of around $-3$ to $-4$\% of a standard deviation
in standardised educational attainment.  Further, these results hold even when
controlling for birth order effects, as can be seen in row 2 of the table.

These results are heterogeneous across birth orders, country income level, and
gender of the affected child (discussion of these latter two groups will be
delayed until the following subsection.)  The magnitude and significance of the 
results is lowest when considering the effect on the first-born child of moving 
from two to three births (the 2+ group), and higher when considering moving from
three to four births.  \texttt{comment re budget constraint???}

Turning to our second innovation, we are able to examine the effect of an
additional birth which takes families above their desired fertility threshold
by interacting the twin instrument with a desired-fertility variable. [EXPLAIN
AND INTERPRET DESIRED FERTILITY RESULTS FROM TABLES 5-7 (row labelled Desired-%
Threshold).]   



\subsection{Estimating Q-Q Bounds}
\label{TWINsscn:boundsRes}
%********************************************************************************
The results from the previous subsection provide consistent estimates of $\beta$
via 2SLS \emph{if} the full set of controls completely account for those 
characteristics and behaviours which predict giving live birth to twins.  
However, given that we have shown that twinning is predicted by a wide range
of health behaviours, and given that maternal health variables in this dataset
are limited, it seems unlikely that all relevant variables are included in
these specifications.  As such, we turn to \citeauthor{Conleyetal2012}'s
(\citeyear{Conleyetal2012}) methodology to estimate bounds for the Q-Q trade-off.

As described in section \ref{TWINsscn:BoundMeth}, this involves the specification
of some prior over the sign that the twin instrument takes in the structural
equation \ref{TWINeqn:Conley}.  Results are displayed in figures 
\ref{TWINfig:ltz2} and \ref{TWINfig:ltz3}, and in table \ref{TWINtab:Conley}.
At each point on the horizontal axis of figures \ref{TWINfig:ltz2} and 
\ref{TWINfig:ltz3}, the bounds for $\beta$ are displayed, along with the
corresponding point estimate under the assumption that $\gamma$ is distributed
$U(0,\delta)$.  Dashed lines present the 95\% confidence interval, while the
solid line represents the point estimate.

Table \ref{TWINtab:Conley} provides these bounds at a particular point on this
graph.  This point corresponds to an assumption of $\gamma \in [0,2\hat\delta]$
and $\gamma\sim U(0,2\hat\delta)$ for the UCI and the LTZ approaches respectively,
where $\hat\delta$ is estimated from equation \ref{TWINeqn:Conley} with DHS data.
These results allow us to make informative statements about the Q-Q trade-off,
at least for those twins born at third and fourth birth orders who affect more
children the birth.  Further, these results suggest that the magnitude of this
trade-off may be large: and indeed much larger than those estimated in existing 
studies as the twin instrument departs from true exogeneity.  For example, the
effect of an additional child at birth order three on first- and second-born
children (ie the 3+ group) has an effect between -1.5\% to as much as -15\% of
a standard deviation in standardised educational attainment if the prior 
(based on DHS estimates) is true.  While it is not the case that informative
statements can always be made given the wide bound on estimates in some cases
(particularly the five plus group), it is important to point out that in each
case as the twin-exogeneity assumption is loosened, the point estimate of the
Q-Q tradeoff becomes more (and considerably) negative.



\section{Conclusion}
%********************************************************************************
Estimation of the Q-Q model using twin births to instrument fertility relies on 
the fact that twin births are exogenous. This implies that both conceiving twins 
\emph{and} taking them to term cannot depend on maternal behaviours and health 
stocks. In this paper we suggest that such an assumption may be too strong to 
hold in practice. We show that healthier mothers are more likely to give birth to 
twins even if twin conception is random, and that the size of this effect is 
considerable.

We re-estimate the Q-Q tradeoff accounting for such an effect.  We thus show
that the existing wisdom (which suggests that the effect of additional births 
on human capital might be minor), likely underestimates the true magnitude of
the Q-Q tradeoff.  Our IV estimates move point estimates from approximately 
0\% of a standard deviation (using the old methodology which does not account
for maternal health) to a significant $-4\%$ of a standard deviation of 
educational attainment.  Further, we suggest that this is likely
a lower bound of the true effect.  Depending upon the degree to which maternal
characteristics predict twin births, the true trade-off may be considerably
larger, perhaps even doubling these IV estimates.  The availability of richer 
maternal health and behaviour data in surveys would allow for us to restrict
the size of these bounds and produce more precise estimates.






\newpage
\section*{Figures}
\begin{figure}[htpb!]
\begin{center}
\caption{Rates of twinning by birth cohort and mother's age (DHS)}
\label{TWINfig:trend}
\includegraphics[scale=0.95]{\twinfolder/Figures/AgeAverage.eps}
\vspace{-8mm}
\end{center}
\end{figure}

\begin{figure}[htpb!]
\begin{center}
\caption{Rates of twinning by birth cohort and birth order (DHS)}
\label{TWINfig:trendbord}
\includegraphics[scale=0.95]{\twinfolder/Figures/BordAverage.eps}
\vspace{-8mm}
\end{center}
\end{figure}

\begin{figure}[htpb!]
\begin{center}
\caption{Birth Weight}
\label{TWINfig:Bwt}
\includegraphics[scale=0.92]{\twinfolder/Figures/Birthweight_op.eps} 
\floatfoot{Note to figure: Each plot (above and below the horizontal axis) is a standard
histogram of birthweights.  The birth weight of twins is displayed above the horizontal 
axis, while the birth weight of singletons is displayed below the axis.  Mirroring of this
histogram is for ease of presentation.}
\end{center}
\end{figure}

\begin{figure}[htpb!]
\begin{center}
\caption{Size reported (retrospectively) by mother}
\label{TWINfig:size}
\includegraphics[scale=0.92]{\twinfolder/Figures/Size.eps}
\vspace{-8mm}
\end{center}
\end{figure}


%\begin{figure}[htpb!]
%\begin{center}
%\begin{wrapfigure}{r}{5cm}
%\rotcaption{Total Recorded Births and Fetal Deaths, 2006-2011}
%\label{TEENfig:BirthDeath}
%\includegraphics[scale=0.8, angle=90]{\teenfolder/Figures/BirthDeath.eps} 
%\end{center}
%\end{figure}

%\begin{figure}[htpb!]
%\begin{center}
%\caption{Pill Prescriptions and Availability by Time}
%\vspace{-5mm}
%\label{TEENfig:Pilltime}
%\includegraphics[scale=0.54]{\teenfolder/Figures/Pill.eps} 
%\end{center}
%\end{figure}


\clearpage

\section*{Tables}
\begin{table}[htpb!]							
\caption{Twin versus Non-Twin Descriptives}		
\label{tab:twinDesc}	
\begin{center}
\begin{tabular}{lccccc} \toprule
\textbf{Variable} &   \textbf{Obs.} & \textbf{Mean} & \textbf{Std. Dev.} &  \textbf{Min} &  \textbf{Max} \\ \midrule
\multicolumn{6}{l}{\textsc{Non-Twins}} \\
Birthweight &    531020 &   3187.346 &   632.6284  &      501  &     5499\\
Premature &    172299  &  0.0223 &    0.1477 &          0   &       1\\
Antenatal Checks &    941622  &  4.088 &    3.481 &          0  &       20\\
Caesarean &   1063192  &  0.0732 &     0.260 &          0        &  1\\
Breastfeeding &   1058024  &  14.100 &    9.117 &          0     &    47\\
Infant Mortality &   3575793 &   0.009 &    0.0955 &          0  &        1\\
Child Mortality &   2649717  &  0.0168 &    0.1288 &          0   &       1\\
Education &   3772448 &   2.582 &    3.631 &          0       &  23\\
School Z-Score &   2393790  &  0.000 &    1.000 &  -9.999701 &  41.227 \\
No Educ. &   2193941  &  0.1453 &    0.3524 &          0      &    1\\
High School &   1413658  &  0.4567 &    0.4981 &          0     &     1\\
Male &   3847827  &  0.5223 &   0.4994 &          0        &  1\\ \midrule
\multicolumn{6}{l}{\textsc{Twins}} \\
Birthweight &     13688  &   2491.28  &  661.708 &       544  &     5454\\
Premature &      3412  &  0.0911 &    .2878 &          0     &     1\\
Antenatal Checks &     15699  &  4.5625 &    3.832 &          0 &        20\\
Caesarean &     24976  &  0.1585 &    0.3652 &          0        &  1\\
Breastfeeding &     22359  &  12.967 &    9.566 &          0     &    47\\
Infant Mortality &     69679  &  0.0500 &    0.2180 &          0   &       1\\
Child Mortality &     49585  &  0.0752 &    0.2638 &          0     &     1\\
Education &     74726 &   2.138 &    3.302 &          0       &  19\\
School Z-Score &     44383  & -0.0292 &    0.963 &  -8.221  & 10.271 \\
No Educ. &     40170   & 0.1624  &   0.3688 &          0       &   1\\
High School &     24737   & 0.3989 &    0.4897 &          0     &     1\\
Male &     76132  &  0.5049 &    0.4999 &        0       &   1\\ \bottomrule
\multicolumn{6}{p{8cm}}{\textsc{Notes:}} \\ \midrule
\end{tabular}\end{center}\end{table} 


\begin{landscape}\begin{table}[htpb!] 
\caption{Probability of Giving Birth to Twins} \label{TWINtab:twinreg1} 
\begin{center}\begin{tabular}{lcccccc} \toprule \toprule 
&(1)&(2)&(3)&(4)&(5)&(6)\\
Twin*100&All&\multicolumn{2}{c}{Income}&\multicolumn{2}{c}{Time}&Prenatal\\
 \cmidrule(r){3-4} \cmidrule(r){5-6} 
&&Low inc&Middle inc&1990-2013&1972-1989&\\\midrule
\begin{footnotesize}\end{footnotesize}&\begin{footnotesize}\end{footnotesize}&\begin{footnotesize}\end{footnotesize}&\begin{footnotesize}\end{footnotesize}&\begin{footnotesize}\end{footnotesize}&\begin{footnotesize}\end{footnotesize}&\begin{footnotesize}\end{footnotesize}\\
Age&0.596***&0.615***&0.554***&0.647***&0.326***&0.631***\\
&(0.029)&(0.036)&(0.050)&(0.033)&(0.075)&(0.040)\\
Age Squared&-0.008***&-0.008***&-0.008***&-0.009***&-0.003**&-0.009***\\
&(0.001)&(0.001)&(0.001)&(0.001)&(0.001)&(0.001)\\
Age First Birth&-0.053***&-0.093***&0.004&-0.052***&-0.056***&-0.041***\\
&(0.009)&(0.012)&(0.014)&(0.010)&(0.019)&(0.013)\\
Education (years)&0.042**&0.089***&-0.005&0.048**&0.022&-0.068**\\
&(0.017)&(0.022)&(0.029)&(0.020)&(0.034)&(0.028)\\
Education squared&-0.002&-0.006***&0.001&-0.002&0.000&0.003\\
&(0.001)&(0.002)&(0.002)&(0.002)&(0.003)&(0.002)\\
Height&0.058***&0.057***&0.059***&0.062***&0.042***&0.058***\\
&(0.004)&(0.005)&(0.007)&(0.005)&(0.008)&(0.007)\\
BMI&0.048***&0.063***&0.039***&0.046***&0.055***&0.044***\\
&(0.006)&(0.009)&(0.009)&(0.007)&(0.011)&(0.011)\\
Prenatal (Doctor)&&&&&&0.906***\\
&&&&&&(0.128)\\
Prenatal (Nurse)&&&&&&0.067\\
&&&&&&(0.108)\\
Prenatal (None)&&&&&&-0.497***\\
&&&&&&(0.132)\\
&&&&&&\\R-squared&0.01&0.01&0.01&0.01&0.01&0.01\\
Observations &1930653&1201555&729098&1524947&405706&615935\\
\hline\hline\multicolumn{7}{p{14.3cm}}{\begin{footnotesize}\textsc{Notes:} All specifications include a full set of year of birth and  country dummies, and are estimated as linear probability models.  Twin is multiplied by 100 for presentation.  Height is measured in cm  and BMI is weight in kg divided by height in metres squared. l  Prenatal care variables are only recoreded for recent births.  As  such, column (6) is estimated only for that subset of births where  these observations are made.
$^{*}$p$<$0.1; $^{**}$p$<$0.05; $^{***}$p$<$0.01
 \end{footnotesize}}\\ \hline \normalsize \end{tabular}\end{center}\end{table}\end{landscape} 


%\begin{table}[htpb!]							
\caption{Probability of Giving Birth to Multiple Children (Chile)}							
\label{tab:twinreg2}							
\begin{center}							
\begin{tabular}{lclc} \toprule							
	&	(1)	&		&		\\
VARIABLES	&	Twin	&		&		\\ \midrule
\vspace{4pt} & \begin{footnotesize}\end{footnotesize} & \begin{footnotesize}\end{footnotesize} & \begin{footnotesize}\end{footnotesize} \\							
Mother age birth	&	0.00451***	&	Preg Normal Weight	&	0.00820**	\\
	\vspace{4pt} & \begin{footnotesize}	(0.00129)	\end{footnotesize} & \begin{footnotesize}		\end{footnotesize} & \begin{footnotesize}	(0.00417)	\end{footnotesize} \\
Mother's Age$^2$	&	-7.38e-05***	&	Preg Overweight	&	0.0110**	\\
	\vspace{4pt} & \begin{footnotesize}	(2.23e-05)	\end{footnotesize} & \begin{footnotesize}		\end{footnotesize} & \begin{footnotesize}	(0.00490)	\end{footnotesize} \\
Indigenous	&	-0.0111***	&	Preg Obese	&	-0.00908	\\
	\vspace{4pt} & \begin{footnotesize}	(0.00384)	\end{footnotesize} & \begin{footnotesize}		\end{footnotesize} & \begin{footnotesize}	(0.00695)	\end{footnotesize} \\
Poor	&	0.00903**	&	Preg Anemia	&	0.00854*	\\
	\vspace{4pt} & \begin{footnotesize}	(0.00421)	\end{footnotesize} & \begin{footnotesize}		\end{footnotesize} & \begin{footnotesize}	(0.00481)	\end{footnotesize} \\
Meduc Preschool	&	-0.0106	&	Preg No Checkup	&	-0.0174***	\\
	\vspace{4pt} & \begin{footnotesize}	(0.0136)	\end{footnotesize} & \begin{footnotesize}		\end{footnotesize} & \begin{footnotesize}	(0.00668)	\end{footnotesize} \\
Meduc Primary	&	0.00572	&	Preg Smoked	&	-0.00611	\\
	\vspace{4pt} & \begin{footnotesize}	(0.0135)	\end{footnotesize} & \begin{footnotesize}		\end{footnotesize} & \begin{footnotesize}	(0.00400)	\end{footnotesize} \\
Meduc Secondary1	&	0.00885	&	Preg drugs 1	&	-0.00157	\\
	\vspace{4pt} & \begin{footnotesize}	(0.0136)	\end{footnotesize} & \begin{footnotesize}		\end{footnotesize} & \begin{footnotesize}	(0.0160)	\end{footnotesize} \\
Meduc Secondary2	&	0.00738	&	Preg drugs 2	&	-0.0184***	\\
	\vspace{4pt} & \begin{footnotesize}	(0.0133)	\end{footnotesize} & \begin{footnotesize}		\end{footnotesize} & \begin{footnotesize}	(0.00352)	\end{footnotesize} \\
Meduc Technical1	&	0.0126	&	Preg alcohol 1	&	-0.00151	\\
	\vspace{4pt} & \begin{footnotesize}	(0.0144)	\end{footnotesize} & \begin{footnotesize}		\end{footnotesize} & \begin{footnotesize}	(0.00549)	\end{footnotesize} \\
Meduc Technical2	&	0.0152	&	Preg alcohol 2	&	-0.0185***	\\
	\vspace{4pt} & \begin{footnotesize}	(0.0145)	\end{footnotesize} & \begin{footnotesize}		\end{footnotesize} & \begin{footnotesize}	(0.00296)	\end{footnotesize} \\
Meduc Tertiary1	&	0.0182	&	Preg public hosp.	&	-0.0117***	\\
	\vspace{4pt} & \begin{footnotesize}	(0.0145)	\end{footnotesize} & \begin{footnotesize}		\end{footnotesize} & \begin{footnotesize}	(0.00351)	\end{footnotesize} \\
Meduc Tertiary2	&	0.0251*	&	Rural	&	-7.58e-05	\\
	\vspace{4pt} & \begin{footnotesize}	(0.0150)	\end{footnotesize} & \begin{footnotesize}		\end{footnotesize} & \begin{footnotesize}	(0.00450)	\end{footnotesize} \\
\vspace{4pt} & \begin{footnotesize}\end{footnotesize} & \begin{footnotesize}\end{footnotesize} & \begin{footnotesize}\end{footnotesize} \\							\\
Constant	&	-0.0394	&	Observations	&	14,900 \\
	\vspace{4pt} & \begin{footnotesize}	(0.0270)	\end{footnotesize} & \begin{footnotesize}	R-squared	\end{footnotesize} & \begin{footnotesize}	0.005	\end{footnotesize}\\ \midrule
\multicolumn{4}{p{11cm}}{\begin{footnotesize}Notes to table: Data comes from the Encuesta Longitudinal de Primera Infancia (ELPI) from Chile.  Education at each level are dummy variables, ‘no education’ is the omitted base. Regional controls and child age fixed effects are omitted for clarity.  Heteroscedasticity robust standard errors are presented in parenthesis. *** p$<$0.01, ** p$<$0.05, * p$<$0.1\end{footnotesize}} \\ \bottomrule
\end{tabular}\end{center}\end{table}							


\input{"\twinfolder/Tables/ChileTwin.tex"}

\input{"\twinfolder/Tables/ScotlandTwin.tex"}

\clearpage
\newpage

\bibliography{./BiBBase1}

\newpage
\appendix
\section*{Appendices}

\section{Appendix Tables}
%\input{"\twinfolder/Tables/DesiredTest.tex"}

\input{"\twinfolder/Tables/Countries.tex"}

%\begin{table}[!htbp] \centering 
\caption{Instrumental Variables Estimates: Five Plus} \vspace{4mm} 
\label{TWINtab:IVFiveplus} 
\begin{tabular}{lcccc} \toprule \toprule 
&Base&&&\\
&Controls&Socioec&Health&Obs.\\\midrule
\multicolumn{5}{l}{\textsc{Pre-Twins}}\\ 
&&&&\\
\multicolumn{5}{l}{\textbf{All Families}}\\ 
Fertility&-0.015&-0.011&-0.022&359,164\\
         &(0.022)&(0.021)&(0.020)&\\
&&&&\\
\multicolumn{5}{l}{\textbf{All Families (bord dummies)}}\\ 
Fertility&-0.015&-0.013&-0.024&359,164\\
         &(0.022)&(0.020)&(0.020)&\\
&&&&\\
\multicolumn{5}{l}{\textbf{Low-Income Countries}}\\ 
Fertility&0.000&-0.005&-0.017&237,694\\
         &(0.026)&(0.024)&(0.023)&\\
&&&&\\
\multicolumn{5}{l}{\textbf{Middle-Income Countries}}\\ 
Fertility&-0.047&-0.019&-0.028&121,470\\
         &(0.040)&(0.039)&(0.039)&\\
&&&&\\
\multicolumn{5}{l}{\textbf{Desired-Threshold}}\\ 
Fertility&0.003&0.004&-0.006&359,164\\
         &(0.024)&(0.022)&(0.022)&\\
Fertility$\times$desire&-0.020**&-0.018**&-0.018**&\\
         &(0.009)&(0.008)&(0.008)&\\
\midrule\multicolumn{5}{l}{\textsc{Twins and Pre-Twins}}\\ 
&&&&\\
\multicolumn{5}{l}{\textbf{All Families}}\\ 
Fertility&-0.019&-0.032*&-0.042**&462,413\\
         &(0.020)&(0.019)&(0.019)&\\
&&&&\\
\multicolumn{5}{l}{\textbf{All Families (bord dummies)}}\\ 
Fertility&-0.015&-0.018&-0.028&462,413\\
         &(0.021)&(0.019)&(0.019)&\\
\midrule\multicolumn{5}{l}{\textsc{First Stage (Pre-Twins)}}\\ 
&&&&\\
\multicolumn{5}{l}{\textbf{All Families}}\\ 
Twins&0.846***&0.831***&0.835***&359,164\\
         &(0.026)&(0.025)&(0.025)&\\
\hline\multicolumn{5}{p{10cm}}{\begin{footnotesize}\textsc{Notes:} Five-plus refers to all first- to fourth-born children in families with five or more children.  Each cell presents the coefficient of a 2SLS regression where fertility is instrumented by twinning at birth order five.  Base controls include child age, mother's age, and mother's age at birth fixed effects plus country and year-of-birth FEs.  The sample is made up of all children aged between 6-18 years from families in the DHS who fulfill five-plus requirements. Birth order dummies are included only if explicitly stated.  First-stage results in the final panel correspond to the second stage in row 1.  Full first stage results for each row are available in table \ref{TWINtab:FS}. Standard errors are clustered by mother. 
$^{*}$p$<$0.1; $^{**}$p$<$0.05; $^{***}$p$<$0.01\end{footnotesize}}
\\\bottomrule\normalsize\end{tabular}\end{table} 


\begin{table}[!htbp] \centering 
\caption{Instrumental Variables Estimates: Female and Male Children} 
\vspace{4mm}\label{TWINtab:IVgend} 
\begin{tabular}{lcccccc} \toprule \toprule 
&\multicolumn{3}{c}{Females}&\multicolumn{3}{c}{Males}\\ 
\cmidrule(r){2-4} \cmidrule(r){5-7} 
&2+&3+&4+&2+&3+&4+ \\ \midrule 
\multicolumn{7}{l}{\textsc{Pre-Twins}}\\ 
&&&&\\
\multicolumn{7}{l}{\textbf{All Families}}\\ 
Fertility&-0.023&-0.050*&-0.056**&-0.027&-0.033&-0.010\\
&(0.035)&(0.028)&(0.026)&(0.034)&(0.027)&(0.027)\\
&&&&\\
\multicolumn{7}{l}{\textbf{All Families (bord dummies)}}\\ 
Fertility&-0.027&-0.054*&-0.051**&-0.020&-0.036&-0.010\\
&(0.035)&(0.027)&(0.026)&(0.034)&(0.027)&(0.027)\\
&&&&\\
\multicolumn{7}{l}{\textbf{Low-Income Countries}}\\ 
Fertility&0.022&-0.028&-0.027&0.005&-0.044&-0.003\\
&(0.039)&(0.034)&(0.030)&(0.042)&(0.033)&(0.033)\\
&&&&\\
\multicolumn{7}{l}{\textbf{Middle-Income Countries}}\\ 
Fertility&-0.104&-0.100**&-0.093*&-0.058&-0.033&-0.023\\
&(0.070)&(0.048)&(0.050)&(0.056)&(0.045)&(0.044)\\
&&&&\\
\multicolumn{7}{l}{\textbf{Desired-Threshold}}\\ 
Fertility&-0.041&-0.045&-0.041&-0.015&-0.029&-0.017\\
&(0.037)&(0.032)&(0.029)&(0.042)&(0.030)&(0.032)\\
Fertility$\times$desire&0.013&-0.006&-0.004&-0.004&-0.005&0.005\\
&(0.022)&(0.013)&(0.010)&(0.017)&(0.013)&(0.009)\\
\midrule\multicolumn{5}{l}{\textsc{Twins and Pre-Twins}}\\ 
&&&&\\
\multicolumn{7}{l}{\textbf{All Families}}\\ 
Fertility&-0.072***&-0.086***&-0.062***&-0.083***&-0.059**&-0.019\\
         &(0.025)&(0.022)&(0.022)&(0.026)&(0.024)&(0.024)\\
&&&&\\
\multicolumn{7}{l}{\textbf{All Families (bord dummies)}}\\ 
Fertility&-0.063**&-0.079***&-0.058**&-0.053**&-0.044*&-0.013\\
         &(0.028)&(0.023)&(0.023)&(0.027)&(0.024)&(0.025)\\
\midrule\multicolumn{7}{l}{\textsc{First Stage (Pre-Twins)}}\\ 
&&&&\\
\multicolumn{7}{l}{\textbf{All Families}}\\ 
Twins&0.808***&0.811***&0.845***&0.793***&0.809***&0.837***\\
         &(0.034)&(0.031)&(0.030)&(0.033)&(0.031)&(0.030)\\

\midrule\multicolumn{7}{p{13.4cm}}{\begin{footnotesize}\textsc{Notes:} Each cell presents the coefficient from a 2SLS regression of standardised educational attainment on fertility.  2+, 3+ and 4+ refer to the birth orders of children included in the regression.  For a full description of these groups see tables \ref{TWINtab:IVTwoplus}, \ref{TWINtab:IVThreeplus} and \ref{TWINtab:IVFourplus}.  Each regression includes full controls including maternal health and socioeconomic variables.  The sample is made up of all children aged between 6-18 years from families in the DHS who fulfill birth order and gender requirements indicated in the header.  Standard errors are clustered by mother.$^{*}$p$<$0.1; $^{**}$p$<$0.05; $^{***}$p$<$0.01 
\end{footnotesize}}
\\\bottomrule\normalsize\end{tabular}\end{table} 


\begin{landscape}\begin{table}[htpb!]\caption{First Stage Results} 
\label{TWINtab:FS}\begin{center}\begin{tabular}{lccccccccc}
\toprule \toprule 
&\multicolumn{3}{c}{2+}&\multicolumn{3}{c}{3+}&\multicolumn{3}{c}{4+}\\ \cmidrule(r){2-4} \cmidrule(r){5-7} \cmidrule(r){8-10} 
\textsc{Fertility}&Base&+S&+S\&H&Base&+S&+S\&H&Base&+S&+S\&H\\ \midrule 
\begin{footnotesize}\end{footnotesize}& 
\begin{footnotesize}\end{footnotesize}& 
\begin{footnotesize}\end{footnotesize}& 
\begin{footnotesize}\end{footnotesize}& 
\begin{footnotesize}\end{footnotesize}& 
\begin{footnotesize}\end{footnotesize}& 
\begin{footnotesize}\end{footnotesize}& 
\begin{footnotesize}\end{footnotesize}& 
\begin{footnotesize}\end{footnotesize}& 
\begin{footnotesize}\end{footnotesize}\\ 
\multicolumn{10}{l}{\textbf{Pre-Twins}}\\ 
Twin&0.798***&0.839***&0.841***&0.796***&0.821***&0.824***&0.841***&0.860***&0.863***\\
&(0.030)&(0.028)&(0.028)&(0.027)&(0.026)&(0.026)&(0.027)&(0.026)&(0.026)\\
\begin{footnotesize}\end{footnotesize}&\begin{footnotesize}\end{footnotesize}&\begin{footnotesize}\end{footnotesize}&\begin{footnotesize}\end{footnotesize}&\begin{footnotesize}\end{footnotesize}&\begin{footnotesize}\end{footnotesize}&\begin{footnotesize}\end{footnotesize}&\begin{footnotesize}\end{footnotesize}&\begin{footnotesize}\end{footnotesize}&\begin{footnotesize}\end{footnotesize}\\\multicolumn{10}{l}{\textbf{Pre-Twins (+bord)}}\\ 
Twin&0.796***&0.838***&0.839***&0.793***&0.825***&0.828***&0.842***&0.863***&0.865***\\
&(0.030)&(0.028)&(0.028)&(0.027)&(0.026)&(0.026)&(0.026)&(0.026)&(0.026)\\
\begin{footnotesize}\end{footnotesize}&\begin{footnotesize}\end{footnotesize}&\begin{footnotesize}\end{footnotesize}&\begin{footnotesize}\end{footnotesize}&\begin{footnotesize}\end{footnotesize}&\begin{footnotesize}\end{footnotesize}&\begin{footnotesize}\end{footnotesize}&\begin{footnotesize}\end{footnotesize}&\begin{footnotesize}\end{footnotesize}&\begin{footnotesize}\end{footnotesize}\\\multicolumn{10}{l}{\textbf{Low-Income}}\\ 
Twin&0.844***&0.861***&0.861***&0.804***&0.822***&0.824***&0.863***&0.866***&0.868***\\
&(0.037)&(0.036)&(0.036)&(0.033)&(0.032)&(0.032)&(0.032)&(0.032)&(0.032)\\
\begin{footnotesize}\end{footnotesize}&\begin{footnotesize}\end{footnotesize}&\begin{footnotesize}\end{footnotesize}&\begin{footnotesize}\end{footnotesize}&\begin{footnotesize}\end{footnotesize}&\begin{footnotesize}\end{footnotesize}&\begin{footnotesize}\end{footnotesize}&\begin{footnotesize}\end{footnotesize}&\begin{footnotesize}\end{footnotesize}&\begin{footnotesize}\end{footnotesize}\\\multicolumn{10}{l}{\textbf{Middle-Income}}\\ 
Twin&0.741***&0.803***&0.806***&0.774***&0.810***&0.815***&0.790***&0.844***&0.846***\\
&(0.051)&(0.045)&(0.045)&(0.047)&(0.044)&(0.044)&(0.047)&(0.043)&(0.043)\\
\begin{footnotesize}\end{footnotesize}&\begin{footnotesize}\end{footnotesize}&\begin{footnotesize}\end{footnotesize}&\begin{footnotesize}\end{footnotesize}&\begin{footnotesize}\end{footnotesize}&\begin{footnotesize}\end{footnotesize}&\begin{footnotesize}\end{footnotesize}&\begin{footnotesize}\end{footnotesize}&\begin{footnotesize}\end{footnotesize}&\begin{footnotesize}\end{footnotesize}\\\multicolumn{10}{l}{\textbf{Desired-Threshold}}\\ 
Twin&0.775***&0.821***&0.822***&0.749***&0.786***&0.788***&0.846***&0.863***&0.865***\\
&(0.036)&(0.033)&(0.033)&(0.030)&(0.029)&(0.029)&(0.028)&(0.028)&(0.028)\\
Twin$\times$desire&0.085&0.064&0.066&0.220***&0.161**&0.165**&-0.031&-0.016&-0.011\\
&(0.067)&(0.062)&(0.062)&(0.070)&(0.066)&(0.066)&(0.085)&(0.079)&(0.080)\\
\begin{footnotesize}\end{footnotesize}&\begin{footnotesize}\end{footnotesize}&\begin{footnotesize}\end{footnotesize}&\begin{footnotesize}\end{footnotesize}&\begin{footnotesize}\end{footnotesize}&\begin{footnotesize}\end{footnotesize}&\begin{footnotesize}\end{footnotesize}&\begin{footnotesize}\end{footnotesize}&\begin{footnotesize}\end{footnotesize}&\begin{footnotesize}\end{footnotesize}\\\multicolumn{10}{l}{\textbf{Twins and Pre-twins}}\\ 
Twin&0.747***&0.801***&0.806***&0.815***&0.831***&0.835***&0.861***&0.863***&0.868***\\
&(0.028)&(0.026)&(0.026)&(0.028)&(0.027)&(0.027)&(0.027)&(0.026)&(0.026)\\
\begin{footnotesize}\end{footnotesize}&\begin{footnotesize}\end{footnotesize}&\begin{footnotesize}\end{footnotesize}&\begin{footnotesize}\end{footnotesize}&\begin{footnotesize}\end{footnotesize}&\begin{footnotesize}\end{footnotesize}&\begin{footnotesize}\end{footnotesize}&\begin{footnotesize}\end{footnotesize}&\begin{footnotesize}\end{footnotesize}&\begin{footnotesize}\end{footnotesize}\\
\midrule\multicolumn{10}{p{18.0cm}}{\begin{footnotesize}\textsc{Notes:} Each cell represents the coefficient from the first-stage of a two-stage regression.  The first-stage represents the effect of twinning at parity $N$ on total fertility where $N$ is 2, 3 or 4 for 2+, 3+ and 4+ groups respectively.  The 2+ group includes all first borns in families with at least 2 births, the 3+ group includes first and second borns in families with at least 3 births, and the 4+ group includes all first to third borns in families with at least four births.  In each regressions the sample is made up of all children aged between 6-18 years from families in the DHS who fulfill these birth order conditions.  Controls in each case are identical to those described in table \ref{TWINtab:IVTwoplus}.  Standard errors are clustered at the level of the mother.$^{*}$p$<$0.1; $^{**}$p$<$0.05; $^{***}$p$<$0.01 
\end{footnotesize}} \\ \bottomrule 
\end{tabular}\end{center}\end{table}\end{landscape}



\end{spacing}
\end{document}
