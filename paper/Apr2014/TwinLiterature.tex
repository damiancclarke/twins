The occurrence of twin births is a commonly used for identification in economic
studies.  Specifically, twin births as a shock to total fertility have been used
to motivate the estimation of the quantity-quality trade-off 
\citep{RosenzweigWolpin1980,Blacketal2005,Angristetal2010,Caceres2006}, the 
effects of childbearing on female labour force participation 
\citep{RosenzweigWolpin1980b,Jacobsenetal1999,AngristEvans1998}, and the effects 
of unwed childbearing on marriage market outcomes, poverty and welfare receipt 
\citep{BronarsGrogger1994}.\footnote{Twins have been widely used in the economic, 
medical, biology and psychology literature in a number of ways.  In this paper we 
focus only on the use of twin births as an instrument for total fertility, and 
not on the so called `twin studies', which base inference on between-twin 
comparisons using maternal fixed effects.}

The methodology employed in all of these studies requires that twinning be (at 
least conditionally) exogenous.  Typically then, the previous literature has made 
one of a series of assumptions.  The earliest twin studies of the Q-Q trade-off 
\citep{RosenzweigWolpin1980,RosenzweigWolpin1980b} pointed out that twinning 
is both a positive function of total parity, and of maternal age.  By focusing
on twins at first birth and conditioning on maternal age, they thus produced 
consistent estimates of the tradeoff under the assumption that beyond maternal 
age and parity, twin births were entirely random.  A more recent wave of studies 
including \citet{Blacketal2005,Caceres2006,Lietal2008,Angristetal2010} take this 
one step further, assuming that twinning is conditionally exogenous after 
controlling for additional family characteristics such as a mother's race and her 
educational attainment.  They thus condition on a mother's age, race, and 
education, as well as on the birth order of the child in first and second stage 
regressions.

In some cases the validity of such assumptions is probed by regressing twinning
on observable family outcomes, or testing for the equality of means of certain 
characteristics between twin and non-twin families. \citet{Blacketal2005}, 
\citet{Lietal2008} and \citet{Sanhueza2009} report joint $F$-tests suggesting that
twinning is not related to parental education in their data samples, while 
\citet{RosenzweigZhang2009} report $t$-tests showing equality of means across twin 
and non-twin groups. However, as is well known and acknowledged in each case, any 
such tests are at best partial evidence in support of instrumental validity.  While 
twins can be shown to be unrelated to observable or measured characteristics, 
similar tests cannot be run for variables which are either unobservable, or not 
recorded in survey data. We return to this point in the following sections.

Finally, recent studies aim to control for the fact that multiple births are 
correlated with fertility treatments.  Typically such a treatment requires either 
focusing on offspring born before the introduction of fertility treatments 
\citep{Caceres2006, Angristetal2010}, or, in the case of sufficiently rich data, 
removing families undergoing fertility treatment from estimation samples 
\citep{Braakman2014}.  Once again, consistent estimation in this case is based
on the assumption that beyond fertility treatment and family controls listed 
above, twin births are as good as random.

Critiques of the twin instrument have focused on parental behaviours in response
to twins, rather than on the likelihood that parental behaviours (or endowments)
may affect the likelihood of twinning.  \citet{RosenzweigZhang2009}  question the
effect that close (or indeed no) birth spacing and an endowment effect---where 
parental behaviours respond to the lower health at birth of twins compared to single 
births\footnote{Using data from the United States, \citet{Almondetal2005} document 
that twins have substantially lower birth weight, lower APGAR scores, higher use of 
assisted ventilation at birth and lower gestion period than singletons.}---has on 
investments in pre-twin siblings.  They demonstrate that if parents behave in such
a manner, bounds for the Q-Q trade-off can be calculated.  This hypothesis is 
tested in \citet{Angristetal2010}, and applied in \citet{FitzsimonsMalde2010}.



%REMOVED.  SHOULD MAKE THIS POINT MORE EXTENSIVELY IN INTRO.
%Previous twin literature has not controlled for maternal health variables which
%may increase the likelihood of live twin births.  There is, however, considerable
%evidence that maternal health and maternal behaviours during pregnancy have a 
%substantial effect on birth and later life outcomes \citep{Almondetal2011,
%BhalotraRawlings2013,Barker1995}.  As we will discuss in the coming sections,

