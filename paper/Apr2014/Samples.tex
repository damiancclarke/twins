We work with a number of estimation samples.  For tests in which twinning is the 
outcome variable of interest, our sample consists of all children who remain 
living in the household with their mother, and hence for whom we have full 
covariates.  For tests in which child death is the outcome, we use the full 
sample of under 18 year-olds, whether or not they remain in the household of 
their mother.  Such a sample is necessary given that children who died prior to 
the date of the survey no longer appear in our household sample.

In all tests of the Q-Q trade-off using twins, we follow existing literature in 
defining birth-order-specific estimation samples conditional on the total number 
of a mother's births. These samples are referred to as the 2+, 3+, 4+ and 5+ 
samples.  These samples are defined $\forall\ n \in \{2,3,4,5\}$ such that they 
include first-born to $n-1$ born children in families with at least $n$ births.%
\footnote{Existing studies such as \citet{Angristetal2010} focus mainly on the 2+ 
and 3+ samples.  Given the higher fertility in the DHS data, we also include 
higher birth-order groups.}  As an example, the 2+ sample consists of first-borns 
in families with at least two births, and the 3+ sample consists of first- and 
second-borns in families with at least 3 births.  Such a sample decision is 
important when estimating the Q-Q trade-off using twinning as an instrument.  
Given that family size is endogenously chosen by parents and rates of twin birth
are not constant by birth-order, twin-births will occur more frequently in 
families that have a higher fertility preference. This point is addressed by 
(among others) \citet{RosenzweigWolpin1980} and \citet{Blacketal2005} who first 
suggested combining $n$+ groups with twinning at birth order $n$ as a way to 
ensure that twin and non-twin families in the sample would have similar fertility 
preferences.

Table \ref{TWINtab:Samples} presents the sample size along with basic descriptives 
for each of the samples.  Full descriptives are provided in the sub-section which 
follows. For each of the 2+ to 5+ samples we only focus on children aged over six 
given that our outcome of interest requires the child to have (at least) reached
primary-school age.

