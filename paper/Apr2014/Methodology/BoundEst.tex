%*******************************************************************************
%Description of methodology of \citet{Altonjietal2005} perhaps?

In the previous subsection, Q-Q estimation using twins is motivated in equations 
(\ref{TWINeqn:secondstage}) and (\ref{TWINeqn:firststage}).  Consistent IV
estimation imposes the (strong) prior belief that twin births can be excluded 
from the second stage equation, or that the sign of $\gamma$ in the following 
is equal to zero:
\begin{equation}
\label{TWINeqn:Conley}
Quality_{ij}=\beta fert_j + \gamma twin_j + u_{ij}.
\end{equation}
As we discuss above, this will not be the case if maternal health controls 
omitted from (\ref{TWINeqn:secondstage}) are correlated with both the 
likelihood of taking twin conceptions to term, and with eventual measures of 
child quality.

However, even in cases such as this where we are not confident that $\gamma=0$,
we can still estimate bounds on the Q-Q tradeoff if we are confident in making
some statement of prior belief about the distribution from which $\gamma$ is 
drawn.  \citet{Conleyetal2012} describe such a process, which they refer to as 
\emph{plausible exogeneity}.  We invoke this terminology here, and refer to 
twins as a plausibly exogenous event, implying that we have reason to believe 
that $\gamma$ may be close to, but not necessarily precisely equal to, zero.

In this paper we estimate $\beta$ under a range of assumptions regarding the
true nature of $\gamma$.  Firstly we estimate $\beta$ by simply assuming a
support assumption for $\gamma$: specifically that $\gamma$ falls between 0 
(implying instrumental validuty) and some (positive) number $\delta$:
\begin{equation}
\label{TWINeqn:uci}
\gamma \in [0,\delta ].
\end{equation}
This is a relatively weak assumption, however, as \citet{Conleyetal2012} show,
it allows for us to recover a `union of confidence intervals' (hereafter UCI) 
for estimates of $\beta$ over the entire support of $\gamma$.  We also estimate 
by imposing a stronger prior: specifically we fully specify the distribution of
$\gamma$ as:
\begin{equation}
\label{TWINeqn:ltz}
\gamma \sim U(0,\delta).
\end{equation}
This stronger assumption allows for a tighter estimate of the bounds on $\beta$.
\citet{Conleyetal2012} provide a full derivation of this result, and we follow
them in referring to this as a local-to-zero (LTZ) approximation.

Assumptions (\ref{TWINeqn:uci}) and (\ref{TWINeqn:ltz}) depend upon the values
of $\delta$ which we believe hold in the case of twinning and the Q-Q equation.
As such, we estimate both specifications over a wide range of values for 
$\delta$, however our preferred values (those which we present in tables in this 
paper) are that $\gamma \in [0,2\hat\gamma]$ and $\gamma \sim U(0,2\hat\gamma)$.
We thus first estimate $\hat\gamma$ by running the structural equation 
(\label{TWINeqn:Conley}), and then plug these estimates back into the 
\citeauthor{Conleyetal2012} plausibly exogenous approach to estimate bounds for
$\beta$.

%*******************************************************************************
