As a result, 2SLS estimation is employed, where fertility is instrumented using
twin births.\footnote{Other instruments and methodologies are also used 
including gender mix of children \citep{Angristetal2010}, policy experiments 
\citep{Qian2009}, and historical time series variation in schooling 
\citep{BleakleyLange2009}}  The corresponding first stage is:
\begin{equation}
\label{TWINeqn:firststage}
fert_{j}=\alpha_0+\alpha_1 twins_{j}+\bm{X}\bm{\alpha}_2+\nu_{j},
\end{equation}
where $twins_j$ is an indicator for whether the $n^{th}$ birth in a family is a 
twin birth. As described in section \ref{TWINsscn:samples}, the sample in each 
case is the so-called $n+$ group, consisting of children born before birth $n$ 
in families with at least $n$ births.  As such the twins themselves are excluded 
from the estimation sample.\footnote{Typically, the argument is made that twins 
are different to single births, and hence should not be compared in analysis}
As an alternative specification we do include children from the $n^{th}$ birth 
(twins in twin families), however generally focus on the typical $n+$ sample as
our principal sample.

Consistent estimation of $\hat\beta_1$ can thus proceed provided (among other
things) that instrumental validity holds:
\begin{equation}
\label{TWINeqn:IVvalid}
P\lim N^{-1}twin_ju_{ij}=0 
\end{equation}
Given the unobservable nature of the error term $u$, we are unable to test 
(\ref{TWINeqn:IVvalid}).  There is however nothing which stops us from 
partially testing (\ref{TWINeqn:IVvalid}) by removing a subset of observable
components from $u$.  The error term $u$ in (\ref{TWINeqn:secondstage}) is a 
function of a large number of elements:
\begin{equation}
\label{TWINeqn:IVbias}
\begin{split}
u=f(\text{maternal health, fertility behaviour, positive pregnancy investments,}  \\
\text{parental education, negative pregnancy investments,}\ldots)
\end{split}
\end{equation}
for now we do not specify the signs on any of these variables.  While many of 
the relevant elements are either completely or partially unobservable to the 
econometrician, some of these variables, such as maternal education and health
can be measured.  Thus a partial test of the twin methodology consists of 
estimating the following regression.
\begin{equation}
\label{TWINeqn:twinreg}
twin_{j}=\gamma_0 + \bm{X}\bm{\gamma}_1 + \bm{S}\bm{\gamma}_2
                  + \bm{H}\bm{\gamma}_3 + \varepsilon_{j}.
\end{equation}
Here $\bm{X}$ refers to the initial vector of family and child controls, $\bm{S}$
to additional family socioeconomic variables such as income and parental 
education, and $\bm{H}$ to maternal health variables.  

If twin birth is indeed an event which is as good as random, the coefficients
on maternal health and family socieoeconomic variables in the above regression
should not be significantly different to zero.  We thus test the following 
hypothesis in our DHS data:
\begin{equation}
\label{TWINeqn:twintest}
H_0: \bm{\gamma}_2 = \bm{\gamma}_3 = 0.
\end{equation}
Rejection of the null would raise difficulties in proceeding with IV estimation 
using the twin instrument.  Of course, if the rejection of the null were only due 
to one or a number of \emph{observable} element(s) which predicted twinning, 
these variables could simply be included in the first and second stages above, 
much like occurs with maternal age and race in the existing literature.  However, 
more generally it would be difficult to be argue for instrumental validity if 
twinning is shown to depend upon (a limited set of) measurable family 
characteristic or choice variables, while many similar variables are not observed.

Given the biological demands placed on a mother pregnant with twins, we may 
expect that healthier mothers, or mothers with more resources to invest in their
pregnancy are more likely to take twin conceptions to term.  Similarly, we may
suspect that mothers more able to invest their pregnancy will also be more able
to invest in their child's human capital after birth.  If this is the case, we
would see that (at the very least) $\bm{\gamma_2}>0$.

If we assume additive separability of the elements in the omitted error term, we
can re-write $u_{ij}$ from (\ref{TWINeqn:firststage}) and (\ref{TWINeqn:IVbias}) 
as:
\[ u_{ij}=u^S_{ij}+u^H_{ij}+u^*_{ij}. \]
Here $u^S_{ij}$ and $u^H_{ij}$ correspond to the identical (observable) elements
included as $\bm{S}$ and $\bm{H}$ in (\ref{TWINeqn:twinreg}), while $u^*_{ij}$
represent the remaining (unobserved) components.  We can thus re-write our IV
estimate for $\beta_1$ as:
\begin{equation}
\label{TWINeqn:betabias}
\hat\beta_1^{IV} = \beta_1 + P\lim N^{-1}twin_ju^S_{ij} + 
P\lim N^{-1}twin_ju^H_{ij} + P\lim N^{-1}twin_ju^*_{ij}
\end{equation}
Typically, this is the coefficient estimated in the existing twin literature 
which assumes that twinning is a conditionally exogenous event.  If, however, the 
likelihood of taking twin conceptions to term increases for healthier and/or 
wealthier mothers, we should include $\bm{S}$ and $\bm{H}$ in the first and 
second stages, giving
\begin{equation}
\label{TWINeqn:betacloser}
\hat\beta_1^{IV,S+H} = \beta_1 + P\lim N^{-1}twin_ju^*_{ij},
\end{equation}
where the superscript $S+H$ signifies that socioeconomic and health variables 
have been included as additional controls.  What's more, if both the likelihood
that a family takes twins to term and a family's investment in child human capital 
are positively correlated with positive health behaviours and other positive 
socioeconomic variables such as parental education, we would expect that:
\[
\hat\beta_1^{IV}>\hat\beta_1^{IV,S}>\hat\beta_1^{IV,S+H}>\beta_1.
\]
It should be noted in the above series of inequalities that even conditional upon
socioeconomic and health variables, IV estimation will \emph{not} result in a
consistent estimate of $\beta_1$ if twinning is correlated with unobservable
elements in $u^*_{ij}$.  We return to this point, and how to bound $\beta_1$ in
the following sub-section.

