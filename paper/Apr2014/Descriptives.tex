Table \ref{TWINtab:sumstats} provides summary statistics of this data. Fertility 
and maternal characteristics are described at the level of the mother, while 
child education and survival are described at the level of the child (the 
education and full samples respectively).  The number of observations at each 
level is provided at the bottom of the table.

Survey countries are classified according to country income level in order to 
allow for a disaggregation of Q-Q results by income group.\footnote{This 
classification is obtained from the World Bank, with DHS surveyed countries 
falling into two broad groups based on their GNI per capita at the moment of the 
DHS survey.  These groups consist of countries classed as low-income economies, 
and countries classed as middle-income economies (either lower-middle or 
upper-middle). Details regarding this classification can be found in Appendix 
Table \ref{TWINtab:countries}.}  We present summary statistics by birth type
(singleton or twin), and by country income status.  Twin births make up 1.85\% 
of all births.  A simple comparison of means suggests that healthy mothers (as 
proxied by height, BMI and probability of being underweight) are more likely to 
give birth to twins, and that twin births are more likely to occur in low-income
countries.  This apparent contradiction can be explained given that twins are
(both mechanically and biologically) a positive function of fertility, and 
fertility is higher in the low-income sample.  Figure \ref{TWINfig:bord} 
describes this positive relationship: while twins account for less than 1\% of 
all first-borns, they account for greater than 4\% of all tenth-born children.
As expected, twin families are larger than non-twin families. Figure 
\ref{TWINfig:births} describes total fertility in twin and non-twin families.  
The distribution of family size in families where at least one twin birth has 
occurred dominates the corresponding distribution for all-singleton families.  
This is expected given imperfect fertility control and---even were fertility 
perfectly controlled by families---given that some twins will occur on a family's 
final desired birth.  Such a result is required for instrumental relevance when
using twining to estimate a Q-Q trade-off.

Child `quality' is measured using each child's educational attainment.  Our
principal outcome variable is a standarised score for schooling (Z-Score). This
Z-Score is calculated by comparing each child's total years of completed 
education to his or her birth cohort in their country of residence.  This allows
us to express all effect-sizes in terms of a one standard deviation increase 
in total educational attainment.

