%********************************************************************************
The results from the previous subsection provide consistent estimates of $\beta$
via 2SLS \emph{if} the full set of controls completely account for those 
characteristics and behaviours which predict giving live birth to twins.  
However, given that we have shown that twinning is predicted by a wide range
of health behaviours, and given that maternal health variables in this dataset
are limited, it seems unlikely that all relevant variables are included in
these specifications.  As such, we turn to \citeauthor{Conleyetal2012}'s
(\citeyear{Conleyetal2012}) methodology to estimate bounds for the Q-Q trade-off.

As described in section \ref{TWINsscn:BoundMeth}, this involves the specification
of some prior over the sign that the twin instrument takes in the structural
equation \ref{TWINeqn:Conley}.  Results are displayed in figures 
\ref{TWINfig:ltz2} and \ref{TWINfig:ltz3}, and in table \ref{TWINtab:Conley}.
At each point on the horizontal axis of figures \ref{TWINfig:ltz2} and 
\ref{TWINfig:ltz3}, the bounds for $\beta$ are displayed, along with the
corresponding point estimate under the assumption that $\gamma$ is distributed
$U(0,\delta)$.  Dashed lines present the 95\% confidence interval, while the
solid line represents the point estimate.

Table \ref{TWINtab:Conley} provides these bounds at a particular point on this
graph.  This point corresponds to an assumption of $\gamma \in [0,2\hat\delta]$
and $\gamma\sim U(0,2\hat\delta)$ for the UCI and the LTZ approaches respectively,
where $\hat\delta$ is estimated from equation \ref{TWINeqn:Conley} with DHS data.
These results allow us to make informative statements about the Q-Q trade-off,
at least for those twins born at third and fourth birth orders who affect more
children the birth.  Further, these results suggest that the magnitude of this
trade-off may be large: and indeed much larger than those estimated in existing 
studies as the twin instrument departs from true exogeneity.  For example, the
effect of an additional child at birth order three on first- and second-born
children (ie the 3+ group) has an effect between -1.5\% to as much as -15\% of
a standard deviation in standardised educational attainment if the prior 
(based on DHS estimates) is true.  While it is not the case that informative
statements can always be made given the wide bound on estimates in some cases
(particularly the five plus group), it is important to point out that in each
case as the twin-exogeneity assumption is loosened, the point estimate of the
Q-Q tradeoff becomes more (and considerably) negative.
