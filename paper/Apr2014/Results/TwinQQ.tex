%********************************************************************************
As is typically found in empirical studies of the Q-Q tradeoff, correlations 
between family size and child outcome variables are negative, and strongly 
significant. Table \ref{TWINtab:OLS} shows OLS estimates of child `quality' on
total fertility. These results suggest that an additional sibling is associated
with an approximatelty 0.1 sd decrease in standardised schooling outcomes.

Of course, this empirically observed relationship between the quantity of an 
individual's siblings and their measured `quality' does not necessarily imply 
that such a trade-off exists if parental decisions regarding the production of 
child quality and quantity are jointly made and possibly influenced by unobserved
factors \citep{BeckerTomes1976}.  Principally here we are concerned with 
unobserved parental behaviours which may favour both lower family size and higher 
child quality.  The OLS results are consistent with such a result, as the 
inclusion of maternal education and maternal health controls---likely correlated 
with desires for smaller family size and higher investments per child---reduce the 
magnitude of this observed trade-off.

%********************************************************************************
\subsubsection{Q-Q Tradeoff: Estimates Using Twin Births}
Rather than focusing on OLS estimates which are likely to be biased, we turn 
to estimates which rely on twin births to identify the Q-Q tradeoff.  As we 
outline in section \ref{TWINsscn:twinRes}, the assumption of `as good as random' 
twin births is unlikely to hold, even when conditioning on the augmented set of 
controls proposed in (\ref{TWINeqn:twinreg}).  If this is the case, we will 
also be unable to consistently estimate $\beta_1$ using twin births.

However, it is likely that the $\beta_1$ that we estimate using twin births will 
provide us with a strict lower bound of the magnitude of the Q-Q trade-off as
outlined in (\ref{TWINsscn:methodQQ}). We expect that the bias in this estimate 
is due to those mothers who invest more in their children in utero, or who have
greater initial health endowments, being more likely to give birth to twins 
resulting in larger family sizes.  At the same time, we expect healthier mothers 
to invest more in their children after birth resulting in higher quality children.
By relegating health variables to the error term, these two positive correlations 
will result in a positive bias on the fertility coefficient estimated via IV.  In 
order to determine the effect that these omitted variables have on estimates of 
the Q-Q trade-off, we turn to results for equation (\ref{TWINeqn:secondstage}), 
both first omitting, and the including, maternal health and socioeconomic 
variables.

We present IV estimates of the Q-Q trade-off in table \ref{TWINtab:IVAll}.  The 
main specification is displayed in the top row with separate columns for the 2+, 
3+ and 4+ sample groups.  In each case the base case (controlling for maternal 
and child age, country, and year of birth) results in insignificant, and at times 
weakly positive, estimates of the effect of an additional birth on a child's
educational attainment.  These results suggest that the inclusion of maternal 
health and socioeconomic controls may be of considerable importance.  Despite
the lack of results when using the `typical' set of twin controls, including 
health (columns 2, 4 and 6) reduces point estimates on fertility from an effect
of approximately 0\% of a standard deviation, to $-3$ or $-4$\% of a standard 
deviation in standardised educational attainment.  Further, conditioning on
maternal education results in slightly more precise estimates, suggesting a
statistically significant\footnote{Or close to statistically significant in the 
case of the 2+ sample.} Q-Q trade-off of at least 3 or 4\%.

%********************************************************************************
\subsubsection{Heterogeneity of the Q-Q Trade-off}
Estimates of the Q-Q trade-off are heterogeneous across birth orders, country 
income level, and gender of the child affected by the additional birth.  Such a 
result is not surprising if parents perceive that returns to education vary for 
different children or in different circumstances, and invest in human capital in 
line with this.  

The magnitude and significance of the results is lowest when considering the 
effect on the first-born child of moving from two to three births (the 2+ group), 
and higher when considering moving from three to four births or four to five 
births.  However, in lower fertility environments the effect is, as expected, 
concentrated on lower birth orders. The third row of table \ref{TWINtab:IVAll} 
suggests that in middle-income countries that the effect is largest on first 
borns, and progressively smaller, but still considerable, at higher birth orders.

Estimates of the magnitude of the Q-Q trade-off by country income level suggest
that the trade-off is considerably larger in \emph{middle} rather than low-income
countries.  In low-income countries point estimates on fertility suggest 
(insignificant) trade-offs centred around 2-3\% of a standard deviation, while 
in middle-income countries results are significant, and considerably larger,
reaching as much as 9\% of a standard deviation: only slightly lower than OLS
estimates for this group. [ADD TEST HERE LOOKING AT COSTS OF EDUCATION?  HIGHER
IN COUNTRIES WHERE EDUCATION COSTS MORE? SONIA, DO YOU HAVE THE EDUCATION COST 
DATABASE THAT I THINK YOU MENTIONED TO ME ONCE?]

Similarly, effects of the Q-Q trade-off are considerably different depending
upon a child's gender.  In table \ref{TWINtab:IVgend} we present specifications 
estimated separately by the gender of the index child.  These results suggest
that females may bear the brunt of additional births, with estimates being 
negative and significant for girls, while insignificant for boy children.  Such 
a result suggests that parents may engage in redistributory behaviour.  This 
idea has been extensively studied by \citet{RosenzweigZhang2009}, however their 
analysis focuses on the possibility that parents reinforce positive birth 
endowments based on child health, not based on child gender as we find here.

In order for the results we produce to be comparable to recent literature we
thus far have only focused on children \emph{preceding} twins.  As an 
alternative specification we also include the $n^{th}$ birth in the analysis
sample.\footnote{In the case of of the 2+ group we thus compare first and 
second born children from families which did not have a twin at the second
birth to first and second born children from families which did have a twin
at the second birth.}  Given the lower birth endowments of twins (and potential
that reinforcing behaviour moves resources away from twins as per 
\citet{RosenzweigZhang2009}), we expect to see that the trade-off is 
significantly stronger when including twins in the sample.  These results are 
displayed in the final row table \ref{TWINtab:IVAll}, and suggest that, if we 
focus not just on pre-twins, that the trade-off of an additional birth is large: 
on the order of 5-8\% of a standard deviation. Finally, in row 4 of the table
we perform an alternative consistency check.  Rather than using the full 
fertility reported by a family, we adjust birth order and fertility to only 
account for those children who survived beyond 1 year of age.  Once again, we 
find that the estimated Q-Q trade-off is larger, and of similar magnitude to 
the results when including twin births.  However, given that child survival 
could itself be thought of as a `quality' variable, it is not clear that this
result is representative of the entire population of children.  Like the results
based on twins and pre-twins, this result, while interesting, is less
conservative than our main specification.
