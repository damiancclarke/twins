%********************************************************************************
As is typically found in empirical studies of the Q-Q tradeoff, correlations 
between family size and child outcome variables are negative, and strongly 
significant. Table \ref{TWINtab:OLS} shows OLS estimates of child `quality' on
total fertility. These results suggest that an additional sibling is associated
with an approximatelty 0.1 sd decrease in standardised schooling outcomes.


Of course, this empirically observed relationship between an individual's sibship 
size and their measured `quality' does not necessarily imply that such a 
trade-off exists if parental decisions regarding the production of child quality 
and quantity are jointly made and possibly influenced by unobserved factors 
\citep{BeckerTomes1976}.  Principally here we are concerned with unobserved 
parental behaviours which may favour both lower family size and higher child 
quality. \citet{Qian2009} suggests that such a mechanism will exist where parents 
who value education more highly also decide to have less children.  The OLS 
results are consistent with such a result, as the inclusion of maternal education 
and maternal health controls---likely correlated with desires for smaller 
family size and higher investments per child---reduce the magnitude of this 
observed tradeoff.

%********************************************************************************
\subsubsection{Q-Q Tradeoff: Estimates Using Twin Births}
Rather than focusing on OLS estimates which are likely to be biased, we turn 
to estimates which rely on twin births to identify the Q-Q tradeoff.  As we 
outline in section \ref{TWINsscn:twinRes}, the assumption of `as good as random' 
twin births is unlikely to hold, even when conditioning on the augmented set of 
controls proposed in (\ref{TWINeqn:twinreg}).  If this is the case, we will be 
unable to consistently estimate $\beta$ using twin births.  

However, it is likely that the $\beta$ that we estimate using twin births will 
provide us with a strict lower bound of the magnitude of the Q-Q trade-off. 
Given that we expect that the bias in this estimate is due to those mothers who 
invest more in their children in utero (or with greater initial health endowments) 
being more likely to give birth to twins (and hence having larger family sizes), 
and at the same time we expect healthier mothers to invest more in their children
after birth resulting in higher quality children, then relegating health 
variables to the error term will result in a positive bias on the fertility 
coefficient.  In order to determine the effect that these omitted variables have 
on estimates of the Q-Q trade-off, we turn to results for equation
(\ref{TWINeqn:secondstage}), both omitting and not omitting maternal health and
socioeconomic variables.

The main specification is presented in row one of tables 
\ref{TWINtab:IVTwoplus}-\ref{TWINtab:IVFourplus}.  These results suggest that
the inclusion of maternal health and socioeconomic controls may be of 
considerable importance.  In each case (for 2+, 3+ and 4+ groups), the base
case which includes no twin predictor variables as controls results in an 
insignificant (and at times weakly positive) coefficient on fertility.  However,
once including the full set of health and socioeconomic variables, this 
coefficient becomes significantly more negative, resulting (at least in the
3+ and 4+ case) in point estimates of around $-3$ to $-4$\% of a standard deviation
in standardised educational attainment.  Further, these results hold even when
controlling for birth order effects, as can be seen in row 2 of the table.

These results are heterogeneous across birth orders, country income level, and
gender of the affected child (discussion of these latter two groups will be
delayed until the following subsection.)  The magnitude and significance of the 
results is lowest when considering the effect on the first-born child of moving 
from two to three births (the 2+ group), and higher when considering moving from
three to four births.  \texttt{comment re budget constraint???}

Turning to our second innovation, we are able to examine the effect of an
additional birth which takes families above their desired fertility threshold
by interacting the twin instrument with a desired-fertility variable. [EXPLAIN
AND INTERPRET DESIRED FERTILITY RESULTS FROM TABLES 5-7 (row labelled Desired-%
Threshold).]   

