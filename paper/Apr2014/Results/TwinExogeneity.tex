%*******************************************************************************
Table \ref{TWINtab:comp} presents a test of balance of maternal characteristics
between mothers who have and who have not given birth to twins.  This table
presents a comparison of means, and a two-tailed $t$-test of the null that 
mothers of twins are not statistically different from mothers of non-twins.
These comparisons suggest that twin mothers are older at the moment of birth,
begin child-bearing earlier, have more children, and are more (less) likely to
come from top (bottom) wealth quintiles.  These comparisons of means, however,
are not surprising given the well established fact that twins occur with higher
frequency as mothers age, and at higher parities \citep{Hall2003}. Given that
mother's who have more births (and hence who are more likely to have twins) are
different in other dimensions, these tests may capture the effect of selection
into higher fertility, rather than twinning itself.  It is, however, worth 
pointing out that table \ref{TWINtab:comp} suggests that healthier mothers as
proxied by height and body mass index (BMI) are unconditionally more likely to 
have twins.  Similar comparisons of all births suggest that healthier mothers
on average have less births.

In order to examine partial correlations, we regress a child's twin status
(1 if a twin, 0 otherwise) on their mother's health, education, and a range
of other demographic and family characteristics.\footnote{In our principal
specification, the full set of controls are country, child year of birth, and
age dummies; a cubic function of mother's age at time of birth; mother's age
at time of first birth; mother's education and education squared; and mother's
height and BMI.  We cluster standard errors at the level of the mother.}  In 
table \ref{TWINtab:twinreg1} we report the results from specification 
(\ref{TWINeqn:twinreg}). These results suggest that twin births are not random, 
even after conditioning on maternal age and child birth order as is typical in 
recent twin literature. The inclusion of a full set of country and year-of-birth 
dummies (not displayed in table \ref{TWINtab:twinreg1}) will capture any 
systematic trend in the frequency of twin births across time or regions, and 
country dummies will absorb all time invariant differences in the probability 
of a twin birth across countries.  The estimated coefficients and signs support 
the idea discussed in section \ref{TWINscn:method} that higher `investments' 
(for example in maternal health) required to maintain multiple healthy fetuses 
\emph{in utero} may result in non-random twin births. Initially results from the 
pooled DHS data are presented as this provides a particularly large sample with 
which to test the hypothesis of twin exogeneity.  This is represented in table 
\ref{TWINtab:twinreg1} column (1) and provides considerable evidence that live 
multiple births respond to family `choice' variables such as education (tests 
for the joint significance of both socioeconomic variables and health variables 
are rejected with p-values of 0.0000).

These results are robust to all alternative specifications examined. Significant
and quantitatively similar results are found if a logit model is estimated 
rather than a linear probability model, and when running separate models for 
twinning at each birth order.  Similarly, if we run the regression at the level
of the mother or include any combination of fertility measures, similar 
patterns are observed.

The fact that maternal health is correlated with twinning is supported by 
medical literature, although is not a point which has been incorporated into 
prior economic studies of twinning.  \citet{Hall2003} for example suggests that 
follicle-stimulating hormone (FSH) is associated with an increased likelihood 
of twinning, and is found in higher concentrations in older, heavier and taller 
mothers.  Further, she suggests ``that adequate maternal folic acid consumption 
could affect the number of twins coming to term'' (see p.\ 741, and further 
discussion in \citet{Lietal2003}).  Given that twinning also increases in cases 
where the mother undergoes fertility treatment, we run a similar regression for 
children born in a period not potentially affected by IVF.\footnote{In order to 
be conservative, we estimate for the period preceeding 1990, the date which 
coincides with the first reported successful use of IVF in South Africa, an 
early-adopter among DHS countries.}  These results are included in columns (4) 
and (5).  Pre- and post-1990 results are qualitatively similar although education 
is no longer significant prior to 1990 (in the smaller sample). Mother's height 
and weight, and family socioeconomic variables remain economically and 
statistically significant.

If the reason non-random twin births are observed is due to insufficient investment 
in the developing fetus, it seems likely that twin `selection' will be more 
pronounced in lower income settings, and settings where the mother is less well 
resourced during gestation.  This is tested in columns (2) and (3), where it is 
shown that the violation of the twin exclusion restriction is particularly strong 
in low income countries.  Here maternal health is a more important predictor, and 
the explained portion of this set of variables is larger that in middle-income 
countries.\footnote{The low $R^2$ in these regressions is not at all surprising 
given that twin conception can be thought of as an approximately random process.
The fact that socioeconomic and health variables have \emph{any} power in 
explaining twin birth however is sufficient to invalidate IV estimations if these 
or other relevant predictors are not controlled for.} Finally, we add measures for 
the availability of prenatal care (for the subset of children for whom this variable 
is recorded).  The omitted case are those who report prenatal care from relatives
or traditional health care providers.  We find evidence to suggest that those 
receiving prenatal care from a doctor are more likely to give live birth to twins,
and those who receive no prenatal care are less likely to twin.  We treat these
results with precaution however, given that accessing prenatal care is a choice
variable, and likely endogenous.

These results call into question the veracity of the conditional exogeneity (or 
`as good as random') assumption regarding twin births required to estimate 
$\beta_1$ consistently in (\ref{TWINeqn:secondstage}).  This implies that omitting 
factors such as family socioeconomic and maternal health variables is likely to 
result in inconsistent IV estimates if these variables are also correlated with 
child `quality' outcomes.  It seems unlikely, furthermore, that controlling for 
the factors that we have shown to predict twinning will allow us to recover 
consistent IV estimates.  Maternal health stocks are both difficult to measure, 
and multi-dimensional in nature.\footnote{In ongoing work, we find that the 
probability of twinning falls with maternal alcohol consumption, smoking, drug 
taking and a range of other behaviours in data from the USA, Chile, UK and 
Scotland.}  In the following subsections we examine the effect that omitting these
factors is likely to have on IV estimates of the Q-Q trade-off.
