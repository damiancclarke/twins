%*******************************************************************************
Table \ref{TWINtab:comp} presents a test of balance of maternal characteristics
between mothers who have and who have not given birth to twins.

In table \ref{TWINtab:twinreg1} we report the results from specification 
(\ref{TWINeqn:twinreg}). These results suggest that twin births are 
not random, even after conditioning on maternal age and child birth order as is
typical in recent twin literature. The inclusion of a full set of country and 
year-of-birth dummies (not displayed in table \ref{TWINtab:twinreg1}) will 
capture any systematic trend in the frequency of twin births across time or 
regions, and country dummies will absorb all time invariant differences in the 
probability of a twin birth across countries.  The estimated coefficients and 
signs support the idea discussed in section \ref{TWINscn:method} that higher 
`investments' (for example in maternal health) required to maintain multiple 
healthy fetuses \emph{in utero} may result in non-random twin births. Initially 
results from the pooled DHS data are presented as this provides a particularly 
large sample with which to test the hypothesis of twin exogeneity.  This is 
represented in table \ref{TWINtab:twinreg1} column (1) and provides considerable 
evidence that live multiple births respond to family `choice' variables such as 
education (tests for the joint significance of both socioeconomic variables and 
health variables are rejected with p-values of 0.0000).


The fact that maternal health is correlated with twinning is supported by medical 
literature, although is not a point that has been incorporated into prior 
economic studies of twinning.  \citet{Hall2003} for example suggests that 
follicle-stimulating hormone (FSH) is associated with an increased likelihood 
of twinning, and is found in higher concentrations in older, heavier and taller 
mothers.  Further, she suggests ``that adequate maternal folic acid consumption 
could affect the number of twins coming to term'' (see p.\ 741, and further 
discussion in \citet{Lietal2003}).  Given that twinning also increases in cases 
where the mother undergoes fertility treatment, we run a similar regression for 
children born in a period not potentially affected by IVF.\footnote{In order to 
be conservative, we estimate for the period preceeding 1990, the date which 
coincides with the first reported successful use of IVF in South Africa, an 
early-adopter among DHS countries.}  These results are included in columns (4) 
and (5), and although education is now no longer always significant, mother's 
height and weight, and family socioeconomic variables remain economically and 
statistically significant.

If the reason non-random twin births are observed is due to insufficient investment 
in the developing fetus, it seems likely that twin `selection' will be more 
pronounced in lower income settings, and settings where the mother is less well 
resourced during gestation.  This is tested in columns (2) and (3), where it is 
shown that the violation of the twin exclusion restriction is particularly strong 
in low income countries.  Here maternal health is a more important predictor, and 
the explained portion of this set of variables is larger that in middle-income 
countries.\footnote{The low $R^2$ in these regressions is not at all surprising 
given that twin conception can be thought of as an approximately random process.
The fact that socioeconomic and health variables have \emph{any} power in 
explaining twin birth however is sufficient to invalidate IV estimations if these 
or other relevant predictors are not controlled for.}

These results call into question the veracity of the conditional exogeneity (or 
`as good as random') assumption required to estimate $\beta$ consistently in 
(\ref{TWINeqn:secondstage}).  This implies that omitting factors such as family 
income...
