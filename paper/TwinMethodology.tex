\documentclass{article}[11pt,subeqn]

%\usepackage{fancyhdr}
%\pagestyle{fancy}
\usepackage{color}
\usepackage[pdftex]{graphicx}
\usepackage{setspace}
\usepackage{framed}
\usepackage{lastpage}
\usepackage{amsmath}
\usepackage[utf8]{inputenc}
\title{Child Quantity versus Quality: Are Twin Births Exogenous?}
\author{Sonia Bhalotra\thanks{University of Bristol, \href{mailto:s.bhalotra@bristol.ac.uk}{s.bhalotra@bristol.ac.uk}.} \and Damian Clarke\thanks{The University of Oxford, \href{mailto:damian.clarke@economics.ox.ac.uk}{damian.clarke@economics.ox.ac.uk}.}}

\setlength\topmargin{-0.575in}
%\setlength\headheight{15pt}
%\setlength\headwidth{6.05in}
\setlength\textheight{9.2in}
\setlength\textwidth{6.2in}
\setlength\oddsidemargin{0.18in}
\setlength\evensidemargin{-0.5in}
\setlength\parindent{0.25in}
\setlength\parskip{0.25in}

\usepackage{natbib}
\bibliographystyle{abbrvnat}
\bibpunct{(}{)}{;}{a}{,}{,}

\usepackage{hyperref}

\usepackage{lscape}
\usepackage{rotating}
\usepackage{multirow}
\usepackage{rotating,capt-of}
\usepackage{array}

%\usepackage{lineno}
\usepackage[update,prepend]{epstopdf}

\usepackage[font=sc]{caption}

%NEW COMMANDS
\newcommand{\Lagr}{\mathcal{L}}
\newcommand{\vect}[1]{\mathbf{#1}}
\newcolumntype{P}[1]{>{\raggedright}p{#1\linewidth}}

\usepackage{appendix}
\usepackage{booktabs}
\usepackage{cleveref}

%\fancyhead{}
%\fancyfoot{}
%\fancyhead[L]{\textsc{Extended Essay}}
%\fancyhead[C]{\textsc{MSc. Economics for Development}}
%\fancyhead[R]{\textsc{Damian C. Clarke}}
%\fancyfoot[C]{\textsc{\thepage\ of  \pageref{LastPage}}}
%\fancyfoot[R]{\texttt{ \Large DRAFT}}



\begin{document}
%\linenumbers
\begin{spacing}{1.25}
 
\section{Twin Methodology}
\label{scn:EF}

Before turning to tests of the Q-Q model, we are interested in testing the (conditional) exogeneity of twin births.  Although birth exogeneity is 
inherently untestable as there are many unobserved factors which could interact with birth outcomes, it is possible to test whether the probability of twin birth depends upon observable 
characteristics of the mother and the family environment.  Whilst the dependence of twinning on observable characteristics offers no conclusive evidence 
that it will also depend upon unobservables, it will cast doubt on the validity of the twin instrumentation strategy---particularly when the vector of 
observable characteristics is small.  If, for example, twin birth depends upon observed measures of maternal health such as height 
and weight, it also seems reasonable to suggest that unmeasured or unobserved health measures such as pre-natal medical visits and a mother's behaviour 
and diet are also related to the likelihood of live twin births. 

Using a linear probability model\footnote{We also test this with a probit model, however the implications of each model are identical for our results.} 
the following twin birth equation is estimated: 
\begin{equation}
\label{eqn:twinpred}
P(Twin_{ijk}=1)=\delta_1 B_i + W'_j\vect{\delta_\vect{W}}+ T'_{i}\vect{\delta_\vect{T}} +  \phi_k + \varepsilon_{ijk}.
\end{equation}
Here $Twin_{ijk}$ takes the value one if child $i$ from family $j$ and country $k$ is a twin, and zero otherwise.  A full set of year of birth ($T$) and country
dummies ($\phi$) are included.  The observable parameters of interest $W'_j$ are family level variables representing characteristics such as maternal age and
education, maternal height and BMI, and a variable representing the family's socioeconomic quintile (based on assets).  We are interested in testing the
hypothesis that all $\delta_\vect{W}=0$, as rejection of this would provide suggestive evidence of the endogeneity of multiple births in the sense discussed
above.

After examining the validity of the twin instrument, we test the Q-Q model in a number of ways: firstly,  we use the instrumental strategy followed by the majority of 
studies discussed in the previous section, and then using a variation of the technique initially discussed by \citet{RosenzweigZhang2009}. Most recent empirical studies 
examining the Q-Q tradeoff follow \cite{Angristetal2010} in estimating the following two-stage strategy:
\begin{subequations}
\label{eqn:2SLS}
\begin{eqnarray}
\label{eqn:QQ}
Q_{ij}&=&\beta_0+\beta_1\widehat{Fert}_j+\beta_2B_{i}+X'_{ij}\vect{\beta_{\vect{X}}}+W'_j\vect{\beta_\vect{W}}+u_{ij} \label{eqn:2SLSa}\\
Fert_{j}&=&\gamma_0+\gamma_1Twin_{ij}+\gamma_2B_{i}+X'_{ij}\vect{\gamma_\vect{X}}+W'_j\vect{\gamma_\vect{W}}+v_{ij}. \label{eqn:2SLSb}
 \end{eqnarray}
\end{subequations}
Here $Fert_j$ refers to the number of children in family $j$, $B_i$ birth order of child $i$, $X_{ij}$ a vector of individual characteristics, and $W_j$ 
a vector of family characteristics such as income and parental education.  Along with instrumental relevance, this strategy requires that two separate 
assumptions hold regarding the error term $u_{ij}$ for estimates of $\beta_1$ to be unbiased.  Firstly, $u_{ij}$ must be uncorrelated with factors which
predict twin births and which are contained in $v_{ij}$.  This is the typical validity assumption of an IV strategy, and is indirectly examined by 
(\ref{eqn:twinpred}) above.  If we believe that this is \emph{not} the case, rather than estimating the true tradeoff $\beta_1$, we will estimate:
\[
 \hat{\beta}_1=\beta_1+(Z'X)^{-1}Z'u
\]
where the second part of this term refers to the typical omitted variable bias, and the direction of the bias depends upon the correlation between
twins and fertility (which we expect and observe to be positive), and the unobserved correlation between twinning and unobservables.  If, for example,
we believe that we are not capturing the full health behaviour of a mother, and that healthier mothers are more likely to have twins, we will then
have that our estimated $\hat{\beta}_1$ overstates the true Q-Q tradeoff.

However, \emph{even} in the case where the instrumental validity assumption holds, $\beta_1$ will not provide an unbiased estimate of the Q-Q
tradeoff if parents make reinforcing investments in higher endowment single-birth children and shift resources due to the close birth-spacing of twins.
For this reason we estimate the following equation (where our notation is similar to that of \citet{RosenzweigZhang2009}):
\begin{equation}
 \label{eqn:RZ}
 Q_{ij}=\eta_0TF_j+\eta_1(TF_j\times pretwin_{ij})+\eta_2pretwin_{ij}+X'_{ij}\vect{\eta_\vect{X}}+W'_j\vect{\eta_\vect{W}}+\varepsilon_{ij}.
\end{equation}
This specification examines educational outcome variables for children who are born in twin families ($TF_j$) and non-twin families, and separates by
twins and non-twin births ($pretwin_{ij}$).  Effectively, we compare children born in those families where a twin is born on the last birth with children
in similar families where a singleton birth occurs.  Given that we are comparing families with twins born on the last birth to those with singleton births,
each family has $n$ birth occurrences, however twin families have $n+1$ children compared to the $n$ children in singleton families due to twinning on the 
$n^{th}$ birth.  As we discuss in section 
\ref{scn:lit}, we are interested in separately identifying the effect of an extra birth on both twins and on their siblings, given that these estimates
will give us estimates of the bounds of the Q-Q tradeoff under the assumption that twin births are exogenous.  These estimates are provided by $\eta_0$
and $\eta_0+\eta_1$ respectively.

Explain twin groups.  Show how RZ biased.

\textcolor{blue}{Angrist et al may be worth reporting as it is current state of literature.  We reiterate how it is wrong in two ways.}


\bibliography{BiBBase1} 
\end{spacing}
\end{document}



