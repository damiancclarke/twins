\documentclass{article}[11pt,subeqn]

\usepackage{fancyhdr}
\pagestyle{fancy}
\usepackage[pdftex]{graphicx}
\usepackage{setspace}
\usepackage{framed}
\usepackage{lastpage}
\usepackage{amsmath}
\usepackage[utf8]{inputenc}
\title{Child Quantity versus Quality: Are Twin Births Exogenous?}
\author{Damian C. Clarke}

\setlength\topmargin{-0.375in}
\setlength\headheight{20pt}
\setlength\headwidth{6.5in}
\setlength\textheight{8.7in}
\setlength\textwidth{6.5in}
\setlength\oddsidemargin{0in}
\setlength\evensidemargin{-0.5in}
\setlength\parindent{0.25in}
\setlength\parskip{0.25in}

\usepackage{natbib}
\bibliographystyle{abbrvnat}
\bibpunct{(}{)}{;}{a}{,}{,}

\usepackage{lscape}
\usepackage{rotating}
\usepackage{multirow}
\usepackage{rotating,capt-of}
\usepackage{array}

\usepackage{lineno}

%NEW COMMANDS
\newcommand{\Lagr}{\mathcal{L}}
\newcommand{\vect}[1]{\mathbf{#1}}
\newcolumntype{P}[1]{>{\raggedright}p{#1\linewidth}}

\usepackage{appendix}
\usepackage{booktabs}
\usepackage{cleveref}

\fancyhead{}
\fancyfoot{}
\fancyhead[L]{Extended Essay}
\fancyhead[C]{MSc. Economics for Development}
\fancyhead[R]{Damian C. Clarke}
\fancyfoot[C]{\thepage\ of  \pageref{LastPage}}
\fancyfoot[R]{DRAFT}

\begin{document}
\linenumbers
\begin{spacing}{1.25}

\maketitle
\begin{abstract}
The justification of family planning policies in low income countries may draw some theoretical support 
if the production of child quality by a household is a function of the number siblings also competing
for investment.  This hypothesis has been tested using twins as an exogenous shock to family size.  Here, I test
the exogeneity of twins to family size, and derive the bias that may occur if twinning is actually
an endogenous event.  
\end{abstract}

\section{Introduction}
\citet{Malthus1798} famously introduces a theory in which population growth enters into competition with economic   
well-being. Contemporary microeconomic models have described the existence of similar mechanisms at a household level, albeit with less disastrous consequences.  Early work by \citet{BeckerLewis1973} and \citet{BeckerTomes1976} propose that a household's ability to produce child `quality' is a function of the quantity of children.  The evidence in favour of this quantity-quality (Q-Q) relationship is significant.  Surveys at both a macro- and microeconomic level have documented the existence of a negative correlation between achievement measures and family size (see for example \cite{Desai1995}, or \cite{Hanushek1992}).

TAKE OUT THESE TWO PARAGRAPHS (THEY ADD LITTLE AND AT BEST SHOULD BE IN LIT REVIEW).  REPLACE WITH A DISCUSSION OF WHY Q-Q TRADEOFF MAY EXIST AND WHY THIS IS PARTICULARLY SO IN THE DEVELOPED WORLD.
At the household level, the Q-Q model is supported by the frequently observed empirical relationship between an individual's sibship size and a
range of measures of their `quality' such as completed years of schooling, mortality, or labour force participation.
The estimation of the causal relationship between quality and quantity is less clear. 

Recent empirical results have called into question the validity of the Q-Q trade-off hypothesis.  The majority of
these studies however have been based upon data in developed countries, where often educational investments
required by parents are highly subsidised by the State.  It seems important to test this hypothesis in developing
countries, where the production function for child quality may be substantially different, requiring more intensive
investments by the family.  A number of recent papers have used Chinese data and found results
more in line with the Q-Q model.  Here, I extend this analysis to a wider range of developing countries, providing
a novel examination relevant to populations facing a stricter (average) household budget constraint.
\vspace{-5mm}
\section{Literature}
\label{scn:lit}
\vspace{-5mm}
The use of twin births to address the problem of endogeneity in the Q-Q model seems to have been initially proposed by \citet{RosenzweigWolpin1980}.  They derive the theoretical requirements to estimate the size of the trade-off when the shadow price of child quality depends on the number of children and vice versa.  By relying on the assumption that multiple births are an exogenous shock to family size (once accounting for the total number of a mother's pregnancies) they estimate the effect of a twin birth upon the educational attainment of children in the twins' family.  This and alternate empirical specifications discussed in this section are decribed in table \ref{tab:litrev} in appendix \ref{scn:litrev}.  

Subsequent papers employing a twin-birth methodology have proposed a number of strategies which enable them to obtain consistent estimates of the Q-Q trade-off while relaxing Rosenzweig and Wolpin's exogeneity assumption.  \citet{Blacketal2005} extend the controls to account for the fact that the probability of multiple birth increases with maternal age as described by \citet{Jacobsenetal1999} and others.  They include a set of parental age and education controls, however note that they are unable to reject the hypothesis that parental education has no effect on the probability of multiple birth.  Likewise, \citet{Caceres2006} includes controls for mother's age, race, and education, suggesting that the use of these and pre-1980 US Census data should be sufficient to approximate conditional exogeneity\footnote{The use of pre-1980 Census data seems important as this predates the widespread introduction of fertility drugs.  The use of fertility drugs is associated with higher a probability of multiple births, and resulting concerns that the orthogonality assumption will be violated if users of fertility treatment are non-random.}.  Finally, \citet{Angristetal2010} recognise that twin birth varies with maternal age at birth and race, including twinning as one of three instruments. 

Angrist et al.\ join recent work from \citet{RosenzweigZhang2009} in questioning the validity of twin instrumentation in another sense.  These authors suggest that the error term in the Q-Q equation is unlikely to be orthogonal to the instrument given that twinning imposes predictable and unobserved family responses in investment decisions.  Particularly, these studies question the effect that close birth spacing and an endowment effect responding to the lower health at birth of twins compared to single births\footnote{Using data from the United States, \citet{Almondetal2005} document that twins have substantially lower birth weight, lower APGAR scores, higher use of assisted ventilation at birth and lower gestion period than singletons.} has on investments in pre-twin siblings.  Despite this critique, Rosenzweig and Zhang suggest that it is still possible to compute an upper and a lower bound of the Q-Q trade-off.

More recently, instrumentation using twin births has been applied to estimate a Q-Q model in the developing world. \cite{FitzsimonsMalde2010}, \citet{Sanhueza2009} and \citet{Lietal2008} have applied a similar methodology to that of Angrist et al., examining twin births in Mexico, Chile and China respectively.  These studies find mixed results depending upon the country under examination and once again, while considering the invalidity of the twin exclusion restriction in the Q-Q model in terms of maternal education, do not examine this in terms of non-random twin births due to maternal health. 














%A number of instruments have been proposed to isolate the effect of family size on child quality.  These aim to identify a valid exclusion restriction which acts as a %shock to child quantity, without otherwise affecting parental preferences or other unobservables.  \citet{ConleyGlauber2006} propose that gender mix acts as an %appropriate instrument, with parents more likely to stop having children if the first two births are of opposing genders.  \citet{Lee2008} also proposes a %gender-%based instrument, using the son preference in South Korean households as an indicator of the likelihood of subsequent births, while \citet{Qian2009} takes %advantage of a relaxation of family planning policies in China. 















%Angrist Lavy Schlosser: Because twins probably differ from nontwins for reasons both observed and unobserved, we prefer empirical strategies that look at the effects %of twins on older siblings.
%On balance, the results reported here offer little evidence for an effect of family size on schooling, work, or earnings, although we do find some effects on girls' marital %status, age at marriage, and fertility
%787-788 worthwhile discussion of compliers etc.
%"Because twinning is as good as randomly assigned, causal effects for the latter population [the treated] are the same as causal effects on all compliers."
%RE INSTRUMENTAL VALIDITY AND EXCLUSION RESTRICTION:
%We also replicated the common finding that twin births are associated with older maternal age. For example, the mothers of first and second borns who had twins at
%second or third birth were .3–.5 years older at first birth than those who had singletons. Twinning is not otherwise associated with subject demographics, with one %exception: in the 1995 sample of 2+ subjects, twin rates are higher for younger cohorts.
%OUTCOMES
%The outcome variables measure human capital, economic wellbeing, and social circumstances. In particular, we look at measures of subjects’ educational attainment %(highest grade completed and indicators of high school completion and college attendance), labor market status (indicators of work last year and hours worked last %week), earnings (monthly earnings and the natural log of earnings for full-time workers), marital status (indicators of being married at census day and married by age %21), and fertility.
%INTUITION
%2SLS estimates of this equation capture siblings' weighted average response to the birth of an additional child for those whose parents were induced to have an %additional child by the instrument at hand.


%BlackDevereuxSalvanes
%Q-Q MODEL
%A key element of the quantityquality model is an interaction between quantity and quality in the budget constraint that leads to rising marginal costs of quality
%with respect to family size; this generates a trade-off between quality and quantity.
%Casual evidence suggests that children from larger families have lower average education levels. However, is it true that having a larger family has a causal effect on the %“quality” of the children? Or is it the case that families who choose to have more children are (inherently) different, and the children would have lower education
%regardless of family size?
%IMPORTANCE OF BIRTH ORDER
%Like most previous studies, we find a negative correlation between family size and children’s educational attainment. However, when we include indicators for birth order, %the effects of family size are reduced to almost zero.  The evidence suggests that family size itself has little impact on the quality of each child but more likely impacts %only the marginal children through the effect of birth order
%In the estimates so far, we may be confounding the effects of family size with those of birth order. (p. 680)...  The family size effects are reduced to close to zero—the %coefficient on the linear term is now 0.01—with the addition of the birth order dummies.
%WHO TO EXAMINE
%The TWIN indicator is equal to 1 if the nth birth is a multiple birth and equal to 0 if the nth birth is a singleton. We restrict the sample to families with at least n births and %study the outcomes of children born before the nth birth. (see 681 for complete discussion of this)
%VALIDITY OF TWIN INSTRUMENT
%Important discussion on page 682.  Suggest that in Norway twinning NOT related to parent's education (estimate at each parity).  Suggest modern fertility drugs were %introduced later than the data in their sample
%TWINS MAY HAVE DIRECT EFFECT ON SIBLING OUTCOMES BEYOND JUST INCREASING FAMILY SIZE
%More generally, to the extent that twins are not excludable and have a negative direct effect on the educational attainment of the other children in the family (perhaps %because they are more likely to be in poor health), our estimates of the effect of family size using the twins instrument will be biased toward finding larger negative
%effects.
%FIRST STAGE
%The first stage is very strong and suggests that a twin birth increases completed family size by about 0.7 to 0.8. As expected, twins at higher parity have a larger effect %on family size, presumably because they are more likely to push families above their optimal number of children.
%TWIN RESULTS (p. 683)
%OUTCOMES
% Principal: education
%Labour market: We examine the effects of family size and birth order on the following labor market outcomes: the earnings of all labor market participants, the earnings %of full-time employees only, and the probability of being a full-time employee
%Also examine prob of having a teen birth

%Caceres (2006)
% Description of rescuing twin births without twin identifier (pp. 741-742)
%WHY RESTRICT SAMPLE TO OLDER?
%Therefore, the sample is restricted to oldest siblings in the household that are not from a multiple birth, since being part of a multiple birth or being a younger sibling of %twins or other higher order multiple birth is conditional on the occurrence of multiple births in the household (post-treatment).
% INSTRUMENT VALIDITY
%Despite the fact that the orthogonality assumption is nontestable (orthogonality of multiple births to child quality preferences), the random nature of multiple births, %the choice of the observational unit under analysis (oldest child in the household who does not belong to a multiple birth), the selection of the 1980 U.S. Census as %data set,8 the inclusion of other variables that are correlated with the incidence of multiple births such as age of the mother, race, and mothers’ education (Table 2), %as well as the analysis of the impact of twinning in a specific birth make it more likely that this assumption holds

%Rosenzweig & Wolpin (1980) 
% EXOGENEITY OF TWINS (see top of p. 237)
%Nevertheless, we ran Tobit regressions (because of the concentration of observations at zero) of TR on farm size, non-earnings income, and the schooling levels of %each parent based on our total sample of farm families. Application of the chi2 test indicated that we could not reject the hypothesis of joint insignificance at %conventional levels (chi2(4) = 2.49). The same set of variables, however, explained a statistically significant (F4,1628 = 5.17) but small (R2 = 005) proportion of the %variance in children ever born.
% USE a twins variable which standardises for parity as an approximate way to correct for the fact that families with higher than avearge number of children are likely 
%to select in to multiple births.




\section{Data}
\label{scn:data}
\vspace{-5mm}
In order to address this question, the dataset used must include various measures of both mother and child
outcomes. In order to estimate specification (\ref{eqn:twin}), information must be available on a mother's
health and socioeconomic outcomes along with details of parity -- principally birth order, the presence
of multiple births, and the gender of each child.  It is important that this data be representative at a
national level.  This regression will be estimated using data from the Demographic and Health Surveys (DHS),
a nationally representative survey currently administered in approximately 90 developing countries.

In order to estimate the Q-Q model (\ref{eqn:model}), additional data requirements exist.  Principally,
information must be available regarding `quality' outcomes for children, as well as regarding the mother's
health and economic status.  This requires that children are observed late enough in life to have suitable
quality measures.  This data will be obtained from Chile's Social Protection Survey (EPS for its initials
is Spanish), a nationally representative panel survey, which follows all members of a household.  The panel
nature of this survey will allow for the inclusion of lagged health and socioeconomic characteristics,
allowing for the mother's characteristics earlier in her children's lives to measured.  In order to test
the power of these findings in light of earlier empirical work, I will recreate past analyses with the
data used (such as Caceres, 2006), giving some indication of the degree of importance of any biases observed.

\begin{table}[ht]
\begin{center}
\begin{tabular}{lcc} \toprule
 VARIABLES & Height Recorded & Missing \\ \midrule
Twin & 0.0240 & 0.0202 \\
\begin{footnotesize}\end{footnotesize} & \begin{footnotesize}(0.153)\end{footnotesize} & \begin{footnotesize}(0.141)\end{footnotesize} \\
Fertility &  5.36 & 5.45\\
\begin{footnotesize}\end{footnotesize} & \begin{footnotesize}(2.81)\end{footnotesize} & \begin{footnotesize}(2.83)\end{footnotesize} \\
Educfyrs & 3.40 & 3.58\\
\begin{footnotesize}\end{footnotesize} & \begin{footnotesize}(4.18)\end{footnotesize} & \begin{footnotesize}(4.11)\end{footnotesize} \\
Contraceptive\_use &  1.69 &  1.59 \\
\begin{footnotesize}\end{footnotesize} & \begin{footnotesize}(1.42)\end{footnotesize} & \begin{footnotesize}(1.37)\end{footnotesize} \\
Contraceptive\_know &  2.79 & 2.54 \\
\begin{footnotesize}\end{footnotesize} & \begin{footnotesize}(0.757)\end{footnotesize} & \begin{footnotesize}(0.900)\end{footnotesize} \\
Poor1 & 0.0882 & 0.105\\
\begin{footnotesize}\end{footnotesize} & \begin{footnotesize}(0.284)\end{footnotesize} & \begin{footnotesize}(0.307)\end{footnotesize} \\ \midrule
Observations & 2,139,076 &  2,093,323  \\ \bottomrule
\end{tabular}
\caption{Summary statistics of missing versus non-missing height}
\label{tab:missing}
\end{center}
\end{table}


\section{Empirical Framework}
A 2SLS estimation strategy to estimate the causal effect $\alpha$ in (\ref{eqn:model}) involves the following
regressions:
\begin{subequations}
\label{eqn:2SLS}
\begin{eqnarray}
Q_{ij}&=&\beta_0+\beta_1Size_j+\beta_2B_{j}+\beta_3X_{ij}+\beta_4W_j+u_{ij} \label{eqn:2SLSa}\\
Size_{ij}&=&\gamma_0+\gamma_1Twin_{ij}+\gamma_2B_{j}+\gamma_3X_{ij}+\gamma_4W_j+v_{ij} \label{eqn:2SLSb}
 \end{eqnarray}
\end{subequations}
where $Size_j$ refers to number of children in the family, $B_j$ birth order of the child, $X_{ij}$ a vector
of individual characteristics and $W_j$ a vector of family characteristics such as income and parents' education.
In what follows, exogenous variables in equation \ref{eqn:2SLSa} will be denoted $\mathbf{x_2}$, the
endogenous variable $Size_j$ as $x_1$, the instrumental variable twin as $z_1$, and the outcome variable
$Q_{ij}$ as $y$.  Then, the regressors from equation \ref{eqn:2SLSa} are represented as
$\vect{x}=[x_1\ \vect{x}_2']'$ and the instruments from \ref{eqn:2SLSb} as $\vect{z}=[z_1\ \vect{x}_2']'$.

In order to derive the consistency of the estimators $\vect{\beta}$, we start from the typical instrumental variables estimator:
\begin{eqnarray}
\label{eqn:IVconsist}
\vect{\beta}_{IV}=(\vect{Z}'\vect{X})^{-1}\vect{Z}'\vect{y} \nonumber
\end{eqnarray}
where $\vect{Z}$ is an $N \times 4$ matrix with $i$\textsuperscript{th} row $\vect{z}_i'$.  Determining consistency proceeds via
the susbstitution of the structural equation for $\vect{y}$:
\begin{eqnarray}
\label{eqn:IVconsist2}
\vect{\beta}_{IV}&=&(\vect{Z}'\vect{X})^{-1}\vect{Z}'[\vect{X\beta}+\vect{u}] \nonumber\\
&=&\vect{\beta}+(\vect{Z}'\vect{X})^{-1}\vect{Z}'\vect{u}\nonumber\\
&=&\vect{\beta}+(N^{-1}\vect{Z}'\vect{X})^{-1}N^{-1}\vect{Z}'\vect{u}\nonumber
\end{eqnarray}
where the final line is included in order to demonstrate consistency via the use of the law of large numbers\footnote{This relies
on the fact that $N^{-1}\vect{Z}'\vect{X}=N^{-1}\sum_i\vect{x}_i\vect{x}_i'$ and likewise for $N^{-1}\vect{Z}'\vect{u}$.  A
complete exposition is provided in Cameron and Trivedi (2006).}.  Consistency of the IV estimator requires that
\begin{eqnarray}
\label{eqn:IVconsist3}
\mathrm{plim} N^{-1}\vect{Z'u}&=&0, \hspace{5mm}\mathrm{and} \\
\mathrm{plim} N^{-1}\vect{Z'X}&\neq&0,
\end{eqnarray}
a set of conditions very similar to those outlined in equation \ref{eqn:IV}.

This paper is concerned with the potential endogeneity of twin births.  Where twin births are endogenous, responding to
characteristics such as mother's health and education and family income, omitting these factors in the structural equation
\ref{eqn:2SLSa} will lead to biased and inconsistent estimates of $\vect{\beta}_{IV}$.

\section{Results}
\label{sec:results}
\subsection{Twin Endogeneity}
\label{sec:twinendog}
Blah blah blah.  Table \ref{tab:twinreg1} presents regressions for the probability of multiple birth using the full sample of DHS data.
\begin{equation}
\label{eqn:twinpred}
P(t_i=1)=\gamma_1 y_i + \gamma_2 h_i + \gamma_3 m_i +X_i'\beta + \epsilon_i.
\end{equation}

\begin{table}[ht]
\begin{center}
\begin{tabular}{lcc} \toprule
  & (1) & (2) \\
 VARIABLES & twind & twind \\ \midrule
 \begin{footnotesize}\end{footnotesize} & \begin{footnotesize}\end{footnotesize} & \begin{footnotesize}\end{footnotesize} \\
 fertility & 0.0825*** & 0.00490*** \\
  \begin{footnotesize}\end{footnotesize} & \begin{footnotesize}(0.00330)\end{footnotesize} & \begin{footnotesize}(0.000352)\end{footnotesize} \\
  educfyrs & 0.0125*** & 0.000704*** \\
  \begin{footnotesize}\end{footnotesize} & \begin{footnotesize}(0.00204)\end{footnotesize} & \begin{footnotesize}(0.000116)\end{footnotesize} \\
  contracept\_use & 0.00660 & 0.000218 \\
  \begin{footnotesize}\end{footnotesize} & \begin{footnotesize}(0.00505)\end{footnotesize} & \begin{footnotesize}(0.000275)\end{footnotesize} \\
  contracept\_know & 0.0123* & 0.000688 \\
  \begin{footnotesize}\end{footnotesize} & \begin{footnotesize}(0.00727)\end{footnotesize} & \begin{footnotesize}(0.000418)\end{footnotesize} \\
  height & 0.0125*** & 0.000673*** \\
  \begin{footnotesize}\end{footnotesize} & \begin{footnotesize}(0.00111)\end{footnotesize} & \begin{footnotesize}(5.13e-05)\end{footnotesize} \\
  bmi & 0.00233 & 9.72e-05 \\
  \begin{footnotesize}\end{footnotesize} & \begin{footnotesize}(0.00226)\end{footnotesize} & \begin{footnotesize}(0.000126)\end{footnotesize} \\
  poor1 & -0.117*** & -0.00567*** \\
  \begin{footnotesize}\end{footnotesize} & \begin{footnotesize}(0.0394)\end{footnotesize} & \begin{footnotesize}(0.00192)\end{footnotesize} \\
  Constant & -4.559*** & -0.114*** \\
  \begin{footnotesize}\end{footnotesize} & \begin{footnotesize}(0.175)\end{footnotesize} & \begin{footnotesize}(0.00773)\end{footnotesize} \\
  \begin{footnotesize}\end{footnotesize} & \begin{footnotesize}\end{footnotesize} & \begin{footnotesize}\end{footnotesize} \\
  Observations & 2,083,707 & 2,083,707 \\
  R-squared &  & 0.009  \\ \midrule
\multicolumn{3}{c}{\setstretch{0.9}\begin{footnotesize}Robust standard errors in parentheses\end{footnotesize}} \\
\multicolumn{3}{c}{\begin{footnotesize}*** p$<$0.01, ** p$<$0.05, * p$<$0.1\end{footnotesize}} \\ 
\multicolumn{3}{p{7.2cm}}{\setstretch{0.9}\begin{footnotesize}\textsc{Note:} Column (1) represents probit with MFX at the mean.  Column
(2) is OLS.\end{footnotesize}}\\
\bottomrule
\end{tabular}
\end{center}
\label{tab:twinreg1}
\caption{Probability of giving birth to multiple children}
\end{table}
\setstretch{1.25}
\subsection{Bias in IV Regressions}
\subsubsection{Monte Carlo Simulation}
\label{scn:MCS}
The results from section \ref{sec:twinendog} suggest that twin birth is not exogenous. As a result, IV estimates failing to control for the full set of 
relevant variables in the second stage will produce inconsistent and biased estimates of the importance of family size on child quality [REHASH THIS SENTENCE.  IT BRINGS UNNECESSARY CHANCE OF TROUBLE].  As a first approach at quantifying the importance of this bias, Monte Carlo simulations are run allowing for correlation between twin birth and previously uncontrolled factors
to vary.\footnote{Whilst it may seem unlikely that the correlation between twin births and these unobserved factors will vary over time and hence that
simulation of a range of values is un-ilustrative, these simulations may be useful in abstracting to other situations.  This analysis is identical to the
case where multiple births occurr more frequently where in-vitro fertilisation (IVF) is used.  As IVF treatment often involves the implantation of 
multiple embryos, multiple births are more likely here compared with traditional methods of fertilisation (cite).  These simulations then could be 
thought as analogous to examining the role increasing use of IVF plays on the consistency of twin estimates where only birth records are available, 
but details regarding fertilisation are unknown.}  This allows for the examination of the importance of the instrumental validity assumption, essentially
by simulating values of equation (\ref{eqn:IVconsist3}) which increasingly diverge from 0. 

I simulate an ($n \times 3)$ vector of $N=100,000$ standard normal error terms are for the following system of equations.\\
\begin{eqnarray}
\label{eqn:MC1}
y_i=\beta_0+\beta_1 x_i + \epsilon_{1i} \nonumber\\
x_i=\gamma_0 + \gamma_1 z_i + \epsilon_{2i} \nonumber\\
z_i=1[\alpha_0+\epsilon_{3i}>0], \nonumber
\end{eqnarray} 
where simulated $\epsilon_j$ follow:
\begin{equation}
\begin{bmatrix}
\epsilon_1\\
\epsilon_2\\
\epsilon_3\\
\end{bmatrix}
\sim \mathcal{N}
\left(\begin{bmatrix}
0\\
0\\
0\\
\end{bmatrix}
,
\begin{bmatrix}
\sigma_{\epsilon_1}^2 &  \rho_{\epsilon_1\epsilon_2} &  \rho_{\epsilon_1\epsilon_3}\\
\rho_{\epsilon_2\epsilon_1} & \sigma_{\epsilon_2}^2 &  \rho_{\epsilon_2\epsilon_3} \\
\rho_{\epsilon_3\epsilon_1}&  \rho_{\epsilon_3\epsilon_2}& \sigma_{\epsilon_3}^2 \\
\end{bmatrix}\right)
\end{equation}


\subsubsection{Empirical Evidence}
\label{scn:EE}
\subsection{Twin Selection}
Sections \ref{scn:MCS} and \ref{scn:EE} demonstrate that the incidence of non-observed twin predictor variables will invalidate the instrumental variable identification strategy.  Monte Carlo simulations suggest that even a relatively minor correlation between these unobserved terms and the stochastic component of the Q-Q equation can result in estimators which perform worse than traditional OLS estimators.  However, if we are able to collect data on these variables which influence twin-selection\footnote{I refer to twin ``selection" here not because families choose whether or not to have twins, but rather because certain characteristics make it appear as if families are more likely to select in to multiple births.  The usefelness of this term will become apparent in the analysis which follows, which draws from prior literature on selection on observables versus selection on unobservables.  Once again, a parallel may be drawn with the use of twins as an instrument in situations where IVF treatment is available; in this case it seems less farfetched to speak of ``selection'' of multiple births.} a 2SLS methodology can still be used consistently by including these predictors as controls in the first and second stage.  The credibility of this strategy however relies on the ability to perfectly observe determinant factors which increase the propensity of twin birth.  If we believe that models such as (\ref{eqn:twinpred}) are structural representations of twin birth, we can simply include all relevant predictors, and recover consistent estimates of family size.

It seems unlikely that survey data will allow for all such relevant predictors to be observed however.  While mother's weight and height and family socio-economic characteristics are observable, many aspects of a mother's pre-natal behaviour are not observed by the econometrician.  Dietary habits and other household investments in (pre-natal) child health may be both difficult to observe, and correlated with child quality and quantity preferences.  In this case, a problem of selection on unobservables exists for twin births.  The inclusion of ovbservable predictors of twin births in the 2SLS specification will not correct for this bias.

 



\section{Conclusion}

\newpage
\bibliography{BiBBase}

\newpage
\appendix
\section{Twin Estimation Strategies}
\label{scn:litrev}

\vspace{19.2cm}	

\begin{center}
\begin{rotate}{90}
\begin{tabular}{lp{4mm}lll}\toprule
Author& &  Estimation Strategy & \multicolumn{2}{c}{Controls} \\ \midrule
&& & \hspace{5mm}1\textsuperscript{st} Stage Controls & 2\textsuperscript{nd} Stage Controls \\  \cmidrule(r){4-4} \cmidrule{5-5}
Rosenzweig and Wolpin (1980) & &
$ED=\omega_0+\omega_1 TR+u_1$&
\begin{small}None.\end{small}&
\begin{small}N\slash A\end{small}.
\\
\begin{footnotesize}\end{footnotesize}&\begin{footnotesize}\end{footnotesize}&\begin{footnotesize}\end{footnotesize}&\begin{footnotesize}\end{footnotesize}&\begin{footnotesize}\end{footnotesize}\\
\multirow{2}{*}{Black \emph{et al.} (2005)} & &
(1) $FSIZE=\alpha_0 + \alpha_1 T + X\alpha_2 + v$ &
\begin{small}Full family size and  birth \end{small}&
\begin{small}Covariates from\end{small}
\\
& &
(2) $ED=\beta_0+\beta_1 FSIZE + X\beta_2 + \epsilon$ &
\begin{small}order dummies. Demogra-\end{small}  &
\begin{small}stage 1.\end{small}
\\
& & &
\begin{small}phic controls: age, sex, pa-\end{small}&
\\
& & &
\begin{small}rental age and education.\end{small}&
\\
\begin{footnotesize}\end{footnotesize}&\begin{footnotesize}\end{footnotesize}&\begin{footnotesize}\end{footnotesize}&\begin{footnotesize}\end{footnotesize}&\begin{footnotesize}\end{footnotesize}\\
\multirow{2}{*}{C\'aceres (2006)} & &
(1) $n_i=X'_i\mu + \rho mb_i + v_i$ &
\begin{small}Dummies by age, state of\end{small}&
\begin{small}Covariates from\end{small}
\\
& &
(2) $y_i=\alpha + \gamma n_i + X'_i\beta + \epsilon_i$ &
\begin{small}residence, mother's educ- \end{small}&
\begin{small}stage 1.\end{small}
\\
& & &
\begin{small}ation, race, mother's age\end{small}&
\\
& & &
\begin{small}and sex.\end{small}&
\\
\begin{footnotesize}\end{footnotesize}&\begin{footnotesize}\end{footnotesize}&\begin{footnotesize}\end{footnotesize}&\begin{footnotesize}\end{footnotesize}&\begin{footnotesize}\end{footnotesize}\\
\multirow{2}{*}{Angrist \emph{et al.} (2006)} &  &
(1)  $c_1=X'_i\beta+\alpha t_{2i}+\eta_i$ &
\begin{small}Mother's age, mother's \end{small} & 
\begin{small}Covariates $X_i$\end{small}
\\
& &
(2) $y_i=W'_i\mu+\rho c_i + \epsilon_i$ &  
\begin{small}age at first birth, census \end{small} &
\begin{small}from stage 1.\end{small}
\\
& & &
\begin{small}year, parental birth place.\end{small}&
\\
\bottomrule 
\multicolumn{5}{p{19.2cm}}{\setstretch{0.9}\begin{footnotesize}\textsc{Note:} Nomenclature employed here is as faithful as possible to the original studies.  $TR$ refers to twin ratio, the number of twin births per mother divided by number of pregnancies, and $ED$ represents a child's standardised educational attainment. $FSIZE$ refers to sibship size and $T$ a dummy for twin birth on the $n$\textsuperscript{th} birth, restricting the analysis to children with birth order $<n$. The variable $n_i$ defined by C\'aceres once again refers to sibship size, and $mb_i$ multiple births. Finally, $c_i$ refers to child $i$'s sibship size, with $t_{2i}$ denoting multiple second births.\end{footnotesize}}\\
\end{tabular}
\end{rotate}
\captionof{table}{Prior empirical specifications}
\label{tab:litrev}
\end{center}
\setstretch{1.25}

\section{Full Survey Data}
\label{scn:surveys}
\begin{table}[ht!]
\caption{Surveys with data for education}
\vspace{-7mm}
\label{tab:survey}
\begin{center}
\begin{tabular}{l l c c c c}
\multicolumn{6}{c}{\textsc{and household characteristics}}\\
& & & & &  \\
\toprule
Country	& Income &	Wave 1	&	Wave 2	&	Wave 3	&	Wave 4	\\
\midrule
Armenia	&	LM		&	2000	&	2005	&		&		\\
Bangladesh	&	LI	&	1993	&	1996	&	2004	&		\\
Benin	&	LI		&	1996	&	2001	&		&		\\
Bolivia	&	LM		&	1994	&	1998	&	2003	&		\\
Burkina Faso	&	LI &	1992	&	1998	&	2003	&		\\
Cambodia	&	LI	&	2000	&	1991	&	1998	&	2004	\\
Chad	&	LI		&	1996	&	2004	&		&		\\
Comoros	&	LI		&	1996	&		&		&		\\
Congo	&	LM		&	2005	&		&		&		\\
Cote D'Ivoire	&	LM	&	1994	&	1998	&		&		\\
Dominican Republic	&	UM	&	1991	&	1996	&	2002	&		\\
Egypt	&	LM	&	1992	&	1995	&	2000	&	2005	\\
Ghana	&	LM		&	1993	&	1998	&	2003	&		\\
Guinea	&	LI		&	1999	&	2005	&		&		\\
Haiti	&	LI		&	1994	&	2000	&		&		\\
Honduras	&	LM	&	2005	&		&		&		\\
India	&	LM		&	1999	&		&		&		\\
Indonesia	&	LM	&	1994	&	1997	&		&		\\
Kazakhstan	&	UM	&	1995	&		&		&		\\
Kenya	&	LI		&	1993	&	1998	&	2003	&		\\
Lesotho	&	LM		&	2004	&		&		&		\\
Madagascar	&	LI	&	1992	&	1997	&	2004	&		\\
Malawi	&	LI		&	1992	&	2000	&	2004	&		\\
Mali	&	LI		&	1995	&	2001	&		&		\\
Morocco	&	LM		&	1992	&	2003	&		&		\\
Mozambique	&	LI	&	1997	&	2003	&		&		\\
Namibia	&	UM		&	1992	&	2000	&		&		\\
Nepal	&	LI		&	1996	&	2001	&		&		\\
Nicaragua	&	LM	&	1997	&	2001	&		&		\\
Niger	&	LI		&	1992	&	1998	&		&		\\
Nigeria	&	LM		&	1999	&	2003	&		&		\\
Pakistan	&	LM	&	1991	&		&		&		\\
Peru	&	UM		&	1996	&	2000	&		&		\\
Philippines	&	LM	&	1993	&	1998	&	2003	&		\\
Rwanda	&	LI		&	1992	&	2000	&	2005	&		\\
Senegal	&	LM		&	1992	&	2005	&		&		\\
South Africa	&	UM	&	1998	&		&		&		\\
Tanzania	&	LI	&	1992	&	1996	&	2004	&		\\
Togo	&	LI		&	1998	&		&		&		\\
Turkey	&	UM		&	1993	&	1998	&		&		\\
Uganda	&	LI		&	1995	&	2000	&		&		\\
Vietnam	&	LM		&	1997	&	2002	&		&		\\
Zambia	&	LM		&	1992	&	1996	&	2001	&		\\
Zimbabwe	& LI	&	1994	&	1999	&		&		\\
\midrule
\multicolumn{6}{p{10.5cm}}{\footnotesize\textsc{Note:} LI refers to Low Income, LM to Lower-Middle income, and UM to Upper-Middle.
Classifications are according to the World Bank (2012).}\\

\bottomrule
\end{tabular}
\end{center}
\end{table}

\end{spacing}
\end{document}

________________________________________________________________________________________________________________________________



































Naive OLS is
likely to overestimate the importance of the degree of causality.  Where unobserved factors such
as parental preferences drive both quality of children and household size, estimates of the
trade-off will be biased.  Various plausible channels for this mechanism have been proposed, such
as a preference for lower children by more highly educated parents (Qian, 2005). Frequently, data
on multiple births is used to recover a causal estimate of the Q-Q trade-off. Rosenzweig and Wolpin
(1980) discussed the theoretical validity of instrumenting for family size ($N$) with twin birth ($t$),
with this method subsequently receiving considerable treatment in the empirical literature.  The
validity of this estimation strategy relies upon typical IV assumptions, namely
\begin{equation}
\label{eqn:IV}
\begin{aligned}
\text{cov}(t,N)\neq0 \\
\text{cov}(t,u)=0
\end{aligned}
\end{equation}
where u refers to the unobserved error term in the QQ model:
\begin{equation}
\label{eqn:model}
Q_i=\alpha N_i + X_i'\beta + u_i.
\end{equation}

I propose to examine the exogeneity of the instrumental strategy outlined in (\ref{eqn:IV}).
I will ask whether multiple births truly are an exogenous shock to family size, or whether these are
influenced by other socioeconomic factors not considered in prior studies.  Fundamentally, I am
interested in determining whether a mother's likelihood of giving birth to multiple infants depends upon
her household income ($y_i$), investment in own health ($h_i$), or other individual or household
characteristics such as marital status ($m_i$).  These results are quite simple to obtain using a limited
dependent variable regression model (\emph{e.g.}\ a probit) to estimate:
\begin{equation}
\label{eqn:twin}
P(t_i=1)=\gamma_1 y_i + \gamma_2 h_i + \gamma_3 m_i +X_i'\beta + \epsilon_i.
\end{equation}
Prior literature has failed to take account of this threat to instrumental validity, generally assuming
that twin births are exogenous (beyond race and birth-order).

It seems plausible that even in the case that twin conception is exogenous to a household or mother's
socioeconomic characteristics, the likelihood that a mother carries to full term and gives birth to
multiple live infants will not be exogenously determined.  I am interested in testing the null that
$\gamma_1=0$, $\gamma_2=0$ and $\gamma_3=0$, individually and collectively against the alternative
that $\vec{\gamma}>0$.

If this hypothesis is borne out empirically, previous Q-Q model estimates obtained instrumenting for
multiple birth, but not cotrolling for a mother's socioeconomic characteristics, will be biased. In this
essay, the bias will be derived mathematically, and tested empirically.  Using data from Chile (see
section \ref{scn:data}) I will estimate (\ref{eqn:model}) via OLS, IV (not controlling for additional
mother characteristics) and IV controlling for these characteristics.  This will allow me to empirically
determine if the bias in previous estimates exists, is important, and agrees with theoretical
derivations.



\setstretch{1.1}
\begin{eqnarray}
\label{eqn:MC1}
y_i=\beta_0+\beta_1 x_i + \epsilon_{1i} \nonumber\\
x_i=\gamma_0 + \gamma_1 z_i + \epsilon_{2i} \nonumber\\
z_i=1[\alpha_0+\epsilon_{3i}>0], \nonumber
\end{eqnarray} 
where simulated $\epsilon_j$ follow:
\begin{equation}
\left(
\begin{array}{c}
\epsilon_1\\
\epsilon_2\\
\epsilon_3\\
\end{array}
\right)
\sim \mathcal{N}\left(
\Biggl(
\begin{array}{c}
0\\
0\\
0\\
\end{array}
\Biggl),
\Biggl(\begin{array}{c}
\sigma_{\epsilon_1}^2 \quad  \rho_{\epsilon_1\epsilon_2} \quad  \rho_{\epsilon_1\epsilon_3}\\
\rho_{\epsilon_2\epsilon_1} \quad \sigma_{\epsilon_2}^2 \quad  \rho_{\epsilon_2\epsilon_3} \\
\rho_{\epsilon_3\epsilon_1}\quad  \rho_{\epsilon_3\epsilon_2}\quad \sigma_{\epsilon_3}^2 \\
\end{array}\Biggl)
\right)
\end{equation}
\subsubsection{Empirical Evidence}
\label{scn:EE}
\setstretch{1.25}
