\documentclass{article}[11pt,subeqn]

\usepackage{fancyhdr}
\pagestyle{fancy}
\usepackage[pdftex]{graphicx}
\usepackage{setspace}
\usepackage{framed}
\usepackage{lastpage}
\usepackage{amsmath}
\usepackage[utf8]{inputenc}
\title{Child Quantity versus Quality: Are Twin Births Exogenous?}
\author{Damian C. Clarke}

\setlength\topmargin{-0.375in}
\setlength\headheight{15pt}
\setlength\headwidth{6.05in}
\setlength\textheight{9in}
\setlength\textwidth{6.05in}
\setlength\oddsidemargin{0.18in}
\setlength\evensidemargin{-0.5in}
\setlength\parindent{0.25in}
\setlength\parskip{0.25in}

\usepackage{natbib}
\bibliographystyle{abbrvnat}
\bibpunct{(}{)}{;}{a}{,}{,}

\usepackage{hyperref}

\usepackage{lscape}
\usepackage{rotating}
\usepackage{multirow}
\usepackage{rotating,capt-of}
\usepackage{array}

\usepackage{lineno}
\usepackage[update,prepend]{epstopdf}

\usepackage[font=sc]{caption}

%NEW COMMANDS
\newcommand{\Lagr}{\mathcal{L}}
\newcommand{\vect}[1]{\mathbf{#1}}
\newcolumntype{P}[1]{>{\raggedright}p{#1\linewidth}}

\usepackage{appendix}
\usepackage{booktabs}
\usepackage{cleveref}

\fancyhead{}
\fancyfoot{}
\fancyhead[L]{\textsc{Extended Essay}}
\fancyhead[C]{\textsc{MSc. Economics for Development}}
\fancyhead[R]{\textsc{Damian C. Clarke}}
\fancyfoot[C]{\textsc{\thepage\ of  \pageref{LastPage}}}
\fancyfoot[R]{\texttt{ DRAFT -- Please do not circulate.}}

\begin{document}
\linenumbers
\begin{spacing}{1.25}

%\begin{titlepage}
%\begin{center} 
%\textbf{\Large Child Quantity versus Quality: Are Twin Births Exogenous?}\\
%\vspace{3cm}
%\emph{Thesis submitted in partial fulfillment of the requirements for the\\
%Degree of Master of Science in Economics for Development\\ at the University
%of Oxford}\\
%\vspace{0.2cm}
%\today\\
%\vspace{3.5cm}
%By\\
%\vspace{2.5cm}
%2806032\\
%\vspace{1cm}
%\texttt{\Large DRAFT -- Please do not circulate.}\\
%\vfill
%Word count: 9,100
%\end{center}
%\end{titlepage}

\begin{titlepage}
\begin{center} 
\textbf{\Large Child Quantity versus Quality: Are Twin Births Exogenous?}\\
\vspace{3cm}
\emph{Thesis submitted in partial fulfillment of the requirements for the\\
Degree of Master of Science in Economics for Development}\\ 
\vspace{0.2cm}
\vspace{3.5cm}
By\\
\vspace{2.5cm}
Damian C. Clarke\\
Balliol College, Oxford\\
\today\\
\vspace{5cm}
%\texttt{\Large DRAFT -- Please do not circulate.}\\
%\vfill
%Word count: 9,100
\end{center}
\end{titlepage}

%\maketitle

\end{spacing}
\begin{spacing}{1.5}

\begin{abstract}
Given the endogenous nature of a family's fertility decisions, demonstrating the existence of a trade-off between child quality and child quantity requires the identification of a 
valid exclusion restriction in a quality-quantity model.  Prior work has suggested that multiple births are an appropriate manner to estimate this relationship.  I show that twin 
births are \emph{not} exogenous in a developing country setting, invalidating the identifying assumption of this methodology in this situation.  Instead twin birth depends on a 
range of observable (and potentially unobservable) 
characteristics of the mother, such as height, BMI, and family income.  The effect of this result on typical 2SLS estimates is then examined via Monte Carlo simulation, and 
empirically using pooled results from the Demographic and Health Surveys, a dataset consisting of more that 1,000,00 children from 44 developing countries.  Given the poor 
performance of these estimates under simulation, an alternative methodology is employed to determine the plausibility of family size having any effect on child quality.  This
methodology suggests that higher sibship produces a larger delay in educational attainment and total years of schooling, but suggests that family size has less effect on child
attendance.  
\end{abstract}

\section{Introduction}
\label{scn:intro}
Economists, and presumably policy makers in developing countries, are interested in determining whether a larger family size causes lower `quality' offspring.  Such a consideration
 has a long history in economic thought, with implications for optimal population decisions at both the macroeconomic and family level.  
%The existence of a trade-off between family size and child quality is an enduring hypothesis in economics.  
\citet{Malthus1798} famously introduced a theory in which population 
growth entered into competition with economic well-being.  Despite the apparent intractability of Malthus' initial theory, contemporary microeconomic models have sought to use 
the idea of a quality-quantity trade-off in order to explain the observed (negative) quality gradient which exists across fertility rates.  Early work by \citet{BeckerLewis1973} 
and \citet{BeckerTomes1976} propose that a household's ability to produce child `quality' is a function of the quantity of children.  The evidence in favour of this quantity-quality 
(Q-Q) relationship is significant.  Surveys at both a macro- and microeconomic level have documented the existence of a negative correlation between a wide range of achievement 
measures and family size (see for example \cite{Desai1995}, or \cite{Hanushek1992}).

However, the empirically observed relationship between an individual's sibship size and their measured `quality' does not imply that such a trade-off exists if parental decisions 
regarding the production of child quality and quantity are jointly driven by other unobserved factors.  Concerns regarding parental heterogeneity and omitted variable 
bias\footnote{Principally here we are concerned with unobserved parental behaviours which may favour both lower family size and higher child quality.  \citet{Qian2009} 
suggests that such a mechanism will exist where parents who value education more highly also decide to have less children.} have spawned an entire literature which aims to 
isolate the causal effect of sibship size.  In order to determine whether increases in child quantity actually cause families to skimp on quality, exogenous shocks to family 
size must be exploited.  The economic literature has suggested a number of ways that this can be done, with one of the most common being the unexpected rise in family size 
resulting from a multiple or twin birth.\footnote{The literature which uses this methodology will be discussed in section \ref{scn:lit}.  Other strategies which have been proposed 
to identify the Q-Q model involve gender mix and parental stopping rules \citep{ConleyGlauber2006}, son-preference \citep{Lee2008}, and natural experiments based upon the 
relaxation of government fertility policies \citep{Qian2009}.   In what follows, I only focus on the use of multiple births as an instrument.}

The use of twins as an instrument requires that twinning truly is an exogenous shock to family size.  \citet{Kahnetal2003} suggest that twin conception may be a random event absent mother's age, 
parity,\footnote{As per related medical literature, parity here refers to the number of previous livebirths \citep{Elwood1978}.} and race.\footnote{For now I abstract from 
the issue of in vitro fertilisation (IVF).  This topic will be returned to at various points in the coming sections.} However, exogeneity of twin births implies a stricter 
biological relationship than exogeneity of twin conception.  Even where conception is conditionally exogenous, the subsequent success of twin gestation and birth must not interact with maternal or family characteristics for the instrument to be valid.  Recent work in the medical literature 
suggests that this may not be the case. \citet{Shinagawaetal2005} find that metabolic demands in mothers during the third trimester of twin pregnancy is about 10\% higher 
than mothers with a singleton pregnancy. \citet{Philipson2008}  also suggests that twin and mother survival at birth depends (more so than singleton births) upon the availability 
of costly technology and professionals.  Both of these results imply that a family's probability of twinning will depend upon the resources invested during (and even preceeding) 
gestation.  This point is particularly salient for low-income economies.  Anemia, fetal death, and maternal death are much more common in developing countries and vary depending 
upon income and place of residence \citep{Rush2000}.  Where comprehensive pre-natal programs are available for mothers, as is the case in many industrialised countries (for 
example the program \emph{Women, Infants and Children}, a program which targets low-income pregnant and postpartum women in the USA), this 
may reduce the effect of ill-health or low socioeconomic status on a mother's likelihood of carrying twins to full term by providing compensatory investments.

Despite the higher physiological and economic costs of twin birth, the economic literature is yet to fully incorporate these findings into twin studies.  This essay proposes to examine the plausible exogeneity of twin births in a developing country setting.  This is an important hypothesis to test given that its failure to
hold is a direct threat to the validity of instrumental estimations of the Q-Q model proposed by a range of papers.  Furthermore, it seems to be yet untested in the
empirical literature on the Q-Q tradeoff beyond a brief examination of twin births and maternal education levels in a developed country setting (see \citet{Blacketal2005}
and \citet{Angristetal2010}).  Although these studies found that twinning is not related to maternal education, there is no reason to believe \emph{a priori} that this
will be the case when turning to developing countries setting, or with a more complete set of covariates including maternal health in more developed countries. 

Whether a Q-Q trade-off occurs is an important question for developing economies.  \citet{Clelandetal2006} suggest that the main motivation for the introduction of family 
planning policies in Asia was as a manner to ``enhance prospects for socioeconomic development by reducing population growth''.  Family planning is also still considered a 
concern according to policy makers in the developing world.  A recent survey of national governments\footnote{The United Nations Inquiry among Governments on Population and 
Development.} suggests that fertility was perceived as too high in 50\% of developing countries, with this figure rising to 86\% among the least developed countries (United 
Nations, \citeyear{UN2010}).   Whilst there are a range of other motivations for family planning policies, in what follows this paper aims to examine the perception that 
reducing the number of children results in an increase in per-child investments at a family level.  I will proceed as follows; section \ref{scn:lit} discusses the use of 
the twin instrument in prior literature.  Section \ref{scn:EF} then discusses the basic empirical framework to be employed in this paper, with section \ref{scn:data} discussing 
the data employed in this empirical analysis.  Results are presented in section \ref{scn:results}; I examine the exogeneity of twin births in subsection \ref{scn:twinendog}, 
and then provide a simulated and empirical analysis of the bias potential endogeneity produces in subsection \ref{scn:bias}. Subsection \ref{scn:selection} provides details of an 
alternative method of quantifying the Q-Q trade-off which avoids the suspect exclusion restriction.  Finally, section \ref{scn:conclusion} concludes. 

\section{Twin Studies}
\label{scn:lit}
The use of twin births to address the problem of endogeneity in the Q-Q model seems to have been initially proposed by \citet{RosenzweigWolpin1980}.  
They derive the theoretical requirements to estimate the size of the trade-off when the shadow price of child quality depends on the number 
of children and vice versa.  By relying on the assumption that multiple births are an exogenous shock to family size (once accounting for 
the total number of a mother's pregnancies) they estimate the effect of a twin birth upon the educational attainment of children in the twins' 
family.  This and alternate empirical specifications discussed in this section are decribed in table \ref{tab:litrev} in appendix \ref{scn:litrev}.  

Subsequent papers employing a twin-birth methodology have proposed a number of strategies which enable them to obtain consistent estimates of 
the Q-Q trade-off while relaxing Rosenzweig and Wolpin's exogeneity assumption.  \citet{Blacketal2005} extend the controls to account for the 
fact that the probability of multiple birth increases with maternal age as described by \citet{Jacobsenetal1999} and others.  They include a 
set of parental age and education controls, however note that they are unable to reject the hypothesis that parental education has no effect 
on the probability of multiple birth.  Likewise, \citet{Caceres2006} includes controls for mother's age, race, and education, suggesting that 
the use of these and pre-1980 US Census data should be sufficient to approximate conditional exogeneity\footnote{The use of pre-1980 Census 
data seems important as this predates the widespread introduction of fertility drugs.  The use of fertility drugs is associated with higher a 
probability of multiple births, and resulting concerns that the orthogonality assumption will be violated if users of fertility treatment are 
non-random.}.  Finally, \citet{Angristetal2010} recognise that twin birth varies with maternal age at birth and race, including twinning as 
one of three instruments. 

Angrist et al.\ join recent work from \citet{RosenzweigZhang2009} in questioning the validity of twin instrumentation in another sense.  These 
authors suggest that the error term in the Q-Q equation is unlikely to be orthogonal to the instrument given that twinning imposes predictable 
and unobserved family responses in investment decisions.  Particularly, these studies question the effect that close birth spacing and an endowment 
effect responding to the lower health at birth of twins compared to single births\footnote{Using data from the United States, \citet{Almondetal2005} 
document that twins have substantially lower birth weight, lower APGAR scores, higher use of assisted ventilation at birth and lower gestion period 
than singletons.} has on investments in pre-twin siblings.  Despite this critique, Rosenzweig and Zhang suggest that it is still possible to compute 
an upper and a lower bound of the Q-Q trade-off.

More recently, instrumentation using twin births has been applied to estimate a Q-Q model in the developing world. \cite{SouzaPonczek2012}, 
\cite{FitzsimonsMalde2010}, \citet{Sanhueza2009} and \citet{Lietal2008} have applied a similar methodology to that of Angrist et al., 
examining twin births in Brazil, Mexico, Chile and China respectively.  These studies find mixed results depending upon the country under 
examination and once again, while considering the invalidity of the twin exclusion restriction in the Q-Q model in terms of maternal education, 
do not examine this in terms of non-random twin births due to maternal health. 



\vspace{-5mm}
\section{Empirical Framework}
\label{scn:EF}
A two-stage least squares strategy to estimate the causal effect of family size on child quality using twin births as an instrument involves the following system of equations:
\begin{subequations}
\label{eqn:2SLS}
\begin{eqnarray}
\label{eqn:QQ}
Q_{ij}&=&\beta_0+\beta_1Fert_j+\beta_2B_{i}+X'_{ij}\vect{\beta_{\vect{X}}}+W'_j\vect{\beta_\vect{W}}+u_{ij} \label{eqn:2SLSa}\\
Fert_{j}&=&\gamma_0+\gamma_1Twin_{ij}+\gamma_2B_{i}+X'_{ij}\vect{\gamma_\vect{X}}+W'_j\vect{\gamma_\vect{W}}+v_{ij}. \label{eqn:2SLSb}
 \end{eqnarray}
\end{subequations}
Here $Fert_j$ refers to number of children in the family, $B_i$ birth order of the child, $X_{ij}$ a vector
of individual characteristics and $W_j$ a vector of family characteristics such as income and parents' education.
In what follows, exogenous variables in equation \ref{eqn:2SLSa} will be denoted $\mathbf{x_2}$, the
endogenous variable $Size_j$ as $x_1$, the instrumental variable twin as $z_1$, and the outcome variable
$Q_{ij}$ as $y$.  Then, the regressors from equation \ref{eqn:2SLSa} are represented as
$\vect{x}=[x_1\ \vect{x}_2']'$ and the instruments from \ref{eqn:2SLSb} as $\vect{z}=[z_1\ \vect{x}_2']'$.

In order to derive the asymptotic properties of the estimators of $\vect{\beta}$, we start from the typical instrumental variables estimator
$\vect{\hat{\beta}}_{IV}=(\vect{Z}'\vect{X})^{-1}\vect{Z}'\vect{y} $
where $\vect{Z}$ and $\vect{X}$ are $N \times k$ matrices with $i$\textsuperscript{th} row $\vect{z}_i'$ and $\vect{x}_i'$ respectively.  
Determining consistency proceeds via the susbstitution of the structural equation for $\vect{y}$:
\begin{eqnarray}
\label{eqn:IVderive}
\vect{\hat{\beta}}_{IV}&=&(\vect{Z}'\vect{X})^{-1}\vect{Z}'[\vect{X\beta}+\vect{u}] \nonumber\\
&=&\vect{\beta}+(\vect{Z}'\vect{X})^{-1}\vect{Z}'\vect{u}\nonumber\\
&=&\vect{\beta}+(N^{-1}\vect{Z}'\vect{X})^{-1}N^{-1}\vect{Z}'\vect{u}
\end{eqnarray}
where the final line is included in order to demonstrate consistency via the use of the law of large numbers\footnote{This relies
on the fact that $N^{-1}\vect{Z}'\vect{X}=N^{-1}\sum_i\vect{x}_i\vect{x}_i'$ and likewise for $N^{-1}\vect{Z}'\vect{u}$.  A
complete exposition can be found in Cameron and Trivedi (2006).}.  Consistency of the IV estimator requires that
\begin{eqnarray}
\mathrm{plim} N^{-1}\vect{Z'u}&=&0, \hspace{5mm}\mathrm{and} \label{eqn:IVconsist3}\\ 
\mathrm{plim} N^{-1}\vect{Z'X}&\neq&0, \label{eqn:IVrelevance}
\end{eqnarray}
the typical IV assumptions of consistency (\ref{eqn:IVconsist3}) and relevance (\ref{eqn:IVrelevance}).  

This paper is concerned 
with the potential endogeneity of twin births.  Where twin births are endogenous, responding to characteristics such as mother's 
health and education and family income, omitting these factors in the structural equation \ref{eqn:2SLSa} will lead to biased and 
inconsistent estimates of $\vect{\hat{\beta}}_{IV}$.  Suppose that relevant (observed or unobserved) factors are omitted from
(\ref{eqn:2SLSa}) and (\ref{eqn:2SLSb}) thus being relegated to the error terms $u_{ij}$ and $v_{ij}$.  If these omitted variables 
are correlated with twin birth ($Twin$), this will invalidate the IV strategy via the violation of condition (\ref{eqn:IVconsist3}).  This
inconsistency follows directly from a rearrangement of (\ref{eqn:IVderive}):
\begin{equation}
\label{eqn:inconsistency}
\vect{\hat{\beta}}_{IV}-\beta=(N^{-1}\vect{Z}'\vect{X})^{-1}N^{-1}\vect{Z}'\vect{u}.
\end{equation}
If twin birth is indeed endogenous and its determinants are omitted from our IV estimation strategy, the magnitude of the inconsistency in 
(\ref{eqn:inconsistency}) will depend upon two things.  Firstly, (inversely) upon the correlation between the exogenous instrumental
variables and the explanatory variables $\vect{x}$, and secondly upon the correlation between the exogenous instrumental component
$\vect{z}$ and u.  We will turn to these points in section \ref{scn:results}.

\section{Data}
\label{scn:data}
\vspace{-5mm}
The estimation of the Q-Q model requires information regarding various measures of both mother and child outcomes. In order to estimate 
specification (\ref{eqn:2SLSa}-\ref{eqn:2SLSb}), observations of child `quality' outcomes plus a mother's full birth rota (including a 
measure of twin or singleton births) are required.  In order to test the hypothesis of twin endogeneity, stricter data requirements 
must be met.  Along with child outcomes, the mother's health, and family socioeconomic characteristics must be observed.

I take advantage of the comprehensive information available on maternal and child outcomes in the Demographic and Health Surveys (DHS) 
to estimate the Q-Q model.  The DHS are a nationally representative set of surveys administered in approximately 90 developing countries.  
The entire set of surveys between 1991 and 2005 has been pooled, resulting in 98 surveys from 44 countries.  A full list of surveys by 
country and year is available as table \ref{tab:survey} in Appendix \ref{scn:surveys}.  DHS survey data on the educational achievement 
of each member in surveyed houeholds has been merged with maternal characteristics including weight, body mass index (BMI), and 
household level socioeconomic variables.  This merge results in 1,698,262 matched children with both educational and maternal health
outcomes.  Of this sample, 31,983 (or 1.89\%) correspond to children born in multiple births.

Table \ref{tab:missing} provides summary statistics by birth type (single or multiple) and by country type.  Countries are classified 
according to country income level in order to allow for a disaggregation of Q-Q results by country group.  This classification is obtained
from the World Bank\footnote{\url{http://data.worldbank.org/about/country-classifications},  accessed May 10, 2012.}, with DHS
surveyed countries falling into three groups: low-income economies (GNI of \$1,005 or less), lower-middle-income (\$1,006-\$3,975)
and upper-middle-income (\$3,976-\$12,275).  Details regarding this classification are provided in table \ref{tab:survey}.  In approximately 
one third of the surveys conducted the mother's height (and BMI) is not recorded and as a result these observations are not used in the 
Q-Q analysis.  \texttt{COMMENT}


\begin{table}[ht]
\caption{Summary statistics by birth type}
\label{tab:missing}
\vspace{-7mm}
\begin{center}
\begin{tabular}{lccp{5mm}ccp{5mm}cc} 
%\multicolumn{3}{c}{\textsc{versus non-missing height}}\\
%& & \\
\toprule
  & \multicolumn{2}{c}{Low Income} & & \multicolumn{2}{c}{Lower Middle}& & \multicolumn{2}{c}{Upper Middle} \\ \cmidrule(r){2-3} \cmidrule(r){5-6} \cmidrule(r){8-9} 
 & Single & Twins && Single & Twins&& Single & Twins \\ \midrule
\textsc{Fertility} & & &&&&&& \\																
\begin{footnotesize}\end{footnotesize}	&	\begin{footnotesize}\end{footnotesize}	&	\begin{footnotesize}\end{footnotesize}	&	\begin{footnotesize}\end{footnotesize} &	\begin{footnotesize}\end{footnotesize}	&	\begin{footnotesize}\end{footnotesize}	&	\begin{footnotesize}\end{footnotesize} &	\begin{footnotesize}\end{footnotesize}	&	\begin{footnotesize}\end{footnotesize}		\\
bord	&	3.65	&	4.96	&	&	3.19	&	4.29	&	&	2.85	&	3.71		\\
\begin{footnotesize}\end{footnotesize}	& \begin{footnotesize} (2.38)\end{footnotesize} & \begin{footnotesize} (2.52)\end{footnotesize} & \begin{footnotesize} 	\end{footnotesize} & \begin{footnotesize} (2.19)\end{footnotesize} & \begin{footnotesize} (2.42)\end{footnotesize} & \begin{footnotesize} 	\end{footnotesize} & \begin{footnotesize} (2.05)\end{footnotesize} & \begin{footnotesize} (2.28)\end{footnotesize}	\\
fert	&	5.39	&	6.88	&	&	4.80	&	6.02	&	&	4.12	&	5.11		\\
\begin{footnotesize}\end{footnotesize}	& \begin{footnotesize} (2.70)\end{footnotesize} & \begin{footnotesize} (2.68)\end{footnotesize} & \begin{footnotesize} 	\end{footnotesize} & \begin{footnotesize} (2.59)\end{footnotesize} & \begin{footnotesize} (2.70)\end{footnotesize} & \begin{footnotesize} 	\end{footnotesize} & \begin{footnotesize} (2.48)\end{footnotesize} & \begin{footnotesize} (2.59	)\end{footnotesize}	\\
agemay	&	25.50	&	27.91	&	&	25.06	&	27.17	&	&	24.98	&	26.73		\\
\begin{footnotesize}\end{footnotesize}	& \begin{footnotesize} (6.44)\end{footnotesize} & \begin{footnotesize} (6.13)\end{footnotesize} & \begin{footnotesize} 	\end{footnotesize} & \begin{footnotesize} (5.91)\end{footnotesize} & \begin{footnotesize} (5.82)\end{footnotesize} & \begin{footnotesize} 	\end{footnotesize} & \begin{footnotesize} (5.85)\end{footnotesize} & \begin{footnotesize} (5.77)\end{footnotesize}	\\
\textsc{Socioeconomic} & & &&&&&& \\																
\begin{footnotesize}\end{footnotesize}	&	\begin{footnotesize}\end{footnotesize}	&	\begin{footnotesize}\end{footnotesize}	&	\begin{footnotesize}\end{footnotesize} &	\begin{footnotesize}\end{footnotesize}	&	\begin{footnotesize}\end{footnotesize}	&	\begin{footnotesize}\end{footnotesize} &	\begin{footnotesize}\end{footnotesize}	&	\begin{footnotesize}\end{footnotesize}		\\
educfyrs	&	2.50	&	2.54	&	&	4.53	&	4.64	&	&	6.28	&	6.33		\\
\begin{footnotesize}\end{footnotesize}	& \begin{footnotesize} (3.43)\end{footnotesize} & \begin{footnotesize} (3.47)\end{footnotesize} & \begin{footnotesize} 	\end{footnotesize} & \begin{footnotesize} (4.62)\end{footnotesize} & \begin{footnotesize} (4.81)\end{footnotesize} & \begin{footnotesize} 	\end{footnotesize} & \begin{footnotesize} (4.33)\end{footnotesize} & \begin{footnotesize} (4.45)\end{footnotesize}	\\
eduyears	&	1.34	&	1.12	&	&	2.96	&	2.46	&	&	3.26	&	2.97		\\
\begin{footnotesize}\end{footnotesize}	& \begin{footnotesize} (2.50)\end{footnotesize} & \begin{footnotesize} (2.23)\end{footnotesize} & \begin{footnotesize} 	\end{footnotesize} & \begin{footnotesize} (3.82)\end{footnotesize} & \begin{footnotesize} (3.51)\end{footnotesize} & \begin{footnotesize} 	\end{footnotesize} & \begin{footnotesize} (3.85)\end{footnotesize} & \begin{footnotesize} (3.65	)\end{footnotesize}	\\
attendance	&	0.33	&	0.32	&	&	0.44	&	0.43	&	&	0.52	&	0.50		\\
\begin{footnotesize}\end{footnotesize}	& \begin{footnotesize} (0.47)\end{footnotesize} & \begin{footnotesize} (0.47)\end{footnotesize} & \begin{footnotesize} 	\end{footnotesize} & \begin{footnotesize} (0.50)\end{footnotesize} & \begin{footnotesize} (0.50)\end{footnotesize} & \begin{footnotesize} 	\end{footnotesize} & \begin{footnotesize} (0.50)\end{footnotesize} & \begin{footnotesize} (0.50)\end{footnotesize}	\\
poor1	&	0.00	&	0.01	&	&	0.13	&	0.10	&	&	0.15	&	0.13		\\
\begin{footnotesize}\end{footnotesize}	& \begin{footnotesize} (0.06)\end{footnotesize} & \begin{footnotesize} (0.09)\end{footnotesize} & \begin{footnotesize} 	\end{footnotesize} & \begin{footnotesize} (0.34)\end{footnotesize} & \begin{footnotesize} (0.30)\end{footnotesize} & \begin{footnotesize} 	\end{footnotesize} & \begin{footnotesize} (0.35)\end{footnotesize} & \begin{footnotesize} (0.34)\end{footnotesize}	\\
\textsc{Health} & & &&&&&& \\															
\begin{footnotesize}\end{footnotesize}	&	\begin{footnotesize}\end{footnotesize}	&	\begin{footnotesize}\end{footnotesize}	&	\begin{footnotesize}\end{footnotesize} &	\begin{footnotesize}\end{footnotesize}	&	\begin{footnotesize}\end{footnotesize}	&	\begin{footnotesize}\end{footnotesize} &	\begin{footnotesize}\end{footnotesize}	&	\begin{footnotesize}\end{footnotesize}		\\
height	&	157.30	&	158.71	&	&	154.96	&	156.95	&	&	153.08	&	154.39		\\
\begin{footnotesize}\end{footnotesize}	& \begin{footnotesize} (8.49)\end{footnotesize} & \begin{footnotesize} (7.67)\end{footnotesize} & \begin{footnotesize} 	\end{footnotesize} & \begin{footnotesize} (9.70)\end{footnotesize} & \begin{footnotesize} (15.53	)\end{footnotesize} & \begin{footnotesize} 	\end{footnotesize} & \begin{footnotesize} (6.68)\end{footnotesize} & \begin{footnotesize} (6.51)\end{footnotesize}	\\
bmi	&	21.78	&	22.33	&	&	24.51	&	25.50	&	&	25.67	&	25.80		\\
\begin{footnotesize}\end{footnotesize}	& \begin{footnotesize} (3.60)\end{footnotesize} & \begin{footnotesize} (3.79)\end{footnotesize} & \begin{footnotesize} 	\end{footnotesize} & \begin{footnotesize} (5.65)\end{footnotesize} & \begin{footnotesize} (5.86)\end{footnotesize} & \begin{footnotesize} 	\end{footnotesize} & \begin{footnotesize} (4.52)\end{footnotesize} & \begin{footnotesize} (4.64)\end{footnotesize}	\\
\\ \midrule																
Observations & 582,168 & 13,617 & & 850,863 & 14,804 & & 233,258 & 3,562 \\ 
Obs (Health) & 438,642 & 10,502 & & 515,952 & 9,592 & & 138,779 & 1,942 \\ 
\bottomrule																


\end{tabular}
\end{center}
\end{table}

Child quality is measured using individual educational results collected during DHS surveys.   Survey results are available for `children' 
between the ages of 0 and 41\footnote{Children refer to any offspring of 
mothers living in surveyed households.}, with approximately 75\% of surveyed children 
under the age of 16. Three distinct `quality' variables are examined.  Years of completed education and a dummy indicator of school attendance
are examined in relevant age groups of the population.  Years of education is examined only for those individuals over the age of 16 years, and
attendance is restricted to the subgroup of children between the ages of 6 to 16.     As both of these 
variables would be expected to vary by age, a complete set of age and year of birth dummies are included in all specifications of the Q-Q model.  
Finally, for all individuals over 6 years of age a standardized schooling variable is created.  This is a school Z-score, comparing each child's
schooling attainment with the mean of all individuals in her country and cohort (age) group, and dividing by the standard deviation.  

Recent studies attempting to estimate a child Q-Q specification have restricted their instrumental analysis to singleton births preceeding 
twins.  I follow \citet{Angristetal2010} in defining a number of samples based upon total family fertility.  As per Angrist et al.\ the first sample 
consists of all firstborn children in families in with two or more births and the second sample all first- and second-born children in families 
with at least three total births.  These samples are denominated 2+ and 3+ respectively.  Given the higher fertility rates in the DHS sample I also 
define similar groups for higher parity familise; the 4+, 5+, 6+ and 7+ groups.  These follow logically from the previous two groups.  The estimation 
of instrumental regressions over wider groups of compliers is an important consideration in countries where observed family sizes are larger.  The 
sample size of each group is $N_{2+}= 307,897$, $N_{3+}=445,660$, $N_{4+}= 455,352$, $N_{5+}=407,775$, $N_{6+}=340,980$ and $N_{7+}=264,338$.  
%As each child quality variable is restricted to certain age-groups, the final size of each sample in each regression specification is a portion of this $N$.  
The motiviation 
in restricting to children born before twins is to isolate a plausibly exogenous shock in terms of family investment behaviour.  For those 
individuals born before twins parents will already have formed their optimal family size target and carry out investment per child in line 
with this.  Upon the arrival of twins, the optimal family size may be exceeded in certain cases, resulting in a rearrangement of investment 
patterns.  For those children born after twins such an effect does not exist.  As parents have already internalised the prior arrival of the 
multiple birth event further births can be thought of as once again ocurring according to the prior endogenous mechanism.

  
\section{Results}
\label{scn:results}
\subsection{Twin Endogeneity}
\label{scn:twinendog}
The fundamental assumption for IV validity in a twin estimation of the Q-Q model is conditional exogeneity of twins in line with (\ref{eqn:IVconsist3}).
Primarily I am interested in testing whether the probability of twin birth is correlated with the unobserved error term $u$ in (\ref{eqn:2SLSa}).  
Although this is inherently untestable, it is possible to test whether the probability of twin birth depends upon a range of previously omitted observable 
characteristics of the mother and the family environment.  Whilst the dependence of twinning on observable chartacteristics is no evidence that it will also 
depend upon unobservables, it will cast doubt on the validity of the twin instrumentation strategy -- particularly where the vector of observable 
characteristics is small.  If, for example, the probability of twin birth depends upon observed measures of a mother's health such as height and weight, 
it also seems reasonable to suggest that unmeasured or unobserved health measures such as pre-natal medical visits and a mother's behaviour and diet
are also related to the likelihood of live twin births. 
 
Using a linear probability model the following twin birth equation is estimated: 
\begin{equation}
\label{eqn:twinpred}
P(Twin_{ijk}=1)=\delta_1 B_i + W'_j\vect{\delta_\vect{W}}+ T'_{i}\vect{\delta_\vect{T}} +  \phi_k + \varepsilon_{ijk}
\end{equation}
Here $Twin_{ijk}$ takes the value one if child $i$ from family $j$ and country $k$ is a twin, and zero otherwise.  Full year of birth ($T$) and country
dummies ($\phi$) are included.  The observable parameters of interest $W'_j$ are family level variables representing characteristics such as maternal age and
education, maternal height and BMI, and a dummy variable if the family is classed as poor (the lowest quintile) in the DHS sample data. The results from this 
regression are included as table \ref{tab:twinreg1}.

\begin{table}[ht]
\caption{Probability of giving birth to}
\vspace{-7mm}
\label{tab:twinreg1}
\begin{center}
\begin{tabular}{lcccc} 
\multicolumn{5}{c}{\textsc{multiple children}}\\
& & \\
\toprule
 & (1) & (2) & (3) & (4) \\
\textsc{Twin=1} & Pooled & Low & Lower- & Upper- \\ 
 & Sample & Income & Middle & Middle \\  \midrule
\vspace{4pt} & \begin{footnotesize}\end{footnotesize} & \begin{footnotesize}\end{footnotesize} & \begin{footnotesize}\end{footnotesize} & \begin{footnotesize}\end{footnotesize} \\
birth order & 0.544** & 0.698** & 0.462** & 0.341** \\
\vspace{4pt} & \begin{footnotesize}(0.0128)\end{footnotesize} & \begin{footnotesize}(0.0222)\end{footnotesize} & \begin{footnotesize}(0.0175)\end{footnotesize} & \begin{footnotesize}(0.0332)\end{footnotesize} \\
mother's age & -0.0360** & -0.0744** & -0.0227** & 0.00782 \\
\vspace{4pt} & \begin{footnotesize}(0.00415)\end{footnotesize} & \begin{footnotesize}(0.00734)\end{footnotesize} & \begin{footnotesize}(0.00574)\end{footnotesize} & \begin{footnotesize}(0.00995)\end{footnotesize} \\
mother educ 0 & -0.947** & -0.737** & -0.815** & -0.897** \\
\vspace{4pt} & \begin{footnotesize}(0.0650)\end{footnotesize} & \begin{footnotesize}(0.148)\end{footnotesize} & \begin{footnotesize}(0.0851)\end{footnotesize} & \begin{footnotesize}(0.173)\end{footnotesize} \\
mother educ 1--4 & -0.847** & -0.632** & -0.739** & -0.721** \\
\vspace{4pt} & \begin{footnotesize}(0.0668)\end{footnotesize} & \begin{footnotesize}(0.152)\end{footnotesize} & \begin{footnotesize}(0.0910)\end{footnotesize} & \begin{footnotesize}(0.140)\end{footnotesize} \\
mother educ 5--6 & -0.665** & -0.611** & -0.496** & -0.544** \\
\vspace{4pt} & \begin{footnotesize}(0.0674)\end{footnotesize} & \begin{footnotesize}(0.156)\end{footnotesize} & \begin{footnotesize}(0.0942)\end{footnotesize} & \begin{footnotesize}(0.128)\end{footnotesize} \\
mother educ 7--10 & -0.408** & -0.0715 & -0.378** & -0.480** \\
\vspace{4pt} & \begin{footnotesize}(0.0651)\end{footnotesize} & \begin{footnotesize}(0.152)\end{footnotesize} & \begin{footnotesize}(0.0884)\end{footnotesize} & \begin{footnotesize}(0.126)\end{footnotesize} \\
height & 0.0413** & 0.0463** & 0.0389** & 0.0274** \\
\vspace{4pt} & \begin{footnotesize}(0.00257)\end{footnotesize} & \begin{footnotesize}(0.00416)\end{footnotesize} & \begin{footnotesize}(0.00368)\end{footnotesize} & \begin{footnotesize}(0.00682)\end{footnotesize} \\
BMI & 0.0266** & 0.0429** & 0.0247** & 0.00784 \\
\vspace{4pt} & \begin{footnotesize}(0.00375)\end{footnotesize} & \begin{footnotesize}(0.00771)\end{footnotesize} & \begin{footnotesize}(0.00494)\end{footnotesize} & \begin{footnotesize}(0.00818)\end{footnotesize} \\
Poor & -0.0884 & 2.720** & -0.112 & -0.250* \\
\vspace{4pt} & \begin{footnotesize}(0.0511)\end{footnotesize} & \begin{footnotesize}(0.570)\end{footnotesize} & \begin{footnotesize}(0.0577)\end{footnotesize} & \begin{footnotesize}(0.106)\end{footnotesize} \\
%Constant & -5.614** & -5.425** & -6.364** & -4.648** \\
% & \begin{footnotesize}(0.486)\end{footnotesize} & \begin{footnotesize}(0.787)\end{footnotesize} & \begin{footnotesize}(0.707)\end{footnotesize} & \begin{footnotesize}(1.195)\end{footnotesize} \\
\vspace{4pt} & \begin{footnotesize}\end{footnotesize} & \begin{footnotesize}\end{footnotesize} & \begin{footnotesize}\end{footnotesize} & \begin{footnotesize}\end{footnotesize} \\
Observations & 1,111,232 & 447,365 & 523,146 & 140,721 \\
 $R^2$ & 0.009 & 0.011 & 0.009 & 0.005 \\ \midrule
\multicolumn{5}{c}{\begin{footnotesize} Robust standard errors clustered by country \end{footnotesize}} \\
\multicolumn{5}{c}{\begin{footnotesize} ** p$<$0.01, * p$<$0.05 \end{footnotesize}} \\
\bottomrule
\multicolumn{5}{p{10cm}}{\setstretch{0.9}\begin{footnotesize}\textsc{Note:} All specifications include full year of birth and country
dummies and are estimated as linear probability models.  Results are robust to the inclusion of birth order dummies (rather than a single discrete 
variable), however are excluded here for simplicity.  Twin is multiplied by 100 to simplify coefficients and standard errors.  Height is measured
in cm and Poor is a dummy variable based upon household possessions.\end{footnotesize}}\\
\end{tabular}
\end{center}
\end{table}
\setstretch{1.5}

These results suggest that twin births are not random, even after conditioning on mother's age and child birth order.  The
inclusion of a full set of country and year of birth dummies (not reported in table \ref{tab:twinreg1}) should capture any trend in
probability of twin birth across time or regions.  The estimated coefficients and signs support the idea discussed in section \ref{scn:intro}
that higher `investments' (for example in mother's health) required to maintain multiple healthy fetuses \emph{in utero} may result in 
non-random twin births. 
Initially results from the pooled DHS data are consulted as this provides a particularly large sample with which to test the hypothesis of
twin exogeneity.  This is represented in table \ref{tab:twinreg1} column 1 and provides considerable evidence that live multiple
births respond to family `choice' variables such as education (tests for the joint significance of both socioeconomic variables and health
variables are rejected with p-values of 0.0000).

However, if the reason non-random twin births are observed is due to insufficient investment in the developing fetus, it seems likely that
the twin `selection' equation is likely to be more pronounced in lower income settings, and settings where less external support is available
to the mother during gestation. \footnote{\texttt{DAMIAN: IT WOULD BE INTERESTING TO TEST THIS WITH SOME RESOURCE SHOCK WHICH ARRIVES DURING
THE THIRD TRIMESTER (EG).}}  This is tested in columns (2)-(4), where it is shown that the violation of the twin exclusion restriction is particularly
strong in low income countries.  Here maternal health is a more important predictor, and the explained portion of this set of variables is more than
two times that in upper-middle-income countries\footnote{The low $R^2$ in these regressions is not at all surprising given that twin contraception
can be thought of as an approximately random process.  The fact that socioeconomic and health variables have \emph{any} power in explaining
twin birth however is sufficient to invalidate IV estimations if these or other relevant predictors are not controlled for.}.

These results call into question the veracity of assumption (\ref{eqn:IVconsist3}), and imply that a 2SLS regression omitting factors such as family
income, maternal health and maternal education would be inconsistent, at least in the case of the data from the DHS surveys.  In what follows
I will examine the implication of this finding first by examining the importance of omitted variables in a simulated Q-Q model based upon
correlations similar to those from the available DHS data, and then by examining empirically the performance of these IV estimates.  
\subsection{Bias in IV Regressions}
\label{scn:bias}
\subsubsection{Monte Carlo Simulation}
\label{scn:MCS}
The results from section \ref{scn:twinendog} suggest that twin birth is not exogenous when conditioning on only a limited set of variables. As a result, 
IV estimates failing to control for the full set of 
relevant factors in the second stage will produce inconsistent estimates of the importance of family size on child quality.  As a first 
approach at quantifying the importance of this bias, Monte Carlo simulations are run allowing for correlation between twin birth and previously uncontrolled factors
to vary.\footnote{Whilst it may seem unlikely that the correlation between twin births and these unobserved factors will vary over time and hence that
simulation of a range of values is un-ilustrative, these simulations may be useful in abstracting to other situations.  This analysis is identical to the
case where multiple births occurr more frequently where in-vitro fertilisation (IVF) is used.  As IVF treatment often involves the implantation of 
multiple embryos, multiple births are more likely here compared with traditional methods of fertilisation \citep{Beraletal1990}.  These simulations then could be 
thought as analogous to examining the role increasing use of IVF plays on the consistency of twin estimates where only birth records are available, 
but details regarding fertilisation are unknown.}  This allows for the examination of the importance of the instrumental validity assumption, essentially
by simulating values of equation (\ref{eqn:IVconsist3}) which increasingly diverge from zero. 

I simulate an ($n \times 3)$ vector of $N=100,000$ standard normal error terms for the following system of equations.
\begin{eqnarray}
\label{eqn:MC1}
y_i=\beta_0+\beta_1 x_i + \varepsilon_{1i} \nonumber\\
x_i=\gamma_0 + \gamma_1 z_i + \varepsilon_{2i} \nonumber\\
z_i=1[\alpha_0+\varepsilon_{3i}>0], \nonumber
\end{eqnarray} 
where simulated $\varepsilon_j$ follow:
\begin{equation}
\begin{bmatrix}
\varepsilon_1\\
\varepsilon_2\\
\varepsilon_3\\
\end{bmatrix}
\sim \mathcal{N}
\left(\begin{bmatrix}
0\\
0\\
0\\
\end{bmatrix}
,
\begin{bmatrix}
\sigma_{\varepsilon_1}^2 &  \rho_{\varepsilon_1\varepsilon_2} &  \rho_{\varepsilon_1\varepsilon_3}\\
\rho_{\varepsilon_2\varepsilon_1} & \sigma_{\varepsilon_2}^2 &  \rho_{\varepsilon_2\varepsilon_3} \\
\rho_{\varepsilon_3\varepsilon_1}&  \rho_{\varepsilon_3\varepsilon_2}& \sigma_{\varepsilon_3}^2 \\
\end{bmatrix}\right).
\end{equation}

This system of equations follows a Q-Q type setup, in which $y_i$ can be thought of as a quality variable such as years of schooling, and $x_i$ represents
sibship size which depends upon twin births $z_i$. The true population value of $\beta_1$ is defined as -0.15 and the value of $\gamma_1$ is 0.5.  These
values imply that the Q-Q hypothesis is correct ($\beta_1<0$), and that parents partially offset twin births by reducing future births ($0<\gamma_1<1$).
Whilst the true value for $\beta_1$ is not known, the sample value of $\gamma_1$ can be plausibly estimated from the DHS data.  The  value chosen is consistent
with a range of twin births at different parities in the sample data.
A negative correlation is defined to exist between $\varepsilon_1$ and $\varepsilon_2$ ($\rho_{\varepsilon_1\varepsilon_2}=-0.3$) in line with omitted variables such as
parental preference for education which drive both lower family size and higher educational attainment, and a positive correlation between $z$ and $x$ for
instrumental relevance. %($\rho_{\varepsilon_2\varepsilon_3}=0.67$). 
Finally $\rho_{\varepsilon_1\varepsilon_3}$ is allowed to vary from 0 to 0.5.  This allows
us to observe the behaviour of the IV strategy under violations of the instrumental consistency assumption (\ref{eqn:IVconsist3}) which increase in magnitude.

\begin{figure}[!htbp]
\caption{Simulations of IV and OLS consistency}
\label{fig:MC}
\begin{center}
\vspace{-4mm}
\includegraphics[scale=0.36]{Sim4_uz_20120401.eps}
\end{center}
\end{figure}


Simulation results are presented in figure \ref{fig:MC} and suggest that even a small (non-zero) correlation between $z$ and $\varepsilon_1$ can introduce an 
important bias in IV estimates.  Whilst OLS estimates consistently overestimate the importance of the Q-Q tradeoff ($\hat{\beta}_{1,OLS}=-0.2436 (0.0023)$), a correlation
as small as 0.05 between the instrument and the error in the Q-Q equation results in an estimate of no tradeoff when in reality the true effect is negative.\footnote{The
simulation described above assumes that $x$ is a continuous variable.  Alternate simulations were run modeling $x$ as a discrete count variable in order to more closely resemble the fertility size variable in our true data set.  These simulations assumed that $x$ depends upon a latent variable $x^*$, and behaves as follows:
\[ x = \left\{ \begin{array}{ll}
         1 & \mbox{if $x^* < 2$}\\
         2 & \mbox{if $2\leq x^* \leq 3$}\\
        \vdots & \\
        11 & \mbox{if $x > 11$}.\end{array} \right. \]
 The results from section \ref{scn:MCS} were found to be robust to the discretisation of $x$.
}
\vspace{-5mm}
\subsubsection{Empirical Evidence}
\label{scn:EE}
\vspace{-5mm}
The Q-Q specification (\ref{eqn:2SLSa}-\ref{eqn:2SLSb}) is fitted to the pooled DHS data using OLS and IV with a limited set of controls and with the full set of observed 
controls suggested as important in the probability of twin birth equation (table \ref{tab:twinreg1}).  A set of controls consistent with the recent literature on twin births (which 
include socioeconomic variables such as education but omit health variables) are also included, to examine the importance of these omitted factors.  The results for one outcome variable (years of schooling) and controls are presented in table \ref{tab:YrsEduc}.  Here for simplicity only two fertility groups are reported which are representative
of twinning at relatively low- and relatively high-parity births.


%________________________________________________________________________________________________________________________________
\begin{sidewaystable}[!htbp]																					
\caption{Q-Q specification with years of schooling as quality}																					
\vspace{-3mm}																					
\label{tab:YrsEduc}																					
\begin{center}																					
\begin{tabular}{lcccccccccc} \toprule																					
& \multicolumn{5}{c}{2 +} & \multicolumn{5}{c}{5 +} \\ \cmidrule(r){2-6} \cmidrule(r){7-11}																					
& OLS  & OLS & IV & IV  & IV  & OLS  & OLS & IV  & IV  & IV \\ 																					
& No control & All & No control &  Socioec. & + Health & No control & All & No control &  Socioec. & + Health \\ \midrule
\begin{footnotesize}\end{footnotesize}&\begin{footnotesize}\end{footnotesize}&\begin{footnotesize}\end{footnotesize}&\begin{footnotesize}\end{footnotesize}&\begin{footnotesize}\end{footnotesize}&\begin{footnotesize}\end{footnotesize}&\begin{footnotesize}\end{footnotesize}&\begin{footnotesize}\end{footnotesize}&\begin{footnotesize}\end{footnotesize}&\begin{footnotesize}\end{footnotesize}&\begin{footnotesize}\end{footnotesize}\\																					
fertility	&	-0.588**	&	-0.309**	&	0.915	&	0.260	&	0.118	&	-0.492**	&	-0.298**	&	0.0254	&	-0.149	&	-0.246	\\
\vspace{4pt} & \begin{footnotesize}		(0.00962)	\end{footnotesize} & \begin{footnotesize}	(0.00914)	\end{footnotesize} & \begin{footnotesize}	(0.776)	\end{footnotesize} & \begin{footnotesize}	(0.482)	\end{footnotesize} & \begin{footnotesize}	(0.449)	\end{footnotesize} & \begin{footnotesize}	(0.00857)	\end{footnotesize} & \begin{footnotesize}	(0.00810)	\end{footnotesize} & \begin{footnotesize}	(0.286)	\end{footnotesize} & \begin{footnotesize}	(0.247)	\end{footnotesize} & \begin{footnotesize}	(0.238)	\end{footnotesize} \\
mother educ 0	&		&	-4.337**	&		&	-5.585**	&	-5.052**	&		&	-3.344**	&		&	-4.753**	&	-4.353**	\\
\vspace{4pt} & \begin{footnotesize}			\end{footnotesize} & \begin{footnotesize}	(0.0641)	\end{footnotesize} & \begin{footnotesize}		\end{footnotesize} & \begin{footnotesize}	(0.851)	\end{footnotesize} & \begin{footnotesize}	(0.754)	\end{footnotesize} & \begin{footnotesize}		\end{footnotesize} & \begin{footnotesize}	(0.0511)	\end{footnotesize} & \begin{footnotesize}		\end{footnotesize} & \begin{footnotesize}	(0.309)	\end{footnotesize} & \begin{footnotesize}	(0.282)	\end{footnotesize} \\
mother educ 1--4	&		&	-2.874**	&		&	-3.892**	&	-3.488**	&		&	-1.955**	&		&	-3.245**	&	-2.953**	\\
\vspace{4pt} & \begin{footnotesize}			\end{footnotesize} & \begin{footnotesize}	(0.0682)	\end{footnotesize} & \begin{footnotesize}		\end{footnotesize} & \begin{footnotesize}	(0.725)	\end{footnotesize} & \begin{footnotesize}	(0.649)	\end{footnotesize} & \begin{footnotesize}		\end{footnotesize} & \begin{footnotesize}	(0.0546)	\end{footnotesize} & \begin{footnotesize}		\end{footnotesize} & \begin{footnotesize}	(0.251)	\end{footnotesize} & \begin{footnotesize}	(0.231)	\end{footnotesize} \\
mother educ 5--6	&		&	-1.650**	&		&	-2.264**	&	-2.036**	&		&	-0.847**	&		&	-2.003**	&	-1.828**	\\
\vspace{4pt} & \begin{footnotesize}			\end{footnotesize} & \begin{footnotesize}	(0.0665)	\end{footnotesize} & \begin{footnotesize}		\end{footnotesize} & \begin{footnotesize}	(0.455)	\end{footnotesize} & \begin{footnotesize}	(0.411)	\end{footnotesize} & \begin{footnotesize}		\end{footnotesize} & \begin{footnotesize}	(0.0576)	\end{footnotesize} & \begin{footnotesize}		\end{footnotesize} & \begin{footnotesize}	(0.174)	\end{footnotesize} & \begin{footnotesize}	(0.162)	\end{footnotesize} \\
mother educ 7--10	&		&	-0.871**	&		&	-1.229**	&	-1.121**	&		&		&		&	-1.045**	&	-0.971**	\\
\vspace{4pt} & \begin{footnotesize}			\end{footnotesize} & \begin{footnotesize}	(0.0631)	\end{footnotesize} & \begin{footnotesize}		\end{footnotesize} & \begin{footnotesize}	(0.294)	\end{footnotesize} & \begin{footnotesize}	(0.271)	\end{footnotesize} & \begin{footnotesize}		\end{footnotesize} & \begin{footnotesize}		\end{footnotesize} & \begin{footnotesize}		\end{footnotesize} & \begin{footnotesize}	(0.131)	\end{footnotesize} & \begin{footnotesize}	(0.126)	\end{footnotesize} \\
Poor	&		&	-1.937**	&		&	-2.521**	&	-2.243**	&		&	-1.815**	&		&	-2.064**	&	-1.847**	\\
\vspace{4pt} & \begin{footnotesize}			\end{footnotesize} & \begin{footnotesize}	(0.0691)	\end{footnotesize} & \begin{footnotesize}		\end{footnotesize} & \begin{footnotesize}	(0.378)	\end{footnotesize} & \begin{footnotesize}	(0.329)	\end{footnotesize} & \begin{footnotesize}		\end{footnotesize} & \begin{footnotesize}	(0.0522)	\end{footnotesize} & \begin{footnotesize}		\end{footnotesize} & \begin{footnotesize}	(0.170)	\end{footnotesize} & \begin{footnotesize}	(0.155)	\end{footnotesize} \\
Height	&		&	0.0244**	&		&		&	0.0280**	&		&	0.0178**	&		&		&	0.0182**	\\
\vspace{4pt} & \begin{footnotesize}			\end{footnotesize} & \begin{footnotesize}	(0.00279)	\end{footnotesize} & \begin{footnotesize}		\end{footnotesize} & \begin{footnotesize}		\end{footnotesize} & \begin{footnotesize}	(0.00472)	\end{footnotesize} & \begin{footnotesize}		\end{footnotesize} & \begin{footnotesize}	(0.00229)	\end{footnotesize} & \begin{footnotesize}		\end{footnotesize} & \begin{footnotesize}		\end{footnotesize} & \begin{footnotesize}	(0.00287)	\end{footnotesize} \\
BMI	&		&	0.0700**	&		&		&	0.0810**	&		&	0.0864**	&		&		&	0.0875**	\\
\vspace{4pt} & \begin{footnotesize}			\end{footnotesize} & \begin{footnotesize}	(0.00346)	\end{footnotesize} & \begin{footnotesize}		\end{footnotesize} & \begin{footnotesize}		\end{footnotesize} & \begin{footnotesize}	(0.0121)	\end{footnotesize} & \begin{footnotesize}		\end{footnotesize} & \begin{footnotesize}	(0.00287)	\end{footnotesize} & \begin{footnotesize}		\end{footnotesize} & \begin{footnotesize}		\end{footnotesize} & \begin{footnotesize}	(0.00582)	\end{footnotesize} \\
Male	&	-0.104**	&	0.151**	&	-0.142**	&	0.159**	&	0.179**	&	0.124**	&	0.342**	&	0.0792*	&	0.300**	&	0.340**	\\
\vspace{4pt} & \begin{footnotesize}		(0.0380)	\end{footnotesize} & \begin{footnotesize}	(0.0343)	\end{footnotesize} & \begin{footnotesize}	(0.0519)	\end{footnotesize} & \begin{footnotesize}	(0.0455)	\end{footnotesize} & \begin{footnotesize}	(0.0462)	\end{footnotesize} & \begin{footnotesize}	(0.0302)	\end{footnotesize} & \begin{footnotesize}	(0.0279)	\end{footnotesize} & \begin{footnotesize}	(0.0397)	\end{footnotesize} & \begin{footnotesize}	(0.0296)	\end{footnotesize} & \begin{footnotesize}	(0.0286)	\end{footnotesize} \\
mother's age &	0.128**	&	0.0502**	&	0.309**	&	0.0948*	&	0.0824*	&	0.0449**	&	0.0318**	&	0.0834**	&	0.0395*	&	0.0353*	\\
\vspace{4pt} & \begin{footnotesize}		(0.00606)	\end{footnotesize} & \begin{footnotesize}	(0.00559)	\end{footnotesize} & \begin{footnotesize}	(0.0938)	\end{footnotesize} & \begin{footnotesize}	(0.0370)	\end{footnotesize} & \begin{footnotesize}	(0.0343)	\end{footnotesize} & \begin{footnotesize}	(0.00516)	\end{footnotesize} & \begin{footnotesize}	(0.00477)	\end{footnotesize} & \begin{footnotesize}	(0.0219)	\end{footnotesize} & \begin{footnotesize}	(0.0170)	\end{footnotesize} & \begin{footnotesize}	(0.0166)	\end{footnotesize} \\
%Birth Order 2	&		&		&		&		&		&	-3.79e-05	&	-0.0234	&	-0.280*	&	-0.0826	&	-0.0504	\\
%\vspace{4pt} & \begin{footnotesize}			\end{footnotesize} & \begin{footnotesize}		\end{footnotesize} & \begin{footnotesize}		\end{footnotesize} & \begin{footnotesize}		\end{footnotesize} & \begin{footnotesize}		\end{footnotesize} & \begin{footnotesize}	(0.0395)	\end{footnotesize} & \begin{footnotesize}	(0.0364)	\end{footnotesize} & \begin{footnotesize}	(0.160)	\end{footnotesize} & \begin{footnotesize}	(0.133)	\end{footnotesize} & \begin{footnotesize}	(0.129)	\end{footnotesize} \\
%Birth Order 3	&		&		&		&		&		&	0.124**	&	0.0689	&	-0.431	&	-0.0399	&	0.0153	\\
%\vspace{4pt} & \begin{footnotesize}			\end{footnotesize} & \begin{footnotesize}		\end{footnotesize} & \begin{footnotesize}		\end{footnotesize} & \begin{footnotesize}		\end{footnotesize} & \begin{footnotesize}		\end{footnotesize} & \begin{footnotesize}	(0.0458)	\end{footnotesize} & \begin{footnotesize}	(0.0423)	\end{footnotesize} & \begin{footnotesize}	(0.310)	\end{footnotesize} & \begin{footnotesize}	(0.256)	\end{footnotesize} & \begin{footnotesize}	(0.249)	\end{footnotesize} \\
																					
																					
\vspace{4pt} & \begin{footnotesize}\end{footnotesize} &  \begin{footnotesize}\end{footnotesize} & \begin{footnotesize}\end{footnotesize} & \begin{footnotesize}\end{footnotesize} & \begin{footnotesize}\end{footnotesize} & \begin{footnotesize}\end{footnotesize} & \begin{footnotesize}\end{footnotesize} & \begin{footnotesize}\end{footnotesize} & \begin{footnotesize}\end{footnotesize} & \begin{footnotesize}\end{footnotesize} \\																					
Observations & 39,946 & 39,946 & 39,946 & 39,946 & 39,946 & 65,685 & 65,685 & 65,685 & 65,685 & 65,685 \\																					
$R^2$ & 0.254 & 0.367 & & & &0.325 & 0.454 &  & & \\ \midrule																					
\multicolumn{11}{c}{\begin{footnotesize} Robust standard errors in parentheses\end{footnotesize}} \\																					
\multicolumn{11}{c}{\begin{footnotesize} ** p$<$0.01, * p$<$0.05 \end{footnotesize}} \\																					
\bottomrule																					
\multicolumn{11}{p{21cm}}{\setstretch{0.9}\begin{footnotesize}\textsc{Note:} All specifications include a full set of country, age, and year of birth controls, along with complete birth order dummies. Robust standard errors are clustered by country.	No controls refers to the basic IV model which assumes that twins are exogenous conditioning only on mother's age and parity, while Socioec. refers to an IV
model which assumes conditional exogeneity when controlling for education and income along with basic controls.	  Health is the new IV model proposed here where all variables determined to influence twin birth are included. 	
\end{footnotesize}}\\																					
\end{tabular}																					
\end{center}																					
\end{sidewaystable}																					
\setstretch{1.5}																					



%_________________________________________________________________________________________________________________________________

As expected children's completed schooling depends positively upon mother's education (with decreasing returns to higher education), 
and positively upon mother's health (proxied by height and BMI) and age at birth of child.  The quality outcome responds negatively to 
years of education and higher birth order dummies.  The results for the Q-Q model act in a similar manner as predicted in the Monte
Carlo simulations.  Whilst OLS finds a significantly negative effect of higher fertility on quality, IV estimates are significantly
more positive.  Indeed, failure to include the full set of observed maternal controls in the IV estimation results in positive point estimate
for the effect of family size on child quality -- a result which if statisticially significant would imply that children in larger families achieve  
higher education levels, perhaps suggestive of positive spill-overs. 


Table \ref{tab:fertilityALL} describes the effect of fertility on all educational quality variables examined, and in all three country groups.  The results in this table represent one parity group (4+).  Full results for point estimates and confidence intervals across the entire range of parity groups are presented in figure \ref{fig:YrS}, appendix \ref{scn:Graphs}. The coefficients on further control variables are suppressed for ease of presentation. Estimates on all three outcome variables behave as in simulations, and suggest that the non-exogeneity of twins is quantitatively important to the consistency of IV estimates.  This is particularly true for attendance and schooling z-score, where failure to account for the non-random nature of twin birth (the NC column) results in positive (but insignificant) point estimates, whereas the inclusion of full health and socioeconomic controls produces negative, but once again insignificant, point estimates.

Overall the IV results suggest that no Q-Q tradeoff exists \emph{if} twinning is exogenous when conditioning on the full set of controls included in (\ref{eqn:twinpred}).  As 
discussed in section \ref{scn:twinendog}, this relies fundamentally on the ability of observable characteristics to predict twin births.  Where the vector of observables only goes
part way to explaining twin likelihood, the true estimate of the Q-Q tradeoff will be expected to lie between the negative and significant OLS estimate (column b) and the estimate 
from IV which is not statistically different from zero (column d).  This follows given that in the case of OLS it is likely that parental unobervables such as desire for education are likely to be positively correlated with child `quality' but negatively related to family size, thus biasing downward estimates of $\beta$.  The opposite will be the case for IV, where the concern involves omitted twin selection variables.  Variables such as maternal diet during fetal gestation will be both positively related to family size (via higher likelihood of twin birth) and to their child's eventual educational attainment (see for example \citet{Barker1995} for a discussion of the fetal origins hypothesis.)  [\texttt{Perhaps I should show this formally as an appendix or footnote.}]  It is interesting to note that there is some evidence that the Q-Q hypothesis is stronger in poorer country groups.  Estimates of $\beta_{Fert}$ are nearly universally lower in these countries, for both OLS and IV estimation strategies.  I will return to this point in the following section which aims to determine the plausibility that \emph{any} quantity-quality trade-off exists.



\begin{table}[!htbp]														
\caption{Q-Q estimates for all outcome variables}														
\vspace{-3mm}														
\label{tab:fertilityALL}														
\begin{center}														
\begin{tabular}{lcccccc} \toprule														
& (a) & (b) & (c) & (d) & (e) & (f)\\														
& OLS & OLS & IV & IV & IV & Implied \\														
& NC & S+H & NC & S & S+H & Ratio\textsuperscript{a} \\														
\midrule														
\textsc{Years of Schooling} & & & & & &\\														
Pooled &		-0.554**	&	-0.317***	&	0.201	&	0.0544	&	0.0174	&	1.34		\\
\vspace{4pt} &	\begin{footnotesize}	(0.00784)	\end{footnotesize} & \begin{footnotesize}	(0.00744)	\end{footnotesize} & \begin{footnotesize}	(0.358)	\end{footnotesize} & \begin{footnotesize}	(0.296)	\end{footnotesize} & \begin{footnotesize}	(0.290)	\end{footnotesize} & \begin{footnotesize}		\end{footnotesize}	\\
Low Income &		-0.378**	&	-0.216**	&	-0.0785	&	-0.0944	&	0.0252	&	2.33		\\
\vspace{4pt} &	\begin{footnotesize}	(0.0132)	\end{footnotesize} & \begin{footnotesize}	(0.0121)	\end{footnotesize} & \begin{footnotesize}	(0.471)	\end{footnotesize} & \begin{footnotesize}	(0.426)	\end{footnotesize} & \begin{footnotesize}	(0.426)	\end{footnotesize} & \begin{footnotesize}		\end{footnotesize}	\\
Lower Middle &		-0.641**	&	-0.377**	&	0.563	&	0.125	&	0.0225	&	1.43		\\
\vspace{4pt} &	\begin{footnotesize}	(0.0110)	\end{footnotesize} & \begin{footnotesize}	(0.0105)	\end{footnotesize} & \begin{footnotesize}	(0.635)	\end{footnotesize} & \begin{footnotesize}	(0.469)	\end{footnotesize} & \begin{footnotesize}	(0.443)	\end{footnotesize} & \begin{footnotesize}		\end{footnotesize}	\\
Upper Middle &		-0.583**	&	-0.308**	&	-0.0839	&	0.123	&	0.115	&	1.12		\\
\vspace{4pt} &	\begin{footnotesize}	(0.0189)	\end{footnotesize} & \begin{footnotesize}	(0.0185)	\end{footnotesize} & \begin{footnotesize}	(0.744)	\end{footnotesize} & \begin{footnotesize}	(0.707)	\end{footnotesize} & \begin{footnotesize}	(0.704)	\end{footnotesize} & \begin{footnotesize}		\end{footnotesize}	\\
\midrule														
\textsc{P(Attendance)=1} & & & & & &\\														
Pooled &		-0.0272**	&	-0.0111**	&	0.00140	&	-0.00613	&	-0.00901	&	0.689		\\
\vspace{4pt} &	\begin{footnotesize}	(0.000835)	\end{footnotesize} & \begin{footnotesize}	(0.000821)	\end{footnotesize} & \begin{footnotesize}	(0.0135)	\end{footnotesize} & \begin{footnotesize}	(0.0127)	\end{footnotesize} & \begin{footnotesize}	(0.0127)	\end{footnotesize} & \begin{footnotesize}		\end{footnotesize}	\\
Low Income &		-0.00654**	&	0.00371**	&	-0.01	&	-0.0179	&	-0.0245	&	N/A		\\
\vspace{4pt} &	\begin{footnotesize}	(0.00131)	\end{footnotesize} & \begin{footnotesize}	(0.00127)	\end{footnotesize} & \begin{footnotesize}	(0.0203)	\end{footnotesize} & \begin{footnotesize}	(0.0194)	\end{footnotesize} & \begin{footnotesize}	(0.0193)	\end{footnotesize} & \begin{footnotesize}		\end{footnotesize}	\\
Lower Middle &		-0.0476**	&	-0.0276**	&	0.0197	&	0.00461	&	0.00372	&	1.38		\\
\vspace{4pt} &	\begin{footnotesize}	(0.00123)	\end{footnotesize} & \begin{footnotesize}	(0.00121)	\end{footnotesize} & \begin{footnotesize}	(0.0201)	\end{footnotesize} & \begin{footnotesize}	(0.0186)	\end{footnotesize} & \begin{footnotesize}	(0.0184)	\end{footnotesize} & \begin{footnotesize}		\end{footnotesize}	\\
Upper Middle &		-0.0300**	&	-0.0142**	&	-0.00889	&	-0.00213	&	-0.00523	&	0.899		\\
\vspace{4pt} &	\begin{footnotesize}	(0.00211)	\end{footnotesize} & \begin{footnotesize}	(0.00218)	\end{footnotesize} & \begin{footnotesize}	(0.0370)	\end{footnotesize} & \begin{footnotesize}	(0.0375)	\end{footnotesize} & \begin{footnotesize}	(0.0371)	\end{footnotesize} & \begin{footnotesize}		\end{footnotesize}	\\
\midrule														
\textsc{School Z-Score} & & & & & &\\														
Pooled &		-0.125**	&	-0.0778**	&	0.0219	&	-0.00863	&	-0.0162	&	1.65		\\
\vspace{4pt} &	\begin{footnotesize}	(0.00123)	\end{footnotesize} & \begin{footnotesize}	(0.00118)	\end{footnotesize} & \begin{footnotesize}	(0.0277)	\end{footnotesize} & \begin{footnotesize}	(0.0247)	\end{footnotesize} & \begin{footnotesize}	(0.0244)	\end{footnotesize} & \begin{footnotesize}		\end{footnotesize}	\\
Low Income &		-0.102**	&	-0.0730**	&	-0.00256	&	-0.0284	&	-0.0376	&	2.52		\\
\vspace{4pt} &	\begin{footnotesize}	(0.00187)	\end{footnotesize} & \begin{footnotesize}	(0.00176)	\end{footnotesize} & \begin{footnotesize}	(0.0354)	\end{footnotesize} & \begin{footnotesize}	(0.0325)	\end{footnotesize} & \begin{footnotesize}	(0.0321)	\end{footnotesize} & \begin{footnotesize}		\end{footnotesize}	\\
Lower Middle &		-0.142**	&	-0.0875**	&	0.0324	&	-0.0231	&	-0.0303	&	1.61		\\
\vspace{4pt} &	\begin{footnotesize}	(0.00186)	\end{footnotesize} & \begin{footnotesize}	(0.00179)	\end{footnotesize} & \begin{footnotesize}	(0.0447)	\end{footnotesize} & \begin{footnotesize}	(0.0387)	\end{footnotesize} & \begin{footnotesize}	(0.0382)	\end{footnotesize} & \begin{footnotesize}		\end{footnotesize}	\\
Upper Middle &		-0.136**	&	0.0665**	&	0.0924	&	0.119	&	0.114	&	0.957		\\
\vspace{4pt} &	\begin{footnotesize}	(0.00335)	\end{footnotesize} & \begin{footnotesize}	(0.00327)	\end{footnotesize} & \begin{footnotesize}	(0.0849)	\end{footnotesize} & \begin{footnotesize}	(0.0798)	\end{footnotesize} & \begin{footnotesize}	(0.0788)	\end{footnotesize} & \begin{footnotesize}		\end{footnotesize}	\\
\midrule														
\multicolumn{7}{c}{\begin{footnotesize} Robust standard errors in parentheses\end{footnotesize}} \\														
\multicolumn{7}{c}{\begin{footnotesize} ** p$<$0.01, * p$<$0.05 \end{footnotesize}} \\														
\bottomrule														
\multicolumn{7}{p{14.2cm}}{\setstretch{0.9}\begin{footnotesize}\textsc{Note:} All specifications include a full set of country, age, and year of birth controls. Robust standard errors are clustered by country. In column headings, NC refers to No controls, S refers to socioeconomic controls, and S+H refers to all controls (socioec.\ and health).  For a more complete description see the note to table \ref{tab:YrsEduc}.\end{footnotesize}}\\														
\multicolumn{7}{p{14.2cm}}{\setstretch{0.9}\begin{footnotesize}\textsuperscript{a}The implied ratio is a statistic proposed by \citet{Altonjietal2005} to determine the necessary ratio of selection on unobservables to selection on observables in order to explain away the entire OLS coefficient of interest. This is discussed in section \ref{scn:selection} \end{footnotesize}}\\														
\end{tabular}														
\end{center}														
\end{table}														






\subsection{Fertility Selection}
\label{scn:selection}
Sections \ref{scn:MCS} and \ref{scn:EE} demonstrate that the incidence of non-observed twin predictor variables will invalidate the instrumental variable identification strategy.  
Monte Carlo simulations suggest that even a relatively minor correlation between these unobserved terms and the stochastic component of the Q-Q equation can result in estimators 
which perform worse than traditional OLS estimators.  However, if we are able to collect data on these variables which influence twin-selection,\footnote{I refer to twin ``selection" 
here not because families choose whether or not to have twins, but rather because certain characteristics make it appear as if families are more likely to select in to multiple 
births.  Once again, a parallel may be drawn with the use of twins as an instrument in situations where IVF treatment is available; in this case it seems less farfetched to speak 
of ``selection'' of multiple births.  The usefelness of this term will become apparent in the analysis which follows (for the OLS analysis), which draws from prior literature on 
selection on observables versus selection on unobservables.} a 2SLS methodology can still be used consistently by including these predictors as controls in the first and second 
stage.  This is a technique employed in previous literature which has included controls for parity, race and mother's age, along with (more recently) mother's education.  The 
credibility of this strategy however relies on the ability to perfectly observe determinant factors which increase the propensity of twin birth.  If we believe that models such 
as (\ref{eqn:twinpred}) are structural representations of twin birth, we can simply include all relevant predictors, and recover consistent estimates of family size.

It seems unlikely, however, that survey data will allow for all such relevant predictors to be observed.  While mother's weight and height and family socio-economic characteristics 
are observable, many aspects of a mother's pre-natal behaviour are not observed by the econometrician.  Dietary habits and other household investments in (pre-natal) child health 
may be both difficult to observe, and correlated with child quality and quantity preferences.  In this case, a problem of selection on unobservables exists for twin births.  The 
inclusion of observable predictors of twin births in the 2SLS specification---although partially correcting for the resulting inconsistency--- will not correct for it completely.

I follow recent work of \citet{Altonjietal2005, Altonjietal2008}\footnote{And particularly an application by \citet{BellowsMiguel2008}.} who propose an alternative method to examine the significance of a coefficient of interest in the presence of 
potential selection on unobservables.  This involves determining the degree of selection on unobservables which must occur to force the OLS estimate of $\beta_1$ in equation 
(\ref{eqn:2SLSa}) to zero.  This acts to bypass the endogeneity problem which has been demonstrated to exist in the twin birth 2SLS strategy by directly addressing selection on 
unobservables which exists in the Q-Q equation. Suppose that selection of family size is represented by $SELEC$, and that this depends upon a vector of observed factors, $\vect{X}$, and an unobserved residual $\widetilde{SELEC}$:  %In this case, I am interested in determing the degree of unobserved selection which must occur in a family's fertility decision 
\begin{equation}
SELEC=\vect{X'\gamma}+ \widetilde{SELEC}.
\end{equation}
Further, assume that the effect of the controls $\vect{X}$ are only related to total fertility through their effect on $SELEC$.   In order to consistently estimate the effect of 
family size on child quality we are interested in estimating equation 
(\ref{eqn:selectionAC}).  However, given that only a subset of the selection variables are observed we are only able to estimate (\ref{eqn:selectionC}).
\begin{subequations}
\label{eqn:selection}
\begin{eqnarray}
Q_{ij}&=&\alpha SIZE_{j}+\beta SELEC_j+u_{ij} \label{eqn:selectionAC}\\
Q_{ij}&=&\alpha SIZE_{j}+\vect{X'\gamma}+u_{ij} \label{eqn:selectionC}\\
Q_{ij}&=&\alpha SIZE_{j}+u_{ij} \label{eqn:selectionNC}
\end{eqnarray}
\end{subequations}
Obtaining an estimate of $\alpha$ from (\ref{eqn:selectionC}) results in the following:
\begin{equation}
\label{eqn:biasC}
 p\lim \hat\alpha=\alpha + \gamma\frac{cov(SIZE,\widetilde{SELEC})}{var(SIZE)}
\end{equation}
which is inconsistent given the correlation between SIZE and the omitted portion of the selection term.  This estimate will fall between that from (\ref{eqn:selectionNC}), an estimable equation where
no controls are included, and that from (\ref{eqn:selectionAC}), the consistent Q-Q equation which cannot be estimated given the unobserved portion of selection.  It can then be shown that the
difference between estimates of $\alpha$ from (\ref{eqn:selectionC}) and (\ref{eqn:selectionNC}) is:
\begin{equation}
 \label{eqn:biasdif}
\hat\alpha_{Controls} - \hat\alpha_{NoControls} = \gamma\frac{cov(SIZE,\vect{X}'\beta)}{var(SIZE)}.
\end{equation}
As the objective is to determine the size of the correlation between non-observables and observables which would drive the estimate of $\alpha$ to zero, $\alpha$ in (\ref{eqn:biasC})
is set to zero and (\ref{eqn:biasC}) is divided by (\ref{eqn:biasdif}).  This results in the ratio of non-observables to observables; our statistic of interest:
\begin{equation}
 \label{eqn:Aratio}
\frac{\hat\alpha_{Controls}}{\hat\alpha_{Controls} - \hat\alpha_{NoControls}}=\frac{cov(SIZE,\widetilde{SELEC})}{cov(SIZE,\vect{X}'\beta)}
\end{equation}

This ratio describes how strong the covariance must be between the unobserved part of the family selection term and total fertility compared to the covariance between the observed
part and fertility in order for the true effect of fertility on quality to be zero.  \citet{Altonjietal2005} suggest that if this ratio is high and if the observed controls are
representative of the full set of controls, it is unlikely that the true effect is zero given that an important proportion of explanatory power must be contained in the unobservable 
characteristics.  Based upon an enumerated set of assumptions, Altonji et al.\ argue that the a ratio exceeding 1 suggests that the true effect is not zero, however this depends 
fundamentally on the underlying mechanism of selection in the economic process under study and the quantity of observed characteristics in the sample data.

A set of estimates of the Altonji et al.\ ratio (or implied ratio) are provided in column f of table \ref{tab:fertilityALL}.  These results do not provide overwhelming evidence
that additional siblings affect school attendance, but do suggest that higher family size results in lower overall attainment of schooling, both when looking at completed years of schooling, and the standardized variable which compares each child to their country--age cohort.  It is particularly notable that this ratio greatly exceeds 1 in the low income sample of
countries, providing evidence in favour of some Q-Q tradeoff in a low income setting.  An implied ratio greater than 2, as is the case for years of schooling and school z-score in low income countries, implies that observed characteristics such as parental education, place of residence and maternal health can only explain one third of the family size decision, while unobservables must explain at least two thirds.  Such a result is consistent with one of two explanations; either family size decisions are inherently more `unobservable' in low income countries, or the evidence in favour of a Q-Q trade-off is greater in these settings.  Given that there seeems to be no particular reason to suppose that the family fertility decisions are more unobservable in a low-income setting by a sufficient degree to explain away the entire difference in the implied ratio,\footnote{It is possible to suggest some mechanisms by which fertility in low income countries may be more unobservable: fertility may depend upon family labour patterns or perceived health risk, however these would need to account for more than two times the unobservables in lower-middle-income countries.} this provides some evidence that the Q-Q tradeoff may be more than a simple artefact when incomes are low. 

\section{Conclusion}
\label{scn:conclusion}
This essay examines the plausibility of the quantity-quality tradeoff in a developing country setting using a particularly powerful dataset which spans 44 developing countries over 15 years.  In this way, it has added to the previously scant literature examining the Q-Q hypothesis in the developing world.  Fundamentally, it is suggested that instrumental strategies which utilize twins as an exclusion restriction are likely to be invalid, given that multiple births do not appear to be random, depending instead upon mother's weight and BMI, along with other factors previously tested in the literature: education, maternal age, and parity.  Given that this study focuses only on developing countries, it does not suggest that prior instrumentation strategies using twins in a \emph{developed} country setting are invalid, given that higher prenatal investment and more comprehensive public policies may act to compensate children whose mother is less healthy or less educated.  It does however suggest that tests using similar variables in a developed country setting are warranted.  These results may also  be extended when thinking about multiple births in the presence of IVF treatment.

Notwithstanding this concern of endogeneity in multiple births, evidence is provided which suggests that there may be some tradeoff between family size and years of schooling when examining fertility decisions in poor countries.  Recent work by Altonji et al.\ was applied to the Q-Q model, suggesting that nonobservables must be more than two times as important as observables in this model to explain away the entire effect of family size on child `quality'.


\newpage
\bibliography{BiBBase}

\newpage
\appendix
\section{Twin Estimation Strategies}
\label{scn:litrev}

\vspace{19.2cm}	

\begin{center}
\begin{rotate}{90}
\begin{tabular}{lp{4mm}lll}\toprule
Author& &  Estimation Strategy & \multicolumn{2}{c}{Controls} \\ \midrule
&& & \hspace{5mm}1\textsuperscript{st} Stage Controls & 2\textsuperscript{nd} Stage Controls \\  \cmidrule(r){4-4} \cmidrule{5-5}
Rosenzweig and Wolpin (1980) & &
$ED=\omega_0+\omega_1 TR+u_1$&
\begin{small}None.\end{small}&
\begin{small}N\slash A\end{small}.
\\
\begin{footnotesize}\end{footnotesize}&\begin{footnotesize}\end{footnotesize}&\begin{footnotesize}\end{footnotesize}&\begin{footnotesize}\end{footnotesize}&\begin{footnotesize}\end{footnotesize}\\
\multirow{2}{*}{Black \emph{et al.} (2005)} & &
(1) $FSIZE=\alpha_0 + \alpha_1 T + X\alpha_2 + v$ &
\begin{small}Full family size and  birth \end{small}&
\begin{small}Covariates from\end{small}
\\
& &
(2) $ED=\beta_0+\beta_1 FSIZE + X\beta_2 + \varepsilon$ &
\begin{small}order dummies. Demogra-\end{small}  &
\begin{small}stage 1.\end{small}
\\
& & &
\begin{small}phic controls: age, sex, pa-\end{small}&
\\
& & &
\begin{small}rental age and education.\end{small}&
\\
\begin{footnotesize}\end{footnotesize}&\begin{footnotesize}\end{footnotesize}&\begin{footnotesize}\end{footnotesize}&\begin{footnotesize}\end{footnotesize}&\begin{footnotesize}\end{footnotesize}\\
\multirow{2}{*}{C\'aceres (2006)} & &
(1) $n_i=X'_i\mu + \rho mb_i + v_i$ &
\begin{small}Dummies by age, state of\end{small}&
\begin{small}Covariates from\end{small}
\\
& &
(2) $y_i=\alpha + \gamma n_i + X'_i\beta + \varepsilon_i$ &
\begin{small}residence, mother's educ- \end{small}&
\begin{small}stage 1.\end{small}
\\
& & &
\begin{small}ation, race, mother's age\end{small}&
\\
& & &
\begin{small}and sex.\end{small}&
\\
\begin{footnotesize}\end{footnotesize}&\begin{footnotesize}\end{footnotesize}&\begin{footnotesize}\end{footnotesize}&\begin{footnotesize}\end{footnotesize}&\begin{footnotesize}\end{footnotesize}\\
\multirow{2}{*}{Angrist \emph{et al.} (2006)} &  &
(1)  $c_1=X'_i\beta+\alpha t_{2i}+\eta_i$ &
\begin{small}Mother's age, mother's \end{small} & 
\begin{small}Covariates $X_i$\end{small}
\\
& &
(2) $y_i=W'_i\mu+\rho c_i + \varepsilon_i$ &  
\begin{small}age at first birth, census \end{small} &
\begin{small}from stage 1.\end{small}
\\
& & &
\begin{small}year, parental birth place.\end{small}&
\\
\bottomrule 
\multicolumn{5}{p{19.2cm}}{\setstretch{0.9}\begin{footnotesize}\textsc{Note:} Nomenclature employed here is as faithful as possible to the original studies.  $TR$ refers to twin ratio, the number of twin births per mother divided by number of pregnancies, and $ED$ represents a child's standardised educational attainment. $FSIZE$ refers to sibship size and $T$ a dummy for twin birth on the $n$\textsuperscript{th} birth, restricting the analysis to children with birth order $<n$. The variable $n_i$ defined by C\'aceres once again refers to sibship size, and $mb_i$ multiple births. Finally, $c_i$ refers to child $i$'s sibship size, with $t_{2i}$ denoting multiple second births.\end{footnotesize}}\\
\end{tabular}
\end{rotate}
\captionof{table}{Prior empirical specifications}
\label{tab:litrev}
\end{center}
\setstretch{1.25}

\section{Full Survey Data}
\label{scn:surveys}
\vspace{-3mm}
\begin{table}[ht!]
\caption{Surveys with data for education}
\vspace{-7mm}
\label{tab:survey}
\begin{center}
\begin{tabular}{l l c c c c}
\multicolumn{6}{c}{\textsc{and household characteristics}}\\
& & & & &  \\
\toprule
Country	& Income &	Wave 1	&	Wave 2	&	Wave 3	&	Wave 4	\\
\midrule
Armenia	&	LM		&	2000	&	2005	&		&		\\
Bangladesh	&	LI	&	1993	&	1996	&	2004	&		\\
Benin	&	LI		&	1996	&	2001	&		&		\\
Bolivia	&	LM		&	1994	&	1998	&	2003	&		\\
Burkina Faso	&	LI &	1992	&	1998	&	2003	&		\\
Cambodia	&	LI	&	2000	&	1991	&	1998	&	2004	\\
Chad	&	LI		&	1996	&	2004	&		&		\\
Comoros	&	LI		&	1996	&		&		&		\\
Congo	&	LM		&	2005	&		&		&		\\
Cote D'Ivoire	&	LM	&	1994	&	1998	&		&		\\
Dominican Republic	&	UM	&	1991	&	1996	&	2002	&		\\
Egypt	&	LM	&	1992	&	1995	&	2000	&	2005	\\
Ghana	&	LM		&	1993	&	1998	&	2003	&		\\
Guinea	&	LI		&	1999	&	2005	&		&		\\
Haiti	&	LI		&	1994	&	2000	&		&		\\
Honduras	&	LM	&	2005	&		&		&		\\
India	&	LM		&	1999	&		&		&		\\
Indonesia	&	LM	&	1994	&	1997	&		&		\\
Kazakhstan	&	UM	&	1995	&		&		&		\\
Kenya	&	LI		&	1993	&	1998	&	2003	&		\\
Lesotho	&	LM		&	2004	&		&		&		\\
Madagascar	&	LI	&	1992	&	1997	&	2004	&		\\
Malawi	&	LI		&	1992	&	2000	&	2004	&		\\
Mali	&	LI		&	1995	&	2001	&		&		\\
Morocco	&	LM		&	1992	&	2003	&		&		\\
Mozambique	&	LI	&	1997	&	2003	&		&		\\
Namibia	&	UM		&	1992	&	2000	&		&		\\
Nepal	&	LI		&	1996	&	2001	&		&		\\
Nicaragua	&	LM	&	1997	&	2001	&		&		\\
Niger	&	LI		&	1992	&	1998	&		&		\\
Nigeria	&	LM		&	1999	&	2003	&		&		\\
Pakistan	&	LM	&	1991	&		&		&		\\
Peru	&	UM		&	1996	&	2000	&		&		\\
Philippines	&	LM	&	1993	&	1998	&	2003	&		\\
Rwanda	&	LI		&	1992	&	2000	&	2005	&		\\
Senegal	&	LM		&	1992	&	2005	&		&		\\
South Africa	&	UM	&	1998	&		&		&		\\
Tanzania	&	LI	&	1992	&	1996	&	2004	&		\\
Togo	&	LI		&	1998	&		&		&		\\
Turkey	&	UM		&	1993	&	1998	&		&		\\
Uganda	&	LI		&	1995	&	2000	&		&		\\
Vietnam	&	LM		&	1997	&	2002	&		&		\\
Zambia	&	LM		&	1992	&	1996	&	2001	&		\\
Zimbabwe	& LI	&	1994	&	1999	&		&		\\
\midrule
\multicolumn{6}{p{10.5cm}}{\footnotesize\textsc{Note:} LI refers to Low Income, LM to Lower-Middle income, and UM to Upper-Middle.
Classifications are according to the World Bank (2012).}\\

\bottomrule
\end{tabular}
\end{center}
\end{table}

\section{Graphical Results by Country Group}
\label{scn:Graphs}
\begin{figure}[!htbp]
\caption{Years of Schooling Estimates by Sibship and Country Group}
\label{fig:YrS}
\begin{center}
\vspace{-4mm}
\textsc{A: Years of Schooling}
\includegraphics[scale=0.42]{YrsSch_results1.png}
\end{center}
\begin{center}
\vspace{4mm}
\textsc{B: Attendance}
\includegraphics[scale=0.42]{Attend_results1.png}
\end{center}
\begin{footnotesize}\textsc{ Notes to figures:} Estimates are presented by country groups.  Six groups of point estimates and confidence intervals 
are provided corresponding to basic controls (OLS NC and IV NC), basic and socioeconomic controls (OLS S and IV S), and full controls including 
socioeconomic and health variables (OLS H and IV H).  Point estimates are represented by circles, and confidence intervals by vertical bars (these 
are very small in the case of OLS estimates.)  The colour of these symbols represents the magnitude of each point estimate.  For each class of 
estimates and countries, results are presented for the 2+, 3+, 4+, 5+ 6+ and 7+ group (from left to right).  Subfigures present $\hat{\beta}_{fertility}$ 
for different quality variables; years of schooling (A), attendance (B) and school Z-score (C).  Further details are available in section 
\ref{scn:data} (Data).  \end{footnotesize}
\end{figure}

\begin{figure}[!htbp]
%\caption{Years of Schooling Estimates by Sibship and Country Group}
\begin{center}
\vspace{4mm}
\textsc{C: Schooling Z-Score}
\includegraphics[scale=0.42]{ZScore_results1.png}
\end{center}
\begin{footnotesize}\textsc{ Notes to figures:} See preceding page. \end{footnotesize}
\end{figure}
\end{spacing}
\end{document}
\let\clearpage\relax

\section{Deriving Twin Selection}
\label{scn:TSAppend}
While the concern that (\ref{eqn:IVconsist3}) doesn't hold is inherently untestable, it is possible to estimate
the importance of unobservable twin predictors in the Q-Q model as follows.  It is proposed that the Q-Q model 
can be represented by three equations:
\begin{subequations}
\label{eqn:Sderiv}
\begin{eqnarray}
\label{eqn:QQ}
Q_{ij}&=&\beta_0+\beta_1Fert_j+\vect{X}'_j\vect{\beta}+u_{ij} \label{eqn:Sderiva}\\
Fert_{j}&=&\gamma_0+\gamma_1Twin_{i}+\vect{X}'_j\vect{\gamma}+v_{j} \label{eqn:Sderivb} \\
Twin_i&=&\delta_0+\vect{W}'_i\vect{\delta}+\varepsilon_i. \label{eqn:Sderivc}
 \end{eqnarray}
\end{subequations}
As demonstrated in section \ref{scn:MCS}, $\beta_1$ can be estimated consistently via 2SLS if $\text{cov}(Twin_j, u_i)=0$,
which reduces here to $\text{cov}(\varepsilon_j, u_i)=0$.  However, given that the probability of live twin births depend 
upon a mother's observable characteristics and behaviours ($\delta\neq 0$), there is concern that twin births are also 
driven by unobservables $\varepsilon_i$.  This will be a significant concern if these unobservables vary in some predictable
way with child quality.

As fertility is instrumented with the twin dummy, (\ref{eqn:Sderivb}) can be written as:
\begin{eqnarray}
Fert_{j}&=&\gamma_0+\gamma_1[\delta_0+\vect{W}'_i\vect{\delta}+\varepsilon_i]+v_{j}  \nonumber \\
&=& (\gamma_0+ \gamma_1\delta_0)+\vect{W}'_i\gamma_1\vect{\delta}+\gamma_1\varepsilon_i+v_{j} \label{eqn:Sderiv2a}
\end{eqnarray}
where for simplicity the vector $\vect{X}$ is omitted from this and subsequent specifications.  Equation (\ref{eqn:Sderiv2a}) is simply the first 
stage of the 2SLS estimate.  This instrumented first stage can thus be used to estimate $\beta_1$:
\begin{eqnarray}
Q_{ij}&=&\beta_0+\beta_1[(\gamma_0+ \gamma_1\delta_0)+\vect{W}'_i\gamma_1\vect{\delta}+\gamma_1\varepsilon_i+v_{j}]+u_{ij} \nonumber \\
&=& (\beta_0+\gamma_0+ \gamma_1\delta_0)+\vect{W}'_i\beta_1\gamma_1\vect{\delta}+\beta_1\gamma_1\varepsilon_i+\beta_1v_{j}+u_i \label{eqn:Sderiv2}
\end{eqnarray}
where $\beta_1\gamma_1\varepsilon_i+\beta_1v_{j}+u_i$ is the composite error term.  The concern in this essay has been that twin
births depend upon unobservable characteristics and behaviours of the family in a manner which will bias estimates of the effect of fertility on 
quality ($\beta_{1,IV}$).
We are thus interested in testing whether unobservables in the twin equation $\varepsilon_i$ are correlated with $Q_{ij}$, or more precisely
whether $\beta_1\gamma_1\neq 0$ in (\ref{eqn:Sderiv2}).\footnote{Presumably if twins are more likely to be born to mothers who
invest more while the fetus is \emph{in utero}, we will be concerned that $\beta_1\gamma_1> 0$.}  Given that $\varepsilon_i$ is unobserved,
this cannot be tested directly.  However, we \emph{do} observe both $\beta_1\gamma_1\delta$ in (\ref{eqn:Sderiv2}), along with $\delta$ from
(\ref{eqn:Sderivc}).  This allows us to indirectly calculate the coefficient on $\varepsilon_i$ as:
\begin{equation}
\label{eqn:Sderiv3}
\widehat{\beta_1\gamma_1}=\frac{\widehat{\beta_1\gamma_1\delta}}{\hat{\delta}}=\frac{\widehat{\beta_1\gamma_1\delta}}{\widehat{Twin}},
\end{equation}
providing an estimate of the degree of correlation between unobserved twin predictors and child quality.



________________________________________________________________________________________________________________________________
________________________________________________________________________________________________________________________________
________________________________________________________________________________________________________________________________
________________________________________________________________________________________________________________________________
________________________________________________________________________________________________________________________________
________________________________________________________________________________________________________________________________
________________________________________________________________________________________________________________________________
________________________________________________________________________________________________________________________________
________________________________________________________________________________________________________________________________
________________________________________________________________________________________________________________________________
________________________________________________________________________________________________________________________________
________________________________________________________________________________________________________________________________
________________________________________________________________________________________________________________________________
________________________________________________________________________________________________________________________________
It we suppose that twin births depend upon observable and unobservable
characteristics:
\begin{equation}
\label{eqn:Sderiv}

\end{equation}
where we are concerned that unobserved factors $\varepsilon$ are correlated with child quality.  Given the two-step estimation strategy
described in (\ref{eqn:2SLSa}) and (\ref{eqn:2SLSb}), the fertility term in (\ref{eqn:2SLSa}) by the first stage regression as follows:



































Naive OLS is
likely to overestimate the importance of the degree of causality.  Where unobserved factors such
as parental preferences drive both quality of children and household size, estimates of the
trade-off will be biased.  Various plausible channels for this mechanism have been proposed, such
as a preference for lower children by more highly educated parents (Qian, 2005). Frequently, data
on multiple births is used to recover a causal estimate of the Q-Q trade-off. Rosenzweig and Wolpin
(1980) discussed the theoretical validity of instrumenting for family size ($N$) with twin birth ($t$),
with this method subsequently receiving considerable treatment in the empirical literature.  The
validity of this estimation strategy relies upon typical IV assumptions, namely
\begin{equation}
\label{eqn:IV}
\begin{aligned}
\text{cov}(t,N)\neq0 \\
\text{cov}(t,u)=0
\end{aligned}
\end{equation}
where u refers to the unobserved error term in the QQ model:
\begin{equation}
\label{eqn:model}
Q_i=\alpha N_i + X_i'\beta + u_i.
\end{equation}

I propose to examine the exogeneity of the instrumental strategy outlined in (\ref{eqn:IV}).
I will ask whether multiple births truly are an exogenous shock to family size, or whether these are
influenced by other socioeconomic factors not considered in prior studies.  Fundamentally, I am
interested in determining whether a mother's likelihood of giving birth to multiple infants depends upon
her household income ($y_i$), investment in own health ($h_i$), or other individual or household
characteristics such as marital status ($m_i$).  These results are quite simple to obtain using a limited
dependent variable regression model (\emph{e.g.}\ a probit) to estimate:
\begin{equation}
\label{eqn:twin}
P(t_i=1)=\gamma_1 y_i + \gamma_2 h_i + \gamma_3 m_i +X_i'\beta + \varepsilon_i.
\end{equation}
Prior literature has failed to take account of this threat to instrumental validity, generally assuming
that twin births are exogenous (beyond race and birth-order).

It seems plausible that even in the case that twin conception is exogenous to a household or mother's
socioeconomic characteristics, the likelihood that a mother carries to full term and gives birth to
multiple live infants will not be exogenously determined.  I am interested in testing the null that
$\gamma_1=0$, $\gamma_2=0$ and $\gamma_3=0$, individually and collectively against the alternative
that $\vec{\gamma}>0$.

If this hypothesis is borne out empirically, previous Q-Q model estimates obtained instrumenting for
multiple birth, but not cotrolling for a mother's socioeconomic characteristics, will be biased. In this
essay, the bias will be derived mathematically, and tested empirically.  Using data from Chile (see
section \ref{scn:data}) I will estimate (\ref{eqn:model}) via OLS, IV (not controlling for additional
mother characteristics) and IV controlling for these characteristics.  This will allow me to empirically
determine if the bias in previous estimates exists, is important, and agrees with theoretical
derivations.



\setstretch{1.1}
\begin{eqnarray}
\label{eqn:MC1}
y_i=\beta_0+\beta_1 x_i + \varepsilon_{1i} \nonumber\\
x_i=\gamma_0 + \gamma_1 z_i + \varepsilon_{2i} \nonumber\\
z_i=1[\alpha_0+\varepsilon_{3i}>0], \nonumber
\end{eqnarray} 
where simulated $\varepsilon_j$ follow:
\begin{equation}
\left(
\begin{array}{c}
\varepsilon_1\\
\varepsilon_2\\
\varepsilon_3\\
\end{array}
\right)
\sim \mathcal{N}\left(
\Biggl(
\begin{array}{c}
0\\
0\\
0\\
\end{array}
\Biggl),
\Biggl(\begin{array}{c}
\sigma_{\varepsilon_1}^2 \quad  \rho_{\varepsilon_1\varepsilon_2} \quad  \rho_{\varepsilon_1\varepsilon_3}\\
\rho_{\varepsilon_2\varepsilon_1} \quad \sigma_{\varepsilon_2}^2 \quad  \rho_{\varepsilon_2\varepsilon_3} \\
\rho_{\varepsilon_3\varepsilon_1}\quad  \rho_{\varepsilon_3\varepsilon_2}\quad \sigma_{\varepsilon_3}^2 \\
\end{array}\Biggl)
\right)
\end{equation}
\subsubsection{Empirical Evidence}
\label{scn:EE}
\setstretch{1.25}
