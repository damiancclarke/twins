The use of twin births to address the problem of endogeneity in the Q-Q model seems to have been initially proposed by \citet{RosenzweigWolpin1980}.  They derive the theoretical requirements to estimate the size of the trade-off when the shadow price of child quality depends on the number of children and vice versa.  By relying on the assumption that multiple births are an exogenous shock to family size (once accounting for the total number of a mother's pregnancies) they estimate the effect of a twin birth upon the educational attainment of children in the twins' family.  This and alternate empirical specifications discussed in this section are decribed in table \ref{tab:litrev} in appendix \ref{scn:litrev}.  

Subsequent papers employing a twin-birth methodology have proposed a number of strategies which enable them to obtain consistent estimates of the Q-Q trade-off while relaxing Rosenzweig and Wolpin's exogeneity assumption.  \citet{Blacketal2005} extend the controls to account for the fact that the probability of multiple birth increases with maternal age as described by \citet{Jacobsenetal1999} and others.  They include a set of parental age and education controls, however note that they are unable to reject the hypothesis that parental education has no effect on the probability of multiple birth.  Likewise, \citet{Caceres2006} includes controls for mother's age, race, and education, suggesting that the use of these and pre-1980 US Census data should be sufficient to approximate conditional exogeneity\footnote{The use of pre-1980 Census data seems important as this predates the widespread introduction of fertility drugs.  The use of fertility drugs is associated with higher a probability of multiple births, and resulting concerns that the orthogonality assumption will be violated if users of fertility treatment are non-random.}.  Finally, \citet{Angristetal2010} recognise that twin birth varies with maternal age at birth and race, including twinning as one of three instruments. 

Angrist et al.\ join recent work from \citet{RosenzweigZhang2009} in questioning the validity of twin instrumentation in another sense.  These authors suggest that the error term in the Q-Q equation is unlikely to be orthogonal to the instrument given that twinning imposes predictable and unobserved family responses in investment decisions.  Particularly, these studies question the effect that close birth spacing and an endowment effect responding to the lower health at birth of twins compared to single births\footnote{Using data from the United States, \citet{Almondetal2005} document that twins have substantially lower birth weight, lower APGAR scores, higher use of assisted ventilation at birth and lower gestion period than singletons.} has on investments in pre-twin siblings.  Despite this critique, Rosenzweig and Zhang suggest that it is still possible to compute an upper and a lower bound of the Q-Q trade-off.

More recently, instrumentation using twin births has been applied to estimate a Q-Q model in the developing world. \cite{FitzsimonsMalde2010}, \citet{Sanhueza2009} and \citet{Lietal2008} have applied a similar methodology to that of Angrist et al., examining twin births in Mexico, Chile and China respectively.  These studies find mixed results depending upon the country under examination and once again, while considering the invalidity of the twin exclusion restriction in the Q-Q model in terms of maternal education, do not examine this in terms of non-random twin births due to maternal health. 














%A number of instruments have been proposed to isolate the effect of family size on child quality.  These aim to identify a valid exclusion restriction which acts as a %shock to child quantity, without otherwise affecting parental preferences or other unobservables.  \citet{ConleyGlauber2006} propose that gender mix acts as an %appropriate instrument, with parents more likely to stop having children if the first two births are of opposing genders.  \citet{Lee2008} also proposes a %gender-%based instrument, using the son preference in South Korean households as an indicator of the likelihood of subsequent births, while \citet{Qian2009} takes %advantage of a relaxation of family planning policies in China. 















%Angrist Lavy Schlosser: Because twins probably differ from nontwins for reasons both observed and unobserved, we prefer empirical strategies that look at the effects %of twins on older siblings.
%On balance, the results reported here offer little evidence for an effect of family size on schooling, work, or earnings, although we do find some effects on girls' marital %status, age at marriage, and fertility
%787-788 worthwhile discussion of compliers etc.
%"Because twinning is as good as randomly assigned, causal effects for the latter population [the treated] are the same as causal effects on all compliers."
%RE INSTRUMENTAL VALIDITY AND EXCLUSION RESTRICTION:
%We also replicated the common finding that twin births are associated with older maternal age. For example, the mothers of first and second borns who had twins at
%second or third birth were .3–.5 years older at first birth than those who had singletons. Twinning is not otherwise associated with subject demographics, with one %exception: in the 1995 sample of 2+ subjects, twin rates are higher for younger cohorts.
%OUTCOMES
%The outcome variables measure human capital, economic wellbeing, and social circumstances. In particular, we look at measures of subjects’ educational attainment %(highest grade completed and indicators of high school completion and college attendance), labor market status (indicators of work last year and hours worked last %week), earnings (monthly earnings and the natural log of earnings for full-time workers), marital status (indicators of being married at census day and married by age %21), and fertility.
%INTUITION
%2SLS estimates of this equation capture siblings' weighted average response to the birth of an additional child for those whose parents were induced to have an %additional child by the instrument at hand.


%BlackDevereuxSalvanes
%Q-Q MODEL
%A key element of the quantityquality model is an interaction between quantity and quality in the budget constraint that leads to rising marginal costs of quality
%with respect to family size; this generates a trade-off between quality and quantity.
%Casual evidence suggests that children from larger families have lower average education levels. However, is it true that having a larger family has a causal effect on the %“quality” of the children? Or is it the case that families who choose to have more children are (inherently) different, and the children would have lower education
%regardless of family size?
%IMPORTANCE OF BIRTH ORDER
%Like most previous studies, we find a negative correlation between family size and children’s educational attainment. However, when we include indicators for birth order, %the effects of family size are reduced to almost zero.  The evidence suggests that family size itself has little impact on the quality of each child but more likely impacts %only the marginal children through the effect of birth order
%In the estimates so far, we may be confounding the effects of family size with those of birth order. (p. 680)...  The family size effects are reduced to close to zero—the %coefficient on the linear term is now 0.01—with the addition of the birth order dummies.
%WHO TO EXAMINE
%The TWIN indicator is equal to 1 if the nth birth is a multiple birth and equal to 0 if the nth birth is a singleton. We restrict the sample to families with at least n births and %study the outcomes of children born before the nth birth. (see 681 for complete discussion of this)
%VALIDITY OF TWIN INSTRUMENT
%Important discussion on page 682.  Suggest that in Norway twinning NOT related to parent's education (estimate at each parity).  Suggest modern fertility drugs were %introduced later than the data in their sample
%TWINS MAY HAVE DIRECT EFFECT ON SIBLING OUTCOMES BEYOND JUST INCREASING FAMILY SIZE
%More generally, to the extent that twins are not excludable and have a negative direct effect on the educational attainment of the other children in the family (perhaps %because they are more likely to be in poor health), our estimates of the effect of family size using the twins instrument will be biased toward finding larger negative
%effects.
%FIRST STAGE
%The first stage is very strong and suggests that a twin birth increases completed family size by about 0.7 to 0.8. As expected, twins at higher parity have a larger effect %on family size, presumably because they are more likely to push families above their optimal number of children.
%TWIN RESULTS (p. 683)
%OUTCOMES
% Principal: education
%Labour market: We examine the effects of family size and birth order on the following labor market outcomes: the earnings of all labor market participants, the earnings %of full-time employees only, and the probability of being a full-time employee
%Also examine prob of having a teen birth

%Caceres (2006)
% Description of rescuing twin births without twin identifier (pp. 741-742)
%WHY RESTRICT SAMPLE TO OLDER?
%Therefore, the sample is restricted to oldest siblings in the household that are not from a multiple birth, since being part of a multiple birth or being a younger sibling of %twins or other higher order multiple birth is conditional on the occurrence of multiple births in the household (post-treatment).
% INSTRUMENT VALIDITY
%Despite the fact that the orthogonality assumption is nontestable (orthogonality of multiple births to child quality preferences), the random nature of multiple births, %the choice of the observational unit under analysis (oldest child in the household who does not belong to a multiple birth), the selection of the 1980 U.S. Census as %data set,8 the inclusion of other variables that are correlated with the incidence of multiple births such as age of the mother, race, and mothers’ education (Table 2), %as well as the analysis of the impact of twinning in a specific birth make it more likely that this assumption holds

%Rosenzweig & Wolpin (1980) 
% EXOGENEITY OF TWINS (see top of p. 237)
%Nevertheless, we ran Tobit regressions (because of the concentration of observations at zero) of TR on farm size, non-earnings income, and the schooling levels of %each parent based on our total sample of farm families. Application of the chi2 test indicated that we could not reject the hypothesis of joint insignificance at %conventional levels (chi2(4) = 2.49). The same set of variables, however, explained a statistically significant (F4,1628 = 5.17) but small (R2 = 005) proportion of the %variance in children ever born.
% USE a twins variable which standardises for parity as an approximate way to correct for the fact that families with higher than avearge number of children are likely 
%to select in to multiple births.



